\chapterimage{chap36.jpg}
\chapter{October}
\section{Week \Rmnum{1}}
\textcolor{orange}{October 1}

\begin{example}[][Exam: 36.1.1]
	设 $\displaystyle{I=\int_{0}^{\sqrt{\pi}}x\cos x^2dx}$,
$\displaystyle{J=\int_{0}^{\sqrt{\pi}}\cos x^2dx}$,
$\displaystyle{K=\int_{0}^{\pi}\sqrt{x}\cos xdx}$, 比较 $I,J,K$ 的大小
\end{example}

\begin{solution}

	\begin{eqnarray*}
		I & = & \int_{0}^{\sqrt{\pi}}x\cos x^2dx=\dfrac{\sin x^2}{2}|_{0}^{\sqrt{\pi}}=0 \\
		J & = & \int_{0}^{\sqrt{\pi}}\cos x^2dx=\int_{0}^{\pi}\dfrac{\cos t}{2\sqrt{t}}dt>0 \\
	    K & = & \int_{0}^{\pi}\sqrt{x}\cos xdx<0
	\end{eqnarray*}

	综上所述,我们可以得到: $ K < I < J$
\end{solution}

\begin{example}[][Exam: 36.1.2]
	已知曲线 $L: y=x^2-1(-1\leq x\leq 2)$, 方向从 $A(-1,0)$ 到 $B(2,3)$,
求 $\displaystyle{\int_{L}\dfrac{xdy-ydx}{x^2+y^2}}$
\end{example}

\begin{solution}

	\begin{eqnarray*}
		\int_{L}\dfrac{xdy-ydx}{x^2+y^2}
		& = & \int_{-1}^{2}\dfrac{x^2+1}{x^2+(x^2-1)^2}dx\\
		& = & \int_{-1}^{2}\dfrac{x^2+1}{x^4-x^2+1}dx\\
		& = & \int_{-1}^{2}\dfrac{1+\frac{1}{x^2}}{x^2+\frac{1}{x^2}-1}dx\\
		& = & \int_{-1}^{2}\dfrac{1}{1+(x-\frac{1}{x})^2}d(x-\frac{1}{x})\\
		& = & \int_{-1}^{0^{-}}\dfrac{1}{1+(x-\frac{1}{x})^2}d(x-\frac{1}{x})+\int_{0^{+}}^{2}\dfrac{1}{1+(x-\frac{1}{x})^2}d(x-\frac{1}{x})\\
		& = & \arctan(x-\frac{1}{x})\big|_{x=-1}^{x=0^{-}}+\arctan(x-\frac{1}{x})\big|_{x=0^{+}}^{x=2}\\
		& = & \arctan\dfrac{3}{2}+\pi
	\end{eqnarray*}
\end{solution}


\textcolor{orange}{October 2}

\begin{example}[][Exam: 36.1.3]
	$f(x)=\lim\limits_{n\to+\infty}\sqrt[n]{1+x^n+\left( \dfrac{x^2}{2}\right) ^n}(x>0)$, 求 $\int f(x)dx$
\end{example}

\begin{solution}
 
	$$f(x)=
	\begin{cases}
		1, &x\in(0,1]\\
		x, &x\in(1,2]\\
		\dfrac{x^2}{2}, &x\in(2,+\infty)  
	\end{cases}
	\Rightarrow 
	\int f(x)=
	\begin{cases}
		x+C, &x\in(0,1] \\
		\dfrac{x^2+1}{2}+C, &x\in(1,2] \\
		\dfrac{x^3+7}{6}+C, &x\in(2,+\infty)
	\end{cases}$$
\end{solution}

\begin{example}[][Exam: 36.1.4]
	$f(x)$ 连续,$f(x+2)-f(x)=\sin x, \displaystyle{\int_{0}^{2}f(x)dx=0}$, 求 $\displaystyle{\int_{1}^{3}f(x)dx}$
\end{example}

\begin{solution}

	\begin{eqnarray*}
		\int_{1}^{3}f(x)dx
		& = & \int_{1}^{3}f(x)dx-\int_{0}^{2}f(x)dx \\
		& = & \int_{2}^{3}f(x)dx-\int_{0}^{1}f(x)dx \\
		& = & \int_{0}^{1}f(x+2)dx-\int_{0}^{1}f(x)dx \\
		& = & \int_{0}^{1}[f(x+2)-f(x)]dx \\
		& = & \int_{0}^{1}\sin xdx \\
		& = & 1-\cos 1
	\end{eqnarray*}
\end{solution}
\begin{anymark}[注]
	构造辅助函数: $\displaystyle{F(x)=\int_{x}^{x+2}f(t)dt}$  
	$$\begin{cases}
		F'(x) = f(x+2)-f(x) = \sin x\\
		F(0)  = 0
	\end{cases}\Rightarrow
	\begin{cases}
		F(x) = -\cos x+1 \\
		\displaystyle {F(1) = \int_{1}^{3}f(x)dx=1-\cos 1}
	\end{cases}$$
	\end{anymark}


\textcolor{orange}{October 3}

\begin{example}[][Exam: 36.1.5]
	设 $f(x)$ 在 $[0,1]$ 上连续,在 $(0,1)$ 内可导,且 $f(0)=1,f(1)=0$, 证明:

(1). $\exists \xi_{1},\xi_{2}\in(0,1)(\xi_{1}\neq \xi_{2}),\ s.t.\ f'(\xi_{1})+f'(\xi_{2})=-2$

(2). $\exists \eta,\zeta\in(0,1)(\eta\neq \zeta),\ s.t.\ f'(\eta)f'(\zeta)=1$
\end{example}

\begin{solution}

	(1). 取 $c=f(\dfrac{1}{2})$,我们在 $(0,\dfrac{1}{2})$ 和 $(\dfrac{1}{2},1)$上
	分别对 $f(x)$ 使用拉格朗日中值定理:  
	
	$$\begin{cases}
		\exists \xi_{1}\in(0,\dfrac{1}{2}),\ s.t.\ 2(c-1) = f'(\xi_{1}) \\
		\exists \xi_{2}\in(\dfrac{1}{2},1),\ s.t.\ -2c = f'(\xi_{2})
	\end{cases}
	\Rightarrow f(\xi_{1})+f(\xi_{2})=-2$$
	
	(2). 构造辅助函数: $F(x)=f(x)-x$  
	$$\begin{cases}
		F(0) = f(0) = 1 > 0 \\
		F(1) = f(1) - 1 = -1 < 0
	\end{cases}\Rightarrow \text{零点定理: }
	\exists c\in(0,1),\ s.t.\ F(c) = f(c) - c = 0$$
	
	对 $f(x)$ 分别在 $(0,c)$ 和 $(c,1)$ 上使用拉格朗日中值定理:  
	$$\begin{cases}
		\exists\eta\in(0,c),\ s.t.\ \dfrac{f(c)-1}{c} = f'(\eta)\\
		\exists\zeta\in(c,1),\ s.t.\ \dfrac{-f(c)}{1-c} = f'(\zeta)
	\end{cases}\Rightarrow 
	f'(\eta)f'(\zeta) = \dfrac{c-1}{c}\dfrac{-c}{1-c} = 1$$
\end{solution}

\begin{example}[][Exam: 36.1.6]
	设 $\displaystyle{f(x)=\int_{-1}^{x}t\cos tdt,x\in(-\dfrac{\pi}{2},\dfrac{\pi}{2})}$,
则曲线 $y=f(x)$ 与 $x$ 轴所围成的图形面积为:  
\begin{itemize}
	\item A. $2\int_{0}^{1}x\sin xdx$
	\item B. $2\int_{0}^{1}x^2\sin xdx$
	\item C. $2\int_{0}^{1}x\cos xdx$
	\item D. $2\int_{0}^{1}x^2\cos xdx$
\end{itemize}
\end{example}

\begin{solution}

	$f(x)$ 为偶函数,且 $f'(x)=x\cos x$, $f(1)=f(-1)=0$  

	$$\begin{cases}
		x\in(-\dfrac{\pi}{2},0) & f'(x)<0\\
		x\in(0,\dfrac{\pi}{2}) & f'(x)>0
	\end{cases}\Rightarrow
	 f(x) \text{在}(-\dfrac{\pi}{2},0)\text{上单调递减},
	 \text{在}(0,\dfrac{\pi}{2})\text{上单调递增}$$
	
	\begin{eqnarray*}
		S & = & |\int_{-1}^{1}f(x)dx|\\
		  & = & -2\int_{0}^{1}f(x)dx\\
		  & = & -2[xf(x)]\big|_{x=-1}^{x=1} + 2\int xf'(x)dx\\
		  & = & 2\int x^{2}\cos xdx
	\end{eqnarray*}
\end{solution}


\textcolor{orange}{October 4}

\begin{example}[][Exam: 36.1.7]
	设 $\varGamma$ 是柱面 $x^{2}+y^{2}=1$ 与平面 $z=x+y$ 的交线,
从 $z$ 轴正向往负向看去为逆时针, 计算积分 $\displaystyle{\oint_{\varGamma}xzdx+xdy+\dfrac{y^2}{2}dz}$
\end{example}

\begin{solution}

	我们有:
	$$\int_{0}^{2\pi}\sin^{2k+1}xdx = \int_{0}^{2\pi}\cos^{2k+1}xdx = 0(k=0,1,2,\cdots)$$
	
	令 $\begin{cases}
		x = \cos \theta \\
		y = \sin \theta \\
		z = \sin \theta + \cos \theta \\
		\theta \in [0,2\pi]
	\end{cases}$  曲线积分:
	\begin{eqnarray*}
		  I & = & \int_{0}^{2\pi}\left[ \cos\theta(\sin\theta+\cos\theta)d(\cos\theta)
		   	      + \cos\theta d(\sin\theta) + \dfrac{\sin^{2}\theta}{2}d(\sin\theta+\cos\theta)\right]\\
			& = & \int_{0}^{2\pi}\cos\theta(\cos^{2}\theta-1)d\theta + \int_{0}^{2\pi}(\sin^{2}\theta-1)\sin\theta
				  + \int_{0}^{2\pi}\cos^{2}\theta d\theta \\
			& + & \dfrac{1}{2}\int_{0}^{2\pi}(1-\cos^{2}\theta)\cos\theta d\theta - \dfrac{1}{2}\int_{0}^{2\pi}\sin^{3}\theta d\theta\\
			& = & \int_{0}^{2\pi}\cos^{2}\theta d\theta\\
			& = & \int_{0}^{2\pi}\dfrac{1+\cos2\theta}{2}d\theta\\
			& = & \pi
	\end{eqnarray*}
\end{solution}

\begin{example}[][Exam: 36.1.8]
	$$\int\dfrac{1}{\sin^{3} x+\cos^{3} x}dx$$
\end{example}

\begin{solution}  
	\begin{eqnarray*}
		I & = & \int \dfrac{1}{(\sin x+\cos x)(\sin^2 x+\cos^2 x-\sin x\cos x)}dx\\
		  & = & \int \dfrac{\sin x+\cos x}{(\sin x+\cos x)^{2}(\sin^2 x+\cos^2 x-\sin x\cos x)}dx\\
		  & = & \int \dfrac{d(\sin x-\cos x)}{(1+2\sin x+\cos x)(\sin^2 x+\cos^2 x-\sin x\cos x)}\\
		  & = & \int \dfrac{2d(\sin x-\cos x)}{\left[2-(\sin x-\cos x)^2\right]\left[1+(\sin x-\cos x)^2\right]}\\
		  & = & \int \dfrac{2du}{(2-u^2)(1+u^2)}\\
		  & = & \dfrac{2}{3}\left[\int\dfrac{1}{2-u^2}du+\int\dfrac{1}{1+u^2}du\right]\\
		  & = & \dfrac{2}{3}\left[\dfrac{1}{2\sqrt{2}}\ln|\dfrac{\sqrt{2}+u}{\sqrt{2}-u}|+\arctan u\right]\\
		  & = & \dfrac{1}{3\sqrt{2}}\ln\big|\dfrac{\sqrt{2}+\sin x-\cos x}{\sqrt{2}-\sin x+\cos x}\big|+\dfrac{2}{3}\arctan(\sin x-\cos x)+C
	\end{eqnarray*}
\end{solution}


\textcolor{orange}{October 5}

\begin{example}[][Exam: 36.1.9]
	$\displaystyle{F(x)=\int_{x}^{x+2\pi}e^{\sin t}\sin tdt}$, 求 $F(x)$
\end{example} 

\begin{solution}

	引入: $f(x)=e^{\sin x}\sin x$,$f(x)$ 为周期函数, 周期 $T = 2\pi$

	\begin{eqnarray*}
		F(x) & = & \int_{x}^{x+2\pi}e^{\sin t}\sin tdt\\
		     & = & \int_{x}^{0}e^{\sin t}\sin tdt + \int_{0}^{2\pi}e^{\sin t}\sin tdt + \int_{2\pi}^{x+2\pi}e^{\sin t}\sin tdt\\
		     & = & -\int_{0}^{x}e^{\sin t}\sin tdt + \int_{0}^{2\pi}e^{\sin t}\sin tdt + \int_{0}^{x}e^{\sin t}\sin tdt\\
		     & = & \int_{0}^{2\pi}e^{\sin t}\sin tdt\\
		     & = & \int_{0}^{\pi}(e^{\sin x}-e^{-\sin x})\sin xdx>0\\
	\end{eqnarray*}
	综上所述, $F(x)=C>0$
\end{solution}

\begin{example}[][Exam: 36.1.10]
	设 $f(x)$ 在 $[0,1]$ 上连续, 且 $\displaystyle{I=\int_{0}^{1}f(x)dx\neq 0}$, 证明:
$$\exists \xi,\eta\in(0,1)(\xi\neq \eta),\ s.t.\ \dfrac{1}{f(\xi)}+\dfrac{1}{f(\eta)}=\dfrac{2}{I}$$
\end{example}

\begin{solution}

	构造辅助函数: $\displaystyle{F(x)=\int_{0}^{x}f(t)dt}$,$F(0)=0,F(1)=I\neq 0$
	
	介值定理:  
	$$\exists c\in(0,1),\ s.t.\ \ F(c)=\dfrac{I}{2}$$
	
	$F(x)$ 在区间 $(0,c)$ 和 $(c,1)$ 上使用拉格朗日中值定理:  
	
	$$\begin{cases}
		\exists \xi  \in (0,c),\ s.t.\  \dfrac{F(c)}{c} = F'(\xi) = f(\xi) \\
		\exists \eta \in (c,1).\ s.t.\  \dfrac{F(1)-F(c)}{1-c} = F'(\eta) = f(\eta)
	\end{cases} \Rightarrow 
	\dfrac{1}{f(\xi)} + \dfrac{1}{f(\eta)} = \dfrac{2c}{I} + \dfrac{2(1-c)}{I} = \dfrac{I}{2}$$
	
	综上所述,我们得到: $\exists \xi,\eta\in(0,1)(\xi\neq \eta),\ s.t.\ \dfrac{1}{f(\xi)}+\dfrac{1}{f(\eta)}=\dfrac{2}{I}$
\end{solution}


\textcolor{orange}{October 6}

\begin{example}[][Exam: 36.1.11]
	设函数 $f(x)$ 在 $(-\infty,+\infty)$ 内有定义, 在 $(-\infty,0)\cup(0,+\infty)$ 内可导,
$\lim\limits_{x\to 0}f'(x)$ 存在,$f(x)$ 在 $x=0$ 处可导的充分条件为:  
\begin{itemize}
	\item A $\lim\limits_{x\to 0}\dfrac{f(x)}{x}$ 存在
	\item B $\lim\limits_{x\to 0}\dfrac{f'(x)}{x}$ 存在
	\item C $f(x)$ 在点 $x=0$ 处连续
	\item D $\int_{0}^{x}f(t)dt$ 在点 $x=0$ 处可导
\end{itemize}
\end{example}

\begin{solution}
	\begin{itemize}
		\item A $\lim\limits_{x\to 0}\dfrac{f(x)}{x}$ 存在 $\Rightarrow \lim\limits_{x\to 0}f(x)$ 存在
		\item B $\lim\limits_{x\to 0}\dfrac{f'(x)}{x}$ 存在 $\Rightarrow \lim\limits_{x\to 0}f'(x)$ 存在
		\item C 
		$$
		\begin{cases}
			\lim\limits_{x\to 0}f(x) = f(0) \\
			\lim\limits_{x\to 0}f'(x) = k\\
		    \lim\limits_{x\to 0}\dfrac{f(x)-f(0)}{x} = \lim\limits_{x\to 0}f'(x)
		\end{cases}
		\Rightarrow f'(0) = \lim\limits_{x\to 0}f'(x) = k
		$$
		\item D $\int_{0}^{x}f(t)dt$ 在点 $x=0$ 处可导 $\Rightarrow f(0)$ 存在
	\end{itemize}
\end{solution}

\begin{example}[][Exam: 36.1.12]
	$$\int x\arctan x\cdot \ln(1+x^2)dx$$
\end{example}

\begin{solution}

	\begin{eqnarray*}
		I & = & \dfrac{1}{2}\int \arctan x\cdot \ln(1+x^2)d(x^2+1)\\
		  & = & \dfrac{1}{2}(x^2+1)\arctan x\ln(x^2+1)-\dfrac{1}{2}\int (x^{2}+1)\left[\dfrac{\ln(1+x^{2})}{1+x^{2}}+\dfrac{2x\arctan x}{1+x^{2}}\right]dx\\
		  & = & \dfrac{1}{2}(x^2+1)\arctan x\ln(x^2+1)-\dfrac{1}{2}\int \ln(x^2+1)dx-\int x\arctan xdx\\
		  & = & \dfrac{1}{2}(x^2+1)\arctan x\ln(x^2+1)-\dfrac{1}{2}x\ln(x^2+1) + \int \dfrac{x^{2}}{1+x^{2}}dx\\
		  & + & \dfrac{1}{2}\int \dfrac{x^{2}}{1+x^{2}}dx - \dfrac{1}{2}x^2\arctan x\\
		  & = & \dfrac{x^{2}+1}{2}\arctan x\ln(x^{2}+1) -\dfrac{x}{2}\ln(x^{2}+1) - \dfrac{x^{2}}{2}\arctan x +\dfrac{2}{2}(x-\arctan x) + C
	\end{eqnarray*}
\end{solution}


\textcolor{orange}{October 7}

\begin{example}[][Exam: 36.1.13]
	设$\varGamma = 
\begin{cases}
	x = 2\sqrt{1-y^2}\\
	z = x+y
\end{cases}$, 从 $z$ 轴正向往负向看去为逆时针, 计算曲线积分 
$\displaystyle{\int\limits_{\varGamma}\dfrac{ydx+zdy+xdz}{x^2+y^2+z^2}}$
\end{example}

\begin{solution}

	令 $\begin{cases}
		x = 2\cos \theta\\
		y = \sin \theta\\
		z = \sin \theta + 2\cos \theta\\
		\theta \in [-\dfrac{\pi}{2},\dfrac{\pi}{2}]
	\end{cases}$ 曲线积分:

	\begin{eqnarray*}
		I & = & \int_{-\frac{\pi}{2}}^{\frac{\pi}{2}}\dfrac{\sin\theta d(2\cos\theta) + (2\cos\theta+\sin\theta) d(\sin\theta) + 2\cos\theta d(2\cos\theta+\sin\theta)}{4\cos^{2}\theta+\sin^{2}\theta+(2\cos\theta + \sin\theta)^{2}}\\
		  & = & \dfrac{1}{2}\int_{-\frac{\pi}{2}}^{\frac{\pi}{2}}\dfrac{2+6\cos 2\theta-3\sin 2\theta}{3\cos 2\theta+2\sin\theta+5}d\theta\\
		  & = & \dfrac{1}{4}\int_{-\pi}^{\pi}\dfrac{2+6\cos x-3\sin x}{3\cos x+2\sin x+5}dx\\
		  & = & \dfrac{1}{4}\int_{-\pi}^{\pi}\dfrac{A(3\cos x +2\sin x+5)+B(2\cos x-3\sin x)}{3\cos x+2\sin x+5}dx\\
		  & = & \dfrac{1}{4}\left[ \int_{-\pi}^{\pi}\dfrac{12}{13}dx+\dfrac{11}{13}\int_{-\pi}^{\pi}\dfrac{2\cos x-3\sin x}{3\cos x+2\sin x+5}-\dfrac{34}{13}\int_{-\pi}^{\pi}\dfrac{1}{3\cos x+2\sin x+5}dx\right]\\
		  & = & \dfrac{6}{13}\pi+0-\dfrac{17}{26}\int_{-\pi}^{\pi}\dfrac{1}{3\cos x+2\sin x+5}dx\\
		  & = & \dfrac{6}{13}\pi - \dfrac{17}{26}\int_{-\infty}^{+\infty} \dfrac{1}{(t+1)^{2}+3}dt\ (t=\tan\dfrac{x}{2})\\
		  & = & \dfrac{(36-17\sqrt{3})\pi}{78}
	\end{eqnarray*}
\end{solution}

\begin{example}[][Exam: 36.1.14]
	设函数 $f(x)$ 在 $(-\infty,+\infty)$ 上是以 $T$ 为最小正周期的连续奇函数,下列函数中不是周期函数的个数:  
\begin{itemize}
	\item A. $\int_{a}^{x}f(t)dt$
	\item B. $\int_{-x}^{a}f(t)dt$
	\item C. $\int_{-x}^{x}tf(t)dt$
	\item D. $\int_{-x}^{x}t^2f(t)dt$
\end{itemize}
\end{example}

\begin{solution}

	$f(x)$ 是周期函数, 且为奇函数 $
	\begin{cases}
		\displaystyle{\int_{-\frac{T}{2}}^{\frac{T}{2}}f(x)dx = 0}\\
		\displaystyle{\int_{0}^{T}f(t)dt = \int_{x}^{x+T}f(t)dt}
	\end{cases}$
	
	\begin{itemize}
		\item A. $\displaystyle{F(x) = \int_{a}^{x}f(t)dt}$
		$$
		\begin{cases}
			\displaystyle{F(x+T) = \int_{a}^{x+T}f(t)dt}\\
			\displaystyle{F(x+T)-F(x) = \int_{x}^{x+T}f(t)dt=\int_{-\frac{T}{2}}^{\frac{T}{2}}f(x)dx = 0}
		\end{cases}
		$$
		\item B. $\displaystyle{F(x)=\int_{-x}^{a}f(t)dt}$
		$$
		\begin{cases}
			\displaystyle{F(x+T) = \int_{-x-T}^{a}f(t)dt}\\
			\displaystyle{F(x+T)-F(x) = \int_{-x-T}^{-x}f(t)dt = \int_{-\frac{T}{2}}^{\frac{T}{2}}f(x)dx = 0}
		\end{cases}
		$$ 
		\item C. $\displaystyle{F(x)=\int_{-x}^{x}tf(t)dt}$
		$$
		\begin{cases}
			\displaystyle{F(x+T) = \int_{-x-T}^{x+T}tf(t)dt}\\
			\displaystyle{F(x+T)-F(x) = 2\int_{x}^{x+T}tf(t)dt\neq 0}
		\end{cases}
		$$ 
		\item D. $\displaystyle{F(x)=\int_{-x}^{x}t^{2}f(t)dt = 0}$
	\end{itemize}
	
	综上所述, 上述函数只有 $C$ 不是周期函数, $ABD$ 均为周期函数.
\end{solution}


\section{Week \Rmnum{2}}
\textcolor{blue}{October 8}

\begin{example}[][Exam: 36.2.1]
	设随机变量 $(X,Y)$ 服从二维正态分布,$X\sim N(1,3^2),Y\sim N(0,4^2)$, 且满足 $\rho_{XY}=-\dfrac{1}{2}$,$Z=\dfrac{X}{3}+\dfrac{Y}{2}$

(1). 求 $E(Z)$ 与 $D(Z)$

(2). 求 $\rho_{XZ}$

(3). 证明 $X$ 与 $Z$ 是否独立
\end{example}

\begin{solution}
	
\end{solution}


\begin{example}[][Exam: 36.2.2]
	$$\iint\limits_{D}\dfrac{1}{\sqrt{xy}}dxdy, D = \{(x,y)|(\dfrac{x}{2}+\dfrac{y}{4})^2\leq \dfrac{x}{6},x,y\geq 0\}$$
\end{example}
\begin{solution}
	\begin{eqnarray*}
		I & = & \int_{0}^{\frac{2}{3}}dx\int_{0}^{4\sqrt{\frac{x}{6}}-2x}\dfrac{1}{\sqrt{xy}}dy\\
		  & = & 2\int_{0}^{\frac{2}{3}}\dfrac{1}{\sqrt{x}}\sqrt{4\sqrt{\frac{x}{6}}-2x}dx\\
		  & = & 8\sqrt{6}\int_{0}^{\frac{1}{3}}\sqrt{t-3t^2}dt\ (t=\sqrt{\dfrac{x}{6}})\\
		  & = & \dfrac{2\sqrt{2}}{3}\int_{-\frac{\pi}{2}}^{\frac{\pi}{2}}\cos^2\theta d\theta\ (6t-1=\sin\theta)\\
		  & = & \dfrac{\sqrt{2}\pi}{3}
	\end{eqnarray*}
\end{solution}
\begin{anymark}[注]
	令 
	$$\begin{cases}
		\sqrt{x} = m \\
		\sqrt{y} = n
	\end{cases} \quad \quad 
	D' = \{(m,n) |\dfrac{(m-\frac{1}{\sqrt{6}})^2}{2}+\dfrac{n^2}{4}=\dfrac{1}{12}(m,n>0)\}$$
	
	雅可比行列式: $J = 
	\begin{vmatrix}
		\dfrac{\partial x}{\partial m} & \dfrac{\partial x}{\partial n} \\
		\dfrac{\partial y}{\partial m} & \dfrac{\partial y}{\partial n}
	\end{vmatrix} = 4mn$

	$$dxdy = 4mndmdn\Rightarrow \iint\limits_{D}\dfrac{1}{\sqrt{xy}}dxdy = \iint\limits_{D'}4dmdn$$
	
	$$I = 4S_{D'} = 2ab\pi = 2\pi\times\dfrac{1}{\sqrt{6}}\times\dfrac{1}{\sqrt{3}}=\dfrac{\sqrt{2}\pi}{3}$$
\end{anymark}


\textcolor{blue}{October 9}

\begin{example}[][Exam: 36.2.3]
	$$\int \dfrac{\sqrt{x-1}\arctan \sqrt{x-1}}{x}dx$$
\end{example}

\begin{solution}

	令 $
	\begin{cases}
		\sqrt{x-1} = t\\
		x = t^{2} + 1\\
		dx = 2tdt
	\end{cases}$ 不定积分:  
	\begin{eqnarray*}
		I & = & \int \dfrac{2t^2\arctan t}{t^2+1}dt\\
		  & = & 2\int \arctan tdt-2\int\dfrac{\arctan t}{1+t^2}dt\\
		  & = & 2t\arctan t-\ln(1+t^2)-(\arctan t)^2\\
		  & = & 2\sqrt{x-1}\arctan \sqrt{x-1}-\ln x-(\arctan\sqrt{x-1})^2+C
	\end{eqnarray*}
\end{solution}

\begin{example}[][Exam: 36.2.4]
	设函数 $f(x)$ 在 $(-\infty,+\infty)$ 内满足 $f(x)=f(x-\pi)+\sin x$, 且 $f(x)=x,x\in[0,\pi)$, 求 $\displaystyle{\int_{\pi}^{3\pi}f(x)dx}$
\end{example}

\begin{solution}

	$$\begin{cases}
		f(x) = f(x-\pi)+\sin x, & x\in \mathbb{R}\\
		f(x) = x, &x\in[0,\pi)
	\end{cases}  
	\Rightarrow
	f(x) = x+\sin x-\pi,x\in[\pi,2\pi)$$
	
	我们有:  
	\begin{eqnarray*}
		\int_{\pi}^{3\pi}f(x)dx & = & \int_{\pi}^{3\pi}[f(x-\pi)+\sin x]dx\\
								& = & \int_{\pi}^{3\pi}f(x-\pi)dx\\
								& = & \int_{0}^{2\pi}f(x)dx\\
								& = & \int_{0}^{\pi}xdx+\int_{\pi}^{2\pi}[x+\sin x-\pi]dx\\
								& = & \pi^2-2
	\end{eqnarray*}
\end{solution}


\textcolor{blue}{October 10}

\begin{example}[][Exam: 36.2.5]
	$$\int \dfrac{\cos^3 x-2\cos x}{1+\sin^2 x+\sin^4 x}dx$$
\end{example}

\begin{solution}

	令 $t=\sin x$ 不定积分为:  
	\begin{eqnarray*}
		I & = & -\int \dfrac{t^2+1}{1+t^2+t^4}dt\\
		  & = & -\int \dfrac{1+\frac{1}{t^2}}{1+t^2+\frac{1}{t^2}}dt\\
		  & = & -\int \dfrac{d(t-\frac{1}{t})}{(t-\frac{1}{t})^2+3}\\
		  & = & -\dfrac{1}{\sqrt{3}}\arctan(\dfrac{t-\frac{1}{t}}{\sqrt{3}})+C\\
		  & = & \dfrac{1}{\sqrt{3}}\arctan(\dfrac{\cos^2 x}{\sqrt{3}\sin x})dx+C
	\end{eqnarray*}
\end{solution}

\begin{example}[][Exam: 36.2.6]
	下列积分中, 与积分 $\displaystyle{I=\int_{0}^{1}\dfrac{1}{2}xe^{-\sqrt{x}}dx}$ 值最接近的是
\begin{itemize}
	\item A. $\int_{0}^{1}\sqrt{x}e^{-x}dx$
	\item B. $\int_{0}^{1}xe^{-x}dx$
	\item C. $\int_{0}^{1}x^2e^{-x}dx$
	\item D. $\int_{0}^{1}x^4e^{-x}dx$
\end{itemize}
\end{example}

\begin{solution}
	\begin{eqnarray*}
		I & = & \int_{0}^{1}\dfrac{1}{2}xe^{-\sqrt{x}}dx\\
		  & = & \int_{0}^{1}t^3e^{-t}dt\\
		  & = & \int_{0}^{1}x^3e^{-x}dx
	\end{eqnarray*}
	  
	$$\int_{0}^{1}\sqrt{x}e^{-x}dx>\int_{0}^{1}xe^{-x}dx>\int_{0}^{1}x^{2}e^{-x}dx>\int_{0}^{1}x^{3}e^{-x}dx>\int_{0}^{1}x^{4}e^{-x}dx$$
	
	$$\begin{cases}
		x^{2} - x^{3} > x^{3} - x^{4}\\
		\displaystyle{\int_{0}^{1}(x^{2}-x^{3})e^{-x}dx > \int_{0}^{1}(x^{3}-x^{4})e^{-x}dx}
	\end{cases}$$
	
	综上所述, 与积分 $I$ 最近接的是 $\displaystyle{\int_{0}^{1}x^{4}e^{-x}dx}$
\end{solution}


\textcolor{blue}{October 11}

\begin{example}[][Exam: 36.2.7]
	$f(x)=\dfrac{\left(\sqrt[n]{x}-1\right) ^n}{x+1}$, 求 $f^{(n)}(1)(n\geq 2)$
\end{example}

\begin{solution}

	$$\lim\limits_{x\to 1}\dfrac{f(x)}{(x-1)^n}=\dfrac{1}{2}\lim\limits_{x\to 1}\dfrac{(\sqrt[n]{x}-1)^n}{(x-1)^n}=\dfrac{1}{2n^{n}}$$
	
	$f(x)$ 在 $x = 1$ 处泰勒展开式:

	$$f(x) = \sum\limits_{k=1}^{n}\dfrac{f^{(k)}(x)}{k!}(x-1)^{k} + o[(x-1)^{n}]$$ 

	$$\lim\limits_{x\to 1}\dfrac{f(x)}{(x-1)^n}=\lim\limits_{x\to 1}\dfrac{\sum\limits_{k=1}^{n}\frac{f^{(k)}(x)}{k!}(x-1)^k+o[(x-1)^n]}{(x-1)^n}$$
	
	上述极限存在,我们可以得到:  $f^{(k)}=0(k=1,2,\cdots,n-1)$
	$$\lim\limits_{x\to 1}\dfrac{f(x)}{(x-1)^n}=\dfrac{f^{(n)}(1)}{n!}\Rightarrow f^{(n)}(1)=\dfrac{n!}{2n^{n}}$$
\end{solution}

\begin{example}[][Exam: 36.2.8]
	设 $f(x)$ 在 $[0,1]$ 上二阶导数连续, $f(1)=f'(1)=0, D=\{(x,y)|0\leq x\leq 1,0\leq y\leq 1\}$, 证明:

(1). $\displaystyle{\iint\limits_{D}f(x)dxdy=\iint\limits_{D}x^{2}yf''(x)dxdy}$

(2). $\exists \xi,\eta\in(0,1),\ s.t.\ \xi^{2}f''(\xi) = 2f'(\eta)(\xi-1)$
\end{example}

\begin{solution}

	(1). 
	\begin{eqnarray*}
		\text{Left} & = & \iint\limits_{D}f(x)dxdy\\
				    & = & \int_{0}^{1}f(x)dx\int_{0}^{1}dy\\
		            & = & \int_{0}^{1}f(x)dx\\
		            & = & xf(x)|_{x=0}^{x=1}-\int_{0}^{1}xf'(x)dx\\
		            & = & -\int_{0}^{1}xf'(x)dx
	\end{eqnarray*}
	\begin{eqnarray*}
		\text{Right} & = & \iint\limits_{D}x^2yf''(x)dxdy\\
					 & = & \int_{0}^{1}x^2f''(x)dx\int_{0}^{1}ydy\\
		             & = &\int_{0}^{1}\dfrac{x^2}{2}df'(x)\\
		             & = &\dfrac{x^2}{2}f'(x)|_{x=0}^{x=1}-\int_{0}^{1}xf'(x)dx\\
		             & = &-\int_{0}^{1}xf'(x)dx
	\end{eqnarray*}
	
	综上所述, $\text{Left}=\text{Right} \Rightarrow \displaystyle{\iint\limits_{D}f(x)dxdy=\iint\limits_{D}x^2yf''(x)dxdy}$
	
	(2).
	
	构造辅助函数 $g(x) = \int_{0}^{x} x^{2}f''(x)-2f(x)$, $g(0) = g(1) = 0$
	
	罗尔定理:  
	$$\exists \xi\in(0,1),\ s.t.\ g'(\xi) = 0\Rightarrow \xi^2f''(\xi)-2f(\xi) = 0$$
	
	拉格朗日中值定理:  
	$$\exists\eta\in(\xi,1),\ s.t.\ f(\xi)-f(1)=f'(\eta)(\xi-1)$$
	
	综上所述,我们得到:  $\exists \xi,\eta\in(0,1),\ s.t.\ \xi^2f''(\xi)=2f'(\eta)(\xi-1)$
\end{solution}


\textcolor{blue}{October 12}

\begin{example}[][Exam: 36.2.9]
	$$\int_{-\pi}^{\pi}\dfrac{x\sin x\left( \arctan e^x+\int_{0}^{x}e^{t^2}dt\right) }{1+\cos^2 x}dx$$
\end{example}
\begin{solution}

	$$f(x) = \int_{0}^{x}e^{t^{2}}dt, f(-x) = -f(x)$$
	
	\begin{eqnarray*}
		I & = & \int_{-\pi}^{\pi}\dfrac{x\sin x\arctan e^x}{1+\cos^2 x}dx\\
		  & = & \int_{-\pi}^{\pi}\dfrac{x\sin x\arctan e^{-x}}{1+\cos^2 x}dx\\
	   2I & = & \pi\int_{0}^{\pi}\dfrac{x\sin x}{1+\cos^2 x}dx\\
		  & = & \pi\int_{0}^{\pi}\dfrac{(\pi-x)\sin x}{1+\cos^2 x}dx\\
	   4I & = & \pi^2\int_{0}^{\pi}\dfrac{\sin x}{1+\cos^2 x}dx\\
	   4I & = & \dfrac{\pi^3}{2}\\
		I & = & \dfrac{\pi^3}{8}
	\end{eqnarray*}
\end{solution}

\begin{example}[][Exam: 36.2.10]
	若 $\int_{0}^{+\infty}\dfrac{\arctan(ax)}{x^n}dx$ 收敛,求 $n$ 的取值范围
\end{example}

\begin{solution}

	暇点为 $x = 0$ 和 $x = +\infty$
	
	(1). $x=0$ $\dfrac{\arctan (ax)}{x^n}\sim ax^{1-n}$ 收敛 $\Rightarrow 1-n>-1\Rightarrow n<2$
	
	(2). $x=+\infty$ $\dfrac{\arctan (ax)}{x^n}\sim \frac{\pi}{2}x^{-n}$ 收敛 $\Rightarrow -n<-1\Rightarrow n>1$
	
	综上所述, $n\in(1,2)$
\end{solution}


\textcolor{blue}{October 13}

\begin{example}[][Exam: 36.2.11]
	设 $0<a<1,\displaystyle{I_{1}=\int_{0}^{\frac{\pi}{4}}\dfrac{\sin ax}{\sin x}dx}$,
$\displaystyle{I_{2}=\int_{0}^{\frac{\pi}{4}}\dfrac{\tan ax}{\tan x}dx}$,
比较 $I_{1},I_{2}$ 和 $\dfrac{\pi a}{4}$ 的大小
\end{example}

\begin{solution}

	构造辅助函数: 
	$$\begin{cases}
		f(x) = \sin(ax) - a\sin x, & f(0) = 0 \\
		g(x) = \tan(ax) - a\tan x, & g(0) = 0
	\end{cases}\Rightarrow 
	\begin{cases} 
		f'(x) = a[\cos(ax) - cos x] > 0\\
		g'(x) = \dfrac{a[\cos^{2}x-\cos^{2}(ax)]}{\cos^{2}x\cos^{2}(ax)} < 0
	\end{cases}\Rightarrow 
	\begin{cases}
		\frac{\sin(ax)}{\sin x} > a\\
		\frac{\tan(ax)}{\tan x} < a
	\end{cases}$$
	
	$$\begin{cases}
		I_{1} > \dfrac{\pi a}{4}\\
		I_{2} < \dfrac{\pi a}{4}
	\end{cases}\Rightarrow I_{2} < \dfrac{\pi a}{4} < I_{1}$$
\end{solution}

\begin{example}[][Exam: 36.2.12]
	$$\iint\limits_{D}\dfrac{\tan^3x+y}{\sqrt{x^2+y^2}}dxdy$$
	其中 $D=\{(x,y)|y\geq |x|,(x^2+y^2)^3\leq y^4\}$
\end{example}


\begin{solution}
	$D$ 关于 $x = 0$ 对称, 且 $f(x,y) = \dfrac{\tan^{3}x}{\sqrt{x^{2}+y^{2}}}$ 满足 $f(x,y) = -f(-x,y)$
	
	$$\displaystyle{\iint\limits_{D}f(x,y)dxdy} = 0$$  
	\begin{eqnarray*}
		I & = & \iint\limits_{D}\dfrac{y}{\sqrt{x^2+y^2}}dxdy\\
		  & = & \int_{\frac{\pi}{4}}^{\frac{3\pi}{4}}\sin\theta d\theta\int_{0}^{\sin^{2}\theta}r dr\\
		  & = & \dfrac{1}{2}\int_{\frac{\pi}{4}}^{\frac{3\pi}{4}}\sin^5\theta d\theta\\
		  & = & \int_{\frac{\pi}{4}}^{\frac{\pi}{2}}\sin^5\theta d\theta\\
		  & = & \dfrac{43\sqrt{2}}{120}
	\end{eqnarray*}
\end{solution}


\textcolor{blue}{October 14}

\begin{example}[][Exam: 36.2.13]
	已知 $\displaystyle{\int_{1}^{+\infty}\left( \dfrac{2x^2+bx+a}{2x^2+ax}-1\right)dx=0}$, 求 $a,b$
\end{example}

\begin{solution}

	$\displaystyle{\int_{1}^{+\infty}\dfrac{(b-a)x+a}{2x^2+ax}dx=0}$ 收敛
	
	(1). $x=+\infty$,假设$a\neq b\Rightarrow \dfrac{(b-a)x+a}{2x^2+ax}\sim \dfrac{1}{x}\text{发散}$,我们得到:  $a=b$
	
	原积分等价于:  $$\int_{1}^{+\infty}\dfrac{a}{2x^2+ax}dx=\int_{1}^{+\infty}(\dfrac{1}{x}-\dfrac{2}{2x+a})dx=0\Rightarrow \ln|\dfrac{x}{2x+a}|_{x=1}^{x=+\infty}=0\Rightarrow a=b=0$$
\end{solution}

\begin{example}[][Exam: 36.2.14]
	$$\lim\limits_{n\to +\infty}\left[\sum\limits_{k=1}^{n}\dfrac{\ln(n+k)}{n+\frac{1}{k}}-\ln n\right]$$
\end{example}

\begin{solution}

	夹逼准则:

	$$\sum\limits_{k=1}^{n}\dfrac{\ln(n+k)}{n+1}-\ln n<I<\sum\limits_{k=1}^{n}\dfrac{\ln(n+k)}{n}-\ln n$$

	\begin{eqnarray*}
		\text{Left}  & = & \lim\limits_{n\to +\infty}\sum\limits_{k=1}^{n}\dfrac{\ln(n+k)}{n+1}-\ln n\\
		             & = & \lim\limits_{n\to +\infty}\left\lbrace \dfrac{n}{n+1}\dfrac{1}{n}\sum\limits_{k=1}^{n}[\ln(n+k)-\ln n]-\dfrac{\ln n}{n+1}\right\rbrace \\
					 & = & \lim\limits_{n\to +\infty}\dfrac{n}{n+1}\cdot\lim\limits_{n\to +\infty}\dfrac{1}{n}\sum\limits_{k=1}^{n}\ln(1+\frac{k}{n})-
					 \lim\limits_{n\to +\infty}\dfrac{\ln n}{n+1}\\
		             & = & \int_{0}^{1}\ln(1+x)dx\\
		             & = & 2\ln2-1\\
		\text{Right} & = & \lim\limits_{n\to +\infty}\sum\limits_{k=1}^{n}\dfrac{\ln(n+k)}{n}-\ln n\\
		             & = & \lim\limits_{n\to +\infty}\dfrac{1}{n}\sum\limits_{k=1}^{n}[\ln(n+k)-\ln n]\\
		             & = & \lim\limits_{n\to +\infty}\dfrac{1}{n}\sum\limits_{k=1}^{n}\ln(1+\frac{k}{n})\\
		             & = & \int_{0}^{1}\ln(1+x)dx\\
		             & = & 2\ln2-1
	\end{eqnarray*}
	
	综上所述, 极限 $I=2\ln2-1$
\end{solution}


\section{Week \Rmnum{3}}
\textcolor{cyan}{October 15}

\begin{example}[][Exam: 36.3.1]
	$$\int_{0}^{+\infty}\dfrac{dx}{(1+x^2)(1+x^4)}$$
\end{example}

\begin{solution}

	倒代换:

	\begin{eqnarray*}
		I & = & \int_{0}^{+\infty}\dfrac{t^4}{(1+t^2)(1+t^4)}dt \\
	   2I & = & \int_{0}^{+\infty}\dfrac{1}{1+x^2}dx = \arctan x\big|_{x = 0}^{x = +\infty} = \dfrac{\pi}{2}\\
		I & = & \dfrac{\pi}{4}
	\end{eqnarray*}
\end{solution}

\begin{example}[][Exam: 36.3.2]
	设随机变量 $Y=min\{|X|,1\}$, 其中 $X$ 为随机变量, 且密度函数为 
	$f(x)=\dfrac{k}{1+x^2}(x\in \mathbb{R}, k\equiv C)$, 下列说法不正确的是:  
\begin{itemize}
	\item A. $k=\dfrac{1}{\pi}$
	\item B. $E(X)=0$
	\item C. $Y$ 没有概率密度
	\item D. $E(Y)=\dfrac{\ln(2e^{\frac{\pi}{2}})}{\pi}$
\end{itemize}
\end{example}

\begin{solution}
	
\end{solution}


\textcolor{cyan}{October 16}

\begin{example}[][Exam: 36.3.3]
	求 $y=e^{-x}\sqrt{\sin x}(0\leq x<+\infty)$ 绕 $x$ 轴旋转一周的旋转体体积
\end{example}
\begin{solution}

	我们可以得到:  
	\begin{eqnarray*}
		V&=&\int_{0}^{+\infty}\pi y^2dx\\
		&=&\lim\limits_{n\to+\infty}\sum\limits_{k=0}^{n}\int_{2k\pi}^{(2k+1)\pi}\pi e^{-2x}\sin xdx\\
		&=&\lim\limits_{n\to+\infty}A(1+e^{-4\pi}+e^{-8\pi}+\cdots+e^{-4n\pi})\\
		&=&\dfrac{A\pi}{1-e^{-4\pi}}\\
		A&=&=\int_{0}^{\pi}e^{-2x}\sin xdx=\dfrac{e^{-2\pi}+1}{5}\\
		V&=&\dfrac{e^{-2\pi}+1}{5(1-e^{-4\pi})}=\dfrac{e^{2\pi}\pi}{5(e^{2\pi}-1)}
	\end{eqnarray*}
\end{solution}

\begin{example}[][Exam: 36.3.4]
	已知 $\int_{0}^{+\infty}\dfrac{\ln(1+x)}{x^{\alpha}}dx$ 收敛, 求 $\alpha$ 的取值范围
\end{example}

\begin{solution}

	原积分可能存在的暇点为$x=0,x=+\infty$
	
	(1). $x=0$时,$\dfrac{\ln(1+x)}{x^{\alpha}}\sim x^{1-\alpha}$收敛$\Rightarrow 1-\alpha>-1\Rightarrow \alpha<2$
	
	(2). $x=+\infty$时,$\dfrac{\ln(1+x)}{x^{\alpha}}\sim x^{-\alpha}$收敛$\Rightarrow -\alpha<-1\Rightarrow \alpha>1$
	
	我们得到:  $\alpha\in(1,2)$
\end{solution}


\textcolor{cyan}{October 17}

\begin{example}[][Exam: 36.3.5]
	设随机变量 $X,Y$ 相互独立, 且 $X\sim N(0,1), Y\sim B(n,p), 0<p<1$, 且 $X+Y$ 的分布函数:  
\begin{itemize}
	\item A. 是连续函数
	\item B. 恰有 $n+1$ 个间断点
	\item C. 恰有 $1$ 个间断点
	\item D. 有无穷个间断点
\end{itemize}
\end{example}

\begin{solution}
	
\end{solution}

\begin{example}[][Exam: 36.3.6]
	微分方程 $\cos^4 x\dfrac{d^2y}{dx^2}+2\cos^2 x(1-\sin x\cos x)\dfrac{dy}{dx}+y=e^{-\tan x}$,
	求该微分方程在$t=\tan x$变换下所得的$y$对$t$的微分方程,并求出其通解
\end{example}


\begin{solution}

	我们可以得到:  $$\left\lbrace
	\begin{array}{l}
		t=\tan x\\
		\sin x=\dfrac{t}{\sqrt{1+t^2}}\\
		\cos x=\dfrac{1}{\sqrt{1+t^2}}\\
		\dfrac{dy}{dx}=\dfrac{dy}{dt}\dfrac{dt}{dx}=\dfrac{(1+t^2)dy}{dt}\\
		\dfrac{d^2y}{dx^2}=\dfrac{\dfrac{dy}{dx}}{dt}\dfrac{dt}{dx}=\dfrac{(1+t^2)^2d^2y}{dt^2}-\dfrac{2t(1+t^2)dy}{dt}
	\end{array}
	\right. $$
	
	我们可以得到原微分方程等价于:  
	$$\dfrac{d^2y}{(1+t^2)^2dx^2}+\dfrac{2dy}{(1+t^2)dx}-\dfrac{2tdy}{(1+t^2)^2dx}+y=e^{-t}$$
	
	我们可以得到:  
	$$\dfrac{d^2y}{dt^2}+2\dfrac{dy}{dt}+y=e^{-t}\Rightarrow y''(t)+2y'(t)+y(t)=e^{-t}$$
	
	我们得到特征方程:  $\lambda^2+2\lambda+1=0\Rightarrow \lambda_{1}=\lambda_{2}=-1$
	
	我们可以得到:  $y=(C_{1}x+C_{2})e^{-x}+y^{*}\Rightarrow y^{*}=Ax^2e^{-x}\Rightarrow y^{*}=\dfrac{1}{2}x^2e^{-x}$
	
	综上所述,微分方程的通解为:  $y=(C_{1}x+C_{2})e^{-x}+\dfrac{1}{2}x^2e^{-x}$
\end{solution}


\textcolor{cyan}{October 18}

\begin{example}[][Exam: 36.3.7]
	$$\iint\limits_{D}\arcsin(2\sqrt{x-x^2})dxdy, D=\{(x,y)|0\leq x\leq 1,0\leq y\leq x\}$$
\end{example}


\begin{solution}

	原二重积分可化为:  
	\begin{eqnarray*}
		I&=&\int_{0}^{1}x\arcsin(2\sqrt{x-x^2})dx\\
		&=&\int_{-\frac{\pi}{2}}^{\frac{\pi}{2}}\dfrac{\arcsin(\cos\theta) \cos\theta(\sin\theta+1)}{4}d\theta\\
		&=&\int_{-\frac{\pi}{2}}^{\frac{\pi}{2}}\dfrac{\arcsin(\cos\theta)\cos\theta}{4}d\theta\\
		&=&\dfrac{1}{2}\int_{0}^{\frac{\pi}{2}}(\frac{\pi}{2}-\theta)\cos\theta d\theta\\
		&=&\dfrac{\pi}{4}-\dfrac{1}{2}\int_{0}^{\frac{\pi}{2}}\theta d\sin\theta\\
		&=&\dfrac{1}{2}
	\end{eqnarray*}
\end{solution}

\begin{example}[][Exam: 36.3.8]
	设 $f(x)=\lim\limits_{n\to+\infty}\dfrac{x^{n+1}-x^2}{x^n+1}$,$F(x)=\int_{0}^{x}f(t)dt$,下列说法正确的是:  
\begin{itemize}
	\item A. $f(x)$有$1$个间断点,$F(x)$有$1$个不可导点
	\item B. $f(x)$有$1$个间断点,$F(x)$有$2$个不可导点
	\item C. $f(x)$有$2$个间断点,$F(x)$有$1$个不可导点
	\item D. $f(x)$有$2$个间断点,$F(x)$有$2$个不可导点
\end{itemize}
\end{example}

\begin{solution}

	我们可以得到:  $f(x)=\left\lbrace
	\begin{array}{l}
		x,\ x<-1\\
		-x^2,\ -1<x<1\\
		0,\ x=1\\
		x,\ x>1
	\end{array}
	\right. $
	
	$f(x)$在$x=-1$处无定义,$x=-1$是可去间断点,$x=1$是跳跃间断点.
	
	$F(x)=\int_{0}^{x}f(t)dt$处处连续,仅在$x=1$处不可导.
\end{solution}


\textcolor{cyan}{October 19}

\begin{example}[][Exam: 36.3.9]
	$x\geq 0$, 连续函数 $f(x)$ 的原函数 $F(x)$ 非负, 且满足方程$\int_{0}^{x^2}f(x^2)f(t)dt=\dfrac{1}{2}(\sqrt{1+x^2}-1), F(0)=0$, 求 $f(x)$
\end{example} 
\begin{solution}

	我们设$F(x)=\int_{0}^{x}f(t)dt$,我们令$x^2=u$,我们得到:  
	$$f(u)\int_{0}^{u}f(t)dt=\dfrac{1}{2}(\sqrt{1+u}-1)\Rightarrow F'(x)F(x)=\dfrac{1}{2}(\sqrt{1+x}-1)\Rightarrow \dfrac{1}{2}[F^2(x)]'=\dfrac{1}{2}(\sqrt{1+x}-1)$$
	
	我们得到:  
	$$F^2(x)=\dfrac{2}{3}(1+x)^{\frac{3}{2}}-x+C,F(0)=0\Rightarrow \left\lbrace
	\begin{array}{l}
		F(x)=\sqrt{\dfrac{2}{3}(1+x)^{\frac{3}{2}}-x-\dfrac{2}{3}}\\
		f'(x)=\dfrac{\sqrt{1+x}-1}{2\sqrt{\dfrac{2}{3}(1+x)^{\frac{3}{2}}-x-\dfrac{2}{3}}}
	\end{array}
	\right. $$
\end{solution}

\begin{example}[][Exam: 36.3.10]
	证明: $$\sum\limits_{n=1}^{+\infty}\dfrac{1}{n^2}=\dfrac{\pi^2}{6}$$
\end{example}

\begin{solution}

	我们有:  
	$$\sin x=x-\dfrac{x^3}{3!}+\dfrac{x^5}{5!}+\cdots+\dfrac{(-1)^{n}x^{2n+1}}{(2n+1)}+\cdots$$
	
	我们有:  $f(x)=\dfrac{\sin x}{x}=1-\dfrac{x^2}{3!}+\dfrac{x^4}{5!}+\cdots+\dfrac{(-1)^{n}x^{2n}}{(2n+1)}+\cdots$
	
	我们知道$f(x)$有零点$x=\pm\pi,\pm 2\pi,\cdots\Rightarrow f(x)=A(x-\pi)(x+\pi)(x-2\pi)(x+2\pi)\cdots$
	
	
	我们有:  
	$$f(x)=A(x^2-\pi^2)(x^2-4\pi^2)\cdots,\text{令}x=0,\text{我们有:  }A(-\pi^2)(-4\pi^2)\cdots=1\Rightarrow A=\dfrac{1}{(-\pi^2)(-4\pi^2)\cdots}$$
	
	我们有:  $$f(x)=\left\lbrace
	\begin{array}{l}
		1-\dfrac{x^2}{3!}+\dfrac{x^4}{5!}+\cdots+\dfrac{(-1)^{n}x^{2n}}{(2n+1)}+\cdots\\
		(1-\dfrac{x^2}{\pi^2})(1-\dfrac{x^2}{4\pi^2})\cdots
	\end{array}
	\right. \Rightarrow x^2\text{项系数相等}\Rightarrow -\dfrac{1}{6}=\sum\limits_{n=1}^{+\infty}(-\dfrac{1}{n^2\pi^2})$$
	
	我们可以得到:  $\sum\limits_{n=1}^{+\infty}\dfrac{1}{n^2}=\dfrac{\pi^2}{6}$
\end{solution}


\textcolor{cyan}{October 20}

\begin{example}[][Exam: 36.3.11]
	设 $f(x)$ 在 $[1,+\infty)$ 上有连续的一阶导数, $f'(x)=\dfrac{1}{1+f^{2}(x)}\left[\sqrt{\dfrac{1}{x}}-\sqrt{\ln(1+\dfrac{1}{x})} \right]$,
	证明: $\lim\limits_{x\to +\infty}f(x)$ 存在

\end{example}
\begin{solution}

	我们由:  $\lim\limits_{x\to +\infty}f(x)-f(1)=\int_{1}^{+\infty}f'(x)dx\Rightarrow $我们只需要证明:  $\int_{1}^{+\infty}f'(x)dx$收敛
	
	我们有:  
	\begin{eqnarray*}
		\int_{1}^{+\infty}f'(x)dx&=&\int_{1}^{+\infty}\dfrac{1}{1+f^{2}(x)}\left[\sqrt{\dfrac{1}{x}}-\sqrt{\ln(1+\dfrac{1}{x})} \right]dx\\
		&\leq&\int_{1}^{+\infty}\left[\sqrt{\dfrac{1}{x}}-\sqrt{\ln(1+\dfrac{1}{x})} \right]dx
	\end{eqnarray*}
	
	我们由拉格朗日中值定理得到:  
	$$\lim\limits_{x\to+\infty}\left[\sqrt{\dfrac{1}{x}}-\sqrt{\ln(1+\dfrac{1}{x})} \right]=\lim\limits_{x\to+\infty}\dfrac{1}{2\sqrt{\xi}}\left(\dfrac{1}{x}-\ln(1+\dfrac{1}{x}) \right)=\lim\limits_{x\to+\infty}\dfrac{x^{-\frac{3}{2}}}{4}$$
	
	我们由比较判别法可以得到:  $\left\lbrace 
	\begin{array}{l}
		\int_{1}^{+\infty}\dfrac{1}{x^p}dx\text{收敛},p>1\\
		\int_{1}^{+\infty}\dfrac{1}{4x^{\frac{3}{2}}}dx\text{收敛}
	\end{array}
	\right. $
	
	综上所述,我们得到:  $\lim\limits_{x\to +\infty}f(x)$存在
\end{solution}

\begin{example}[][Exam: 36.3.12]
	函数 $f(x)$ 在 $[0,+\infty)$ 上可导, $f(0)=1$ 且满足等式:  
$$f'(x)+f(x)-\dfrac{1}{x+1}\int_{0}^{x}f(t)dt=0$$

(1). 求 $f'(x)$

(2). 证明: $x\geq 0, e^{-x}\leq f(x)\leq 1$

\end{example}

\begin{solution}

	(1). 我们可以得到:  
	$$(x+1)f'(x)+(x+1)f(x)-\int_{0}^{x}f(t)dt=0$$
	
	我们对上式子求导可以得到:  
	$$(x+1)f''(x)+(x+2)f''(x)=0\Rightarrow \left\lbrace 
	\begin{array}{l}
		(x+1)e^{x}f'(x)=C\\
		f'(0)=-1
	\end{array}
	\right. \Rightarrow f'(x)=-\dfrac{1}{(1+x)e^{x}}$$
	
	(2). 我们有:  $$\left\lbrace 
	\begin{array}{l}
		F(x)=f(x)-1\\
		G(x)=f(x)-e^{-x}
	\end{array}
	\right.$$
	$$\downdownarrows$$ 
	
	$$\left\lbrace 
	\begin{array}{l}
		F'(x)=f'(x)=-\dfrac{1}{(1+x)e^{x}}<0\\
		F(0)=0\\
		G'(x)=f'(x)+e^{-x}=\dfrac{x}{(1+x)e^{x}}>0\\
		G(0)=0
	\end{array}
	\right.$$ 
	$$\downdownarrows$$ 
	$$\left\lbrace 
	\begin{array}{l}
		F(x)\text{单调递减}\\
		F(x)\leq F(0)=1\\
		G(x)\text{单调递增}\\
		G(x)\geq G(0)\Rightarrow f(x)\geq e^{-x}
	\end{array}
	\right. $$
\end{solution}


\textcolor{cyan}{October 21}

\begin{example}[][Exam: 36.3.13]
	设 $f(x)=\int_{-1}^{x}(1-|t|)dt(x\geq -1)$, 求曲线 $y=f(x)$ 与 $x$ 轴所围成的面积
\end{example}

\begin{solution}

	我们可以得到$f(x)$表达式:  
	$$f(x)=\left\lbrace 
	\begin{array}{l}
		\dfrac{(x+1)^2}{2},\ x\in[-1,0]\\
		\dfrac{2x-x^2+1}{2},\ x\in(1,+\infty)
	\end{array}
	\right. $$
	
	我们可以得到:  
	$$S=\int_{0}^{1}f(x)dx+\int_{0}^{1+\sqrt{2}}f(x)dx=\dfrac{1}{6}+\dfrac{5}{6}+\dfrac{2\sqrt{2}}{3}=1+\dfrac{2\sqrt{2}}{3}$$
\end{solution}

\begin{example}[][Exam: 36.3.14]
	设函数 $f(x)$ 连续, $f'(0)$ 存在, $\forall x,y\in\mathbb{R}, f(x+y)=\dfrac{f(x)+f(y)}{1-4f(x)f(y)}$, 且 $f'(0)=\dfrac{1}{2}$,求$f(x)$
\end{example}

\begin{solution}

	我们有:  $f(0)=\dfrac{2f(0)}{1-4f^2(0)}\Rightarrow f(0)=0$
	
	我们利用导数定义:  
	\begin{eqnarray*}
		f'(x)&=&\lim\limits_{\Delta x\to 0}\dfrac{f(x+\Delta x)-f(x)}{\Delta x}\\
		&=&\lim\limits_{\Delta x\to 0}\dfrac{\dfrac{f(x)+f(\Delta x)}{1-4f(x)f(\Delta x)}-f(x)}{\Delta x}\\
		&=&(1+4f^{2}(x))\lim\limits_{\Delta x\to 0}\dfrac{f(\Delta x)}{\Delta}\\
		&=&\dfrac{1+4f^{2}(x)}{2}
	\end{eqnarray*}
	
	我们可以得到微分方程的解:  
	$$\arctan[2f(x)]=x+C,f(0)=0\to f(x)=\dfrac{1}{2}\tan x$$
\end{solution}


\section{Week \Rmnum{4}}
\textcolor{purplea}{October 22}

\begin{example}[][Exam: 36.4.1]
	设函数 $f(x)$ 在 $[0,+\infty)$ 上具有二阶连续导数,
	$f(0)=f'(0)=0,f''(0)=1, \forall x>0, u(x)$ 表示曲线 $y=f(x)$ 在点 $(x,f(x))$ 处的切线在 $x$ 轴上的截距,
	当 $x\to 0^{+}$ 时,下列等价无穷小不成立的是:  
\begin{itemize}
	\item A. $f(x)\sim \dfrac{x^2}{2}$
	\item B. $x\cdot f(u(x))\sim \dfrac{u(x)\cdot f(x)}{2}$
	\item C. $\int_{0}^{x}u(t)dt\sim \dfrac{x^2}{4}$
	\item D. $\int_{0}^{f(x)}u(t)dt\sim \dfrac{x^4}{4}$
\end{itemize}
\end{example}

\begin{solution}

	我们求出$f(x)$在$(x,f(x))$处的切线方程:$y-f(x)=f'(x)(x'-x)$
	
	我们有:
	$$u(x)=x'=x-\dfrac{f(x)}{f'(x)}$$
	
	我们利用泰勒展开式,将$f(x)$展开:
	$$\left\lbrace 
	\begin{array}{l}
		f(x)=f(0)+f'(0)x+\dfrac{f''(0)}{2}x^2+o(x^2)\\
		f'(x)=f'(0)+f''(0)x+o(x)
	\end{array}
	\right. \to \left\lbrace 
	\begin{array}{l}
		f(x)=\dfrac{1}{2}x^2+o(x^2)\\
		f'(x)=x+o(x)
	\end{array}
	\right. $$
	
	当$x\to 0^{+}$时,$f(x)\sim \dfrac{1}{2}x^2,u(x)\sim \dfrac{1}{2}x$
	
	我们得到:
	$$\left\lbrace 
	\begin{array}{l}
		x\cdot f(u(x))\sim \dfrac{u(x)\cdot f(x)}{2}\\
		\int_{0}^{x}u(t)dt\sim \dfrac{x^2}{4}\\
		\int_{0}^{f(x)}u(t)dt\sim \dfrac{x^4}{16}
	\end{array}
	\right. $$
\end{solution}

\begin{example}[][Exam: 36.4.2]
	比较 $\int_{0}^{\frac{\pi}{2}}\dfrac{\sin x}{1+x^2}dx$ 和 $\int_{0}^{\frac{\pi}{2}}\dfrac{\cos x}{1+x^2}dx$ 的大小
\end{example}

\begin{solution}

	我们不妨设:$I=\int_{0}^{\frac{\pi}{2}}\dfrac{\sin x}{1+x^2}dx,J=\int_{0}^{\frac{\pi}{2}}\dfrac{\cos x}{1+x^2}dx$,我们有:
	\begin{eqnarray*}
		I-J&=&\int_{0}^{\frac{\pi}{2}}\dfrac{\sin x-\cos x}{1+x^2}dx\\
		&=&\int_{0}^{\frac{\pi}{2}}\dfrac{\sin(x-\frac{\pi}{4})}{1+x^2}dx\\
		&=&\int_{-\frac{\pi}{4}}^{\frac{\pi}{4}}\dfrac{\sin t}{1+(t+\frac{\pi}{4})^2}dt\\
		&=&\int_{-\frac{\pi}{4}}^{0}\dfrac{\sin t}{1+(t+\frac{\pi}{4})^2}dt+\int_{0}^{\frac{\pi}{4}}\dfrac{\sin t}{1+(t+\frac{\pi}{4})^2}dt\\
		&=&\int_{0}^{\frac{\pi}{4}}\sin t[\dfrac{1}{1+(t+\frac{\pi}{4})^2}-\dfrac{1}{1+(\frac{\pi}{4}-t)^2}]dt\\
		&=&\int_{0}^{\frac{\pi}{4}}\dfrac{-\pi t\sin t}{[1+(t+\frac{\pi}{4})^2][1+(\frac{\pi}{4}-t)^2]}dt<0
	\end{eqnarray*}
	
\end{solution}


\textcolor{purplea}{October 23}

\begin{example}[][Exam: 36.4.3]
	求曲线 $y=\dfrac{x^2}{1+x^2}$ 与其渐近线所围区域绕该渐近线旋转所得旋转体体积
\end{example}

\begin{solution}

	我们可以得到$f(x)$的渐近线为$y=1$,因此我们有:
	\begin{eqnarray*}
		V&=&\pi\int_{-\infty}^{+\infty}\dfrac{1}{(1+x^2)^2}dx\\
		&=&2\pi\int_{0}^{\frac{\pi}{2}}\cos^2\theta d\theta\\
		&=&\dfrac{\pi^2}{2}
	\end{eqnarray*}
\end{solution}

\begin{example}[][Exam: 36.4.4]
	设 $A$ 是 $n$ 阶矩阵, $A$ 的第 $i$ 行第 $j$ 列元素为 $a_{ij}$,满足 $a_{ij}=i\cdot j$,下列命题正确的是:  
\begin{itemize}
	\item A. $r(A)=1$
	\item B. 矩阵$A$不可相似对角化
	\item C. 矩阵$A$的特征值之和为$\sum\limits_{k=1}^{n}k$
	\item D. 矩阵$A$的特征值之和为$\sum\limits_{k=1}^{n}k^2$
\end{itemize}
\end{example}

\begin{solution}

	我们由:$a_{ij}=i\cdot j\to a_{ij}=a_{ji}=i\cdot j$,即矩阵$A$为实对称矩阵,$A$一定可以相似对角化
	
	我们可以得到:$|A|\neq 0\to r(A)=n$且矩阵$A$的特征值之和为$\sum\limits_{i=1}^{n}i^2$
\end{solution}


\textcolor{purplea}{October 24}

\begin{example}[][Exam: 36.4.5]
	$$\iint\limits_{D}\dfrac{1}{x^4+y^2}dxdy, D=\{(x,y)|y\geq x^2+1\}$$
\end{example}
\begin{solution}

	原二重积分等价于:
	\begin{eqnarray*}
		I&=&\int_{-\infty}^{+\infty}dx\int_{x^2+1}^{+\infty}\dfrac{1}{x^4+y^2}dy\\
		&=&2\int_{0}^{+\infty}\dfrac{1}{x^2}\arctan(\dfrac{x^2}{x^2+1})dx\\
		&=&2\int_{0}^{+\infty}\arctan(\dfrac{1}{t^2+1})dt\\
		&=&
	\end{eqnarray*}
\end{solution}

\begin{example}[][Exam: 36.4.6]
	$$\iint\limits_{D}\dfrac{1}{x^4+y^2}dxdy, D=\{(x,y)|x\geq 1,y\geq x^2\}$$
\end{example}

\begin{solution}

	原二重积分等价于:
	\begin{eqnarray*}
		I&=&\int_{1}^{+\infty}dx\int_{x^2}^{+\infty}\dfrac{1}{x^4+y^2}dy\\
		&=&\int_{1}^{+\infty}\dfrac{\pi}{4x^2}dx\\
		&=&\dfrac{\pi}{4}
	\end{eqnarray*}
\end{solution}


\textcolor{purplea}{October 25}

\begin{example}[][Exam: 36.4.7]
	曲线 $y=x^2$ 与直线 $y=mx(m>0)$在第一象限内所围成的图形绕该直线旋转所形成的旋转体的体积$V$
\end{example}

\begin{solution}

	我们设围成区域中任意一点$(x,y)$,我们有:$d=\dfrac{mx-y}{\sqrt{1+m^2}}$
	\begin{eqnarray*}
		V&=&\iint\limits_{S}2\pi ddxdy\\
		&=&\dfrac{2\pi}{\sqrt{1+m^2}}\int_{0}^{m}dx\int_{x^2}^{mx}(mx-y)dy\\
		&=&\dfrac{\pi m^5}{30\sqrt{1+m^2}}
	\end{eqnarray*}
\end{solution}

\begin{example}[][Exam: 36.4.8]
	设函数 $f(x,y)=
	\begin{cases}
		(xy+a|x|+b\sqrt{|y|})\arctan \dfrac{1}{|x|+y^2} & (x,y)\neq (0,0)\\
		0 & (x,y)=(0,0)
	\end{cases}$,下列说法中正确的是:  
\begin{itemize}
	\item A. $f(x,y)$在$(0,0)$处的连续性和$a,b$的取值有关
	\item B. $f(x,y)$在$(0,0)$处偏导数存在的充要条件是$ab=0$
	\item C. $f(x,y)$在$(0,0)$处可微的充要条件是$f(x,y)$在$(0,0)$处偏导数存在
	\item D. $f(x,y)$在$(0,0)$处可微,则$(0,0)$是$f(x,y)$的极值点
\end{itemize}
\end{example}

\begin{solution}
	
\end{solution}


\textcolor{purplea}{October 26}

\begin{example}[][Exam: 36.4.9]
	设 $f(x)$ 在 $[0,+\infty)$ 上可导, $f'(x)+f^2(x)\geq 0,f(0)=1,f(x)\neq 0$,证明: $f(x)\geq \dfrac{1}{x+1}$
\end{example}

\begin{solution}
	
\end{solution}

\begin{example}[][Exam: 36.4.10]
	设函数 $f(x)$ 在 $[0,1]$ 上二阶可导,且 $\int_{0}^{1}f(x)dx=0$
\begin{itemize}
	\item A. 当$f'(x)<0$时,$f(\dfrac{1}{2})<0$
	\item B. 当$f''(x)<0$时,$f(\dfrac{1}{2})<0$
	\item C. 当$f'(x)>0$时,$f(\dfrac{1}{2})<0$
	\item D. 当$f''(x)>0$时,$f(\dfrac{1}{2})<0$
\end{itemize}
\end{example}

\begin{solution}
	
\end{solution}


\textcolor{purplea}{October 27}

\begin{example}[][Exam: 36.4.11]
	$f(x)=
	\begin{pmatrix}
		1&x&x^2&x^3\\
		1&2&4&8\\
		1&-1&1&-1\\
		1&1&1&1
	\end{pmatrix}$,求曲线 $y=f(x)$ 在 $(-1,2)$ 内存在水平切线的条数
\end{example}

\begin{solution}
	
\end{solution}

\begin{example}[][Exam: 36.4.12]
	已知正值连续函数 $f(x)$ 在 $[0,1]$ 上单调减少,$ \forall a,b(0<a<b<1)$,下列结论不正确的是
\begin{itemize}
	\item A. $a\int_{0}^{b}f(x)dx>b\int_{0}^{a}f(x)dx$
	\item B. $b\int_{0}^{a}f(x)dx>a\int_{0}^{b}f(x)dx$
	\item C. $a\int_{0}^{b}\sqrt{f(x)}dx<b\int_{0}^{a}\sqrt{f(x)}dx$
	\item D. $b\int_{0}^{a}\sqrt{f(x)}dx<a\int_{0}^{b}\sqrt{f(x)}dx$
\end{itemize}
\end{example}

\begin{solution}
	
\end{solution}


\textcolor{purplea}{October 28}

\begin{example}[][Exam: 36.4.13]
	设函数 $\varphi(x,y)$ 的全微分为 $dz=(2x-y^2-2y)dx+(-2xy-2x+y^3+3y)dy$,$f(x,y)$连续,
	且$\lim\limits_{(x,y)\to (0,0)}\dfrac{f(x,y)}{\varphi(x,y)}=-1$
\begin{itemize}
	\item A. 点$(0,0)$是$f(x,y)$的极大值点
	\item B. 点$(0,0)$是$f(x,y)$的极小值点
	\item C. 点$(0,0)$不是$f(x,y)$的极值点
	\item D. 不能确定点$(0,0)$是否为$f(x,y)$的极值点
\end{itemize}
\end{example}

\begin{solution}
	
\end{solution}

\begin{example}[][Exam: 36.4.14]
	求 $\int_{L}\dfrac{|y|}{x^2+y^2+z^2}ds$,其中$L:\begin{cases}
		x^2+y^2+z^2=a^2\\
		x^2+y^2=2ax\\
		a\geq 0
	\end{cases}(a>0)$
\end{example}


\begin{solution}
	
\end{solution}


\textcolor{purplea}{October 29}

\begin{example}[][Exam: 36.4.15]
	设$I_{1}=\int_{0}^{\pi}\dfrac{x\sin^2 x}{1+e^{\cos^2 x}}dx$,$I_{2}=\int_{0}^{\pi}\dfrac{\sin^2 x}{1+e^{\cos^2 x}}dx$,
	$I_{3}=\int_{0}^{\frac{\pi}{2}}\dfrac{\cos^2 x}{1+e^{\sin^2 x}}dx$,比较$I_{1},I_{2},I_{3}$的大小
\end{example}
\begin{solution}
	
\end{solution}

\begin{example}[][Exam: 36.4.16]
	设 $f(u,v)$ 有一阶偏导数,$f(x,1-x)=1,f_{1}^{'}(x,1-x)=x$

(1). 设 $z(t)=f(\cos t,\sin t)$, 计算 $z'(0)$
 
(2). 证明: $f(u,v)$ 在单位圆周上至少存在两个不同的点满足方程: $v\dfrac{\partial f}{\partial u}=u\dfrac{\partial f}{\partial v}$
\end{example}

\begin{solution}
	
\end{solution}


\textcolor{purplea}{October 30}

\begin{example}[][Exam: 36.4.17]
	设 $f(x)$ 在 $[0,1]$ 连续可导,$\int_{0}^{1}f(x)dx=\dfrac{5}{2}, \int_{0}^{1}xf(x)dx=\dfrac{3}{2}$,
	证明: $\exists \xi\in(0,1),\ s.t.\ f'(\xi)=3$
\end{example}
\begin{solution}
	
\end{solution}

\begin{example}[][Exam: 36.4.18]
	$$\lim\limits_{x\to 0 }\dfrac{(1-\cos^3 x)(1-\cos^{17} x)}{\dfrac{x^2}{2}-\int_{0}^{x}\sum\limits_{n=0}^{+\infty}\dfrac{(-1)^n}{3^n(2n+1)!}t^{2n+1}dt}$$
\end{example}

\begin{solution}
	
\end{solution}


\textcolor{purplea}{October 31}

\begin{example}[][Exam: 36.4.19]
	设偶函数 $f(t)$ 具有连续的导函数,且满足 $f(t)=e^{4\pi t^2}+\iint\limits_{x^2+y^2\leq 4t^2}f(\dfrac{\sqrt{x^2+y^2}}{2})dxdy$,
	求方程 $\int_{0}^{x}\sqrt{1+4\pi t^2}dt+\int_{\cos x}^{0}\dfrac{1+4\pi t^2}{f(t)}dt=0$ 在 $(0,+\infty)$ 内根的个数
\end{example}
\begin{solution}
	
\end{solution}

\begin{example}[][Exam: 36.4.20]
	设二阶可导函数 $f(x)$ 满足 $f(0)=f(2)=0, f(1)=a>0$ 且 $f''(x)<0$
\begin{itemize}
	\item A. $\int_{0}^{2}f(x)dx>a$
	\item B. $\int_{0}^{2}f(x)dx<a$
	\item C. $\int_{0}^{1}f(x)dx>\int_{1}^{2}f(x)dx$
	\item D. $\int_{0}^{1}f(x)dx<\int_{1}^{2}f(x)dx$
\end{itemize}
\end{example}

\begin{solution}
	
\end{solution}
