\chapterimage{chap27.jpg}
\chapter{January}
\section{Week \Rmnum{1}}

\hl{\textbf{\textit{January 1}}}

1. 已知 $f(x+1)$ 的定义域为$[0,a]$,($a>0$),求 $f(x)$ 定义域
\myspace{1}
\begin{solution}
	
	$f(x+1)$ 的定义域为$[0,a]$,则 $f(x)$ 的定义域为$[-1,a-1]$
\end{solution}
\myspace{1}

2. 已知 $f(x)=e^{x^{2}},f[\varphi(x)]=1-x$ 且 $\varphi(x)\geq 0$,求 $\varphi(x)$ 并求出定义域
\myspace{1}
\begin{solution}
	
	$f[\varphi(x)]=1-x\Rightarrow e^{\varphi^{2}(x)}=1-x$, 因此: $\varphi(x)=\sqrt{\ln (1-x)}, x\ge 1$
\end{solution}
\myspace{1}

3. 设 
	$$g(x)=
	\begin{cases}
		2-x,x\leq 0\\ x+2,x>0
	\end{cases}$$,$$f(x)=
	\begin{cases}
		x^{2},x< 0\\ -x,x\geq 0
	\end{cases}
	$$
求 $g[f(x)]$
\myspace{1}
\begin{solution}

	$$
	g[f(x)] = 
	\begin{cases}
		2+x^{2}, & x<0\\
		x+2, & x\geq 0
	\end{cases}
	$$
\end{solution}
\myspace{1}

4. 设函数 $$f(x)=
\begin{cases}
	1-2x^{2},x<-1\\
	x^{3},-1\leq x\leq 2\\
	12x-16,x>2
\end{cases}
$$
求 $f(x)$ 的反函数 $g(x)$ 的表达式
\myspace{1}
\begin{solution}
	$$
	g(x) = 
	\begin{cases}
		-\sqrt{\dfrac{1-x}{2}}, & x\in(-\infty,-1)\\
		\displaystyle{\sqrt[3]{x}}, & x\in[-1,8]\\
		\dfrac{x+16}{12}, & x\in(8,+\infty)
	\end{cases}
	$$
\end{solution}
\myspace{1}

5. 证明:定义在 $[-a,a]$ 上的任意一个函数 $f(x)$ 都可以表示为一个奇函数和一个偶函数之和
\myspace{1}
\begin{solution}

	令 $g(x)= \dfrac{f(x)+f(-x)}{2}\quad h(x)=\dfrac{f(x)-f(-x)}{2}$
	$$
	\begin{cases}
		g(-x)=\dfrac{f(-x)+f(x)}{2}=\dfrac{f(x)+f(-x)}{2}=g(x)\\
		h(-x)=\dfrac{f(-x)-f(x)}{2}=\dfrac{f(x)-f(-x)}{2}=-h(x)\\
		f(x) = g(x)+h(x)
	\end{cases}
	$$
	
	其中 $g(x)$ 是偶函数, $h(x)$ 是奇函数
\end{solution}
\myspace{1}

6. 判断函数 $f(x)=x\tan x\cdot e^{\sin x}$的奇偶性、单调性、周期性和有界性
\myspace{1}
\begin{solution}

	$f(x)$ 定义域为 $(-\dfrac{\pi}{2}+k\pi,\dfrac{\pi}{2}+k\pi),k\in \mathbb{Z}$, 关于原点对称
	
	(1). 奇偶性: $f(-x) = x\tan x\cdot e^{-\sin x}\neq -f(x),\quad f(-x)\neq f(x)$

	$f(x)$ 是非奇非偶函数

	(2). 单调性: $f'(x) = \tan x\cdot e^{\sin x}+x\sec^{2}x\cdot e^{\sin x}+x\tan x\cos x\cdot e^{\sin x}=e^{\sin x}\left[\tan x+x\sin x+x\sec^{2}x\right]$

	$f(x)$ 不是单调函数

	(3). 有界性: $x\in(-\dfrac{\pi}{2},\dfrac{\pi}{2})$ 时, $f(x) > \dfrac{x\tan x}{e}$, 函数 $g(x)=\frac{x\tan x}{e}$ 为无界函数

	$f(x)$ 是无界函数

	(4). 周期性: $f(x+2\pi) = (x+2\pi)\tan(x+2\pi)\cdot e^{\sin(x+2\pi)} = (x+2\pi)\tan x\cdot e^{\sin x} \neq f(x)$

	$f(x)$ 不是周期函数
\end{solution}
\myspace{1}

7. 函数 $f(x)=\dfrac{|x|\sin(x-2)}{x(x-1)(x-2)^{2}}$ 在下列哪个区间内有界

\begin{itemize}
	\item A. $(-1,0)$
	\item B. $(0,1)$
	\item C. $(1,2)$
	\item D. $(2,3)$
\end{itemize}
\myspace{1}
\begin{solution}

	(1). $x\to 0^{-}, f(x)\to -\dfrac{\sin 2}{4}; x\to 0^{+}, f(x)\to \dfrac{\sin 2}{4}$

	(2). $x\to 1^{-}, f(x)\to +\infty; x\to 1^{+}, f(x)\to -\infty$

	(3). $x\to 2^{-}, f(x)\to -\infty; x\to 2^{+}, f(x)\to \infty$

	$f(x)$ 在区间 $(-1,0)$ 有界
\end{solution}
\myspace{1}

\hl{\textbf{\textit{January 2}}}

1. 求$\lim\limits_{n\to\infty}\left[\sqrt{1+2+\cdots+n}-\sqrt{1+2+\cdots+(n-1)}\right]$
\myspace{1}
\begin{solution}

	\begin{align*}
		I = & \lim\limits_{n\to\infty}\left[\sqrt{1+2+\cdots+n}-\sqrt{1+2+\cdots+(n-1)}\right]\\
		  = & \lim\limits_{n\to\infty}\left[\sqrt{\dfrac{n(n+1)}{2}}-\sqrt{\dfrac{(n-1)n}{2}}\right]\\
		  = & \lim\limits_{n\to\infty}\dfrac{\left[\dfrac{n(n+1)}{2}-\dfrac{n(n-1)}{2}\right]}{\sqrt{\dfrac{n(n+1)}{2}}+\sqrt{\dfrac{(n-1)n}{2}}}\\
		  = & \lim\limits_{n\to\infty}\dfrac{n}{\sqrt{2}n}\\
		  = & \dfrac{1}{\sqrt{2}} 
	\end{align*}
\end{solution}
\myspace{1}

2. 求极限 $\lim\limits_{x\to-\infty}\dfrac{\sqrt{4x^{2}+x-1}+x+1}{\sqrt{x^{2}+\sin x}}$
\myspace{1}
\begin{solution}

	\begin{align*}
		I = & \lim\limits_{x\to-\infty}\dfrac{\sqrt{4x^{2}+x-1}+x+1}{\sqrt{x^{2}+\sin x}}\\
		  = & \lim\limits_{x\to-\infty}\dfrac{\sqrt{x^{2}(4+\dfrac{1}{x}-\dfrac{1}{x^{2}})}+x(1+\dfrac{1}{x})}{\sqrt{x^{2}(1+\dfrac{\sin x}{x^{2}})}}\\
		  = & \lim\limits_{x\to-\infty}\dfrac{\sqrt{4+\dfrac{1}{x}-\dfrac{1}{x^{2}}}-(1+\dfrac{1}{x})}{\sqrt{1+\dfrac{\sin x}{x^{2}}}}\\
		  = & 1
	\end{align*}
\end{solution}
\myspace{1}

\hl{\textbf{\textit{January 3}}}

1. 求极限 $\lim\limits_{x\to 0}\left(\dfrac{2+e^{\frac{1}{x}}}{1+e^{\frac{4}{x}}}+\dfrac{\sin x}{|x|}\right)$
\myspace{1}
\begin{solution}

	\begin{align*}
		I^{+} = & \lim\limits_{x\to 0^{+}}\left(\dfrac{2+e^{\frac{1}{x}}}{1+e^{\frac{4}{x}}}+\dfrac{\sin x}{|x|}\right)\\
		      = & \lim\limits_{x\to 0^{+}}\dfrac{2+e^{\frac{1}{x}}}{1+e^{\frac{4}{x}}}+\lim\limits_{x\to 0^{+}}\dfrac{\sin x}{x}\\
			  = & 0+1 = 1\\
		I^{-} = & \lim\limits_{x\to 0^{-}}\left(\dfrac{2+e^{\frac{1}{x}}}{1+e^{\frac{4}{x}}}+\dfrac{\sin x}{|x|}\right)\\
			  = & \lim\limits_{x\to 0^{-}}\dfrac{2+e^{\frac{1}{x}}}{1+e^{\frac{4}{x}}}-\lim\limits_{x\to 0^{-}}\dfrac{\sin x}{x}\\
			  = & 2-1 = 1\\
	\end{align*}
	综上所述, 原极限 $I =1$
\end{solution}
\myspace{1}

2. 求极限 $\lim\limits_{x\to+\infty}(\sqrt{x^{2}+x}-\sqrt{x^{2}-x})$
\myspace{1}
\begin{solution}

	原极限等价于:
	\begin{align*}
		I = & \lim\limits_{x\to+\infty}(\sqrt{x^{2}+x}-\sqrt{x^{2}-x})\\
		  = & \lim\limits_{x\to+\infty}\dfrac{(\sqrt{x^{2}+x}-\sqrt{x^{2}-x})(\sqrt{x^{2}+x}+\sqrt{x^{2}-x})}{\sqrt{x^{2}+x}+\sqrt{x^{2}-x}}\\
		  = & \lim\limits_{x\to+\infty}\dfrac{x^{2}+x-(x^{2}-x)}{\sqrt{x^{2}+x}+\sqrt{x^{2}-x}}\\
		  = & \lim\limits_{x\to+\infty}\dfrac{2x}{\sqrt{x^{2}+x}+\sqrt{x^{2}-x}}\\
		  = & \lim\limits_{x\to+\infty}\dfrac{2}{\sqrt{1+\dfrac{1}{x}}+\sqrt{1-\dfrac{1}{x}}}\\
		  = & 1
	\end{align*}
\end{solution}
\myspace{1}

\hl{\textbf{\textit{January 4}}}

1. 求极限 $\lim\limits_{x\to 0}\dfrac{\sqrt{1+\tan x}-\sqrt{1+\sin x}}{x(1-\cos x)}$
\myspace{1}
\begin{solution}

	原极限等价于:
	\begin{align*}
		I = & \lim\limits_{x\to 0}\dfrac{\sqrt{1+\tan x}-\sqrt{1+\sin x}}{x(1-\cos x)}\\
		  = & \lim\limits_{x\to 0}\dfrac{\sqrt{1+\tan x}-\sqrt{1+\sin x}}{x(1-\cos x)}\cdot \dfrac{\sqrt{1+\tan x}+\sqrt{1+\sin x}}{\sqrt{1+\tan x}+\sqrt{1+\sin x}}\\
		  = & \lim\limits_{x\to 0}\dfrac{1+\tan x-1-\sin x}{x(1-\cos x)(\sqrt{1+\tan x}+\sqrt{1+\sin x})}\\
		  = & \lim\limits_{x\to 0}\dfrac{\tan x-\sin x}{x(1-\cos x)(\sqrt{1+\tan x}+\sqrt{1+\sin x})}\\
		  = & \lim\limits_{x\to 0}\dfrac{\dfrac{1}{2}x^{3}}{\dfrac{1}{2}x^{3}(\sqrt{1+\tan x}+\sqrt{1+\sin x})}\\
		  = & \dfrac{1}{2}
	\end{align*}
\end{solution}
\myspace{1}

2. 已知 $\lim\limits_{x\to 0}\dfrac{e^{x^{2}}-\cos 2x}{ax^{b}}=1$,求 $a,b$
\myspace{1}
\begin{solution}

	利用泰勒展开式:
	\begin{align*}
		I = & \lim\limits_{x\to 0}\dfrac{e^{x^{2}}-\cos 2x}{ax^{b}}\\
		  = & \lim\limits_{x\to 0}\dfrac{1+x^{2}+\dfrac{x^{4}}{2}-1+\dfrac{4x^{2}}{2!}-\dfrac{8x^{4}}{4!}}{ax^{b}}\\
		  = & \lim\limits_{x\to 0}\dfrac{3x^{2}-\dfrac{x^{4}}{6}}{ax^{b}}\\
		  = & \lim\limits_{x\to 0}\dfrac{3}{a} = 1
	\end{align*}
	因此我们有: $a = 3, b = 2$
\end{solution}
\myspace{1}

\hl{\textbf{\textit{January 5}}}

1. 已知 $\lim\limits_{x\to x_{0}}\varphi(x)=0$,下列结论正确的个数为
\begin{itemize}
	\item A. $\lim\limits_{x\to x_{0}}\dfrac{\sin\varphi(x)}{\varphi(x)}=1$
	\item B. $\lim\limits_{x\to x_{0}}[1+\varphi(x)]^{\frac{1}{\varphi}}=e$
	\item C. 当$x\to x_{0}$时,$\sin \varphi(x)\sim \varphi(x)$
	\item D. 若$\lim\limits_{u\to 0}f(u)=A$,则 $\lim\limits_{x\to x_{0}}f[\varphi(x)]=A$
\end{itemize}
\myspace{1}
\begin{solution}

	正确的个数: $0$, 令 $\varphi(x) = 0,x\in \mathring{U}(x_{0},\delta)$, $A,B,C,D$ 四个选项均不正确
\end{solution}
\myspace{1}

2. 求极限 $\lim\limits_{x\to 0}\dfrac{\arcsin x-\arctan x}{\sin x-\tan x}$
\myspace{1}
\begin{solution}

	原极限等价于:
	\begin{align*}
		I = & \lim\limits_{x\to 0}\dfrac{\arcsin x-\arctan x}{\sin x-\tan x}\\
		  = & \lim\limits_{x\to 0}\dfrac{\dfrac{1}{2}x^{3}}{-\dfrac{1}{2}x^{3}}\\
		  = & -1 
	\end{align*}
\end{solution}
\myspace{1}
\hl{\textbf{\textit{January 6}}}

1. 求极限 $\lim\limits_{x\to 0}\dfrac{\ln\dfrac{x}{\ln(1+x)}}{x}$
\myspace{1}
\begin{solution}

	原极限等价于:
	\begin{align*}
		I = & \lim\limits_{x\to 0}\dfrac{\ln\dfrac{x}{\ln(1+x)}}{x}\\
		  = & \lim\limits_{x\to 0}\dfrac{x-\ln(1+x)}{x\ln (1+x)}\\
		  = & \lim\limits_{x\to 0}\dfrac{\dfrac{1}{2}x^{2}}{x^{2}}\\
		  = & \dfrac{1}{2}
	\end{align*}
\end{solution}
\myspace{1}

2. 求极限 $\lim\limits_{x\to +\infty}\dfrac{e^{x}}{(1+\dfrac{1}{x})^{x^{2}}}$
\myspace{1}
\begin{solution}
	
	原极限等价于:
	\begin{align*}
		I = & \lim\limits_{x\to +\infty}\dfrac{e^{x}}{(1+\frac{1}{x})^{x^{2}}}\\
		  = & \lim\limits_{x\to +\infty}\dfrac{e^{x}}{e^{x^{2}\ln(1+\frac{1}{x})}}\\
		  = & \lim\limits_{x\to +\infty}e^{x-x^{2}\ln(1+\frac{1}{x})}\\
		  = & \lim\limits_{x\to +\infty}e^{x-x^{2}(\frac{1}{x}-\frac{1}{2x^{2}})}\\
		  = & \lim\limits_{x\to +\infty}e^{\frac{1}{2}}\\
		  = & e^{\frac{1}{2}}
	\end{align*}
\end{solution}
\myspace{1}
\hl{\textbf{\textit{January 7}}}

1. 求极限 $\lim\limits_{x\to 0}\dfrac{\cos x-\cos(\sin x)}{x^{4}}$
\myspace{1}
\begin{solution}
	
	原极限等价于:
	\begin{align*}
		I = & \lim\limits_{x\to 0}\dfrac{\cos x-\cos(\sin x)}{x^{4}} (\text{Lagrange's Mean Value Theorem})\\
		  = & \lim\limits_{x\to 0}\dfrac{-\sin \xi(x-\sin x)}{x^{4}}, \xi\in (\sin x\sim x)\\
		  = & \lim\limits_{x\to 0}\dfrac{-\sin \xi}{6x}\\
		  = & -\dfrac{1}{6}
	\end{align*}

	我们由夹逼定理: $\sin \xi \in (\sin x\sim x)\to \lim\limits_{x\to 0}\dfrac{\sin x}{x} = \lim\limits_{x\to 0}\dfrac{x}{x}=1\Rightarrow \lim\limits_{x\to 0}\dfrac{\sin \xi}{x} = 1$

\end{solution}
\myspace{1}

2. 求极限 $\lim\limits_{x\to +\infty}x^{2}\left[\arctan(x+1)-\arctan x\right]$
\myspace{1}
\begin{solution}
	
	原极限等价于:
	\begin{align*}
		I = & \lim\limits_{x\to +\infty}x^{2}[\arctan(x+1)-\arctan x](\text{Lagrange's Mean Value Theorem})\\
		  = & \lim\limits_{x\to +\infty}x^{2}\dfrac{1}{1+\xi^{2}},\xi\in(x,x+1)\\
		  = & 1
	\end{align*}

	我们由夹逼定理: $\xi\in (x, x+1)\to \lim\limits_{x\to +\infty}\frac{x^{2}}{1+(x^{2})} = \lim\limits_{x\to +\infty}\frac{x^{2}}{1+(x+1)^{2}}=1\Rightarrow \lim\limits_{x\to +\infty}\frac{x^{2}}{1+\xi^{2}} = 1$
\end{solution}
\myspace{1}

\section{Week \Rmnum{2}}
\hl{\textbf{\textit{January 8}}}

1. 求极限 $\lim\limits_{n\to \infty}n^{2}\left[\arctan\dfrac{a}{n}-\arctan \dfrac{a}{n+1}\right]$
\myspace{1}
\begin{solution}
	
	由归结原理,原极限等价于:
	\begin{align*}
		I = & \lim\limits_{x\to \infty}x^{2}[\arctan\dfrac{a}{x}-\arctan \dfrac{a}{x+1}](\text{Lagrange's Mean Value Theorem})\\
		  = & \lim\limits_{x\to \infty}\dfrac{ax^{2}}{x(x+1)}\dfrac{1}{1+\xi^{2}},\xi\in(\dfrac{a}{x},\dfrac{a}{x+1})\\
		  = & a
	\end{align*}

	我们由夹逼定理: $\xi\in (\dfrac{a}{x},\dfrac{a}{x+1})\to \lim\limits_{x\to \infty}\dfrac{a}{x} = \lim\limits_{x\to \infty}\dfrac{a}{x+1}=0\Rightarrow \lim\limits_{x\to \infty}\xi = 0$
\end{solution}
\myspace{1}

2. 求极限 $\lim\limits_{x\to +\infty}[\sin\sqrt{x+1}-\sin\sqrt{x}]$
\myspace{1}
\begin{solution}
	
	原极限等价于:
	\begin{align*}
		I = & \lim\limits_{x\to +\infty}[\sin\sqrt{x+1}-\sin\sqrt{x}](\text{Lagrange's Mean Value Theorem})\\
		  = & \lim\limits_{x\to +\infty}\cos \xi(\sqrt{x+1}-\sqrt{x}),\xi\in(\sqrt{x},\sqrt{x+1})\\
		  = & \lim\limits_{x\to +\infty}\cos \xi(\dfrac{1}{\sqrt{x+1}+\sqrt{x}})\\
		  = & 0
	\end{align*}
\end{solution}
\myspace{1}

\hl{\textbf{\textit{January 9}}}

1. 求极限 $\lim\limits_{x\to 0}\left[\dfrac{1}{\ln(1+x^{2})}-\dfrac{1}{\ln(1+\tan^{2}x)}\right]$
\myspace{1}
\begin{solution}
	
	原极限等价于:
	\begin{align*}
		I = & \lim\limits_{x\to 0}\left[\dfrac{1}{\ln(1+x^{2})}-\dfrac{1}{\ln(1+\tan^{2}x)}\right]\\
		  = & \lim\limits_{x\to 0}\dfrac{\ln(1+\tan^{2}x)-\ln(1+x^{2})}{\ln(1+x^{2})\ln(1+\tan^{2}x)}(\text{Lagrange's Mean Value Theorem})\\
		  = & \lim\limits_{x\to 0}\dfrac{(\tan x+x)(\tan x-x)}{x^{4}\xi}, \xi\in(1+x^{2},1+\tan^{2}x)\\
		  = & \lim\limits_{x\to 0}\dfrac{2x\cdot \dfrac{x^{3}}{3}}{x^{4}}\\
		  = & \dfrac{2}{3}
	\end{align*}

	我们由夹逼定理: $\xi\in(1+x^{2},1+\tan^{2}x)\to \lim\limits_{x\to 0}(1+x^{2}) = \lim\limits_{x\to 0} (1+\tan^{2}x) = 1\Rightarrow \lim\limits_{x\to 0}\xi = 1$
\end{solution}
\myspace{1}

2. 求极限 $\lim\limits_{x\to 0}\left[\dfrac{1}{\ln(1+x^{2})}-\dfrac{1}{\sin^{2}x}\right]$
\myspace{1}
\begin{solution}

	原极限等价于:
	\begin{align*}
		I = & \lim\limits_{x\to 0}\left[\dfrac{1}{\ln(1+x^{2})}-\dfrac{1}{\sin^{2}x}\right]\\
		  = & \lim\limits_{x\to 0}\dfrac{\sin^{2}x-\ln(1+x^{2})}{\ln(1+x^{2})\sin^{2}x}\\
		  = & \lim\limits_{x\to 0}\dfrac{\sin^{2}x-x^{2}+x^{2}-\ln(1+x^{2})}{x^{4}}\\
		  = & \lim\limits_{x\to 0}\dfrac{\sin^{2}x-x^{2}}{x^{4}}+\lim\limits_{x\to 0}\dfrac{x^{2}-\ln(1+x^{2})}{x^{4}}\\
		  = & \lim\limits_{x\to 0}\dfrac{(\sin x+x)(\sin x-x)}{x^{4}}+\lim\limits_{x\to 0}\dfrac{\dfrac{1}{2}x^{4}}{x^{4}}\\
		  = & \dfrac{1}{2}+\lim\limits_{x\to 0}\dfrac{2x\cdot(-\dfrac{1}{6}x^{3})}{x^{4}}\\ 
		  = & \dfrac{1}{6}
	\end{align*}
	
	我们由夹逼定理: $\xi\in(1+\sin^{2}x,1+x^{2})\to \lim\limits_{x\to 0}(1+\sin^{2}x) = \lim\limits_{x\to 0} (1+x^{2}) = 1\Rightarrow \lim\limits_{x\to 0}\xi = 1$
\end{solution}
\myspace{1}
\hl{\textbf{\textit{January 10}}}

1. 求极限 $\lim\limits_{x\to 0}\left(x+2^{x}\right)^{\frac{2}{x}}$
\myspace{1}
\begin{solution}
	
	原极限等价于:
	\begin{align*}
		I = & \lim\limits_{x\to 0}(x+2^{x})^{\frac{2}{x}}\\
		  = & \lim\limits_{x\to 0}e^{\frac{2\ln(x+2^{x})}{x}}\\
		  = & e^{\lim\limits_{x\to 0}\frac{2\ln(x+2^{x})}{x}}\\
		  = & e^{\lim\limits_{x\to 0}\frac{2(x+2^{x}-1)}{x}}\\
		  = & e^{\lim\limits_{x\to 0}\frac{2x}{x}+\lim\limits_{x\to 0}\frac{2x\ln 2}{x}}\\
		  = & e^{2+2\ln 2}\\
		  = & 4e^{2}
	\end{align*}
\end{solution}
\myspace{1}

2. 若 $\lim\limits_{x\to 0}\left(\dfrac{1-\tan x}{1+\tan x}\right)^{\frac{1}{\sin kx}}=e$,求 $k$
\myspace{1}
\begin{solution}
	
	原极限等价于:
	\begin{align*}
		I = & \lim\limits_{x\to 0}\left(\dfrac{1-\tan x}{1+\tan x} \right)^{\frac{1}{\sin kx}} \\
		  = & \lim\limits_{x\to 0}e^{\frac{1}{\sin kx}\ln\frac{1-\tan x}{1+\tan x}}\\
		  = & e^{\lim\limits_{x\to 0}\frac{1}{\sin kx}\ln(1-\frac{2\tan x}{1+\tan x})}\\
		  = & e^{\lim\limits_{x\to 0}-\frac{2\tan x}{(1+\tan x)sin kx}}\\
		  = & e^{\lim\limits_{x\to 0}-\frac{2x}{kx}}\\
		  = & e^{-\frac{2}{k}} = e
	\end{align*}

	综上所述, $k = -2$
\end{solution}
\myspace{1}
\hl{\textbf{\textit{January 11}}}

1.  若 $\lim\limits_{x\to 0}(e^{x}+ax^{2}+bx)^{\frac{1}{x^{2}}}=1$,求 $a,b$
\myspace{1}
\begin{solution}
	
	原极限等价于:
	\begin{align*}
		I = & \lim\limits_{x\to 0}(e^{x}+ax^{2}+bx)^{\frac{1}{x^{2}}}\\
		  = & \lim\limits_{x\to 0}e^{\frac{1}{x^{2}}\ln(e^{x}+ax^{2}+bx)}\\
		  = & e^{\lim\limits_{x\to 0}\frac{e^{x}+ax^{2}+bx-1}{x^{2}}}\quad (\text{Taylor's Formula})\\
		  = & e^{\lim\limits_{x\to 0}\frac{(a+\frac{1}{2})x^{2}+(b+1)x+o(x^{2})}{x^{2}}}\\
		  = & 1
	\end{align*}

	综上所述, $a = -\dfrac{1}{2}, b = -1$
\end{solution}
\myspace{1}

2. 求极限 $\lim\limits_{x\to 0}\left(\dfrac{\arctan x}{x}\right)^{\frac{1}{1-\cos x}}$
\myspace{1}
\begin{solution}
	
	原极限等价于:
	\begin{align*}
		I = & \lim\limits_{x\to 0}\left(\dfrac{\arctan x}{x}\right)^{\frac{1}{1-\cos x}}\\
		  = & \lim\limits_{x\to 0}e^{\frac{1}{1-\cos x}\ln\frac{\arctan x}{x}}\\
		  = & e^{\lim\limits_{x\to 0}\frac{1}{1-\cos x}\ln\frac{\arctan x}{x}}\\
		  = & e^{\lim\limits_{x\to 0}\frac{1}{1-\cos x}\ln(1+\frac{\arctan x-x}{x})}\\
		  = & e^{\lim\limits_{x\to 0}\frac{1}{1-\cos x}\frac{\arctan x-x}{x}}\\
		  = & e^{\lim\limits_{x\to 0}-\frac{2x^{3}}{3x^{3}}}\\
		  = & e^{-\frac{2}{3}}
	\end{align*}
\end{solution}
\myspace{1}
\hl{\textbf{\textit{January 12}}}

1. 求极限 $\lim\limits_{n\to \infty}\left(n\tan\frac{1}{n}\right)^{n^{2}}$
\myspace{1}
\begin{solution}
	
	由归结原理,原极限等价于:
	\begin{align*}
		I = & \lim\limits_{x\to \infty}\left(x\tan\frac{1}{x}\right)^{x^{2}}\\
		  = & \lim\limits_{x\to \infty}e^{x^{2}\ln(x\tan\frac{1}{x})}\\
		  = & e^{\lim\limits_{x\to \infty}x^{2}\ln(x\tan\frac{1}{x})}\\
		  = & e^{\lim\limits_{t\to 0}\frac{\ln(\frac{\tan t}{t})}{t^{2}}}\\
		  = & e^{\lim\limits_{t\to 0}\frac{\tan t-t}{t^{3}}}\\
		  = & e^{\frac{1}{3}}
	\end{align*}
\end{solution}
\myspace{1}

2. 求极限 $\lim\limits_{n\to \infty}\tan^{n}\left( \dfrac{\pi}{4}+\dfrac{1}{n}\right)$
\myspace{1}
\begin{solution}
	
	由归结原理,原极限等价于:
	\begin{align*}
		I = & \lim\limits_{x\to \infty}e^{x\ln(\tan(\frac{\pi}{4}+\frac{1}{x}))}\\
		  = & \lim\limits_{x\to \infty}e^{x\ln(\frac{1+\tan(\frac{1}{x})}{1-\tan(\frac{1}{x})})}\\
		  = & \lim\limits_{t\to 0}e^{\frac{2\tan t}{t(1-\tan t)}}\\
		  = & e^{2}
	\end{align*}
\end{solution}
\myspace{1}
\hl{\textbf{\textit{January 13}}}

1. 求极限 $\lim\limits_{x\to 0}\left( \dfrac{\ln(1+x)}{x}\right)^{\frac{1}{e^{x}-1}}$
\myspace{1}
\begin{solution}
	
	原极限等价于:
	\begin{align*}
		I = & \lim\limits_{x\to 0}e^{\frac{\ln(1+x)-x}{x(e^{x}-1)}}\\
		  = & e^{\lim\limits_{x\to 0}\frac{\ln(1+x)-x}{x^{2}}}\\
		  = & e^{\frac{1}{2}}
	\end{align*}
\end{solution}
\myspace{1}

2. 求极限 $\lim\limits_{x\to 0}\left( \dfrac{(1+x)^{\frac{1}{x}}}{e}\right)^{\frac{1}{x}}$
\myspace{1}
\begin{solution}

	原极限等价于:
	\begin{align*}
		I = & \lim\limits_{x\to 0}e^{\frac{\ln(1+x)-x}{x^{2}}}\\
		  = & e^{\frac{1}{2}}
	\end{align*}
\end{solution}
\myspace{1}
\hl{\textbf{\textit{January 14}}}

1. 求极限 $\lim\limits_{x\to 0}(\cos 2x+2x\sin x)^{\frac{1}{x^{4}}}$
\myspace{1}
\begin{solution}

	原极限等价于:
	\begin{align*}
		I = & \lim\limits_{x\to 0}e^{\frac{2x\sin x+\cos 2x-1}{x^{4}}}\quad (\text{Taylor's Formula})\\
		  = & e^{\lim\limits_{x\to 0}\frac{2x(x-\frac{x^{3}}{6})-2x^{2}+\frac{2}{3}x^{4}}{x^{4}}}\\
		  = & e^{\frac{1}{3}}
	\end{align*}
\end{solution}
\myspace{1}

2. 求极限 $\lim\limits_{x\to \frac{\pi}{4}}\left( \tan x\right) ^{\frac{1}{\cos x-\sin x}}$
\myspace{1}
\begin{solution}

	原极限等价于:
	\begin{align*}
		I = & \lim\limits_{x\to \frac{\pi}{4}}e^{\frac{\tan x-1}{\cos x-\sin x}}\\
		  = & e^{\lim\limits_{x\to \frac{\pi}{4}}\frac{\tan x-1}{\cos x(1-\tan x)}}\\
		  = & e^{\lim\limits_{x\to \frac{\pi}{4}}-\frac{1}{\cos x}}\\
		  = & e^{-\sqrt{2}}
	\end{align*}
\end{solution}
\myspace{1}
\section{Week \Rmnum{3}}
\hl{\textbf{\textit{January 15}}}

1. 求极限 $\lim\limits_{x\to 0}\left(\dfrac{(e^{x}+e^{2x}+\cdots +e^{nx})}{n} \right)^{\frac{1}{x}} $
\myspace{1}
\begin{solution}
	
	原极限等价于:
	\begin{align*}
		I = & \lim\limits_{x\to 0}e^{\frac{\ln \frac{e^{x}+e^{2x}+\cdots +e^{nx}}{n}}{x}}\\
		  = & e^{\lim\limits_{x\to 0}\frac{e^{x}-1+e^{2x}-1+\cdots +e^{nx}-1}{nx}}\\
		  = & e^{\lim\limits_{x\to 0}\frac{\frac{n(n+1)x}{2}}{nx}}\\
		  = & e^{\frac{n+1}{2}}
	\end{align*}
\end{solution}
\myspace{1}

2. 求极限 $\lim\limits_{x\to \infty}\left(\dfrac{x^{n}}{(x+1)(x+2)\cdots(x+n)} \right)^{x} $
\myspace{1}
\begin{solution}

	原极限等价于:
	\begin{align*}
		I = & \lim\limits_{x\to +\infty}\left(1+\dfrac{1}{x}\right)^{-x}\left(1+\dfrac{2}{x}\right)^{-x}\cdots \left(1+\dfrac{n}{x}\right)^{-x}\\
		  = & e^{\lim\limits_{x\to +\infty}-x\ln(1+\frac{1}{x})}e^{\lim\limits_{x\to +\infty}-x\ln(1+\frac{2}{x})}\cdots e^{\lim\limits_{x\to +\infty}-x\ln(1+\frac{n}{x})}\\
		  = & e^{-1}e^{-2}\cdots e^{-n}\\
		  = & e^{-\frac{n(n+1)}{2}}
	\end{align*}
\end{solution}
\myspace{1}
\hl{\textbf{\textit{January 16}}}

1. 求极限 $\lim\limits_{x\to \infty}\left(\sin\frac{1}{x}+\cos\frac{1}{x} \right)^{x} $
\myspace{1}
\begin{solution}

	原极限等价于:
	\begin{align*}
		I = & \lim\limits_{x\to \infty}\left(\sin\frac{1}{x}+\cos\frac{1}{x} \right)^{x}\\
		  = & e^{\lim\limits_{x\to +\infty}x\ln(\sin\frac{1}{x}+\cos\frac{1}{x})}\\
		  = & e^{\lim\limits_{t\to 0}\frac{\ln(\sin t+\cos t)}{t}}\\
		  = & e^{\lim\limits_{t\to 0}\frac{\sin t+\cos t-1}{t}}\\
		  = & e
	\end{align*}
\end{solution}
\myspace{1}

2. 求极限 $\lim\limits_{n\to \infty}n\left[e\left(1+\dfrac{1}{n} \right)^{-n}-1 \right]$
\myspace{1}
\begin{solution}
	
	由归结原理得, 原极限等价于:
	\begin{align*}
		I = & \lim\limits_{x\to \infty}x\left[e\left(1+\dfrac{1}{x} \right)^{-x}-1 \right]\\
		  = & \lim\limits_{x\to \infty}x(e^{1-x\ln(1+\frac{1}{x})}-1)\\
		  = & \lim\limits_{t\to 0}\frac{e^{1-\frac{\ln(1+t)}{t}}-1}{t}\\
		  = & \lim\limits_{t\to 0}\frac{t-\ln(1+t)}{t^{2}}\\
		  = & e^{\frac{1}{2}}
	\end{align*}
\end{solution}
\myspace{1}
\hl{\textbf{\textit{January 17}}}

1. 设 $a>0,a\neq 1$,且 $\lim\limits_{x\to +\infty}x^{p}(a^{\frac{1}{x}}-a^{\frac{1}{x+1}})=\ln a$,求 $p$
\myspace{1}
\begin{solution}

	原极限等价于:
	\begin{align*}
		I = & \lim\limits_{x\to +\infty}x^{p}(a^{\frac{1}{x}}-a^{\frac{1}{x+1}})(\text{Lagrange's Mean Value Theorem})\\
		  = & \lim\limits_{x\to +\infty}\dfrac{x^{p}}{x(x+1)}a^{\xi}\ln a, \xi\in(\dfrac{1}{x+1},\dfrac{1}{x})\\
		  = & \ln a\lim\limits_{x\to +\infty}\dfrac{x^{p}}{x(x+1)}\\
		  = & \ln a
	\end{align*}

	我们由夹逼准则: $\lim\limits_{x\to +\infty} \dfrac{1}{x+1} =\lim\limits_{x\to +\infty} \dfrac{1}{x} = 0\Rightarrow \lim\limits_{x\to +\infty} \xi =0$

	综上所述, 我们有: $p = 2$
\end{solution}
\myspace{1}

2. 求极限 $\lim\limits_{x\to 0}\dfrac{(1-\sqrt{\cos x})(1-\sqrt[3]{\cos x})\cdots(1-\sqrt[n]{\cos x})}{(1-\cos x)^{n-1}}$
\myspace{1} 
\begin{solution}
	
	我们利用等价无穷小 $x\to 0, 1- \sqrt[n]{\cos x}\sim -(\sqrt[n]{1+\cos x-1}-1)\sim \dfrac{1-\cos x}{n}$, 原极限等价于:
	\begin{align*}
		I = & \lim\limits_{x\to 0}\dfrac{(1-\sqrt{\cos x})(1-\sqrt[3]{\cos x})\cdots(1-\sqrt[n]{\cos x})}{(1-\cos x)^{n-1}}\\
		  = & \lim\limits_{x\to 0}\dfrac{(1-\cos x)^{n-1}}{n!(1-\cos x)^{n-1}}\\
		  = & \dfrac{1}{n!}
	\end{align*}
\end{solution}
\myspace{1}
\hl{\textbf{\textit{January 18}}}

1. 求极限 $\lim\limits_{x\to 0}\dfrac{\ln(\sin^{2}x+e^{x})-x}{\ln(x^{2}+e^{2x})-2x}$
\myspace{1} 
\begin{solution}
	
	原极限等价于:
	\begin{align*}
		I = & \lim\limits_{x\to 0}\dfrac{\ln(\sin^{2}x+e^{x})-x}{\ln(x^{2}+e^{2x})-2x}\\
		  = & \lim\limits_{x\to 0}\dfrac{\ln(\frac{\sin^{2}x}{e^{x}}+1)}{\ln(\frac{x^{2}}{e^{2x}}+1)}\\
		  = & \lim\limits_{x\to 0}\dfrac{e^{x}\sin^{2}x}{x^{2}}\\
		  = & 1
	\end{align*}
\end{solution}
\myspace{1}

2. 已知极限 $\lim\limits_{x\to 0}\dfrac{x-\tan x}{x^{k}}=c$,其中 $k,c$ 为常数,且 $c\neq 0$,求 $k,c$
\myspace{1}
\begin{solution}

	原极限等价于:
	\begin{align*}
		I = & \lim\limits_{x\to 0}\dfrac{x-\tan x}{x^{k}}\\
		  = & \lim\limits_{x\to 0}-\dfrac{x^{3}}{3x^{k}}\\
		  = & c
	\end{align*}

	我们有: $k = 3, c = -\dfrac{1}{3}$
\end{solution}
\myspace{1}

\hl{\textbf{\textit{January 19}}}

1. 若 $\lim\limits_{x\to 0}\left(\dfrac{\sin x^{4}}{x^{4}}-\dfrac{f(x)}{x^{3}}\right)=2$,则当 $x\to 0$ 时,$f(x)$ 是 $x$ 的:
\begin{itemize}
	\item A. 等价无穷小
	\item B. 同阶但非等价无穷小
	\item \hl{\textbf{C}}. 高阶无穷小
	\item D. 低阶无穷小
\end{itemize}
\myspace{1}
\begin{solution}

	我们记 
	$\begin{cases}
		I = \lim\limits_{x\to 0}\left(\dfrac{\sin x^{4}}{x^{4}}-\dfrac{f(x)}{x^{3}}\right) = 2\\
		J = \lim\limits_{x\to 0}\dfrac{\sin x^{4}}{x^{4}} = 1
	\end{cases}\Rightarrow \lim\limits_{x\to 0}\dfrac{f(x)}{x^{3}} = -1$

	我们得到: $\lim\limits_{x\to 0}\dfrac{f(x)}{x} = \lim\limits_{x\to 0}\dfrac{f(x)}{x^{4}}\cdot\lim\limits_{x\to 0}x^{3} = 0$

	综上所述, $f(x)$ 是 $x$ 的 高阶无穷小.
\end{solution}
\myspace{1}

2. 当 $x\to 0$ 时,$\alpha(x)=kx^{2}$ 与 $\beta(x)=\sqrt{1+x\arcsin x}-\sqrt{\cos x}$ 时等价无穷小,求 $k$
\myspace{1}
\begin{solution}

	由等价无穷小定义得到:
	\begin{align*}
		I = & \lim\limits_{x\to 0}\dfrac{\beta(x)}{\alpha(x)}\\
		  = & \lim\limits_{x\to 0}\dfrac{\sqrt{1+x\arcsin x}-\sqrt{\cos x}}{kx^{2}}\\
		  = & \lim\limits_{x\to 0}\dfrac{\sqrt{1+x\arcsin x}-1}{kx^{2}}-\lim\limits_{x\to 0}\dfrac{\sqrt{\cos x}-1}{kx^{2}}\\
		  = & \lim\limits_{x\to 0}\dfrac{x\arcsin x}{2kx^{2}}+\lim\limits_{x\to 0}\dfrac{1-\cos x}{2kx^{2}}\\
		  = & \dfrac{3}{4k}\\
		  = & 1
	\end{align*}

	综上所述, 我们有: $k = \dfrac{3}{4}$
\end{solution}
\myspace{1}

\hl{\textbf{\textit{January 20}}}

1. 当 $x\to 0^{+}$ 时,与 $\sqrt{x}$ 等价的无穷小量为:
\begin{itemize}
	\item A. $1-e^{\sqrt{x}}$
	\item \hl{\textbf{B}}. $\ln\dfrac{1+x}{1-\sqrt{x}}$
	\item C. $\sqrt{1+\sqrt{x}}-1$
	\item D. $1-\cos \sqrt{x}$
\end{itemize}
\myspace{1}
\begin{solution}

	我们有:
	$$\begin{cases} 
		1-e^{\sqrt{x}}\sim -\sqrt{x}\\
		1-\cos\sqrt{x} \sim \dfrac{1}{2}x\\
		\sqrt{1+\sqrt{x}}-1 \sim \dfrac{1}{2}\sqrt{x}\\
		\ln\dfrac{1+x}{1-\sqrt{x}}\sim \sqrt{x}
	\end{cases}$$

	我们得到: $\ln \dfrac{1-x}{1-\sqrt{x}}\sim \sqrt{x}$
\end{solution}
\myspace{1}

2. 设 $\alpha_{1}=x(\cos\sqrt{x}-1),\alpha_{2}=\sqrt{x}\ln(1+\sqrt[3]{x}),\alpha_{3}=\sqrt[3]{x+1}-1$,当 $x\to 0^{+}$时,以上 $3$ 个无穷小量从低阶到高阶的排序为:
\begin{itemize}
	\item A. $\alpha_{1},\alpha_{2},\alpha_{3}$
	\item \hl{\textbf{B}}. $\alpha_{2},\alpha_{3},\alpha_{1}$
	\item C. $\alpha_{2},\alpha_{1},\alpha_{3}$
	\item D. $\alpha_{3},\alpha_{2},\alpha_{1}$
\end{itemize}
\myspace{1}
\begin{solution}

	我们有:
	$$\begin{cases}
		\alpha_{1}\sim -\dfrac{1}{2}x^{2}\\
		\alpha_{2}\sim x^{\frac{5}{6}}\\
		\alpha_{3}\sim \dfrac{1}{3}x
	\end{cases}$$

	我们得到: $\alpha_{2} < \alpha_{3} < \alpha_{1}$
\end{solution}
\myspace{1}

\hl{\textbf{\textit{January 21}}}

1. 函数 $f(x)=\dfrac{(e^{\frac{1}{x}}+e)\tan x}{x(e^{\frac{1}{x}}-e)}$ 在 $[-\pi,\pi]$上的第一类间断点是:
\begin{itemize}
	\item \hl{\textbf{A}}. $0$
	\item B. $1$
	\item C. $-\dfrac{\pi}{2}$
	\item D. $\dfrac{\pi}{2}$
\end{itemize}
\myspace{1}
\begin{solution}

	我们有: $f(x)$ 在 $[-\pi,\pi]$ 上无定义的点有 $x=0, x=1, x=\pm\dfrac{\pi}{2}$

	\begin{itemize}
		\item $x\to 0^{+}, f(x)\to 1 ; x\to 0^{-}, f(x)\to -1$
		\item $x\to 1^{+}, f(x)\to -\infty ; x\to 1^{-}, f(x)\to +\infty$
		\item $x\to \dfrac{\pi}{2}^{+}, f(x)\to -\infty ; x\to \dfrac{\pi}{2}^{-}, f(x)\to +\infty$
		\item $x\to -\dfrac{\pi}{2}^{+}, f(x)\to -\infty ; x\to -\dfrac{\pi}{2}^{-}, f(x)\to +\infty$
	\end{itemize}

	综上所述, $x=0$ 是 $f(x)$ 的跳跃间断点, 是第一类间断点
\end{solution}
\myspace{1}

2. 设函数 $f(x)=\dfrac{\ln|x|}{|x-1|}\sin x$,则$f(x)$ 有
\begin{itemize}
	\item \hl{\textbf{A}}. $1$ 个可去间断点,$1$ 个跳跃间断点
	\item B. $1$ 个可去间断点,$1$ 个无穷间断点
	\item C. $2$ 个跳跃间断点
	\item D. $2$ 个无穷间断点
\end{itemize}
\myspace{1}
\begin{solution}

	我们有: $f(x)$ 在定义域上无定义的点有 $x=0, x=1$
	\begin{itemize}
		\item $x\to 0^{+}, f(x)\to 0 ; x\to 0^{-}, f(x)\to 0$
		\item $x\to 1^{+}, f(x)\to \sin 1 ; x\to 1^{-}, f(x)\to -\sin 1$
	\end{itemize}

	综上所述, $f(x)$ 有 $1$ 个可去间断点, $1$ 个跳跃间断点
\end{solution}
\myspace{1}

\section{Week \Rmnum{4}}
\hl{\textbf{\textit{January 22}}}

1. 函数 $f(x)=\dfrac{x^{2}-x}{x^{2}-1}\sqrt{1+\dfrac{1}{x^{2}}}$ 的无穷间断点的个数为:
\begin{itemize}
	\item A. $0$
	\item \hl{\textbf{B}}. $1$
	\item C. $2$
	\item D. $3$
\end{itemize}
\myspace{1}
\begin{solution}

	我们有: $f(x)$ 在定义域上无定义的点有 $x=0, x=\pm 1$
	\begin{itemize}
		\item $x\to 0^{+}, f(x)\to 1 ; x\to 0^{-}, f(x)\to -1$
		\item $x\to 1^{+}, f(x)\to \dfrac{\sqrt{2}}{2} ; x\to 1^{-}, f(x)\to \dfrac{\sqrt{2}}{2}$
		\item $x\to -1^{+}, f(x)\to +\infty ; x\to -1^{-}, f(x)\to -\infty$
	\end{itemize}

	综上所述, $f(x)$ 有 $1$ 个可去间断点, $1$ 个跳跃间断点, $1$ 个无穷间断点
\end{solution}
\myspace{1}

2. 函数 $f(x)=\dfrac{|x|^{x}-1}{x(x+1)\ln|x|}$ 的可去间断点的个数为:
\begin{itemize}
	\item A. $0$
	\item B. $1$
	\item \hl{\textbf{C}}. $2$
	\item D. $3$
\end{itemize}
\myspace{1}
\begin{solution}

	我们有: $f(x)$ 在定义域上无定义的点有 $x=0, x=\pm 1$
	\begin{itemize}
		\item $x\to 0^{+}, f(x)\to 1 ; x\to 0^{-}, f(x)\to 1$
		\item $x\to 1^{+}, f(x)\to \dfrac{1}{2} ; x\to 1^{-}, f(x)\to \dfrac{1}{2}$
		\item $x\to -1^{+}, f(x)\to +\infty ; x\to -1^{-}, f(x)\to -\infty$
	\end{itemize}

	综上所述, $f(x)$ 有 $2$ 个可去间断点, $1$ 个无穷间断点
\end{solution}
\myspace{1}

\hl{\textbf{\textit{January 23}}}

1. 设函数 $f(x)=\lim\limits_{n\to \infty}\dfrac{1+x}{1+x^{2n}}$,讨论函数的间断点,其结论为:
\begin{itemize}
	\item A. 不存在间断点
	\item \hl{\textbf{B}}. 存在间断点 $x=1$
	\item C. 存在间断点 $x=0$
	\item D. 存在间断点 $x=-1$
\end{itemize}
\myspace{1}
\begin{solution}

	我们得到:$$f(x) = \begin{cases}
	0, & x\in(-\infty,-1)\\
	0, & x = -1\\
	x+1, & x\in(-1,1)\\
	1, & x = 1\\
	0, & x\in(1,+\infty)
	\end{cases}$$

	综上所述, $f(x)$ 存在唯一的跳跃间断点 $x = 1$
\end{solution}
\myspace{1}

2. 设函数 $f(x)$ 在 $x=0$ 处连续,且 $\lim\limits_{h\to 0}\dfrac{f(h^{2})}{h^{2}}=1$,则:
\begin{itemize}
	\item A. $f(0)=0$ 且 $f_{-}^{'}(0)$ 存在
	\item B. $f(0)=1$ 且 $f_{-}^{'}(0)$ 存在
	\item \hl{\textbf{C}}. $f(0)=0$ 且 $f_{+}^{'}(0)$ 存在
	\item D. $f(0)=1$ 且 $f_{+}^{'}(0)$ 存在
\end{itemize}
\myspace{1}
\begin{solution}

	我们由 $f(x)$ 在 $x=0$ 处连续得到:
	$$f(0) = \lim\limits_{x\to 0}f(x) = \lim\limits_{x\to 0^{+}}f(x) = \lim\limits_{x\to 0^{-}}f(x)$$

	$$\lim\limits_{h\to 0}\dfrac{f(h^{2})}{h^{2}}=1\Rightarrow 
	\begin{cases}
		\lim\limits_{h^{2}\to 0}f(h^{2}) = \lim\limits_{h\to 0}\dfrac{f(h^{2})}{h^{2}}\cdot \lim\limits_{h\to 0}h^{2}=0\\
		f'_{+}(0) = \lim\limits_{h^{2}\to 0}\dfrac{f(h^{2})-f(0)}{h^{2}}=1
	\end{cases}$$

	综上所述,$f(0) = 0$ 且 $f'_{+}(0)$ 存在 
\end{solution}
\myspace{1}

\hl{\textbf{\textit{January 24}}}

1. 设函数 $f(x)$ 在区间 $(-\delta,\delta)$内有定义,若当 $x\in(-\delta,\delta)$时,恒有$|f(x)|\leq x^{2}$,则$x=0$ 必是$f(x)$的:
\begin{itemize}
	\item A. 间断点
	\item B. 连续而不可导的点
	\item \hl{\textbf{C}}. 可导的点,且 $f'(0)=0$
	\item D. 可导的点,且 $f'(0)\neq 0$
\end{itemize}
\myspace{1}
\begin{solution}

	(1). 连续性: $f(x)$ 在 $x = 0$ 处连续
	\begin{itemize}
		\item $|f(x)|\leq x^{2}, x\in (-\delta,\delta)\Rightarrow |f(0)|\leq 0\Rightarrow f(0) = 0$
		\item 夹逼准则:$\lim\limits_{x\to 0} 0 = \lim\limits_{x\to 0}x^{2} = 0\Rightarrow \lim\limits_{x\to 0} |f(x)| = 0$
		\item $\forall \varepsilon > 0 ,\exists \xi > 0, |f(x)| < \varepsilon \Rightarrow  \lim\limits_{x\to 0} f(x) = 0$
	\end{itemize}

	(2). 可导性: $f(x)$ 在 $x = 0$ 处可导, 且 $f'(0) = 0$
	\begin{itemize}
		\item $|f(x)|\leq x^{2}, x\in (-\delta,\delta)\Rightarrow 0 < \big|\dfrac{f(x)}{x}\big| < |x|$
		\item 夹逼准则:$\lim\limits_{x\to 0} 0 = \lim\limits_{x\to 0}|x| = 0\Rightarrow \lim\limits_{x\to 0} \big|\dfrac{f(x)}{x}\big| = 0$
		\item $\forall \varepsilon > 0 ,\exists \xi > 0, \big|\dfrac{f(x)}{x}\big| < \varepsilon \Rightarrow  \lim\limits_{x\to 0} \dfrac{f(x)}{x} = 0$
		\item $f'(0) = \lim\limits_{x\to 0} \dfrac{f(x)-f(0)}{x} = 0$
	\end{itemize}
\end{solution}
\myspace{1}

2. 设函数 $f(x)=\begin{cases}
	x^{\alpha}\cos\dfrac{1}{x^{\beta}}, & x>0\\
	0,& x\leq 0
\end{cases}(\alpha>0,\beta>0)$,若 $f'(x)$ 在 $x=0$ 处连续,则:
\begin{itemize}
	\item \hl{\textbf{A}}. $\alpha-\beta>1$
	\item B. $0<\alpha-\beta\leq 1$
	\item C. $\alpha-\beta>2$
	\item D. $0<\alpha-\beta\leq 2$
\end{itemize}
\myspace{1}
\begin{solution}

	首先 $f(x)$ 在 $x = 0$ 处连续, $\alpha > 0$, 我们得到:

	$$f'(x) = 
	\begin{cases}
		0, & x< 0\\
		\lim\limits_{x\to 0} x^{\alpha -1}\cos\dfrac{1}{x^{\beta}}, & x = 0\\
		\alpha x^{\alpha -1}\cos \dfrac{1}{x^{\beta}} + \beta x^{\alpha-\beta-1}\sin \dfrac{1}{x^{\beta}}, & x > 0  
	\end{cases}$$

	$f'(x)$ 在 $x = 0$ 处连续 $\Rightarrow \begin{cases} \alpha - 1 >0\\ \alpha - \beta - 1 > 0\end{cases}\Rightarrow \alpha - \beta > 1$
\end{solution}
\myspace{1}

\hl{\textbf{\textit{January 25}}}

1. 曲线 $x+y+e^{2xy}=0$ 在点 $(0,-1)$ 处的切线方程
\myspace{1}
\begin{solution}

	我们设 $\begin{cases}
		F(x,y) = x + y + e^{2xy} = 0\\
		y = y(x)
	\end{cases}$, 我们有:
	$$F'_{x} + F'_{y}\dfrac{dy}{dx} = 0\Rightarrow 1+2ye^{2xy}+(1+2xe^{2xy})\dfrac{dy}{dx} =0$$

	在 $(0,-1)$ 邻域附近, $F'_{y}\neq 0\Rightarrow \dfrac{dy}{dx}\big|_{(0,-1)} = -\dfrac{1+2ye^{2xy}}{1+2xe^{2xy}} = 1$

	综上所述, $(0,-1)$ 处的切线方程为: $x - y - 1 = 0$
\end{solution}
\myspace{1}

2.

(1). 设函数 $u(x),v(x)$ 可导,利用导数定义证明:$[u(x)v(x)]'=u'(x)v(x)+u(x)v'(x)$

(2). 设函数$u_{1}(x),u_{2}(x),\cdots,u_{n}(x)$可导,$f(x)=u_{1}(x)u_{2}(x)\cdots u_{n}(x)$,写出$f(x)$ 的求导公式
\myspace{1}
\begin{solution}

	(1). 我们利用导数定义得到: $f(x) = u(x)v(x)$
	\begin{align*}
		f'(x) = & \lim\limits_{\Delta x\to 0}\dfrac{u(x+\Delta x)v(x+\Delta x)-u(x)v(x)}{\Delta x}\\
		  	  = & \lim\limits_{\Delta x\to 0}\dfrac{u(x+\Delta x)v(x+\Delta x)-u(x+\Delta x)v(x)+u(x+\Delta x)v(x)-u(x)v(x)}{\Delta x}\\
		      = & \lim\limits_{\Delta x\to 0}\dfrac{u(x+\Delta x)v(x+\Delta x)-u(x+\Delta x)v(x)}{\Delta x}+\lim\limits_{\Delta x\to 0}\dfrac{u(x+\Delta x)v(x)-u(x)v(x)}{\Delta x}\\
		      = & v'(x)\lim\limits_{\Delta x\to 0} u(x+\Delta x)+v(x)u'(x)\\
			  = & u(x)v'(x)+u'(x)v(x)
	\end{align*}

	(2). $f'(x) = \sum\limits_{i=1}^{n}u'_{i}(x)\prod\limits_{\substack{1\leq j\leq n\\ i\neq j }}u_{j}$
\end{solution}
\myspace{1}

\hl{\textbf{\textit{January 26}}}

1. 设函数 $f(x)=\begin{cases}
	\ln\sqrt{x},x\geq 1\\2x-1,x<1
\end{cases}$,$y=f(f(x))$,则 $\dfrac{dy}{dx}\big|_{x=e}$
\myspace{1}
\begin{solution}

	(1). 我们得到: $f(f(x)) = 
	\begin{cases}
		\ln\dfrac{\ln x}{2}, & x\in[e^{2},+\infty)\\
		\ln x -1, & x\in [1,e^{2})\\
		4x-3, & x\in(-\infty,1)
	\end{cases}$

	综上所述, $\dfrac{dy}{dx}\big|_{x=e} = \dfrac{1}{e}$

	(2). $$\dfrac{dy}{dx}\big|_{x=e} = f'(u)f'(x)\big|_{\substack{x = e\\u = \ln\sqrt{x}}}=2\cdot \dfrac{1}{2e} =\dfrac{1}{e}$$
\end{solution}
\myspace{1}

2. 设 $y=y(x)$ 是由方程 $xy+e^{y}=x+1$确定的隐函数,则 $\dfrac{d^{2}y}{dx^{2}}\big|_{x=0}$
\myspace{1}
\begin{solution}

	我们令: $F(x,y) = xy+e^{y}-x-1 = 0$, 我们有:
	$$F'_{x} +F'_{y}\dfrac{dy}{dx} = 0\Rightarrow y-1+(x+e^{y})\dfrac{dy}{dx} = 0$$

	我们令: $G(x,y) = y-1+(x+e^{y})\dfrac{dy}{dx} = 0$, 我们有:
	$$\dfrac{dy}{dx}+\dfrac{d^{2}y}{dx^{2}}(x+e^{y})+ (1+e^{y}\dfrac{dy}{dx})\dfrac{dy}{dx}= 0$$

	我们令: $\begin{cases} \dfrac{dy}{dx}\big|_{x=0} =a\\\dfrac{d^{2}y}{dx^{2}}\big|_{x=0} =b \end{cases}$, 我们有:
	$$\begin{cases}
		x = 0,y = 0\\
		a - 1 = 0\\
		a + b + (1+a)a = 0
	\end{cases}\Rightarrow \begin{cases} a = 1\\b = -3\end{cases}$$

	综上所述, $\dfrac{d^{2}y}{dx^{2}}\big|_{x=0} = -3$
\end{solution}
\myspace{1}

\hl{\textbf{\textit{January 27}}}

1. 设函数 $y=y(x)$ 由参数方程$\begin{cases}
	x=t-\ln(1+t)\\y=t^{3}+t^{2}
\end{cases}$所确定,则 $\dfrac{d^{2}y}{dx^{2}}$
\myspace{1}
\begin{solution}

	我们有:
	$$\dfrac{dy}{dx} = \dfrac{dy}{dt}\dfrac{dt}{dx} = \dfrac{(3t^{2}+2t)(1+t)}{t} = (3t+2)(1+t)$$
	$$\dfrac{d^{2}y}{dx^{2}} = \dfrac{d(\dfrac{dy}{dx})}{dt}\dfrac{dt}{dx}=\dfrac{(6t+5)(t+1)}{1+t}$$
\end{solution}
\myspace{1}

2. 设 $y=x^{2}2^{x}$,求 $y^{(n)}$
\myspace{1}
\begin{solution}

	我们由莱布尼茨求导公式: $(uv)^{(n)} = \sum\limits_{i = 0}^{n}\binom{i}{n}u^{(i)}v^{(n-i)}$, 其中 $u(x) = x^{2}, v(x) = 2^{x}$
	\begin{align*}
		y^{(n)} = & (x^{2}2^{x})^{n}\\
		        = & \binom{0}{n}(x^{2})^{(0)}(2^{x})^{(n)}+\binom{1}{n}(x^{2})^{(1)}(2^{x})^{(n-1)}+\binom{2}{n}(x^{2})^{(2)}(2^{x})^{(n-2)}\\
				= & (\ln 2)^{n}x^{2}2^{x} + 2n(\ln 2)^{n-1}x2^{x}+n(n-1)(\ln 2)^{n-2}2^{x}
	\end{align*}
\end{solution}
\myspace{1}

\hl{\textbf{\textit{January 28}}}

1. 设 $y=\dfrac{1}{x^{2}-1}$,求 $y^{(n)}$
\myspace{1}
\begin{solution}

	我们有: $y = \dfrac{1}{2}\left[\dfrac{1}{x-1}-\dfrac{1}{x+1}\right]$, 且:
	$$\begin{cases} 
	f(x) = \dfrac{1}{x-1}\\
	g(x) =\dfrac{1}{x+1}
	\end{cases}\Rightarrow 
	\begin{cases}
	f^{(n)}(x) = \dfrac{(-1)^{n}n!}{(x-1)^{n+1}}\\
	g^{(n)}(x) = \dfrac{(-1)^{n}n!}{(x+1)^{n+1}}
	\end{cases}$$
	
	我们有: $y^{(n)}(x) = \dfrac{(-1)^{n}n!}{2}\left[\dfrac{1}{(x-1)^{n+1}}-\dfrac{1}{(x+1)^{n+1}}\right]$
\end{solution}
\myspace{1}

2. 已知函数 $f(x)$ 具有任意阶导数,且 $f'(x)=[f(x)]^{2}$,则当 $n$ 为大于 $2$ 的正整数时,$f(x)$ 的 $n$ 阶导数 $f^{(n)}(x)$ 是:
\begin{itemize}
	\item \hl{\textbf{A}}. $n![f(x)]^{n+1}$
	\item B. $n[f(x)]^{n+1}$
	\item C. $[f(x)^{2n}]$
	\item D. $n![f(x)]^{2n}$
\end{itemize}
\myspace{1}
\begin{solution}

	我们有:
	$$\begin{cases}
	f'(x) = [f(x)]^{2}\\
	f''(x) = 2f(x)f'(x) = 2[f(x)]^{3}\\
	f^{(3)}(x) = 6[f(x)]^{2}f'(x) = 2\cdot 3[f(x)]^{4}\\
	\cdots\cdots\\
	f^{(n)}(x) = n![f(x)]^{n+1} 
	\end{cases}$$

	综上所述, $f^{(n)}(x) = n![f(x)]^{n+1}$
\end{solution}
\myspace{1}

\hl{\textbf{\textit{January 29}}}

1. 设 $f(x)$ 在 $(-\infty,+\infty)$ 内可导,且对任意 $x_{1},x_{2}$,当 $x_{1}>x_{2}$ 时,都有 $f(x_{1})>f(x_{2})$,则:
\begin{itemize}
	\item A. 对任意 $x$,$f'(x)>0$
	\item B. 对任意 $x$,$f'(-x)\leq 0$
	\item C. 函数 $f(-x)$ 单调增加
	\item \hl{\textbf{D}}. 函数 $-f(-x)$ 单调增加
\end{itemize}
\myspace{1}
\begin{solution}

	我们有: $f'(x)\geq 0$
	\begin{itemize}
		\item $[f(-x)]' = -f'(-x)\leq 0$
		\item $[-f(-x)]' = f'(-x)\geq 0$
	\end{itemize}

	我们有: $f(-x)$ 单调递减, $-f(-x)$ 单调递增
\end{solution}
\myspace{1}

2. 设 $f(x),g(x)$是恒大于零的可导函数,且 $f'(x)g(x)-f(x)g'(x)<0$,当 $a<x<b$ 时,有:
\begin{itemize}
	\item \hl{\textbf{A}}. $f(x)g(b)>f(b)g(x)$
	\item B. $f(x)g(a)>f(a)g(x)$
	\item C. $f(x)g(x)>f(b)g(b)$
	\item D. $f(x)g(x)>f(a)g(a)$
\end{itemize}
\myspace{1}
\begin{solution}

	构造辅助函数: $F(x) = \dfrac{f(x)}{g(x)}, x\in (a,b), F'(x) = \dfrac{f'(x)g(x)-f(x)g'(x)}{[g(x)]^{2}} < 0$

	我们得到: $F(x)$ 在 $(a,b)$ 内单调递减, $F(a) > F(x) > F(b)\Rightarrow 
	\begin{cases}
		f(a)g(x) > f(x)g(a)\\
		f(x)g(b) > f(b)g(x)
	\end{cases}$
\end{solution}
\myspace{1}

\hl{\textbf{\textit{January 30}}}

1. 设 $\lim\limits_{x\to a}\dfrac{f(x)-f(a)}{(x-a)^{n}}=-1$,其中 $n$ 为 大于 $1$ 的整数,则在点 $x=a$ 处:
\begin{itemize}
	\item A. $f(x)$ 的导数存在,且 $f'(a)\neq 0$
	\item B. $f(x)$ 取得极大值
	\item C. $f(x)$ 取得极小值
	\item \hl{\textbf{D}}. $f(x)$ 是否取得极值与 $n$ 的取值有关
\end{itemize}
\myspace{1}
\begin{solution}

	我们有: $\begin{cases}
		\lim\limits_{x\to a} f(x)-f(a) = \lim\limits_{x\to a}\dfrac{f(x)-f(a)}{(x-a)^{n}}\lim\limits_{x\to a}(x-a)^{n} = 0\\
		f'(a) = \lim\limits_{x\to a}\dfrac{f(x)-f(a)}{x-a} =\lim\limits_{x\to a}\dfrac{f(x)-f(a)}{(x-a)^{n}}\lim\limits_{x\to a}(x-a)^{n-1} =0
	\end{cases}$

	(1). 当 $n$ 为偶数时, $x\in (a-\delta,a), f(x) < f(a); x\in (a,a+\delta), f(x) < f(a)$, $x=a$ 是极大值点

	(2). 当 $n$ 为奇数时, $x\in (a-\delta,a), f(x) > f(a); x\in (a,a+\delta), f(x) < f(a)$, $x=a$ 不是极值点

	综上所述, $f(x)$ 是否取极值点与 $n$ 的取值有关
\end{solution}
\myspace{1}

2. 设 $f(x)$ 的导数在 $x=a$处连续,又 $\lim\limits_{x\to a}\dfrac{f'(x)}{x-a}=-1$,则:
\begin{itemize}
	\item A. $x=a$ 是 $f(x)$的极小值点
	\item \hl{\textbf{B}}. $x=a$ 是 $f(x)$的极大值点
	\item C. $(a,f(a))$ 是曲线 $y=f(x)$ 的拐点
	\item D. $x=a$ 不是 $f(x)$ 的极值点,$(a,f(a))$也不是曲线 $y=f(x)$ 的拐点
\end{itemize}
\myspace{1}
\begin{solution}

	我们有: $$\begin{cases}
		f'(a) = \lim\limits_{x\to a} f'(x) = \lim\limits_{x\to a}\dfrac{f'(x)}{x-a}\lim\limits_{x\to a}(x-a) = 0\\
		x\in (a-\delta,a), f'(x) > 0\\
		x\in (a,a+\delta), f'(x) < 0
	\end{cases}\Rightarrow f''(a) = \lim\limits_{x\to a}\dfrac{f'(x)}{x-a} = -1$$

	综上所述, $x = a$ 是 $f(x)$ 的极大值点, $(a,f(a))$ 不是曲线 $f(x)$ 的拐点.
\end{solution}
\myspace{1}

\hl{\textbf{\textit{January 31}}}

1. 曲线 $y=(x-5)x^{\frac{2}{3}}$的拐点坐标为:
\myspace{1}
\begin{solution}

	我们有:
	$$\begin{cases}
		y' = x^{\frac{2}{3}}+\dfrac{2}{3}(x-5)x^{-\frac{1}{3}} = x^{-\frac{1}{3}}(\dfrac{5}{3}x-\dfrac{10}{3})\\
		y''= -\dfrac{1}{3}x^{-\frac{4}{3}}(\dfrac{5}{3}x-\dfrac{10}{3})+\dfrac{5}{3}x^{-\frac{1}{3}} = \dfrac{10}{9}x^{-\frac{4}{3}}(x+1) 
	\end{cases}$$

	令 $y''(x) = 0\Rightarrow x =-1$, 因此拐点坐标 $(-1,-6)$
\end{solution}
\myspace{1}

2. 已知函数 $y=f(x)$ 对一切 $x$ 满足 $xf''(x)+3x[f'(x)]^{2}=1-e^{-x}$,若 $f'(x_{0})=0(x_{0}\neq 0)$,则:
\begin{itemize}
	\item A. $f(x_{0})$ 是 $f(x)$ 的极大值
	\item \hl{\textbf{B}}. $f(x_{0})$ 是 $f(x)$ 的极小值
	\item C. $(x_{0},f(x_{0}))$ 是曲线 $y=f(x)$ 的拐点
	\item D. $f(x_{0})$ 不是 $f(x)$ 的极值,$(x_{0},f(x_{0}))$ 也不是曲线 $y=f(x)$ 的拐点
\end{itemize}
\myspace{1}
\begin{solution}

	我们得到: $f''(x) = \dfrac{1-e^{-x}}{x} -3[f'(x)]^{2}$

	当 $f'(x_{0}) = 0(x_{0}\neq 0)$ 时, $f''(x_{0}) = \dfrac{1-e^{-x_{0}}}{x_{0}} > 0$

	综上所述, $x = x_{0}$ 是 $f(x)$ 的极小值点, 点 $(x_{0},f(x_{0}))$ 不是 $f(x)$ 的拐点.
\end{solution}
\myspace{1}
