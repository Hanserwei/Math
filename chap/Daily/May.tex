\chapterimage{chap31.jpg}
\chapter{May}
\section{Week \Rmnum{1}}
\hl{\textbf{\textit{May 1}}}

1. 已知级数 $\sum\limits_{n=1}^{+\infty}(-1)^n\sqrt{n}\sin \dfrac{1}{n^{\alpha}}$ 绝对收敛,级数 $\sum\limits_{n=1}^{+\infty}\dfrac{(-1)^n}{n^{2-\alpha}}$ 条件收敛,求 $\alpha$ 取值范围.
\myspace{1}
\begin{solution}
	
	$p$ 级数敛散性 : 
	$$\left\lbrace \begin{array}{l}
		2-\alpha>0\\2-\alpha\leq 1
	\end{array}\right. \Rightarrow 1\leq\alpha<2$$
	
	比较判别法的极限形式: 
	$$\lim\limits_{n\rightarrow+\infty}\dfrac{\sqrt{n}\sin\frac{1}{n^{\alpha}}}{\frac{1}{n^{\alpha-\frac{1}{2}}}}=1\Rightarrow \alpha-\frac{1}{2}>1\Rightarrow \alpha >\frac{3}{2}$$
	
	我们得到 $\alpha$ 取值范围 $\dfrac{3}{2}<\alpha<2$
\end{solution}

\myspace{1}

2. $\sum\limits_{n=1}^{+\infty}\dfrac{1}{(2n+1)!}x^{2n+1}$
\myspace{1}
\begin{solution}
	
	$S(x)=\sum\limits_{n=1}^{+\infty}\dfrac{1}{(2n+1)!}x^{2n+1}$,$S'(x)=\sum\limits_{n=0}^{+\infty}\dfrac{1}{2n!}x^{2n}$.
	
	我们得到: $$S(x)+S'(x)=\sum\limits_{n=0}^{+\infty}\frac{1}{n!}x^{n}=e^{x}$$
	
	原问题转化为微分方程: $y'+y=e^x$ 的求解,且 $y(0)=0$
	
	一阶微分方程的求解公式: $$y'+p(x)y=q(x)\Rightarrow y=e^{-\int p(x)dx}(e^{\int p(x)dx}q(x)+C)$$
	
	我们得到:  $S(x)=\dfrac{1}{2}(e^{x}-e^{-x})$
\end{solution}

\myspace{1}

3. 求二重积分 $\iint\limits_{D}y^2dxdy$ 和 $\iint\limits_{D}(x+2y)dxdy$,其中 $D$ 是由参数方程 $\left\lbrace\begin{array}{l}
	x=a(t-\sin t)\\y=a(1-\cos t)
\end{array} \right. ,t\in [0,2\pi]$
\myspace{1}
\begin{solution}
	
	积分区域是摆线,\ $x\in [0,2\pi a]$,二重积分可以化为: 
	$$\iint\limits_{D}y^2dxdy=\int_{0}^{2\pi a}dx\int_{0}^{y(x)}y^2dy=\int_{0}^{2\pi a}\frac{1}{3}y^{3}(x)dx$$
	$$\int_{0}^{2\pi a}\frac{1}{3}y^{3}(x)dx=\int_{0}^{2\pi }\frac{1}{3}a^3(1-\cos t)^3(a-a\cos t)dt=\frac{2}{3}\int_{0}^{\pi}(2\sin^2 t)^4dt$$
	
	华里士公式:  $$\frac{2}{3}\int_{0}^{\pi}(2\sin^2 t)^4dt=\frac{2}{3}\times 16\times\frac{7}{8}\times\frac{5}{6}\times\frac{3}{4}\times\frac{1}{2}\times\frac{\pi}{2}=\frac{35\pi}{12}$$
	
	二重积分$\iint\limits_{D}(x+2y)dxdy$ 可化为: 
	$$\int_{0}^{2\pi a}dx\int_{0}^{y(x)}(x+2y)dy=\int_{0}^{2\pi a}(y^{2}(x)+xy(x))dx$$
	$$I=\int_{0}^{2\pi }[a^2(1-\cos t)^2+a^2(t-\sin t)(1-\cos t)](a-a\cos t)dt$$
	$$I=a^3\int_{0}^{2\pi }(1-\cos t)^3dt+a^3\int_{0}^{2\pi }(1-\cos t)^2(t-\sin t)dt$$
	$$I_{1}=2\int_{0}^{\pi}(2\sin^2 t)^3dt=5\pi$$
	$$I_{2}=\int_{-\pi}^{\pi}(1+\cos x)^2(x+\pi+\sin x)dx=2\pi\int_{0}^{\pi}(1+\cos x)^2dx=3\pi^2$$
\end{solution}

\myspace{1}

\hl{\textbf{\textit{May 2}}}

1. 级数 $\sum\limits_{n=1}^{+\infty}(\dfrac{1}{\sqrt{n}}-\dfrac{1}{\sqrt{n+1}}\sin(n+k))$ 敛散性
\myspace{1}
\begin{solution}
	
	设 $a_{n}=\dfrac{1}{\sqrt{n}}-\dfrac{1}{\sqrt{n+1}}$,$u_{n}<a_{n}$
	
	$\sum\limits_{n=1}^{+\infty}a_{n}$ 部分和 $S_{n}=1-\dfrac{1}{\sqrt{n+1}}$
	
	$$\lim\limits_{n\rightarrow +\infty}S_{n}=1$$
	
	原级数绝对收敛.
\end{solution}

\myspace{1}

2. $\lim\limits_{x\rightarrow 0^{+}}\dfrac{\sqrt{2(\sec x-1)}-\sqrt[3]{6(x-\sin x)}}{\int_{0}^{x^2}\arctan (e^{\sqrt{t}}-1)dt}$
\myspace{1}
\begin{solution}
	
	对于变上限积分,当 $x\rightarrow 0,g(x)\rightarrow 0,h(x)\rightarrow 0$,我们有: 
	$$\lim\limits_{x\rightarrow 0^{+}}\int_{0}^{g(x)}h(t)dt\Rightarrow \lim\limits_{x\rightarrow 0^{+}}\int_{0}^{G(x)}H(t)dt,x\rightarrow 0,f(x)\sim F(x),h(x)\sim H(x)$$
	
	我们得到原极限等价于: 
	$$\lim\limits_{x\rightarrow 0^{+}}\frac{\sqrt{2(\sec x-1)}-\sqrt[3]{6(x-\sin x)}}{\frac{2}{3}x^3}=\lim\limits_{x\rightarrow 0^{+}}\frac{\sqrt{2(\sec x-1)}-\sqrt{2\frac{1}{2}x^2}+\sqrt[3]{6\frac{1}{6}x^3}-\sqrt[3]{6(x-\sin x)}}{\frac{2}{3}x^3}$$
	
	前一个极限: $$\lim\limits_{x\rightarrow 0^{+}}\frac{\sqrt{2(\sec x-1)}-\sqrt{2\frac{1}{2}x^2}}{\frac{2}{3}x^3}=\lim\limits_{x\rightarrow 0^{+}}\frac{2(\sec x-1-\frac{1}{2}x^2)}{\frac{2}{3}x^3(\sqrt{2(\sec x-1)}+x)}$$
	
	我们有: $\lim\limits_{x\rightarrow 0^{+}}\dfrac{\sqrt{2(\sec x-1)}+x}{2x}=1$
	
	前一个极限: $$I_{1}=\frac{3}{2}\frac{1-\cos x-\frac{1}{2}x^2\cos x}{x^4\cos x}=\frac{5}{16}$$
	
	同理可得: $$I_{2}=-\lim\limits_{x\rightarrow 0^{+}}\frac{3}{2}\frac{\sqrt[3]{6\dfrac{x-\sin x}{x^3}}-1}{x^2}=-\lim\limits_{x\rightarrow 0^{+}}\frac{1}{2}\frac{6(x-\sin x)-x^3}{x^5}=\frac{1}{40}$$
	
	原极限为: $I=I_{1}+I_{2}=\dfrac{27}{80}$
\end{solution}

\myspace{1}

\hl{\textbf{\textit{May 3}}}

1. 级数 $\sum\limits_{n=2}^{+\infty}[sin \frac{1}{n}-k\ln(1-\frac{1}{n})]$ 收敛,求 $k$ 的值.
\myspace{1}
\begin{solution}
	
	泰勒公式和级数的比较判别法极限形式
	
	$$\lim\limits_{n\rightarrow+\infty}\frac{sin \frac{1}{n}-k\\ln(1-\frac{1}{n})}{\frac{1}{n}}=\lim\limits_{x\rightarrow+0}\frac{sin x-k\\ln(1-x)}{x}$$
	
	分子利用泰勒展开式得到: 
	$$x\rightarrow 0,\sin x-k\\ln(1-x)\sim x-\frac{x^3}{6}+o(x^3)+kx+k\frac{x^2}{2}+o(x^2)\sim (1+k)x+o(x)$$
	
	我们得到: 
	$$\lim\limits_{n\rightarrow+\infty}\dfrac{sin \dfrac{1}{n}-k\ln(1-\dfrac{1}{n})}{\dfrac{1}{n}}=1+k$$
	
	当且仅当 $k=-1$ 时,级数收敛,因为当 $k\neq -1$时,原级数敛散性和$\sum\limits_{n=1}^{+\infty}\dfrac{1}{n}$一致,级数发散.
\end{solution}

\myspace{1}

2. $\int_{0}^{\frac{\pi}{4}}\left(\dfrac{\sin x-\cos x}{\sin x+\cos x} \right)^3dx $
\myspace{1}
\begin{solution}
	
	原定积分: 
	$$I=\int_{0}^{\frac{\pi}{4}}\left(\frac{-\sin x}{\cos x}\right)^3dx=-\int_{0}^{\frac{\pi}{4}}\tan^{3} xdx$$
	
	令 $tan x=t,x=\arctan t,dx=\dfrac{1}{1+t^2}dt,t\in[0,1]$,原定积分为: 
	$$I=-\frac{1}{2}\int_{0}^{1}\frac{t^2}{1+t^2}dt^2=-\frac{1}{2}\int_{0}^{1}\frac{t}{1+t}dt=\frac{\ln2-1}{2}$$
	
\end{solution}

\myspace{1}

\hl{\textbf{\textit{May 4}}}

1. 级数 $\sum\limits_{n=1}^{+\infty}a_{n}$ 收敛,判断级数 $\sum\limits_{n=1}^{+\infty}|a_{n}|$,$\sum\limits_{n=1}^{+\infty}(-1)^{n}a_{n}$,$\sum\limits_{n=1}^{+\infty}a_{n}a_{n+1}$,$\sum\limits_{n=1}^{+\infty}\dfrac{a_{n}+a_{n+1}}{2}$ 敛散性
\myspace{1}
\begin{solution}
	
	$$a_{n}=(-1)^n\frac{1}{n}\quad b_{n}=(-1)^n\frac{1}{n^{\frac{1}{2}}}$$
	
	$a_{n}$ 收敛,$|a_{n}|$发散;
	
	$a_{n}$ 收敛,$(-1)^na_{n}$发散;
	
	$b_{n}$ 收敛,$b_{n}b_{n+1}$发散;
\end{solution}

\myspace{1}

2. $\int_{0}^{\frac{\pi}{2}}\dfrac{1}{\sin x-\cos x}dx$
\myspace{1}
\begin{solution}
	
	原定积分内有瑕点,原积分等价于: 
	$$I=\int_{0}^{\frac{\pi}{4}}\frac{1}{\sin x-\cos x}dx+\int_{\frac{\pi}{4}}^{\frac{\pi}{2}}\frac{1}{\sin x-\cos x}dx$$
	
	我们有: $\int\dfrac{1}{\sin x-\cos x}dx=\ln|\dfrac{1-\cos(x-\frac{\pi}{4})}{\sin(x-\frac{\pi}{4})}|+C$
	
	我们可以得到原反常积分发散.
\end{solution}

\myspace{1}

3. $\lim\limits_{x\rightarrow 0}\left( -\dfrac{\cot x}{e^{-2x}}+\dfrac{1}{e^{-x}\sin^2 x}-\dfrac{1}{x^2}\right) $
\myspace{1}
\begin{solution}
	
	原极限等价于: 
	$$\lim\limits_{x\rightarrow 0}\frac{x^2e^{x}-\frac{1}{2}x^2\sin 2xe^{2x}-\sin^2 x}{x^4}=-\frac{7}{6}$$
\end{solution}

\myspace{1}

\hl{\textbf{\textit{May 5}}}

1. 判断下列命题是否正确 

(i). $\sum\limits_{n=0}^{+\infty}(u_{2n+1}+u_{2n})$ 收敛,$\sum\limits_{n=0}^{+\infty}u_{n}$ 收敛

(ii). $\lim\limits_{n\rightarrow +\infty}\dfrac{u_{n+1}}{u_{n}}>1$,则 $\sum\limits_{n=0}^{+\infty}u_{n}$发散.
\myspace{1}
\begin{solution}
	
	(i). $u_{n}=(-1)^{n}$, 第一个排除
	
	(ii). 正项级数比较判别法
\end{solution}

\myspace{1}

2. $(1+\sqrt{3})^{n}=a_{n}+b_{n}\sqrt{3}$,求$\lim\limits_{n\rightarrow +\infty}\dfrac{a_{n}}{b_{n}}$
\myspace{1}
\begin{solution}
	
	我们有: 
	$$(1+\sqrt{3})^{n+1}=(1+\sqrt{3})^{n}(1+\sqrt{3})=(a_{n}+b_{n}\sqrt{3})(1+\sqrt{3})$$
	
	我们得到: 
	$$(a_{n}+3b_{n})+(a_{n}+b_{n})\sqrt{3}=a_{n+1}+b_{n+1}\sqrt{3}\Rightarrow\left\lbrace\begin{array}{l}
		a_{n+1}=a_{n}+3b_{n}\\b_{n+1}=a_{n}+b_{n}
	\end{array} \right. $$
	
	我们化简得: 
	$$\frac{a_{n+1}}{b_{n+1}}=\frac{\frac{a_{n}}{b_{n}}+3}{\frac{a_{n}}{b_{n}}+1}$$
	
	不妨设 $x_{n}=\dfrac{a_{n}}{b_{n}}$,我们有$x_{n+1}=1+\dfrac{2}{x_{n}+1}$,$x_{1}=1$
	
	因为 $1<x_{1}<\sqrt{3},x_{n}>1$,我们得到: 
	$$0<|x_{n+1}-\sqrt{3}|=|1+\frac{2}{x_{n}+1}-(1+\frac{2}{\sqrt{3}+1})|=\frac{2}{(x_{n}+1)(\sqrt{3}+1)}|x_{n}-\sqrt{3}|$$
	
	化简得: 
	$$0<|x_{n+1}-\sqrt{3}|<\frac{1}{\sqrt{3}+1}|x_{n}-\sqrt{3}|=\frac{1}{(\sqrt{3}+1)^{n-1}}a_{1}$$
	
	夹逼定理得到: $\lim\limits_{n\rightarrow +\infty}|x_{n}-\sqrt{3}|=0\Rightarrow \lim\limits_{n\rightarrow +\infty}\dfrac{a_{n}}{b_{n}}=\sqrt{3}$
\end{solution}

\myspace{1}

\hl{\textbf{\textit{May 6}}}

1. $\int_{1}^{+\infty}\dfrac{dx}{x^2\sqrt{x^{2}-1}}$
\myspace{1}
\begin{solution}
	
	令 $x=\sec t,t\in [1,\frac{\pi}{2}], dx=\tan t\sec tdt$,我们有: 
	$$I=\int_{0}^{\frac{\pi}{2}}\frac{\sec t\tan t}{\sec^2 t\tan t}dt=\int_{0}^{\frac{\pi}{2}}\cos tdt=1$$
	
\end{solution}

\myspace{1}

2. $\int_{0}^{1}\dfrac{xdx}{(2-x^2)\sqrt{1-x^2}}$

\myspace{1}
\begin{solution}
	
	令 $x=\sin t,t\in[0,\frac{\pi}{2}],dx=\cos tdt$,我们有: 
	$$I=\int_{0}^{\frac{\pi}{2}}\frac{\sin t\cos t}{(2-\sin^2 t)\cos t}dt
	=\int_{0}^{\frac{\pi}{2}}\frac{\sin t}{2-\sin^2 t}dt=
	-\int_{0}^{\frac{\pi}{2}}\frac{d\cos t}{\cos^2 t+1}=-\arctan(\cos t)|_{0}^{\frac{\pi}{2}}=\frac{\pi}{4}$$
\end{solution}

\myspace{1}

3. $y=x^2$ 与$y=mx$ 围成的部分绕着 $y=mx(m>0)$ 旋转一周得到的旋转体体积 $V$
\myspace{1}
\begin{solution}
	
	$$V=\int_{0}^{L}\pi r^2dl=\int_{0}^{m}\pi r^2\sqrt{1+m^2}dx=\frac{\pi}{\sqrt{1+m^2}}\int_{0}^{m}x^2(m-x)^2dx
	=\frac{m^5\pi}{30\sqrt{1+m^2}}$$
\end{solution}

\myspace{1}

\hl{\textbf{\textit{May 7}}}

1. $f(x)$ 在 $(2,4)$上二阶导数连续,$f(3)=0$,求证: $f''(\varepsilon)=3\int_{2}^{4}f(x)dx$
\myspace{1}
\begin{solution}
	
	设 $ F(x)=\int_{2}^{x}f(x)dx$,原命题转化为证明: $$\text{已知}F(2)=0,F'(3)=0,\text{求证}F^{'''}(\varepsilon)=3F(4)$$
	
	$F(x)$ 在$x=3$处的泰勒展开式为: 
	$$F(x)=F(3)+F'(3)(x-3)+\frac{F''(3)}{2}(x-3)^2+\frac{F^{'''}(\varepsilon_{1})}{6}(x-3)^{3}$$
	
	我们得到: 
	$$\left\lbrace 
	\begin{array}{l}
		F(2)=F(3)-F'(3)+\frac{F''(3)}{2}-\frac{F^{'''}(\varepsilon_{1})}{6},\varepsilon_{1}\in (2,3)\\
		F(4)=F(3)+F'(3)+\frac{F''(3)}{2}+\frac{F^{'''}(\varepsilon_{2})}{6},\varepsilon_{2}\in (3,4)
	\end{array}\right. $$

	我们得到: $F(4)=\dfrac{F^{'''}(\varepsilon_{1})+F^{'''}(\varepsilon_{2})}{6}$
	
	由平均值定理得到: $$\exists \varepsilon_{3}\in(\varepsilon_{1},\varepsilon_{2}),\ s.t\ F^{'''}(\varepsilon_{3})=\frac{F^{'''}(\varepsilon_{1})+F^{'''}(\varepsilon_{2})}{2}$$
	
	我们得到: $F(4)=\dfrac{F^{'''}(\varepsilon_{3})}{3}$,证毕
	
\end{solution}

\myspace{1}

2. $f(x)$在$[0,+\infty)$上可导,$f(0)=0$,其反函数为 $g(x)$,若 $\int_{0}^{f(x)}g(t)dt=x^2e^x$,求$f(x)$
\myspace{1}
\begin{solution}
	
	我们对 $\int_{0}^{f(x)}g(t)dt=x^2e^x$左右两边同时对 $x$ 求导: 
	$$g(f(x))f'(x)=(x^2+2x)e^x\Rightarrow f'(x)=(x+2)e^{x}$$
	
	我们得到: $f(x)=(x+1)e^x+C,f(0)=1+C=0,C=-1$
	
	$$f(x)=(x+1)e^{x}-1$$
\end{solution}

\myspace{1}

\section{Week \Rmnum{2}}
\hl{\textbf{\textit{May 8}}}

1. 求幂级数$\sum\limits_{n=1}^{+\infty}(-1)^{n}\dfrac{\ln n}{2^{n}}(x+1)^{2n+1}$ 收敛区间
\myspace{1}
\begin{solution}
	
	求解收敛半径的两种方法: 
	
	(i).$\rho=\lim\limits_{n\rightarrow +\infty}\frac{|a_{n+1}|}{|a_{n}|}$
	
	(ii).$\rho=\lim\limits_{n\rightarrow +\infty}\sqrt[n]{|a_{n}|}$
	
	本题中采用第二种方法: $\rho=\lim\limits_{n\rightarrow +\infty}\sqrt[n]{|a_{n}|}=\lim\limits_{n\rightarrow +\infty}\dfrac{1}{2}\sqrt[n]{\ln n}=\dfrac{1}{2}$
	
	此题中幂级数只有奇数项,收敛半径 $R=\sqrt{\dfrac{1}{\rho}}=\sqrt{2}$
	
	原幂级数收敛区间: $(-1-\sqrt{2},-1+\sqrt{2})$
\end{solution}

\myspace{1}

2. 证明: $\int_{0}^{\frac{\pi}{2}}\left( \dfrac{\sin nx}{\sin x}\right) ^2dx=\dfrac{n\pi}{2}$
\myspace{1}
\begin{solution}
	
	我们不妨设 $a_{n}=\int_{0}^{\frac{\pi}{2}}(\frac{\sin nx}{\sin x})^2dx$,我们有: 
	$$a_{1}=\frac{\pi}{2}$$ $$a_{n+1}-a_{n}=\int_{0}^{\frac{\pi}{2}}\frac{\sin ^2(n+1)x-\sin^2 nx}{\sin^2 x}dx=\int_{0}^{\frac{\pi}{2}}\frac{\sin (2n+1)x}{\sin x}dx$$
	
	我们令 $c_{n}=a_{n+1}-a_{n}$,$c_{0}=\frac{\pi}{2}$
	
	$$c_{n+1}-c_{n}=\int_{0}^{\frac{\pi}{2}}\frac{\sin (2n+3)x-\sin (2n+1)x}{\sin x}dx$$
	
	我们利用和差化积公式得到: 
	
	$$c_{n+1}-c_{n}=\int_{0}^{\frac{\pi}{2}}\frac{\sin [(2n+2)x+x]-\sin [(2n+2)x-x]}{\sin x}dx=\int_{0}^{\frac{\pi}{2}}\frac{2\sin x\cos(2n+2)x}{\sin x}dx=0\Rightarrow c_{n}=\frac{\pi}{2}$$
	
	我们得到: $a_{n}$是等差数列,$a_{n}=\dfrac{n\pi}{2}$
\end{solution}

\myspace{1}

3. 由方程$F(cx-az,cy-bz)=0$确立了函数$z=z(x,y)$,求 $a\dfrac{\partial z}{\partial x}+b\dfrac{\partial z}{\partial y}$
\myspace{1}
\begin{solution}
	
	隐函数求导法则: 
	$G(x,y,z)=F(u,v),\left\lbrace\begin{array}{l}
		u=cx-az\\v=cy-bz
	\end{array} \right. $,我们有: 
	$$\left\lbrace\begin{array}{l}
		\dfrac{\partial z}{\partial x}=-\dfrac{F_{x}'}{F_{z}'}=\dfrac{cF_{1}'}{aF_{1}'+bF_{2}'}\\
		\dfrac{\partial z}{\partial y}=-\dfrac{F_{y}'}{F_{z}'}=\dfrac{cF_{2}'}{aF_{1}'+bF_{2}'}
	\end{array} \right. $$
	$$a\dfrac{\partial z}{\partial x}+b\dfrac{\partial z}{\partial y}=c$$
\end{solution}

\myspace{1}

\hl{\textbf{\textit{May 9}}}

1. 设函数 $f,g$均可微,且$z=f[xy,\ln x+g(xy)]$,求$x\frac{\partial z}{\partial x}-y\frac{\partial z}{\partial y}$
\myspace{1}
\begin{solution}
	$$\left\lbrace 
	\begin{array}{l}
		\dfrac{\partial z}{\partial x}=yf_{1}'+(\frac{1}{x}+yg')f_{2}'\\
		\dfrac{\partial z}{\partial y}=xf_{1}'+xg'f_{2}'
	\end{array}\right. $$
	$$x\frac{\partial z}{\partial x}-y\frac{\partial z}{\partial y}=f_{2}'$$
\end{solution}

\myspace{1}

2. $f'(x)$连续,\quad$|f'(x)|\leq M,\ \int_{0}^{1}f(x)dx=0$,证明: $\forall a\in [0,1],\ |\int_{0}^{a}f(x)dx|\leq \dfrac{M}{8}$
\myspace{1}
\begin{solution}
	
	我们令: $F(x)=\int_{0}^{x}f(t)dt$,原命题等价于: 
	$$|F''(x)|\leq M, F(0)=F(1)=0, \forall a\in [0,1],|F(x)|\leq \frac{M}{8}$$
	
	利用泰勒反向展开: 
	$$\left\lbrace 
	\begin{array}{l}
		F(0)=F(x)+F'(x)(0-x)+\dfrac{F''(\varepsilon_{1})}{2}(0-x)^2 \qquad \circled{1} \\
		F(1)=F(x)+F'(x)(1-x)+\dfrac{F''(\varepsilon_{2})}{2}(1-x)^2 \qquad \circled{2}
	\end{array}\right. $$
	
	我们利用 $(1-x) \circled{1}+x \circled{2}$得到: 
	$$F(x)=-\frac{F''(\varepsilon_{1})}{2}x^2(1-x)-\frac{F''(\varepsilon_{2})}{2}x(1-x)^2$$
	$$|F(x)|\leq \frac{M}{2}[x^2(1-x)+x(1-x)^2]=\frac{M}{8}$$
\end{solution}

\myspace{1}

\hl{\textbf{\textit{May 10}}}

1. 幂级数 $\sum\limits_{n=1}^{+\infty}\frac{3^n+(-2)^n}{n}(x-1)^n$ 的收敛域
\myspace{1}
\begin{solution}
	
	先求幂级数收敛半径: 
	$$\rho=\lim\limits_{n\rightarrow +\infty}\sqrt[n]{\frac{3^n+(-2)^n}{n}}=3\Rightarrow r=\frac{1}{3}$$
	
	原幂级数中心点$x=1$,收敛区间为 $(\frac{2}{3},\frac{4}{3})$,我们验证端点值$x=\frac{2}{3},x=\frac{4}{3}$
	
	当 $x=d\frac{2}{3}$时,原幂级数为$\sum\limits_{n=1}^{+\infty}\dfrac{3^n+(-2)^{n}}{(-3)^{n}n}=\sum\limits_{n=1}^{+\infty}\dfrac{(-1)^n}{n}+\sum\limits_{n=1}^{+\infty}\dfrac{2^n}{n3^n}$,原级数收敛.
	
	当 $x=d\frac{4}{3}$时,原幂级数为$\sum\limits_{n=1}^{+\infty}\dfrac{1}{n}+\sum\limits_{n=1}^{+\infty}\dfrac{(-2)^n}{n3^n}$,原级数发散.
	
	幂级数收敛域为 $[\dfrac{2}{3},\dfrac{4}{3})$
\end{solution}

\myspace{1}

2. 设幂级数 $\sum\limits_{n=1}^{+\infty}a_{n}x^{n}$与 $\sum\limits_{n=1}^{+\infty}b_{n}x^{n}$的收敛半径分别为$\dfrac{\sqrt{5}}{3}$和 $\dfrac{1}{3}$,则幂级数 $\sum\limits_{n=1}^{+\infty}\dfrac{a_{n}^{2}}{b_{n}^{2}}x^{n}$收敛半径为
\myspace{1}
\begin{solution}
	由题意知: 
	$$\left\lbrace 
	\begin{array}{l}
		\lim\limits_{n\rightarrow+\infty}|\dfrac{a_{n+1}}{a_{n}}|=\frac{3}{\sqrt{5}}\\
		\lim\limits_{n\rightarrow+\infty}|\dfrac{b_{n+1}}{b_{n}}|=3
	\end{array}
	\right. $$
	
	后面幂级数收敛半径 $R=\dfrac{1}{\rho}$,我们有: 
	$$\rho=\lim\limits_{n\rightarrow+\infty}|\frac{a^{2}_{n+1}b_{n}^{2}}{a^{2}_{n}b_{n+1}^{2}}|=\frac{9}{5}\frac{1}{9}=\frac{1}{5}\Rightarrow R=5$$
	
	(\textbf{有些许问题})
\end{solution}

\myspace{1}

\hl{\textbf{\textit{May 11}}}

1. $f(x,y)=\left\lbrace
\begin{array}{l}
	xy\dfrac{x^2-y^2}{x^2+y^2},(x,y)\neq (0,0)\\
	0,(x,y)=(0,0)
\end{array} \right. $,计算 $f''_{xy}(0,0)$和 $f''_{yx}(0,0)$
\myspace{1}
\begin{solution}
	$$\left\lbrace
	\begin{array}{l}
		f'_{x}(0,0)=\lim\limits_{x\rightarrow 0}\dfrac{f(x,0)-f(0,0)}{x}=0\\
		f'_{y}(0,0)=\lim\limits_{y\rightarrow 0}\dfrac{f(0,y)-f(0,0)}{y}=0
	\end{array}
	\right. $$
	$$\left\lbrace 
	\begin{array}{l}
		f'_{x}=\dfrac{y(x^2-y^2)}{x^2+y^2}+\dfrac{4x^2y^3}{(x^2+y^2)^2}\\
		f'_{y}=\dfrac{x(x^2-y^2)}{x^2+y^2}-\dfrac{4x^2y^3}{(x^2+y^2)^2}
	\end{array}
	\right. $$
	$$\left\lbrace
	\begin{array}{l}
		f''_{xy}(0,0)=\lim\limits_{y\rightarrow 0}\dfrac{f'_{x}(0,y)-f'_{x}(0,0)}{y}=-1\\
		f''_{yx}(0,0)=\lim\limits_{x\rightarrow 0}\dfrac{f'_{y}(x,0)-f'_{y}(0,0)}{x}=1
	\end{array}
	\right. $$
\end{solution}

\myspace{1}

2. 设函数$f(x)$ 在$[0,1]$ 上二阶可导,且$f(0)=f(1)=0$,$\underset{x\in [0,1]}{min \{f(x)\}=-1}$,证明: $\exists \varepsilon \in(0,1)$,使得$f''(\varepsilon)\geq 8$
\myspace{1}
\begin{solution}
	
	$f(0)=f(1)=0$,$\underset{x\in [0,1]}{min \{f(x)\}=-1}$,由费马定理我们得到: 
	$$\exists x_{0}\in(0,1),f'(x_{0})=0$$
	
	我们利用泰勒展开,$f(x)$在$x=x_{0}$处的泰勒展开式: 
	$$f(x)=f(x_{0})+f'(x_{0})(x-x_{0})+\frac{f''(\eta)}{2}(x-x_{0})^2,\eta \in x\sim x_{0} $$
	
	(1). 当$x=0$时,$f(0)=-1+\dfrac{f''(\eta_{1})}{2}x_{0}^2=0\Rightarrow f''(\eta_{1})=\dfrac{2}{x_{0}^2}$
	
	(2). 当$x=0$时,$f(1)=-1+\dfrac{f''(\eta_{2})}{2}(1-x_{0})^2=0\Rightarrow f''(\eta_{1})=\dfrac{2}{(1-x_{0})^2}$
	
	我们不妨记$f''(\eta)=max(f''(\eta_{1}),f''(\eta_{2}))$,利用不等式的知识,我们得到: 
	$$f''(\eta)\geq \frac{2}{(\frac{1}{2})^2}=8$$
	
	我们得到: $\exists \varepsilon=\eta \in(0,1)$,使得$f''(\varepsilon)\geq 8$
\end{solution}

\myspace{1}

3. 设 $m,n$均是正整数,证明: $\int_{0}^{1}\frac{\sqrt[m]{\ln^{2}(1-x)}}{\sqrt[n]{x}}dx$ 收敛性与$m,n$无关
\myspace{1}
\begin{solution}
	
	令$f(x)=\dfrac{\sqrt[m]{\ln^{2}(1-x)}}{\sqrt[n]{x}}=x^{-\frac{1}{n}}[\ln(1-x)]^{-\frac{2}{m}}$
	
	我们需要讨论$x\rightarrow 0^{+}$和 $x\rightarrow 1^{-}$两个可能的瑕点
	
	(i).$\lim\limits_{x\rightarrow 0^{+}}f(x)=x^{\frac{2}{m}-\frac{1}{n}}=
	\left\lbrace
	\begin{array}{l}
		0,\frac{2}{m}>\frac{1}{n}\\
		1,\frac{2}{m}=\frac{1}{n}\\
		+\infty,\frac{2}{m}<\frac{1}{n}\\
	\end{array}
	\right. $
	
	(ii).$\lim\limits_{x\rightarrow 1^{-}}f(x)=-\infty$
	
	
	综合 (i)(ii),我们知道$x=1$一定是 $f(x)$的瑕点,$x=0$在一定情况下是$f(x)$的瑕点.
	
	$$\int_{0}^{1}f(x)dx=\int_{0}^{\frac{1}{2}}f(x)dx+\int_{\frac{1}{2}}^{1}f(x)dx=I_{1}+I_{2}$$
	
	(1).我们讨论 $I_{2}$的收敛性,我们比较 $f(x)$和$\dfrac{1}{\sqrt[m]{1-x}}$: 
	
	$$\lim\limits_{x\rightarrow 1^{-}}\frac{f(x)}{\frac{1}{\sqrt[m]{1-x}}}=\sqrt[m]{\frac{\ln^{2}(1-x)}{\frac{1}{1-x}}}\overset{t=1-x}{\Rightarrow}\lim\limits_{t\rightarrow 0^{+}}\sqrt[m]{t\ln^2 t}=0$$
	$$\int_{\frac{1}{2}}^{1}f(x)dx<\int_{\frac{1}{2}}^{1}\frac{1}{\sqrt[m]{1-x}}dx$$
	
	后面的定积分收敛,$I_{2}$收敛
	
	(2).我们讨论 $I_{1}$的收敛性,我们比较$f(x)$和 $\dfrac{1}{\sqrt[n]{x}}$
	$$\lim\limits_{x\rightarrow 0^{+}}\frac{f(x)}{\frac{1}{\sqrt[n]{x}}}=\lim\limits_{x\rightarrow 0^{+}}\sqrt[m]{\ln^{2}(1-x)}=0$$
	$$\int_{0}^{\frac{1}{2}}f(x)dx<\int_{0}^{\frac{1}{2}}\frac{1}{\sqrt[n]{x}}dx$$
	
	后面的定积分收敛,$I_{1}$收敛.
	
	$I$积分收敛,与$m,n$的取值无关.
	
\end{solution}

\myspace{1}

\hl{\textbf{\textit{May 12}}}

1. 设数列${a_{n}}$单调减少,$\lim\limits_{n\rightarrow +\infty}a_{n}=0$,$S_{n}=\sum\limits_{k=1}^{n}a_{k}(n=1,2\dots)$无界,求幂级数 $\sum\limits_{n=1}^{+\infty}a_{n}(x-1)^{n}$收敛域.
\myspace{1}
\begin{solution}
	
	我们不难发现幂级数的中心点为$x=1$,数列${a_{n}}$单调减少,$\lim\limits_{n\rightarrow +\infty}a_{n}=0\Rightarrow a_{n}$是正项级数.
	
	(i).当 $x=0$时,我们得到幂级数为 $\sum\limits_{n=1}^{+\infty}(-1)^na_{n}$,莱布尼兹判别法得到级数收敛,由阿贝尔定理我们得到: $x\in(0,2)$,级数收敛.
	
	(ii).当 $x=2$时,我们得到幂级数为 $\sum\limits_{n=1}^{+\infty}a_{n}$,级数发散,由阿贝尔定理我们得到: $x\in (-\infty,0) \cup (2,+\infty)$,级数发散.
	
	综合 (i)(ii),我们的得到幂级数收敛域为$[0,2)$
\end{solution}

\myspace{1}

2.多元函数连续、偏导数、可微、一阶偏导数连续

验证函数$f(x,y)=\left\lbrace
\begin{array}{l}
	y\arctan\dfrac{1}{\sqrt{x^2+y^2}},(x,y)\neq 0\\
	0,(x,y)=(0,0)
\end{array}
\right. $在 $(0,0)$ 处是否连续,是否可微,一阶偏导数的值和是否连续.
\myspace{1}
\begin{solution}
	$$\lim\limits_{\substack{x\rightarrow 0\\ y\rightarrow 0}}f(x,y)=\lim\limits_{\substack{x\rightarrow 0\\ y\rightarrow 0}}y\frac{1}{\sqrt{x^2+y^2}}=\lim\limits_{\substack{x\rightarrow 0\\ y\rightarrow 0}}y\frac{\pi}{2}=0=f(0,0)$$
	
	(i).$f(x,y)$在$(0,0)$处连续.
	$$f'_{x}(0,0)=\lim\limits_{x\rightarrow 0}\frac{f(x,0)-f(0,0)}{x}=0$$
	$$f'_{y}(0,0)=\lim\limits_{y\rightarrow 0}\frac{f(0,y)-f(0,0)}{y}=\arctan\frac{1}{|y|}=\frac{\pi}{2}$$
	
	(ii).$f(x,y)$在$(0,0)$处偏导数为$f'_{x}(0,0)=0,f'_{y}(0,0)=\dfrac{\pi}{2}$
	
	$$\lim\limits_{\substack{x\rightarrow 0\\ y\rightarrow 0}}\frac{f(x,y)-f'_{x}(0,0)(x-0)-f'_{y}(0,0)(y-0)}{\sqrt{x^2+y^2}}=y\dfrac{\arctan\frac{1}{\sqrt{x^2+y^2}}-\dfrac{\pi}{2}}{\sqrt{x^2+y^2}}=0$$
	
	(iii).$f(x,y)$在 $(0,0)$处可微.
	
	公式法求$f(x,y)$偏导数: 
	$$\begin{array}{l}
		f'_{x}=y\dfrac{x^2+y^2}{1+x^2+y^2}(-x)(x^2+y^2)^{-\frac{3}{2}}=-xy\dfrac{(x^2+y^2)^{-\frac{1}{2}}}{1+x^2+y^2}\\
		f'_{y}=\arctan\dfrac{1}{\sqrt{x^2+y^2}}-y^2\dfrac{(x^2+y^2)^{-\frac{1}{2}}}{1+x^2+y^2}
	\end{array}$$
	
	我们得到一阶偏导数在$(0,0)$处的极限: 
	$$\left\lbrace \begin{array}{l}
		\lim\limits_{x\rightarrow0}f'_{x}(x,0)=0=f'_{x}(0,0)\\
		\lim\limits_{y\rightarrow0}f'_{y}(0,y)=\dfrac{\pi}{2}=f'_{y}(0,0)
	\end{array}\right. $$
	
	(iiii).原函数一阶偏导数在$(0,0)$处连续.
\end{solution}

\myspace{1}

\hl{\textbf{\textit{May 13}}}

1. 将 $f(x)=\dfrac{5x-12}{x^2+5x-6}$展开为$x$的幂级数.
\myspace{1}
\begin{solution}
	$$\frac{5x-12}{x^2+5x-6}=\frac{6}{x+6}+\frac{1}{1-x}=\dfrac{1}{1+\dfrac{x}{6}}+\frac{1}{1-x}$$
	
	我们由常见幂级数展开式得到: 
	$$\frac{1}{1-x}=\sum\limits_{n=0}^{+\infty}x^{n},\ -1<x<1$$
	$$\frac{1}{1+x}=\sum\limits_{n=0}^{+\infty}(-1)^{n}x^{n},\ -1<x<1$$
	
	我们可以得到: 
	$$\dfrac{1}{1+\dfrac{x}{6}}=\sum\limits_{n=0}^{+\infty}(-1)^{n}(\frac{x}{6})^{n},-1<\frac{x}{6}<1$$
	
	我们可以得到: 
	$$f(x)=\sum\limits_{n=0}^{+\infty}(-1)^{n}(\frac{x}{6})^{n}+\sum\limits_{n=0}^{+\infty}x^{n}=\sum\limits_{n=1}^{+\infty}(1+\frac{(-1)^n}{6^n})x^n,-1<x<1$$
	
\end{solution}

\myspace{1}

2. 证明: $f'_{x}(x,y)$和$f'_{y}(x,y)$在$(x_{0},y_{0})$处连续,$f(x,y)$在$(x_{0},y_{0})$处可微.
\myspace{1}
\begin{solution}
	
	由题意得: 
	$$\left\lbrace 
	\begin{array}{l}
		\lim\limits_{\substack{x\rightarrow x_{0}\\ y\rightarrow y_{0}}}\dfrac{f(x,y_{0})-f(x_{0},y_{0})}{x-x_{0}}=f'_{x}(x_{0},y_{0})\Rightarrow f(x,y_{0})-f(x_{0},y_{0})=f'_{x}(x_{0},y_{0})(x-x_{0})+\alpha(x,y)\\
		\lim\limits_{\substack{x\rightarrow x_{0}\\ y\rightarrow y_{0}}}\dfrac{f(x_{0},y)-f(x_{0},y_{0})}{y-y_{0}}=f'_{y}(x_{0},y_{0})\Rightarrow f(x_{0},y)-f(x_{0},y_{0})=f'_{y}(x_{0},y_{0})(y-y_{0})+\beta(x,y)
	\end{array}
	\right.$$
	
	我们要证明函数在 $(x_{0},y_{0})$ 处可微,我们只需要证明: 
	
	$$f(x,y)-f(x_{0},y_{0})=f'_{x}(x_{0},y_{0})(x-x_{0})+f'_{y}(x_{0},y_{0})(y-y_{0})+o(\sqrt{(x-x_{0})^2+(y-y_{0})^2})$$
	\begin{eqnarray*}
		f(x,y)-f(x_{0},y_{0})&=&(f(x,y)-f(x_{0},y))+(f(x_{0},y)-f(x_{0},y_{0}))\\
		&=&f'_{x}(\varepsilon_{1},y)(x-x_{0})+f'_{y}(x,\varepsilon_{2})(y-y_{0})\\
		&=&\left[f'_{x}(\varepsilon_{1},y)-f'_{x}(x_{0},y_{0})+f'_{x}(x_{0},y_{0}) \right](x-x_{0})\\ &+&\left[f'_{y}(x,\varepsilon_{2})-f'_{y}(x_{0},y_{0})+f'_{y}(x_{0},y_{0}) \right](y-y_{0})\\
		&=&dz+\alpha_{1}(x,y)+\beta_{1}(x,y)
	\end{eqnarray*}
	我们有: 
	$$\lim\limits_{\substack{x\rightarrow x_{0}\\ y\rightarrow y_{0}}}\alpha_{1}(x,y)=0\ \lim\limits_{\substack{x\rightarrow x_{0}\\ y\rightarrow y_{0}}}\beta_{1}(x,y)=0$$
	证毕.
\end{solution}

\myspace{1}

\hl{\textbf{\textit{May 14}}}

1. 将函数 $f(x)=\ln(1-x-2x^2)$展开为$x$的幂级数,并指出其收敛区间.
\myspace{1}
\begin{solution}
	$$\ln(1-x-2x^2)=\ln(1+x)+\ln(1-2x)$$
	
	我们根据: 
	$$\ln(1+x)=x-\frac{1}{2}x^2+\frac{1}{3}x^3+\dots+(-1)^{n-1}\frac{x^n}{n}=\sum\limits_{n=1}^{+\infty}(-1)^{n-1}\frac{x^n}{n},-1<x\leq 1$$
	
	得到上面式子的展开式: 
	$$\ln(1+x)=\sum\limits_{n=1}^{+\infty}(-1)^{n-1}\frac{x^n}{n},-1<x\leq 1$$
	$$\ln(1-2x)=-\sum\limits_{n=1}^{+\infty}\frac{(2x)^n}{n},-1<-2x\leq 1$$
	
	我们得到 $f(x)$ 的展开式为: 
	$$f(x)=\sum\limits_{n=1}^{+\infty}[\frac{(-1)^n-2^n}{n}]x^n,-\frac{1}{2}<x\leq \frac{1}{2}$$
	
	函数$f(x)$的收敛区间为 $(-\dfrac{1}{2},\dfrac{1}{2})$
\end{solution}

\myspace{1}

2. 设 $f(x),g(x)$在 $x\in[0,1]$ 上的导数连续,且 $f(0)=0,f'(x)\geq 0,g'(x)\geq 0$,证明: 
$\forall a\in[0,1]$,有 $\int_{0}^{a}g(x)f'(x)dx+\int_{0}^{1}f(x)g'(x)dx\geq f(a)g(1)$
\myspace{1}
\begin{solution}
	
	我们设 $F(x)=\int_{0}^{x}g(t)f'(t)dt+\int_{0}^{1}f(t)g'(t)dt-f(x)g(1)$.
	
	我们有: $$F'(x)=g(x)f'(x)-g(1)f'(x)=f'(x)[g(x)-g(1)]$$
	
	我们知道: $f'(x)\geq 0,g'(x)\geq 0\Rightarrow g(x)\leq g(1),x\in[0,1]$
	
	我们得到: $F'(x)\leq 0\Rightarrow F(x)\text{单调递减}$
	
	$$F(x)\geq F(1)=\int_{0}^{1}g(x)df(x)+\int_{0}^{1}f(x)g'(x)dx-f(1)g(1)=-f(0)g(0)=0$$
	
	原命题得证,证毕.
\end{solution}

\myspace{1}

3. 设 $f(x)$在$[0,1]$上连续且单调递减,证明: $\lambda\in (0,1),\int_{0}^{\lambda}f(x)dx>\lambda\int_{0}^{1}f(x)dx$
\myspace{1}
\begin{solution}
	
	我们构造: $F(x)=\dfrac{\int_{0}^{x}f(t)dt}{x}$
	
	我们对 $F(x)$求导得到: 
	$$F'(x)=\frac{xf(x)-\int_{0}^{x}f(t)dt}{x^2}$$
	
	我们令$G(x)=xf(x)-\int_{0}^{x}f(t)dt$,我们得到: 
	$$G'(x)=f(x)+xf'(x)-f(x)=xf'(x),\ x\in[0,1]$$
	
	我们已知 $f(x)\text{在}[0,1]\text{上连续且单调递减},\text{我们可以得到} f'(x)<0,\ x\in(0,1)$
	
	我们得出: $G'(x)<0,\ x\in(0,1)$
	
	$G(x)\text{在}(0,1)\text{上单调递减},G(x)<G(0)=0\Rightarrow F'(x)<0,\ x\in(0,1),F(x)\text{单调递减}$
	我们得到: 
	$$F(x)>F(1)\Rightarrow \frac{\int_{0}^{x}f(t)dt}{x}>\frac{\int_{0}^{1}f(x)dx}{1}$$
	
	即$\forall \lambda\in(0,1),\int_{0}^{\lambda}f(x)dx>\lambda\int_{0}^{1}f(x)dx$
\end{solution}

\myspace{1}

4. $z=(x^2+y^2)f(x^2+y^2),\dfrac{\partial^2 z}{\partial x^2}+\dfrac{\partial^2 z}{\partial y^2}=0,f(1)=0,f'(1)=1$,求$f(x)$表达式
\myspace{1}
\begin{solution}
	$$\left\lbrace 
	\begin{array}{l}
		\dfrac{\partial z}{\partial x}=2xf(x^2+y^2)+2x(x^2+y^2)f'(x^2+y^2)\\
		\dfrac{\partial z}{\partial y}=2yf(x^2+y^2)+2y(x^2+y^2)f'(x^2+y^2)
	\end{array}
	\right. $$
	$$\left\lbrace 
	\begin{array}{l}
		\dfrac{\partial^2 z}{\partial x^2}=2f(x^2+y^2)+4x^2f'(x^2+y^2)+(6x^2+2y^2)f'(x^2+y^2)+4x^2(x^2+y^2)f''(x^2+y^2)\\
		\dfrac{\partial^2 z}{\partial y^2}=2f(x^2+y^2)+4y^2f'(x^2+y^2)+(6y^2+2x^2)f'(x^2+y^2)+4y^2(x^2+y^2)f''(x^2+y^2)
	\end{array}
	\right. $$
	
	我们可以得到: 
	$$\frac{\partial^2 z}{\partial x^2}+\frac{\partial^2 z}{\partial y^2}=4f(u)+12uf'(u)+4u^2f''(u)=0,\ u=x^2+y^2$$
	
	问题转化为: 
	
	$$u^2f''(u)+3uf'(u)+f(u)=0\Rightarrow (u^2f''(u)+2uf'(u))+(uf'(u)+f(u))=[u^2f'(u)+uf(u)]'=0$$
	
	我们得到: $u^2f'(u)+uf(u)=C$,又因为 $f(1)=0,f'(1)=1\Rightarrow C=1$
	
	我们得到一个一阶线性微分方程: $uf'(u)+f(u)=\dfrac{1}{u}$
	
	利用公式法,我们得到: 
	$$(uf(u))'=\frac{1}{u}\Rightarrow uf(u)=\ln u+C_{2}\Rightarrow uf(u)=\ln u $$
	
	我们得到: $f(x)=\dfrac{\ln x }{x}$
	
\end{solution}

\myspace{1}

\section{Week \Rmnum{3}}
\hl{\textbf{\textit{May 15}}}

1. 将 $f(x)=\arctan\dfrac{1+x}{1-x}$展开为$x$的幂级数
\myspace{1}
\begin{solution}
	\begin{eqnarray*}
		f'(x)=\frac{1}{1+x^2}=\sum\limits_{n=0}^{+\infty}(-1)^nx^{2n}, -1<x^2<1
	\end{eqnarray*}

	我们有: 
	
	$$f(x)-f(0)=\int_{0}^{x}\frac{1}{1+x^2}dx=\int_{0}^{x}\sum\limits_{n=0}^{+\infty}(-1)^nx^{2n}dx$$
	$$\int_{0}^{x}\sum\limits_{n=0}^{+\infty}(-1)^nx^{2n}dx=\sum\limits_{n=0}^{+\infty}\int_{0}^{x}(-1)^{n}x^{2n}dx=\sum\limits_{n=0}^{+\infty}(-1)^n\dfrac{x^{2n+1}}{2n+1},$$
	
	$f(0)=\dfrac{\pi}{4}$,我们有: $f(x)=\dfrac{\pi}{4}+\sum\limits_{n=0}^{+\infty}(-1)^n\dfrac{x^{2n+1}}{2n+1},-1<x<1$
	
\end{solution}

\myspace{1}

2.微分方程$y'=\frac{y(1-x)}{x}$通解
\myspace{1}
\begin{solution}
	
	分离变量
	
	原微分方程可以化为: 
	\begin{eqnarray*}
		&(i) &y\neq 0\quad\frac{1}{y}dy=(\frac{1}{x}-1)dx\\
		&(ii)&y=0\quad y'(x)=0,\text{满足条件}
	\end{eqnarray*}
	
	针对(i),我们同时求不定积分得到: 
	$$\ln|y|=\ln|x|-x+C_{1}\Rightarrow \ln|y|=\ln|cxe^{-x}|\Rightarrow y=cxe^{-x}$$
	
	综合(i)(ii),我们得到微分方程通解: $y=Cxe^{-x},C\in \mathbb{R}$
\end{solution}

\myspace{1}

3. 隐函数渐近线 $x^3+y^3=3axy,a>0$,求$y=y(x)$的斜渐近线.
\myspace{1}
\begin{solution}
	
	令 $t=\dfrac{y}{x}\rightarrow y=tx$
	
	我们得到: $\left\lbrace 
	\begin{array}{l}
		x=\dfrac{3at}{1+t^3}\\
		y=\dfrac{3at^2}{1+t^3}
	\end{array}
	\right. $
	当 $x\rightarrow \infty,t\rightarrow -1$
	
	我们得到: 
	
	(i).
	\begin{eqnarray*}
		a&=&\lim\limits_{x\rightarrow +\infty}\frac{y}{x}=\lim\limits_{t\rightarrow -1^{-} }t=-1\\
		b&=&\lim\limits_{x\rightarrow +\infty}(y-ax)=\lim\limits_{t\rightarrow -1^{-}}(\frac{3at^2}{1+t^3}+\frac{3at}{1+t^3})=-a
	\end{eqnarray*}
	
	(ii).
	\begin{eqnarray*}
		a&=&\lim\limits_{x\rightarrow -\infty}\frac{y}{x}=\lim\limits_{t\rightarrow -1^{+} }t=-1\\
		b&=&\lim\limits_{x\rightarrow -\infty}(y-ax)=\lim\limits_{t\rightarrow -1^{+}}(\frac{3at^2}{1+t^3}+\frac{3at}{1+t^3})=-a
	\end{eqnarray*}
	
	斜渐近线为: $y=-x-a$
\end{solution}

\myspace{1}

4. 已知函数 $y=y(x)$在任意点$x$处的增量 $\Delta y=\dfrac{y\Delta x}{1+x^2}+\alpha$,当$\Delta x\rightarrow 0$,$\alpha$是$\Delta x$的高阶无穷小,$y(0)=\pi$,求$y(1)$
\myspace{1}
\begin{solution}
	
	我们由题意得到微分方程: 
	$$\frac{1}{y}dy=\frac{1}{1+x^2}dx\Rightarrow \ln|y|=\arctan x+C\Rightarrow y=Ce^{\arctan x}$$
	
	由 $y(0)=\pi,\Rightarrow C=\pi,\text{我们得到}y(x)\text{表达式}: y(x)=\pi e^{\arctan x}\Rightarrow y(1)=\pi e^{\frac{\pi}{4}}$
\end{solution}

\myspace{1}

\hl{\textbf{\textit{May 16}}}

1. 求幂级数 $\sum\limits_{n=0}^{+\infty}(2n+1)x^n$的收敛域,并求其和函数.
\myspace{1}
\begin{solution}
	
	(i).先求幂级数收敛半径: 
	$$\rho=\lim\limits_{n\rightarrow +\infty}\frac{a_{n+1}}{a_{n}}=\frac{2n+3}{2n+1}=1\Rightarrow R=\frac{1}{\rho}=1$$
	
	(ii).验证两个端点
	
	当 $x=\pm1$时,幂级数对应的级数发散.
	
	我们得到原幂级数的收敛域为 $(-1,1)$.
	
	$$S(x)=2\sum\limits_{n=0}^{+\infty}nx^n+\sum\limits_{n=0}^{+\infty}x^n=2x[\sum\limits_{n=1}^{+\infty}x^n]'+\frac{1}{1-x}=\frac{1+x}{(1-x)^2},-1<x<1$$
\end{solution}

\myspace{1}

2. 微分方程 $\dfrac{dy}{dx}=\dfrac{y}{x}-\dfrac{1}{2}(\dfrac{y}{x})^3$,满足$y|_{x=1}=1$的特解
\myspace{1}
\begin{solution}
	
	令 $z=\dfrac{y}{x},\text{我们得到}: y=xz\Rightarrow \dfrac{dy}{dx}=z+x\dfrac{dz}{dx}$
	
	原微分方程可化为: 
	$$(z+x\frac{dz}{dx})=z-\frac{1}{2}z^3\text{满足} z| _{x=1}=1$$
	
	我们可以得到: $-\dfrac{2}{z^3}dz=\dfrac{1}{x}dx\Rightarrow \dfrac{1}{z^2}=\ln x+c$
	
	$\text{我们有}z| _{x=1}=1,\text{得到}: 1+\ln x=\dfrac{1}{z^2}\Rightarrow y^2=\dfrac{x^2}{1+\ln x}$
	
	我们得到: $\dfrac{dy}{dx}|_{(1,1)}=\dfrac{1}{2}>0\Rightarrow y=\dfrac{x}{\sqrt{1+\ln x}}$
	
\end{solution}

\myspace{1}

3. 求微分方程 $(x+y)dx+(y-x)dy=0\text{满足条件}y(1)=-1\text{的特解}$
\myspace{1}
\begin{solution}
	
	我们对微分方程进行一些简单的变形: 
	$$\frac{dy}{dx}=\dfrac{1+\frac{y}{x}}{1-\frac{y}{x}}$$
	
	我们令 $z=\dfrac{y}{x}\Rightarrow \dfrac{dy}{dx}=z+x\dfrac{dz}{dx}$
	
	原微分方程可以化简为: 
	$$\frac{1}{x}dx=\frac{1-z}{1+z^2}dz\Rightarrow \ln |x|=\arctan z-\ln \sqrt{z^2+1}+C$$
	
	即: $\ln\sqrt{x^2+y^2}=\arctan \dfrac{y}{x}+C\quad ,\text{由}y(1)=-1\text{得到} C=\dfrac{1}{2}\ln 2+\dfrac{\pi}{4}$
	
	我们得到微分方程的特解为: $\ln\sqrt{x^2+y^2}-\arctan \dfrac{y}{x}=\dfrac{1}{2}\ln 2+\dfrac{\pi}{4}$
\end{solution}

\myspace{1}

4. $f(x,y)\text{连续}\lim\limits_{(x,y)\rightarrow (0,0)}\dfrac{f(x,y)-xy}{x^2+y^2}=1,f(0,0)\text{是极大值还是极小值?}$ 
\myspace{1}
\begin{solution}
	
	极小值点,理由如下: 
	$\lim\limits_{(x,y)\rightarrow (0,0)}\dfrac{f(x,y)-xy}{x^2+y^2}=1$,我们可以得到在$(0,0)$的一个去心邻域内,我们有: 
	\begin{eqnarray*}
		f(x,y)&=&xy+(x^2+y^2)(1+\alpha)=\frac{1}{2}(x+y)^2+(x^2+y^2)(\frac{1}{2}+\alpha)\\
		&\text{其中}&\lim\limits_{(x,y)\rightarrow (0,0)}\alpha=0
	\end{eqnarray*}

	我们得到在$(0,0)$的一个邻域内,$f(x,y)\geq 0,f(0,0)\text{是极小值}$
\end{solution}

\myspace{1}

\hl{\textbf{\textit{May 17}}}

1.求幂级数 $\sum\limits_{n=1}^{+\infty}(\dfrac{1}{2n+1}-1)x^{2n}\text{在区间}(-1,1)\text{内的和函数} S(x)$
\myspace{1}
\begin{solution}
	
	$$S(x)=\sum\limits_{n=1}^{+\infty}\frac{x^{2n}}{2n+1}-\sum\limits_{n=1}^{+\infty}x^{2n}=\sum\limits_{n=1}^{+\infty}\frac{x^{2n}}{2n+1}-\frac{x^2}{1-x^2}$$
	
	(i). 当 $x\neq 0$, $S(x)=\dfrac{1}{x}\sum\limits_{n=1}^{+\infty}\dfrac{x^{2n+1}}{2n+1}-\dfrac{x^2}{1-x^2}=\dfrac{1}{x}\int_{0}^{x}\sum\limits_{n=1}^{+\infty}x^{2n}-\dfrac{x^2}{1-x^2}$
	
	$$S(x)=\frac{\ln\frac{1-x}{1+x}}{2x}-\frac{1}{1-x^2}$$
	
	(ii). 当 $x=0$,$S(x)=0$.
	
	综上,$S(x)=\left\lbrace 
	\begin{array}{l}
		\frac{\ln\dfrac{1-x}{1+x}}{2x}-\dfrac{1}{1-x^2},0<|x|<1\\
		0,x=0
	\end{array}
	\right.$
\end{solution}

\myspace{1}

2. 微分方程 $(y+x^3)dx-2xdy=0\text{满足}y|_{x=1}=\dfrac{6}{5}\text{的特解为}$
\myspace{1}
\begin{solution}
	
	我们对微分方程化简: $y'-\dfrac{1}{2x}y=\dfrac{x}{2}\Rightarrow (\frac{y}{\sqrt{x}})'=\dfrac{x^{\frac{3}{2}}}{2}$
	
	我们得到: $y=\dfrac{x^3+C\sqrt{x}}{5},y|_{x=1}=\dfrac{6}{5}\Rightarrow C=5$
	
	原微分方程的解: $y=\dfrac{x^3+5\sqrt{x}}{5}$
\end{solution}

\myspace{1}

3. 微分方程 $xy'+2y=x\ln x\text{满足}y(1)=-\frac{1}{9}\text{的特解为}$
\myspace{1}
\begin{solution}
	
	原微分方程可化为: $y'+\dfrac{2}{x}y=\ln x\Rightarrow (x^2y)'=x^2\ln x$
	
	原微分方程的解为: $y=\dfrac{\int x^2\ln xdx+C}{x^2}\Rightarrow y=\dfrac{x(3\ln x-1)}{9}+\dfrac{C}{x^2}$
	
	我们由$y(1)=-\dfrac{1}{9}$ 得到: $y=\dfrac{x(3\ln x-1)}{9}$
\end{solution}

\myspace{1}

4.$f(x,y)=x^4+2y^2-3x^2y,f(0,0)\text{是极大值还是极小值?}$
\myspace{1}
\begin{solution}
	
	$f(0,0)$ 不是极值点,理由如下: 
	$$f(x,y)=(x^2-\frac{3}{2}y)^2-\frac{1}{4}y^2$$
	
	(i). 当$y=0,f(x,y)\geq 0$
	
	(ii).当 $x^2=\dfrac{3}{2}y,f(x,y)\leq 0$
\end{solution}

\myspace{1}

\hl{\textbf{\textit{May 18}}}

1.$\int_{0}^{+\infty}\left[ \dfrac{\sqrt{\pi}}{2}-\int_{0}^{x}e^{-t^2}dt\right]dx$
\myspace{1}
\begin{solution}
	
	方法 1:  分部积分法
	$$\int_{0}^{+\infty}\left[ \frac{\sqrt{\pi}}{2}-\int_{0}^{x}e^{-t^2}dt\right]dx=x\left[ \frac{\sqrt{\pi}}{2}-\int_{0}^{x}e^{-t^2}dt\right]_{0}^{+\infty}-\int_{0}^{+\infty}xe^{-x^2}dx$$
	$$\lim\limits_{x\rightarrow +\infty}\frac{\frac{\sqrt{\pi}}{2}-\int_{0}^{x}e^{-t^2}dt}{\frac{1}{x}}-\frac{1}{2}d(e^{-x^2})=\frac{1}{2}$$
	
	
	方法 2:  二重积分交换积分次序
	$$\int_{0}^{+\infty}\left[ \frac{\sqrt{\pi}}{2}-\int_{0}^{x}e^{-t^2}dt\right]dx=\int_{0}^{+\infty}\left[\int_{0}^{+\infty}e^{-t^2}dt-\int_{0}^{x}e^{-t^2}dt\right]dx$$
	
	我们得到: 
	$$\int_{0}^{+\infty}(\int_{x}^{+\infty}e^{-t^2}dt)dx=\int_{0}^{+\infty}dt\int_{0}^{t}e^{-t^2}dx=\int_{0}^{+\infty}te^{-t^2}dt=\frac{1}{2}$$
\end{solution}

\myspace{1}

2. 设 $F(x)=f(x)g(x)$,其中函数$f(x),g(x)$在$(-\infty,+\infty)$内满足以下条件: $f'(x)=g(x),g'(x)=f'(x)$,且 $f(0)=0$,$f(x)+g(x)=2e^x$

(i).求$F(x)$ 满足的一阶微分方程

(ii).求$F(x)$ 表达式
\myspace{1}
\begin{solution}
	
	(i).我们有: 
	$$F'(x)=f'(x)g(x)+g'(x)f(x)=f^2(x)+g^2(x)=(f(x)+g(x))^2-2f(x)g(x)$$
	
	我们得到$F(x)$ 满足的微分方程为: $F'(x)+2F(x)=4e^{2x}$
	
	(ii).我们利用一阶线性微分方程公式得到: 
	$$(e^{2x}F(x))'=4e^{4x}\Rightarrow F(x)=\frac{e^{4x}+C}{e^{2x}}$$
	
	由$f(0)=0\Rightarrow F(0)=1+C=0\Rightarrow C=-1$,$F(x)=e^{2x}-e^{-2x}$
\end{solution}

\myspace{1}

3. 求级数 $\sum\limits_{n=2}^{+\infty}\dfrac{1}{(n^2-1)2^n}$
\myspace{1}
\begin{solution}
	
	我们引入幂级数: 
	
	$$\sum\limits_{n=2}^{+\infty}\frac{x^n}{n^2-1}=\sum\limits_{n=2}^{+\infty}\frac{1}{2}(\frac{1}{n-1}-\frac{1}{n+1})x^n=\frac{1}{2}(\sum\limits_{n=2}^{+\infty}\frac{x^n}{n-1}-\sum\limits_{n=2}^{+\infty}\frac{x^n}{n+1})$$
	
	$$\sum\limits_{n=2}^{+\infty}\frac{x^n}{n-1}=x(\sum\limits_{n=2}^{+\infty}\int_{0}^{x}x^{n-2}dx)=x\int_{0}^{x}(\sum\limits_{n=2}^{+\infty}x^{n-2})dx=-x\ln(1-x)$$
	
	$$\sum\limits_{n=2}^{+\infty}\frac{x^n}{n+1}=\frac{\sum\limits_{n=2}^{+\infty}\frac{x^{n+1}}{n+1}}{x}=\frac{\sum\limits_{n=2}^{+\infty}\int_{0}^{x}x^ndx}{x}=\frac{\int_{0}^{x}(\sum\limits_{n=2}^{+\infty}x^n)dx}{x}=-\frac{x}{2}-1-\frac{\ln(1-x)}{x}$$
	
	原幂级数的和函数为 : 
	$$S(x)=\frac{1}{2}(-x\ln(1-x)+\frac{\ln(1-x)}{x}+\frac{x}{2}+1),-1<x<1$$
	
	我们得到: $S(\dfrac{1}{2})=\dfrac{5}{8}-\dfrac{3\ln 2}{4}$
\end{solution}

\myspace{1}

\hl{\textbf{\textit{May 19}}}

1. 已知 $f(x)=\lim\limits_{n\rightarrow  +\infty}\sqrt[n]{2+(2x)^{n}+x^{2n}}\quad (x\geq 0)$,$g(x)=\lim\limits_{n\rightarrow +\infty}\dfrac{1-x^{2n+1}}{1+x^{2n}}$,求$f(g(x))$
\myspace{1}
\begin{solution}
	
	我们易得到: $f(x)=\left\lbrace 
	\begin{array}{l}
		x^2,x\geq 2\\
		2x,\frac{1}{2}<x< 2\\
		1,0\leq x\leq \frac{1}{2}
	\end{array}
	\right. $,\quad $g(x)=\left\lbrace 
	\begin{array}{l}
		-x,|x|> 1\\
		0,x=1\\
		1,-1\leq x<1
	\end{array}
	\right. $.
	
	我们可以得到: 
	$$f(g(x))=\left\lbrace 
	\begin{array}{l}
		x^2,x\leq -2\\
		-2x,-2<x<-1\\
		2,-1\leq x<1\\
		1,x=1	
	\end{array}
	\right. $$
\end{solution}

\myspace{1}

2. 已知 $y_{1}=(1+x^2)^2-\sqrt{1+x^2}$,$y_{2}=(1+x^2)^2+\sqrt{1+x^2}$是微分方程$y'+p(x)y=q(x)$的两个解,求$q(x)$
\myspace{1}
\begin{solution}
	
	由题意得: 
	$$\left\lbrace 
	\begin{array}{l}
		y_{1}'+p(x)y_{1}=q(x)\\
		y_{2}'+p(x)y_{2}=q(x)
	\end{array}
	\right. \Rightarrow (y_{1}-y_{2})'+p(x)(y_{1}-y_{2})=0\Rightarrow p(x)=-\dfrac{(y_{1}-y_{2})'}{y_{1}-y_{2}}$$
	
	我们得到: $p(x)=-\dfrac{x}{1+x^2}$
	
	任意带入一个方程: $q(x)=y_{1}'+p(x)y_{1}=3x(1+x^2)$
\end{solution}

\myspace{1}

3. 设 $z=f(x,y)$具有二阶连续偏导数,$f'_{y}\neq 0$,证明: 对任意的常数 $k$,曲线$f(x,y)=k$ 是直线的充分必要条件为$(f'_{y})^2f''_{xx}-2f'_{x}f'_{y}f''_{xy}+(f'_{x})^2f''_{yy}=0$
\myspace{1}
\begin{solution}
	
	$f(x,y)=k\text{为直线}\Rightarrow f(x,y)=ax+by+c=k$
	
	(i).必要性
	
	$f(x,y)=k\text{为直线}\Rightarrow f''_{xx}=f''_{yy}=f''_{xy}=0$,我们可以得到: $$(f'_{y})^2f''_{xx}-2f'_{x}f'_{y}f''_{xy}+(f'_{x})^2f''_{yy}=0$$
	
	(ii).充分性
	
	我们记$\left\lbrace 
	\begin{array}{l}
		f'_{x}=f'_{1}\\
		f'_{y}=f'_{2}\\
		f''_{xx}=f''_{11}\\
		f''_{xy}=f''_{12}\\
		f''_{yx}=f''_{21}\\
		f''_{yy}=f''_{22}
	\end{array}
	\right. $,我们将$f(x,y)=k$对$x$求导: 
	$$f'_{1}+f'_{2}\frac{dy}{dx}=0$$
	
	再次对等式两边对$x$求导: 
	$$f''_{11}+f''_{12}\frac{dy}{dx}+(f''_{21}+f''_{22}\frac{dy}{dx})\frac{dy}{dx}+f'_{2}\frac{d^2y}{dx^2}=0$$
	
	若要证明$f(x,y)=k$是直线,我们只需要证明: 
	$$\frac{d^2y}{dx^2}=0\Rightarrow f''_{11}+f''_{12}\frac{dy}{dx}+(f''_{21}+f''_{22}\frac{dy}{dx})\frac{dy}{dx}=0$$
	
	由隐函数求导公式: $\dfrac{dy}{dx}=-\dfrac{f'_{1}}{f'_{2}}$,我们化简上式: 
	$$f''_{11}+f''_{12}(-\frac{f'_{1}}{f'_{2}})+(f''_{21}-f''_{22}\frac{f'_{1}}{f'_{2}})(-\frac{f'_{1}}{f'_{2}})=\frac{(f''_{2})^2f''_{11}-2f'_{1}f'_{2}f''_{12}+(f'_{1})^2f''_{22}}{(f''_{2})^2}=0$$
	
	令分子为$0$即可: 
	$$(f''_{2})^2f''_{11}-2f'_{1}f'_{2}f''_{12}+(f'_{1})^2f''_{22}=0\Rightarrow(f'_{y})^2f''_{xx}-2f'_{x}f'_{y}f''_{xy}+(f'_{x})^2f''_{yy}=0$$
\end{solution}

\myspace{1}

\hl{\textbf{\textit{May 20}}}

1.判断级数的敛散性 $\sum\limits_{n=1}^{+\infty}(\sqrt{n+2}-2\sqrt{n+1}+\sqrt{n})$
\myspace{1}
\begin{solution}
	
	我们不妨记$u_{n}=(\sqrt{n+2}-\sqrt{n+1})-(\sqrt{n+1}-\sqrt{n})$
	
	级数的部分和$S_{n}=u_{1}+u_{2}+\cdots+u_{n}$
	\begin{eqnarray*}
		S_{n}&=&(\sqrt{3}-\sqrt{2})-(\sqrt{2}-\sqrt{1})+(\sqrt{4}-\sqrt{3})-(\sqrt{3}-\sqrt{2})+\cdots\\
		&+&(\sqrt{n+2}-\sqrt{n+1})-(\sqrt{n+1}-\sqrt{n})\\
		&=&(\sqrt{n+2}-\sqrt{n+1})-(\sqrt{2}-\sqrt{1})
	\end{eqnarray*}
	
	我们得到: $\lim\limits_{n\rightarrow +\infty}S_{n}=1-\sqrt{2}$,原级数收敛
\end{solution}

\myspace{1}

2. 判断级数的敛散性 $\sum\limits_{n=1}^{+\infty}\dfrac{\sqrt{n+1}-\sqrt{n}}{n^{\alpha}}$
\myspace{1}
\begin{solution}
	我们有: 
	$$\sqrt{n+1}-\sqrt{n}=\dfrac{1}{\sqrt{n+1}+\sqrt{n}}$$
	
	$$n\rightarrow  +\infty,\quad \frac{1}{\sqrt{n+1}+\sqrt{n}}\sim\frac{1}{2\sqrt{n}}$$
	$$\sum\limits_{n=0}^{+\infty}\frac{\sqrt{n+1}-\sqrt{n}}{n^{\alpha}}\sim\sum\limits_{n=0}^{+\infty}\frac{1}{2}\frac{1}{n^{\alpha+\frac{1}{2}}}$$
	
	我们可以得到: 
	$$\sum\limits_{n=1}^{+\infty}\frac{\sqrt{n+1}-\sqrt{n}}{n^{\alpha}}\left\lbrace 
	\begin{array}{l}
		\text{收敛},\alpha>\dfrac{1}{2}\\
		\text{发散},\alpha \leq \dfrac{1}{2}
	\end{array}
	\right. $$
\end{solution}

\myspace{1}

3. 设连续函数 $f(x)$ 在 $(-\infty,+\infty)$ 上单调增加,下列说法正确的是: 
\begin{itemize}
	\item A. $\tan f(x)$ 在 $(-\infty,+\infty)$ 上单调增加
	\item B. $f'(x)>0,\quad x\in(-\infty,+\infty)$
	\item C. $\int_{-1}^{x}\dfrac{f(t)}{1+f^{2}(t)}dt$在$(-\infty,+\infty)$上单调增加
	\item \hl{\textbf{D}}. $\int_{-1}^{e^x}\dfrac{1}{1+f^{2}(t)}dt$在$(-\infty,+\infty)$上单调增加
\end{itemize}
\myspace{1}
\begin{solution}
	
(i). 对于$f(x)=x^3$: $f'(x)\geq 0$,$tan f(x)$在$(-\infty,+\infty)$ 上不单调,$A\text{、}B$错误
	
(ii).令$F(x)=\int_{-1}^{x}\frac{f(t)}{1+f^{2}(t)}dt,F'(x)=\frac{f(x)}{1+f^{2}(x)}$,$f(x)$正负性未知,无法判断函数单调性,$C$错误
	
(iii)令$F(x)=\int_{-1}^{x}\frac{1}{1+f^{2}(t)}dt,F'(x)=\frac{1}{1+f^{2}(x)}>0$,$F(x)$在$(-\infty,+\infty)$上单调递增.
	
正确答案: $D$
\end{solution}

\myspace{1}

4. 下列微分方程是以$y=C_{1}e^x+C_{2}\cos 2x+C_{3}\sin 2x,(C_{1},C_{2},C_{3}\in \mathbb{R})$ 为通解的微分方程为: 
\begin{itemize}
	\item A. $y'''+y''-4y'-4y=0$
	\item B. $y'''+y''+4y'+4y=0$
	\item C. $y'''-y''-4y'+4y=0$
	\item \hl{D}. $y'''-y''+4y'-4y=0$
\end{itemize}
\myspace{1}
\begin{solution}
	
	我们易得: $y=C_{1}e^{x}+C_{4}e^{0}(A\cos 2x+B\sin 2x)$
	
	我们可以得到原三阶微分方程对应的特征方程的三个根 $x_{1}=1,\ x_{2}=2i,\ x_{3}=-2i$,特征方程为: $(r-1)(r^2+4)=0\Rightarrow r^3-r^2+4r-4=0$
	
	我们得到微分方程表达式为: $y'''-y''+4y'-4y=0$,故答案选$D$
\end{solution}

\myspace{1}

5. 设 $f(x)$ 在 $[0,a]$连续可导,证明: $\int_{0}^{a}dx\int_{0}^{x}\dfrac{f'(y)}{\sqrt{(a-x)(x-y)}}dy=\pi[f(a)-f(0)]$
\myspace{1}
\begin{solution}
	
	原二重积分等价于: 
	\begin{eqnarray*}
		\int_{0}^{a}dy\int_{y}^{a}\frac{f'(y)}{\sqrt{(\frac{a-y}{2})^2-(x-\frac{a+y}{2})^2}}dx&=&\int_{0}^{a}f'(y)[\arcsin (\frac{2x-a-y}{a-y})]|_{y}^{a}dy\\
		&=&\pi\int_{0}^{a}f'(y)dy=\pi[f(a)-f(0)]
	\end{eqnarray*}
	
\end{solution}

\myspace{1}

\hl{\textbf{\textit{May 21}}}

1. $f(x)$ 连续且为奇函数,下列函数一定是偶函数的是: 
\begin{itemize}
	\item A. $\int_{0}^{x}du\int_{a}^{u}tf(t)dt$ 
	\item B. $\int_{a}^{x}du\int_{0}^{u}f(t)dt$ 
	\item C. $\int_{0}^{x}du\int_{a}^{u}f(t)dt$ 
	\item \hl{D}. $\int_{a}^{x}du\int_{0}^{u}tf(t)dt$ 
\end{itemize}
\myspace{1}
\begin{solution}
	
	(i).对于$A\text{、}C$,我们交换二重积分的积分次序: 
	$$A\text{: }\int_{0}^{x}du\int_{a}^{0}tf(t)dt+\int_{0}^{x}du\int_{0}^{u}tf(t)dt=x\int_{a}^{0}tf(t)dt+\int_{0}^{x}t^2f(t)dt,\text{前一个是奇函数,后一个是偶函数}$$
	$$C\text{: }\int_{0}^{x}du\int_{a}^{u}f(t)dt=\int_{0}^{x}du\int_{a}^{0}f(t)dt+\int_{0}^{x}du\int_{0}^{u}f(t)dt,\text{前一个是奇函数,后一个是奇函数}$$
	
	(ii).对于$B\text{、}D$,我们交换二重积分的积分次序: 
	
	$$B\text{: }\int_{a}^{0}du\int_{0}^{u}f(t)dt+\int_{0}^{x}du\int_{0}^{u}f(t)dt=\int_{a}^{x}tf(t)dt,\text{奇函数}$$
	$$D\text{: }\int_{a}^{0}du\int_{0}^{u}tf(t)dt+\int_{0}^{x}du\int_{0}^{u}tf(t)dt=\int_{a}^{x}t^2f(t)dt,\text{偶函数}$$
	
	此题答案为: $D$
\end{solution}

\myspace{1}

2. 证明:  $\int_{0}^{1}dx\int_{0}^{1}(xy)^{xy}dy=\int_{0}^{1}x^{x}dx$
\myspace{1}
\begin{solution}
	
	我们令$xy=t,y=\dfrac{1}{x}t$,我们得到原二重积分为: $\int_{0}^{1}dx\int_{0}^{x}\dfrac{1}{x}t^{t}dt$.
	
	我们交换二重积分的积分次序: 
	$$\int_{0}^{1}dt\int_{t}^{1}\frac{1}{x}t^{t}dx=-\int_{0}^{1}t^{t}\ln tdt=-\int_{0}^{1}x^{x}\ln xdx$$
	
	我们只需要证明: 
	$$-\int_{0}^{1}x^{x}\ln xdx=\int_{0}^{1}x^{x}dx\Rightarrow\int_{0}^{1}x^{x}(1+\ln x)dx=\int_{0}^{1}e^{x\ln x}d(x\ln x)=0$$
	
	证毕.
\end{solution}
3.微分方程 $y''-4y'+8y=e^{2x}(1+\cos 2x)$ 的特解可以设为哪种形式: 
\begin{itemize}
	\item A. $Ae^{2x}+e^{2x}(B\cos 2x+C\sin 2x)$ 
	\item B. $Axe^{2x}+e^{2x}(B\cos 2x+C\sin 2x)$ 
	\item \hl{C}. $Ae^{2x}+xe^{2x}(B\cos 2x+C\sin 2x)$ 
	\item D. $Axe^{2x}+xe^{2x}(B\cos 2x+C\sin 2x)$ 
\end{itemize}
\myspace{1}
\begin{solution}
	
	原微分方程对应的特征方程为: $r^2-4r+8=0\Rightarrow r_{1}=2+2i,\ r_{2}=2-2i$
	
	齐次微分方程的通解为: $e^{2x}(A\cos 2x+B\sin 2x)$
	
	对于方程: $y''-4y'+8y=e^{2x}$,特解为: $C_{1}e^{2x}$
	
	对于方程: $y''-4y'+8y=e^{2x}\cos 2x$,特解为: $xe^{2x}(C_{2}\cos 2x+C_{3}\sin 2x)$
	
	我们得到方程的特解形式为: $y=C_{1}e^{2x}+xe^{2x}(C_{2}\cos 2x+C_{3}\sin 2x)$,故此答案选$C$
\end{solution}

\myspace{1}

\section{Week \Rmnum{4}}
\hl{\textbf{\textit{May 22}}}

1. $f(x)$ 连续且为偶函数,下列函数一定是偶函数的是: 
\begin{itemize}
	\item A. $\int_{0}^{x}(x-t^2)f(t)dt$ 
	\item B. $\int_{a}^{x}f(x-t)dt$ 
	\item \hl{C}. $\int_{0}^{x}(x-2t)f(t)dt$ 
	\item D. $\int_{a}^{x}(x-2t)f(t)dt$ 
\end{itemize}
\myspace{1}
\begin{solution}
	
	令$f(x)=1$,我们得到: 
	
	(i).$\int_{0}^{x}(x-t^2)f(t)dt=\int_{0}^{x}(x-t^2)dt=x^2-\dfrac{x^3}{3}\quad \text{非奇非偶函数}$
	
	(ii).$\int_{a}^{x}f(x-t)dt=\int_{a}^{x}dt=x-a,\text{当}a=0\text{时为奇函数}$
	
	(iii).$\int_{0}^{x}(x-2t)f(t)dt=\int_{0}^{x}(x-2t)dt=-\dfrac{x^2}{2},\text{偶函数}$
	
	(iiii). $\int_{a}^{x}(x-2t)f(t)dt=\int_{a}^{x}(x-2t)dt=-\dfrac{x^2}{2}-\dfrac{ax}{2}+a^2,\text{当}a=0\text{时为偶函数}$
	
	故答案为: $C$
\end{solution}

\myspace{1}

2. $\lim\limits_{x\rightarrow +\infty}\dfrac{\int_{1}^{x}t^{-5}dt}{\int_{1}^{x}t^{-3}dt}$
\myspace{1}
\begin{solution}
	
	原极限为: $\lim\limits_{x\rightarrow +\infty}\dfrac{\frac{1}{4}-\frac{1}{4x^4}}{\frac{1}{2}-\frac{1}{2x^2}}=\lim\limits_{x\rightarrow +\infty}\dfrac{1}{2}+\dfrac{1}{2x^2}=\dfrac{1}{2}$
\end{solution}
3. 二阶常系数非齐次线性微分方程 $y''-4y'+3y=2e^{2x}$ 的通解为
\myspace{1}
\begin{solution}
	
	原微分方程对应的特征方程为: $r^2-4r+3=0\Rightarrow r_{1}=1,\ r_{2}=3$
	
	齐次微分方程的通解为: $y=C_{1}e^{x}+C_{2}e^{3x}$
	
	我们设方程的特解为: $y^{*}=Ae^{2x}\Rightarrow A(4-8+3)e^{2x}=2e^{2x}\Rightarrow A=-2$
	
	我们得到原微分方程的通解为: 
	$$y=-2e^{2x}+C_{1}e^{x}+C_{2}e^{3x},C_{1},C_{2}\in\mathbb{R}$$
\end{solution}

\myspace{1}

4.求微分方程 $y''-2y'-e^{2x}=0$ 满足条件 $y(0)=1,y'(0)=1$的解.
\myspace{1}
\begin{solution}
	
	原微分方程对应的特征方程为: $r^2-2r=0\Rightarrow r_{1}=0,\ r_{2}=2$
	
	我们得到齐次微分方程的通解: $y=C_{1}e^{2x}+C_{2}$
	
	我们设微分方程的特解为: $$y^{*}=Axe^{2x}\Rightarrow A(4+4x-2-4x)e^{2x}=e^{2x}\Rightarrow A=\frac{1}{2}$$
	
	我们得到方程的解为: $y=C_{1}e^{2x}+C_{2}+\dfrac{1}{2}xe^{2x}$
	
	又因为$y(0)=1,y'(0)=1\Rightarrow$,我们得到: 
	$$\left\lbrace 
	\begin{array}{l}
		C_{1}+C_{2}=1\\
		2C_{1}+\dfrac{1}{2}=1
	\end{array}
	\right. \Rightarrow \left\lbrace 
	\begin{array}{l}
		C_{1}=\dfrac{1}{4}\\
		C_{2}=\dfrac{3}{4}
	\end{array}
	\right. $$
	
	原微分方程的解: $y=\dfrac{1}{4}e^{2x}+\dfrac{1}{2}xe^{2x}+\dfrac{3}{4}$
\end{solution}

\myspace{1}

5. 设$f(x)$ 在 $(0,+\infty)$ 连续,$f(x)>0$,$\lim\limits_{x\rightarrow  +\infty}f(x)=1$,$\lim\limits_{h\rightarrow 0}\left[\dfrac{f(x+hx)}{f(x)} \right]^{\frac{1}{h}}=e^{\frac{1}{x}} $,求$f(x)$
\myspace{1}
\begin{solution}
	
	我们得到: 
	$$\lim\limits_{h\rightarrow 0}e^{\frac{1}{h}\ln(1+\frac{f(x+hx)-f(x)}{f(x)})}=\lim\limits_{h\rightarrow 0}e^{\frac{1}{h}\frac{f(x+hx)-f(x)}{f(x)}}=e^{\frac{1}{x}}$$
	
	我们进而得到: 
	$$\lim\limits_{h\rightarrow 0}\frac{x}{f(x)}\frac{f(x+x)-f(x)}{hx}=\frac{1}{x}\Rightarrow \frac{f'(x)}{f(x)}=\frac{1}{x^2}$$
	
	即: $\ln f(x)=-\dfrac{1}{x}+C$,又因为$\lim\limits_{x\rightarrow  +\infty}f(x)=1$,两边同时取$x\rightarrow +\infty$ 的极限: 
	$$\lim\limits_{x\rightarrow +\infty}\ln f(x)=\lim\limits_{x\rightarrow +\infty}(-\frac{1}{x}+C)\Rightarrow C=0$$
	
	我们得到: $f(x)=e^{-\frac{1}{x}}$
\end{solution}

\myspace{1}

\hl{\textbf{\textit{May 23}}}

1. $\sum\limits_{n=1}^{+\infty}\dfrac{4^n}{5^n-3^n}$和$\sum\limits_{n=1}^{+\infty}\dfrac{1}{\sqrt{n^3-1}}$敛散性
\myspace{1}
\begin{solution}
	
	采取抓大头的方式,利用级数判别方法中的比较法的极限形式
	
	(i).$\lim\limits_{n\rightarrow+\infty}\dfrac{(\frac{4^n}{5^n-3^n})^n}{(\dfrac{4}{5})^n}=1\Rightarrow \text{原级数和级数}\sum\limits_{n=0}^{+\infty}(\dfrac{4}{5})^n\text{同敛散性}$,原级数收敛
	
	(ii).$\lim\limits_{n\rightarrow+\infty}\dfrac{\frac{1}{\sqrt{n^3-1}}}{\dfrac{1}{\sqrt{n^3}}}=1\Rightarrow \text{原级数和级数}\sum\limits_{n=0}^{+\infty}\dfrac{1}{\sqrt{n^3}}\text{同敛散性}$,原级数收敛
\end{solution}

\myspace{1}

2. 设函数$f(x)$连续,且对于任意的$x\in(-\infty,+\infty)$,恒有$f(x+1)=-f(x)$,下面结论不正确的是: 
\begin{itemize}
	\item A. $f(x)\text{是以2为周期的函数}$ 
	\item B. $\int_{0}^{x}[f(t)-f(-t)]dt\text{是以2为周期的函数}$ 
	\item C. $\int_{0}^{x}f(t)dt-\dfrac{x}{2}\int_{0}^{2}f(t)dt\text{是以2为周期的函数}$ 
	\item \hl{D}. $f'(x)\text{是以2为周期的函数}$ 
\end{itemize}
\myspace{1}
\begin{solution}
	
	我们得到: $\left\lbrace 
	\begin{array}{l}
		f(x+1)+f(x)=0\\
		f(x+2)+f(x+1)=0
	\end{array}
	\right. \Rightarrow f(x+2)=f(x)$
	
	我们得到: $f(x)$是以$2$为周期的周期函数,$A$正确
	
	我们由原函数和导函数周期性关系得到: 
	$$f(x)\text{周期函数},\int_{0}^{x}f(t)dt\text{周期函数}\Leftrightarrow \int_{0}^{T}f(x)dx=0$$
	
	对于$B$选项,我们发现$g(x)=f(x)-f(-x),g(x)\text{是奇函数}$
	
	$g(x)\text{是周期函数}$,且$\int_{-1}^{1}g(x)dx=0\Rightarrow \int_{0}^{x}g(t)dt\text{是周期函数}$
	
	对于$C$选项,我们得到: $\int_{0}^{x}[f(t)-\dfrac{\int_{0}^{2}f(x)dx}{2}]dt$
	
	我们只需要证明: 
	$$\int_{0}^{2}[f(t)-\frac{\int_{0}^{2}f(x)dx}{2}]dt=0$$
	
	不妨设$\int_{0}^{2}f(x)dx=A$,我们有: 
	$$\text{左边}=\int_{0}^{2}[f(t)-\frac{A}{2}]dt=\int_{0}^{2}f(x)dx-A=0\Rightarrow \text{原命题得证}$$
	
	对于$D$选项,我们未知$f(x)$是否可导,故此题答案选$D$
\end{solution}

\myspace{1}

3. 求微分方程: $y''+y=4\sin x$的通解
\myspace{1}
\begin{solution}
	
	特征方程为: $r^2+1=0\Rightarrow r_{1}=i,\ r_{2}=-i$
	
	齐次微分方程的通解为: $y=A\sin x+B\cos x$
	
	非齐次微分方程特解: $y^{*}=x(C_{1}\sin x+C_{2}\cos x)\Rightarrow \left\lbrace 
	\begin{array}{l}
		C_{1}=0\\
		C_{2}=-2
	\end{array}
	\right. $
	
	原微分方程的通解为: $y=-2x\cos x+A\sin x+B\cos x,A\text{、}B\in \mathbb{R}$
\end{solution}

\myspace{1}

4.  求微分方程: $y''+y=\sin x+x\cos 2x$的通解
\myspace{1}
\begin{solution}
	
	特征方程为: $r^2+1=0\Rightarrow r_{1}=i,\ r_{2}=-i$
	
	齐次微分方程的通解为: $y=A\sin x+B\cos x$
	
	非齐次微分方程特解: 
	$$y_{1}^{*}=x(C_{1}\sin x+C_{2}\cos x)\Rightarrow \left\lbrace 
	\begin{array}{l}
		C_{1}=0\\
		C_{2}=-\frac{1}{2}
	\end{array}
	\right. \Rightarrow y_{1}^{*}=-\frac{x}{2}\cos x$$
	$$y_{1}^{*}=(Ax+B)\sin 2x+(Cx+D)\cos 2x)\Rightarrow \left\lbrace 
	\begin{array}{l}
		C=-\frac{1}{3}\\
		A=D=0\\
		B=\frac{4}{9}
	\end{array}
	\right.\Rightarrow y_{2}^{*}=-\frac{x}{3}\cos 2x+\frac{4}{9}\sin 2x $$
	
	原微分方程的通解为: $y=y_{1}^{*}+y_{2}^{*}+A\sin x+B\cos x,A\text{、}B\in \mathbb{R}$
\end{solution}

\myspace{1}

5.$\lim\limits_{x\rightarrow 0 }\int_{0}^{2x}du\int_{0}^{\sqrt{2ux-u^2}}\dfrac{cos(t-u)^2}{\ln(1+x|x|)}dt$
\myspace{1}
\begin{solution}
	
	我们不难发现此题需要分左右极限分别来求: 
	
	$$I_{\text{左}}=\lim\limits_{x\rightarrow 0 }\frac{-\iint\limits_{D_{1}}\cos(t-u)^2d\sigma}{-x^2}$$
	
	由二元函数的积分中值定理: 
	
	$$\iint\limits_{D_{1}}\cos(t-u)^2d\sigma=S_{D_{1}}f(\varepsilon_{1},\varphi_{1}),(\varepsilon_{1},\varphi_{1})\text{在以}(0,x)\text{为半径},x
	\text{为半径的圆内}$$
	$$I_{\text{左}}=\lim\limits_{x\rightarrow 0 }\frac{\frac{x^2\pi}{2}}{-x^2}\cos(\varepsilon_{1}-\varphi_{1})^2=\frac{\pi}{2}$$
	
	$$I_{\text{右}}=\lim\limits_{x\rightarrow 0 }\frac{\iint\limits_{D_{2}}\cos(t-u)^2d\sigma}{x^2}$$
	
	由二元函数的积分中值定理: 
	
	$$\iint\limits_{D_{2}}\cos(t-u)^2d\sigma=S_{D_{2}}f(\varepsilon_{2},\varphi_{2}),(\varepsilon_{2},\varphi_{2})\text{在以}(0,x)\text{为半径},x
	\text{为半径的圆内}$$
	$$I_{\text{右}}=\lim\limits_{x\rightarrow 0 }\frac{\frac{x^2\pi}{2}}{x^2}\cos(\varepsilon_{2}-\varphi_{2})^2=\frac{\pi}{2}$$
	
	综上所述: $I=\frac{\pi}{2}$
\end{solution}

\myspace{1}

\hl{\textbf{\textit{May 24}}}

1. 下列函数中原函数必为周期函数的是: 

\begin{itemize}
	\item A. $|\sin x|$ 
	\item B. $\sin^4 x$ 
	\item C. $\dfrac{1}{1+\sin^2x}$ 
	\item \hl{D}. $\dfrac{\sin x}{1+\sin^4 x}$ 
\end{itemize}
\myspace{1}
\begin{solution}
	
	原函数为周期函数,只需满足函数为周期函数,且$\int_{0}^{T}f(t)dt=0$
	
	对于四个选项: $A,B,C\text{对应的}f(t)\text{在一个周期中函数值大于0},\int_{0}^{T}f(t)dt>0$
	
	此题答案选$D$
\end{solution}

\myspace{1}

2. 设$y=\dfrac{1}{2}e^{2x}+(x-\frac{1}{3})e^{x}$是二阶常系数线性微分方程$y''+ay'+by=ce^x$的一个特解,则: 
\begin{itemize}
	\item \hl{A}. $a=-3,\ b=2,\ c=-1$ 
	\item B. $a=3,\ b=2,\ c=-1$ 
	\item C. $a=-3,\ b=2,\ c=1$ 
	\item D. $a=3,\ b=2,\ c=1$ 
\end{itemize}
\myspace{1}
\begin{solution}
	
	我们可以得到特征方程的两根: $r_{1}=1,\ r_{2}=2\Rightarrow \text{特征方程为: }r^2-3r+2=0\Rightarrow a=-3,\ b=2$
	
	我们将特解$y{*}=xe^{x}$代入微分方程,得到: $c=-1$,故此题选$A$
\end{solution}

\myspace{1}

3. 设$F(a)=\int_{0}^{\frac{\pi}{2}}|\sin x-a\cos x|dx$,求$F(a)$的最小值
\myspace{1}
\begin{solution}
	
	$$F(a)=\sqrt{a^2+1}\int_{0}^{\frac{\pi}{2}}|\sin(x+\varphi)|dx,\text{其中}\left\lbrace 
	\begin{array}{l}
		\cos \varphi=\frac{1}{\sqrt{a^2+1}}\\
		\sin \varphi=-\frac{a}{\sqrt{a^2+1}}
	\end{array}
	\right. $$
	
	(1).$a>0,\ \varphi\in(-\frac{\pi}{2},0)$
	
	$$\frac{F(a)}{\sqrt{a^2+1}}=\int_{\varphi}^{\frac{\pi}{2}+\varphi}|\sin t|dt=2\int_{0}^{-\varphi}\sin tdt+\int_{-\varphi}^{\frac{\pi}{2}+\varphi}\sin tdt=2-\cos \varphi+\sin \varphi=2-\frac{a+1}{\sqrt{a^2+1}}$$
	\begin{eqnarray*}
		F(a)&=&2\sqrt{a^2+1}-(a+1)\\ F'(a)&=&\frac{2a-\sqrt{a^2+1}}{\sqrt{a^2+1}}\\
		\text{当}x&=&\frac{\sqrt{3}}{3},F(a)_{min}=\sqrt{3}-1
	\end{eqnarray*}
	
	(2).$a<0,\ \varphi\in(0,\frac{\pi}{2})$
	
	$$\frac{F(a)}{\sqrt{a^2+1}}=\int_{\varphi}^{\frac{\pi}{2}+\varphi}|\sin t|dt=\cos \varphi+\sin \varphi=\frac{1-a}{\sqrt{a^2+1}}$$
	\begin{eqnarray*}
		F(a)&=&1-a,\ F(a)\geq 1,\ F(a)_{min}=1
	\end{eqnarray*}
	
	综上,$F(a)_{min}=\sqrt{3}-1$
\end{solution}

\myspace{1}

\hl{\textbf{\textit{May 25}}}

1. 设 $f(x)$连续且以$T$为周期,则下列函数以$T$为周期的是: 
\begin{itemize}
	\item A. $\int_{0}^{x}f(t)dt$ 
	\item B. $\int_{-x}^{0}f(t)dt$ 
	\item \hl{C}. $\int_{0}^{x}f(t)dt-\int_{-x}^{0}f(t)dt$ 
	\item D. $\int_{0}^{x}f(t)dt+\int_{-x}^{0}f(t)dt$ 
\end{itemize}
\myspace{1}
\begin{solution}
	
	我们知道如果原函数也为周期函数,那么必有: $\int_{0}^{T}f(t)dt=0$
	
	对于$A,B$选项,我们自然可以排除掉,没有任何附加条件
	
	对于$C,D$选项
	$$C\Rightarrow \int_{0}^{x}[f(t)-f(-t)]dt\text{奇函数}\Rightarrow \int_{-\frac{T}{2}}^{\frac{T}{2}}[f(t)-f(-t)]dt=0$$
	$$D\Rightarrow \int_{0}^{x}[f(t)+f(-t)]dt\text{偶函数}$$
	
	故此题答案选$C$
\end{solution}

\myspace{1}

2. $\int_{0}^{+\infty}\dfrac{dx}{(1+x^2)(1+x^3)}$
\myspace{1}
\begin{solution}
	
	令$x=\dfrac{1}{t},t\in(0,+\infty),dx=-\dfrac{1}{t^2}dt$,原积分等价于: 
	$$I=\int_{0}^{+\infty}\dfrac{t^3}{(1+t^2)(1+t^3)}dt=\int_{0}^{+\infty}\dfrac{x^3}{(1+x^2)(1+x^3)}dx$$
	$$2I=\int_{0}^{+\infty}\dfrac{1+x^3}{(1+x^2)(1+x^3)}dx=\int_{0}^{+\infty}\dfrac{1}{1+x^2}dx=\frac{\pi}{2}$$
	$$I=\int_{0}^{+\infty}\dfrac{dx}{(1+x^2)(1+x^3)}=\frac{\pi}{4}$$
\end{solution}

\myspace{1}

3. 设连续函数$f(x)$满足$f(x)+2\int_{0}^{x}f(t)dt=x^2$,求$f(x)$
\myspace{1}
\begin{solution}
	
	令$F(x)=\int_{0}^{x}f(t)dt,F'(x)=f(x)$,原微分方程等价于: 
	$$F'(x)+2F(x)=x^2\Rightarrow [e^{2x}F(x)]'=x^2e^{2x}$$
	$$e^{2x}F(x)=\int x^2e^{2x}dx\Rightarrow e^{2x}F(x)=\frac{x^2e^{2x}}{2}-\frac{xe^{2x}}{2}+\frac{e^{2x}}{4}+C$$
	我们得到: $$F(x)=\frac{1}{2}x^2-\frac{1}{2}x+\frac{1}{4}+Ce^{-2x}$$
	
	$f(x)=F'(x)=x-\frac{1}{2}-2Ce^{-2x},f(0)=-\frac{1}{2}-2C=0\Rightarrow C=-\frac{1}{4}$
	
	$$f(x)=\frac{1}{2}e^{-2x}+x-\frac{1}{2}$$
	
\end{solution}

\myspace{1}

4. 设连续函数$f(x)$满足$x\int_{0}^{1}f(tx)dt=f(x)+x$,求$f(x)$
\myspace{1}
\begin{solution}
	
	对于$x\int_{0}^{1}f(tx)dt$,令$u=tx,t=\dfrac{u}{x},dt=\dfrac{du}{x}$,原微分方程为: 
	$$\int_{0}^{x}f(u)du=f(x)+x$$
	
	令$F(x)=\int_{0}^{x}f(u)du=f(x)+x,F'(x)=f(x)$,原微分方程等价于: 
	$$F(x)-F'(x)=x\Rightarrow [e^{-x}F(x)]'=-xe^{-x}$$
	$$e^{-x}F(x)=\int(-xe^{-x})dx=(x+1)e^{-x}+C\Rightarrow F(x)=Ce^{x}+x+1$$
	$$f(x)=F'(x)=Ce^{x}+1,f(0)=0\Rightarrow C=-1$$
	$$f(x)=1-e^{x}$$
\end{solution}

\myspace{1}

5. 求抛物线$y=x^2$和直线$x-y-2=0$之间的最短距离
\myspace{1}
\begin{solution}
	
	方法一(不太严谨): 
	
	设抛物线上任意一点坐标$P(x_{0},y_{0})$,点$P$到直线$x-y-2=0$的距离为: 
	$$d=\dfrac{|x_{0}-x_{0}^2-2|}{\sqrt{2}}=\dfrac{(x_{0}-\frac{1}{2})^2+\frac{7}{4}}{\sqrt{2}}$$
	$$d_{min}=\dfrac{7\sqrt{2}}{8}$$
	
	方法二: 
	
	我们设抛物线上一点$P$坐标为$(x,y)$,直线$x-y-2=0$上一点$Q$坐标$(\alpha,\beta)$,我们得到$|PQ|^2=(x-\alpha)^2+(y-\beta)^2$.
	
	我们令$f(x,y,\alpha,\beta,\lambda,\mu)=(x-\alpha)^2+(y-\beta)^2+\lambda(y-x^2)+\mu(\alpha-\beta-2)$
	
	我们令: $$\left\lbrace 
	\begin{array}{l}
		f'_{x}=2(x-\alpha)-2\lambda x=0\\
		f'_{y}=2(y-\beta)+\lambda=0\\
		f'_{\alpha}=-2(x-\alpha)+\mu=0\\
		f'_{\beta}=-2(y-\beta)-\mu=0\\
		f'_{\lambda}=y-x^2=0\\
		f'_{\mu}=\alpha-\beta-2=0
	\end{array}
	\right. $$
	我们解得: $$\left\lbrace 
	\begin{array}{l}
		x=\dfrac{1}{2}\\
		y=\dfrac{1}{4}\\
		\alpha=\dfrac{11}{8}\\
		\beta=-\dfrac{5}{8}\\
		\lambda=-\dfrac{7}{4}\\
		\mu=-\dfrac{7}{4}
	\end{array}
	\right. $$
	$$f(x,y,\alpha,\beta,\lambda,\mu)_{min}=f(\dfrac{1}{2},\dfrac{1}{4},\dfrac{11}{8},-\dfrac{5}{8},-\dfrac{7}{4},-\dfrac{7}{4})=\dfrac{7\sqrt{2}}{8}$$
\end{solution}

\myspace{1}

\hl{\textbf{\textit{May 26}}}

1. 设$f(x)$二阶可导,$f(-x)=-f(x),f(x+1)=f(x),x\in(-\infty,+\infty)$,且$\lim\limits_{x\rightarrow 1}\dfrac{f(x)}{\sin(x-1)}=1$,则
\begin{itemize}
	\item A. $f''(99)\leq f'(100)\leq f(101)$ 
	\item \hl{B}. $f(99)=f(100)<f'(101)$ 
	\item C. $f'(99)\leq f(100)<f''(101)$ 
	\item D. $f(99)< f'(100)=f''(100)$ 
\end{itemize}
\myspace{1}
\begin{solution}
	
	由题意知道: $$f(x),f'(x),f''(x)\text{周期为}1,f(x),f''(x)\text{是奇函数},\text{且}f(1)=0$$
	$$\lim\limits_{x\rightarrow 1}\frac{f(x)}{\sin(x-1)}=\lim\limits_{x\rightarrow 1}\frac{f(x)-f(1)}{x-1}=f'(1)=1$$
	
	我们从而得到: $f(n)=0,f'(n)=1,f''(n)=0$
	
	此题答案选$B$
\end{solution}

\myspace{1}

2. 设连续函数$f(x)$满足$\int_{0}^{x}f(x-t)dt=\int_{0}^{x}(x-t)f(t)dt+e^{-x}-1$,求$f(x)$
\myspace{1}
\begin{solution}
	
	原微分方程等价于: 
	$$\int_{0}^{x}f(t)dt=x\int_{0}^{x}f(t)dt-\int_{0}^{x}tf(t)dt+e^{-x}-1$$
	
	对微分方程左右两边对$x$求导,得到: 
	$$f(x)=\int_{0}^{x}f(t)dt-e^{-x}$$
	
	我们令$F(x)=\int_{0}^{x}f(t)dt,f(x)=F'(x)$,上式等价于: 
	$$F(x)-F'(x)=e^{-x}\Rightarrow [e^{-x}F(x)]'=-e^{-2x}$$
	$$F(x)=Ce^{x}+\frac{1}{2}e^{-x}$$
	$$f(x)=F'(x)=Ce^{x}-\frac{1}{2}e^{-x},f(0)=-1\Rightarrow C=-\frac{1}{2}$$
	$$f(x)=-\frac{1}{2}e^{x}-\frac{1}{2}e^{-x}$$
\end{solution}
3.设连续函数$f(x)$满足$f(x)=\sin x-\int_{0}^{x}(x-t)f(t)dt$,求$f(x)$
\myspace{1}
\begin{solution}
	
	对微分方程左右两边对$x$求导,得到: 
	$$f'(x)=\cos x-\int_{0}^{x}f(t)dt$$
	
	令$F(x)=\int_{0}^{x}f(t)dt,f(x)=F'(x)$,上式等价于: 
	$$F''+F=\cos x$$
	
	特征方程为$r^2+1=0,r_{1}=i,\ r_{2}=-i$,方程通解为$F(x)=C_{1}\cos x+C_{2}\sin x$
	
	特解设为$F(x)=x(A\cos x+B\sin x)$,代入得到: 
	$$A=0,\ B=\frac{1}{2}$$
	$$F(x)=C_{1}\cos x+C_{2}\sin x+\frac{x}{2}\sin x$$
	$$f(x)=F'(x)=C_{2}\cos x-C_{1}\sin x+\frac{1}{2}\sin x+\frac{x}{2}\cos x\Rightarrow \left\lbrace 
	\begin{array}{l}
		f(0)=0\\
		f'(0)=1
	\end{array}
	\right. \Rightarrow \left\lbrace 
	\begin{array}{l}
		C_{1}=0\\
		C_{2}=0
	\end{array}
	\right. $$
	$$f(x)=\frac{1}{2}\sin x+\frac{x}{2}\cos x$$
\end{solution}

\myspace{1}

4. $\int_{0}^{\frac{\pi}{4}}\ln(1+\tan x)dx$ \label{problem: 区间再现}
\myspace{1}
\begin{solution}
	
	典型的区间再现
	
	$$I=\int_{0}^{\frac{\pi}{4}}\ln(1+\tan x)dx=\int_{0}^{\frac{\pi}{4}}\ln(1+\tan(\frac{\pi}{4}-x))dx$$
	$$I=\int_{0}^{\frac{\pi}{4}}(\ln2-\ln(1+\tan x))dx$$
	$$2I=\int_{0}^{\frac{\pi}{4}}\ln 2dx=\frac{\ln 2}{4}\pi$$
	$$I=\int_{0}^{\frac{\pi}{4}}\ln(1+\tan x)dx=\frac{\ln 2}{8}\pi$$
\end{solution}

\myspace{1}

5. 设$f(x)$是连续的正值函数,且单调减少,证明: $\dfrac{\int_{0}^{1}xf^{2}(x)dx}{\int_{0}^{1}xf(x)dx}\leq \dfrac{\int_{0}^{1}f^{2}(x)dx}{\int_{0}^{1}f(x)dx}$
\myspace{1}
\begin{solution}
	
	原命题等价于: 
	$$\int_{0}^{1}xf^{2}(x)dx\int_{0}^{1}f(y)dy-\int_{0}^{1}xf(x)dx\int_{0}^{1}f^{2}(y)dy\leq 0$$
	$$I=\iint\limits_{D}[xf(x)f(y)(f(x)-f(y))]dxdy,D=\{0\leq x\leq 1;\ 0\leq y\leq 1 \}$$
	
	积分区域关于$y=x$对称,我们交换$x,y$位置,得到
	$$2I=\iint\limits_{D}[f(x)f(y)(f(x)-f(y))(x-y)]dxdy,D=\{0\leq x\leq 1;\ 0\leq y\leq 1 \}$$
	
	我们知道$f(x)$单调递减,且$f(x)>0$,我们得到: 
	$$f(x)f(y)>0,\ (x-y)(f(x)-f(y))\leq 0\Rightarrow I\leq 0,\text{证毕}$$
\end{solution}

\myspace{1}

6. $\iint\limits_{D}\dfrac{1}{\arcsin\sqrt{x^2+y^2}}dxdy$,其中$D=\{(x,y)|x^2+y^2\leq 1,\ x^2+y^2\geq y,\ x\geq 0,\ y\geq 0\}$
\myspace{1}
\begin{solution}
	
	利用极坐标公式进行代换,$x=r\cos \theta,\ y=r\sin \theta$,我们得到: 
	$$I=\iint\limits_{D_{1}}\dfrac{r}{\arcsin r}drd\theta,\theta\in(0,\frac{\pi}{2}),\ r\in(\sin\theta,1)$$
	
	上面的积分可以化为: 
	$$I=\int_{0}^{\frac{\pi}{2}}d\theta\int_{\sin\theta}^{1}\dfrac{r}{\arcsin r}dr=\int_{0}^{1}dr\int_{0}^{\arcsin r}\frac{r}{\arcsin r}d\theta=\frac{1}{2}$$
	$$\iint\limits_{D}\dfrac{1}{\arcsin\sqrt{x^2+y^2}}dxdy=\frac{1}{2}$$
\end{solution}

\myspace{1}

\hl{\textbf{\textit{May 27}}}

1. 设$f(x)$在$[0,1]$连续,在$(0,1)$内可导,且$\int_{0}^{1}f(x)dx=1$,证明: $\exists \xi\neq \eta\in(0,1),\text{s.t.}\ f(\xi)+3f(\eta)=4f(\xi)f(\eta)$
\myspace{1}
\begin{solution}
	
	我们令$F(x)=\int_{0}^{x}f(t)dt$,我们有: 
	
	$$F(0)=0,\ F(1)=1,F(x)\text{在}[0,1]\text{连续},\text{在}(0,1)\text{可导}$$
	
	原命题转化为证明: 
	$$\exists \xi\neq \eta\in(0,1),\text{s.t.}\ F'(\xi)+3F'(\eta)=4F'(\xi)F'(\eta)$$
	
	上面的式子我们进行一些变形: 
	$$\exists \xi\neq \eta\in(0,1),\text{s.t.}\ \frac{1}{F'(\eta)}+\frac{3}{F'(\xi)}=4$$
	
	$F(x)\text{在}[0,1]\text{连续},F(0)=0,F(1)=1$,$\exists c\in(0,1)\text{s.t.}\ F(c)=\dfrac{1}{4}$.
	
	由拉格朗日中值定理我们得到: 
	$$\exists\xi\in(0,c),\eta\in(c,1), \text{s.t.}\left\lbrace 
	\begin{array}{l}
		\dfrac{F(c)-F(0)}{c}=F'(\xi)\\
		\dfrac{F(1)-F(c)}{1-c}=F'(\eta)
	\end{array}
	\right. \Rightarrow \left\lbrace 
	\begin{array}{l}
		\dfrac{1}{F'(\xi)}=4c\\
		\dfrac{1}{F'(\eta)}=\frac{4}{3}(1-c)
	\end{array}
	\right. $$
	$$\frac{1}{F'(\xi)}+\frac{3}{F'(\eta)}=4c+4(1-c)=4$$
	$$\exists \xi\neq \eta\in(0,1),\text{s.t.}\ f(\xi)+3f(\eta)=4f(\xi)f(\eta),\text{证毕}$$
\end{solution}

\myspace{1}

2. 函数$\dfrac{|x|e^{\frac{1}{x-1}}\ln(x-1)^2}{x(x-1)(x-2)}$在下列哪个区间内无界: 
\begin{itemize}
	\item A. $(-\infty,0)$ 
	\item B. $(0,1)$ 
	\item \hl{C}. $(1,2)$ 
	\item D. $(2,+\infty)$ 
\end{itemize}
\myspace{1}
\begin{solution}
	$$\lim\limits_{x\rightarrow 0^{-}}\frac{-2x\ln(1-x)}{2ex}=0,\ \lim\limits_{x\rightarrow 0^{+}}\frac{2x\ln(1-x)}{2ex}=0$$
	$$\lim\limits_{x\rightarrow 1^{-}}\frac{2e^{\frac{1}{x-1}}\ln(1-x)}{1-x}=0,\ \lim\limits_{x\rightarrow 1^{+}}\frac{2e^{\frac{1}{x-1}}\ln(x-1)}{1-x}=\infty$$
	$$\lim\limits_{x\rightarrow 2}\frac{2e\ln(x-1)}{x-2}=2e$$
	$$\lim\limits_{x\rightarrow +\infty}\frac{2\ln(x-1)}{(x-1)(x-2)}=0,\ \lim\limits_{x\rightarrow -\infty}\frac{-2\ln(1-)}{(x-1)(x-2)}=0$$
	
	答案:$C$
\end{solution}

\myspace{1}

3. 计算极限$\lim\limits_{\substack{x\rightarrow 0\\ y\rightarrow 0}}\dfrac{\sqrt{1+x^2+y^2}-1}{x^2+y^2}$和$\lim\limits_{\substack{x\rightarrow +\infty\\ y\rightarrow 1}}\left( 1+\dfrac{1}{xy}\right)^{\frac{y^2}{\sin\frac{2}{x}}} $
\myspace{1}
\begin{solution}
	$$I_{1}=\lim\limits_{\substack{x\rightarrow 0\\ y\rightarrow 0}}\dfrac{1+x^2+y^2-1}{(x^2+y^2)(\sqrt{1+x^2+y^2}+1)}=\lim\limits_{\substack{x\rightarrow 0\\ y\rightarrow 0}}\dfrac{1}{\sqrt{1+x^2+y^2}+1}=\frac{1}{2}$$
	
	$$I_{2}=\lim\limits_{\substack{x\rightarrow +\infty\\ y\rightarrow 1}}e^{\frac{y^2}{\sin\frac{2}{x}}}\ln(1+\frac{1}{xy})$$
	$$I_{2}=\lim\limits_{\substack{x\rightarrow +\infty\\ y\rightarrow 1}}e^{\frac{xy^2}{2}\frac{1}{xy}}=e^{\frac{1}{2}}$$
\end{solution}

\myspace{1}

4. $\int_{0}^{\pi}\left( e^{-\cos x}-e^{\cos x}\right)dx $
\myspace{1}
\begin{solution}
	
	区间再现
	
	原积分等价于: 
	$$I=\int_{0}^{\pi}\left( e^{\cos x}-e^{-\cos x}\right)dx$$
	$$2I=0\Rightarrow I=0$$
\end{solution}

\myspace{1}

5. $\int_{-\frac{\pi}{2}}^{\frac{\pi}{2}}\left( \arctan e^{x}\right)\sin^{2}xdx $
\myspace{1}
\begin{solution}
	
	区间再现
	
	原积分等价于: 
	$$I=\int_{-\frac{\pi}{2}}^{\frac{\pi}{2}}(\frac{\pi}{2}-\arctan e^{x})\sin^{2}xdx$$
	$$2I=\frac{\pi}{2}\int_{-\frac{\pi}{2}}^{\frac{\pi}{2}}\sin^{2}xdx=\frac{\pi^{2}}{4}$$
	$$I=\int_{-\frac{\pi}{2}}^{\frac{\pi}{2}}\left( \arctan e^{x}\right)\sin^{2}xdx=\frac{\pi^{2}}{8}$$
\end{solution}

\myspace{1}

6. 设$f(x)=\int_{0}^{x}f(x-t)\sin tdt+x$,求$f(x)$
\myspace{1}
\begin{solution}
	
	我们有: $\int_{0}^{x}f(x-t)\sin tdt=\sin x\int_{0}^{x}f(t)\cos tdt-\cos x\int_{0}^{x}f(t)\sin tdt$
	
	原微分方程等价于: 
	$$f(x)=\sin x\int_{0}^{x}f(t)\cos tdt-\cos x\int_{0}^{x}f(t)\sin tdt+x$$
	
	两边对$x$求导得到: 
	$$f'(x)=1+\cos x\int_{0}^{x}f(t)\cos tdt+\sin x\int_{0}^{x}f(t)\sin tdt$$
	
	再次两边对$x$求导得到: 
	$$f''(x)=f(x)+\cos x\int_{0}^{x}f(t)\sin tdt-\sin x\int_{0}^{x}f(t)\cos tdt\Rightarrow f''(x)=f(x)+x-f(x)$$
	
	$$f''(x)=x\Rightarrow f'(x)=\frac{1}{2}x^2+C_{1}\Rightarrow f(x)=\frac{1}{6}x^3+C_{1}x+C_{2}$$
	
	我们有: $f(0)=0,\ f'(0)=1\Rightarrow C_{1}=1,\ C_{2}=0$
	
	$$f(x)=\frac{1}{6}x^3+x$$
\end{solution}

\myspace{1}

\hl{\textbf{\textit{May 28}}}

1. 下列函数在区间$(0,1)$内无界的是: 
\begin{itemize}
	\item A. $\int_{0}^{x}\dfrac{1}{t^2}e^{-\frac{1}{t}}dt$ 
	\item B. $\int_{0}^{x}\dfrac{\sin t}{t}dt$ 
	\item \hl{C}. $\int_{0}^{x}\dfrac{1}{\sqrt{(1-t)^3}}dt$ 
	\item D. $\int_{0}^{x}\dfrac{1}{(t-1)^2}\sin\dfrac{1}{t-1}dt$ 
\end{itemize}
\myspace{1}
\begin{solution}
	
	$A \quad f(x)=\int_{0}^{x}\frac{1}{t^2}e^{-\frac{1}{t}}dt=e^{-\frac{1}{x}}$
	
	$B \quad \lim\limits_{x\rightarrow 0}\dfrac{\sin x}{x}=1$
	
	$C \quad f(x)=\int_{0}^{x}\dfrac{1}{\sqrt{(1-t)^3}}dt=\dfrac{2}{\sqrt{1-x}}-2, \ f(1)\rightarrow+\infty$
	
	$D \quad f(x)=\int_{0}^{x}\dfrac{1}{(t-1)^2}\sin\dfrac{1}{t-1}dt=\cos \dfrac{1}{x-1}+\dfrac{\pi}{2}\  \text{有界}$
	
	答案:$C$
\end{solution}

\myspace{1}

\hl{\textbf{\textit{May 29}}}

1. $I=\iint\limits_{D}\dfrac{x^3\sin y\cos y e^{\sqrt{x^2+2}}}{\sqrt{x^2\cos^2y+2}\sqrt{x^2+2}}dxdy,\ \text{其中}D=\{(x,y)|\ 0\leq x\leq 1,\ 0\leq y\leq \dfrac{\pi}{2}\}$
\myspace{1}
\begin{solution}
	
	原二重积分可以化为: 
	\begin{eqnarray*}
		I&=&\iint\limits_{D'}\dfrac{r^2\sin \theta\cos \theta e^{\sqrt{r^2+2}}}{\sqrt{r^2\cos^2\theta+2}\sqrt{r^2+2}}rdrd\theta,\ \text{其中}D'=\{(r,\theta)|\ 0\leq r\leq 1,\ \theta\in(0,\dfrac{\pi}{2})\}\\
		&=&\iint\limits_{D''}\dfrac{xye^{\sqrt{x^2+y^2+2}}}{\sqrt{x^2+2}\sqrt{x^2+y^2+2}}dxdy\\
		&=&\int_{0}^{1}\dfrac{x}{\sqrt{x^2+2}}dx\int_{0}^{\sqrt{1-x^2}}\dfrac{ye^{\sqrt{x^2+y^2+2}}}{\sqrt{x^2+y^2+2}}dy\\
		&=&e^{\sqrt{3}}\int_{0}^{1}\dfrac{x}{\sqrt{x^2+2}}dx-\int_{0}^{1}\dfrac{xe^{\sqrt{x^2+2}}}{\sqrt{x^2+2}}dx\\
		&=&e^{\sqrt{3}}(\sqrt{3}-\sqrt{2})-e^{\sqrt{3}}+e^{\sqrt{2}}
	\end{eqnarray*}
\end{solution}

\myspace{1}

2. 已知函数$f(x)=\dfrac{\int_{0}^{x}\ln(1+t^2)dt}{x^{\alpha}}$在$(0,+\infty)$上有界,则$\alpha$的取值范围为: 

\begin{itemize}
	\item A. $(0,+\infty)$ 
	\item B. $(0,3]$ 
	\item C. $(0,2)$ 
	\item \hl{D}. $(1,3]$ 
\end{itemize}
\myspace{1}
\begin{solution}
	
	$f(x)$在$(0,+\infty)$上有界,我们可以得到: 
	$$\lim\limits_{x\rightarrow +\infty}f(x)\text{和}\lim\limits_{x\rightarrow 0}f(x)\text{存在}$$
	$$\lim\limits_{x\rightarrow +\infty}f(x)=\lim\limits_{x\rightarrow +\infty}\dfrac{\ln(1+x^2)}{\alpha x^{\alpha-1}}=\lim\limits_{x\rightarrow +\infty}\dfrac{2x^{3-\alpha}}{\alpha(\alpha-1)(1+x^2)}$$
	$$\lim\limits_{x\rightarrow 0}f(x)=\lim\limits_{x\rightarrow 0}\dfrac{\ln(1+x^2)}{\alpha x^{\alpha-1}}=\lim\limits_{x\rightarrow 0}\dfrac{2x^{3-\alpha}}{\alpha(\alpha-1)}$$
	
	我们得到: 
	$$\left\lbrace 
	\begin{array}{l}
		3-\alpha\geq 0\\
		3-\alpha\leq 2\\
		\alpha\neq 1
	\end{array}
	\right. \Rightarrow 1<\alpha\leq 3$$
	
	综上所述,答案:$D$
\end{solution}

\myspace{1}

\hl{\textbf{\textit{May 30}}}

1. $\int_{0}^{\frac{\pi}{4}}\dfrac{x}{\cos(\frac{\pi}{4}-x)\cos x}dx$
\myspace{1}
\begin{solution}
	
	区间再现
	
	$$I=\int_{0}^{\frac{\pi}{4}}\dfrac{\frac{\pi}{4}-x}{\cos(\frac{\pi}{4}-x)\cos x}dx$$
	$$2I=\int_{0}^{\frac{\pi}{4}}\dfrac{\frac{\pi}{4}}{\cos(\frac{\pi}{4}-x)\cos x}dx=\frac{\sqrt{2}\pi}{4}\int_{0}^{\frac{\pi}{4}}\frac{dx}{\cos^2x+\sin x\cos x}=\frac{\sqrt{2}\pi}{4}\int_{0}^{\frac{\pi}{4}}\frac{d\tan x}{1+\tan x}=\frac{\ln 2\sqrt{2}\pi}{4}$$
	$$I=\int_{0}^{\frac{\pi}{4}}\dfrac{x}{\cos(\frac{\pi}{4}-x)\cos x}dx=\frac{\ln 2\sqrt{2}\pi}{8}$$
\end{solution}

\myspace{1}

2. $\text{计算二重积分}\iint\limits_{D}\dfrac{1}{xy}dxdy,\ D=\{(r,\theta)|\dfrac{\cos \theta}{4}\leq r\leq \dfrac{\cos\theta}{2}, \dfrac{\sin \theta}{4}\leq r\leq \dfrac{\sin\theta}{2}\}$
\myspace{1}
\begin{solution}
	
	二重积分积分区域如下图所示: 
	$$I=2\iint_{D_{ACB}}\frac{1}{xy}dxdy=2\int_{\arctan \frac{1}{2}}^{\frac{\pi}{4}}d\theta\int_{\frac{\cos \theta}{4}}^{\frac{\sin\theta}{2}}\frac{1}{r\sin\theta\cos\theta}dr$$
	$$I=2\int_{\arctan \frac{1}{2}}^{\frac{\pi}{4}}\frac{\ln2\tan \theta}{\sin\theta\cos\theta}d\theta=2\int_{ \frac{1}{2}}^{1}\frac{\ln2t}{t}dt=\ln^2 2$$
	\begin{figure}[ht]
		\centering
		\begin{tikzpicture}[scale=25]
			\begin{scope}
				\clip (0,0) arc (-90:0:1/8) arc (90:180:1/8);
				\fill[pattern=horizontal lines]
				(0,0) arc (-90:0:1/8) --(1/16,1/16) arc (90:0:1/16);
				\fill[pattern=vertical lines]
				(0,0) arc (180:90:1/8) --(1/16,1/16) arc (0:90:1/16);
			\end{scope}
			\draw [->,thick](-0.05,0)--(0.3,0);\draw [->,thick](0,-0.05)--(0,0.3);
			\node[right]at(0.3,0){$x$};\node[left]at(0,0.3){$y$};
			\draw (0,0) arc (-90:90:1/8) (0,0) arc (-90:90:1/16)
			(0,0) arc (180:0:1/8) (0,0) arc (180:0:1/16) (0,0)--(1/8,1/8);
			\node [below]at(0.069,0.0627){$A$};
			\node [above]at(0.136,0.123){$B$};
			\node [right]at(0.105,0.0541){$C$};
			\node [below]at(0.069,0){$O_{1}$};
			\node [below]at(0.136,0){$O_{2}$};
			\node [left]at(0,0.0627){$O_{3}$};
			\node [left]at(0,0.123){$O_{4}$};
		\end{tikzpicture}
		\caption{二重积分区域示意图}
	\end{figure}
\end{solution}

\myspace{1}

3. $\text{当}n\text{充分大时},a-\dfrac{1}{n}<a_{n}<a+\dfrac{1}{n}\text{是数列}a_{n}\text{收敛于}a\text{的什么条件 ?}$
\begin{itemize}
	\item A. $\text{充分必要条件}$
	\item B. $\text{必要条件但非充分条件}$
	\item \hl{C}. $\text{充分条件但非必要条件}$
	\item D. $\text{既非充分也非必要条件}$
\end{itemize}
\myspace{1}
\begin{solution}
	
	(i).充分性: 
	
	$a-\dfrac{1}{n}<a_{n}<a+\dfrac{1}{n},\text{由夹逼定理得到: }$
	$$\lim\limits_{n\rightarrow +\infty}(a-\frac{1}{n})<\lim\limits_{n\rightarrow +\infty}a_{n}<\lim\limits_{n\rightarrow +\infty}(a+\frac{1}{n})$$
	
	我们有: $\lim\limits_{n\rightarrow +\infty}(a-\frac{1}{n})=\lim\limits_{n\rightarrow +\infty}(a+\frac{1}{n})=a$
	因此: 
	$$\lim\limits_{n\rightarrow +\infty}a_{n}=a\Rightarrow\text{充分性成立}$$
	
	(ii).必要性
	
	$\lim\limits_{n\rightarrow +\infty}a_{n}=a\Rightarrow \forall \varepsilon>0,\ \exists N_{0}>0,\text{当}n>N_{0},|a_{n}-a|<\varepsilon$
	
	我们得到: 
	$$n>N_{0},a-\varepsilon<a_{n}<a+\varepsilon,\text{当}\frac{1}{N_{0}+1}<\varepsilon,\text{我们不能得到}a-\frac{1}{n}<a_{n}<a+\frac{1}{n},\text{必要性不成立}$$
	
	答案:$C$
\end{solution}

\myspace{1}

4. $\iint\limits_{D}|x^2+y^2-\sqrt{2}(x+y)|dxdy,\ D=\{(x,y)|x^2+y^2\leq 4\}$
\myspace{1}
\begin{solution}
	
	原二重积分等价于: 
	$$I=I_{1}+2I_{2}=\iint\limits_{D}[x^2+y^2-\sqrt{2}(x+y)]dxdy+2\iint\limits_{D_{1}}[\sqrt{2}(x+y)-(x^2+y^2)]dxdy$$
	$$I_{1}=\int_{0}^{2\pi}d\theta\int_{0}^{2}r(r^2-\sqrt{2}r(\sin \theta+\cos \theta))dr=\int_{0}^{2\pi}(4-\frac{8\sqrt{2}}{3}(\sin\theta+\cos\theta))d\theta=8\pi$$
	$$I_{2}=\iint\limits_{D_{1}}[\sqrt{2}(x+y)-(x^2+y^2)]dxdy,D_{1}: (x-\frac{\sqrt{2}}{2})^2+(y-\frac{\sqrt{2}}{2})^2\leq 1$$
	
	做变量替换: $\left\lbrace 
	\begin{array}{l}
		x-\frac{\sqrt{2}}{2}=R\cos \alpha\\
		y-\frac{\sqrt{2}}{2}=R\sin \alpha
	\end{array}
	\right. \Rightarrow dxdy=\left| \dfrac{\partial (x,y)}{\partial (R,\alpha)}\right| dRd\alpha=RdRd\alpha$
	
	$$I_{2}=\int_{0}^{2\pi}d\alpha\int_{0}^{1} (1-R^2)RdR=\frac{\pi}{2}$$
	$$I=I_{1}+2I_{2}=9\pi$$
\end{solution}

\myspace{1}

\hl{\textbf{\textit{May 31}}}

1.设$f(x)$可积,则下列结论正确的是: 
\begin{itemize}
	\item A.$\text{如果}\lim\limits_{n\rightarrow +\infty}x_{n}=0,\text{且}\lim\limits_{n\rightarrow +\infty}f(x_{n})=A,\text{则}\lim\limits_{x\rightarrow 0}f(x)=A$
	\item B.$\text{如果}\lim\limits_{n\rightarrow +\infty}x_{n}=0,\text{且}\lim\limits_{x\rightarrow 0}f(x)=A,\text{则}\lim\limits_{n\rightarrow +\infty}f(x_{n})=A$
	\item C.$\text{如果}\lim\limits_{n\rightarrow +\infty}f(n)=A,\text{则}\lim\limits_{x\rightarrow +\infty}f(x)=A$
	\item \hl{D}.$\text{如果}\lim\limits_{n\rightarrow +\infty}\int_{0}^{n}\dfrac{|f(t)|}{1+t^2}dt=A,\text{则}\lim\limits_{x\rightarrow +\infty}\int_{0}^{x}\dfrac{|f(t)|}{1+t^2}dt=A$
\end{itemize}
\myspace{1}
\begin{solution}
	
	针对于此题,我们需要区分几个概念: 
	
	$$\lim\limits_{x\rightarrow x_{0}}f(x)=A, \text{只要求在}x\text{邻域内有定义,不要求}x\text{处有定义}$$
	$$\lim\limits_{n\rightarrow+\infty}f(n)=A,f(x)\text{单调},\Rightarrow \lim\limits_{x\rightarrow+\infty}f(x)=A$$
	
	对于$A$,我们只能得到在$x=0$的邻域内的一组特殊的离散点满足极限的定义,其余点未知,我们举一个反例,$x_{n}=\dfrac{1}{n\pi},f(x)=\sin\dfrac{1}{x}$
	
	对于$B$,我们举出一个反例$f(x)=\left\lbrace 
	\begin{array}{l}
		\dfrac{\sin x}{x},x\neq 0\\
		2
	\end{array}
	\right. $,$x_{n}=\left\lbrace 
	\begin{array}{l}
		\dfrac{1}{n},n\text{为奇数}\\
		0,n\text{为偶数}
	\end{array}
	\right. $
	
	对于$C$,不清楚$f(x)$单调性,举一个反例,$f(x)=\sin \pi x$
	
	对于$D$,利用积分的几何意义知道$g(x)=\int_{0}^{x}\dfrac{|f(t)|}{1+t^2}dt$单调递增
	
	答案:$D$.
\end{solution}

\myspace{1}

2. 设$f(u,v)$具有二阶连续偏导数,且满足$\dfrac{\partial^2 f}{\partial u^2}+\dfrac{\partial^2 f}{\partial v^2}=1$,又$g(x,y)=f(xy,\frac{1}{2}(x^2-y^2))$,求$\dfrac{\partial^2 g}{\partial x^2}+\dfrac{\partial^2 g}{\partial y^2}$
\myspace{1}
\begin{solution}
	
	我们令$\left\lbrace 
	\begin{array}{l}
		u=xy\\
		v=\dfrac{1}{2}(x^2-y^2)
	\end{array}
	\right. $
	
	我们有: 
	
	$$\left\lbrace 
	\begin{array}{l}
		\dfrac{\partial g}{\partial x}=\dfrac{\partial f}{\partial u}\dfrac{\partial u}{\partial x}+\dfrac{\partial f}{\partial v}\dfrac{\partial v}{\partial x}=y\dfrac{\partial f}{\partial u}+x\dfrac{\partial f}{\partial v}\\
		
		\\
		\dfrac{\partial g}{\partial y}=\dfrac{\partial f}{\partial u}\dfrac{\partial u}{\partial y}+\dfrac{\partial f}{\partial v}\dfrac{\partial v}{\partial y}=x\dfrac{\partial f}{\partial u}-y\dfrac{\partial f}{\partial v}
	\end{array}
	\right. $$
	$$\left\lbrace 
	\begin{array}{l}
		\dfrac{\partial^2 g}{\partial x^2}=y(y\dfrac{\partial^2 f}{\partial u^2}+x\dfrac{\partial^2 f}{\partial u\partial v})+\dfrac{\partial f}{\partial v}+x(y\dfrac{\partial^2 f}{\partial u\partial v}+x\dfrac{\partial^2 f}{\partial v^2})\\
		
		\\
		\dfrac{\partial^2 g}{\partial y^2}=x(x\dfrac{\partial^2 f}{\partial u^2}-y\dfrac{\partial^2 f}{\partial u\partial v})-\dfrac{\partial f}{\partial v}-y(x\dfrac{\partial^2 f}{\partial u\partial v}-y\dfrac{\partial^2 f}{\partial v^2})
	\end{array}
	\right. $$
	
	我们得到: 
	$$\dfrac{\partial^2 g}{\partial x^2}+\dfrac{\partial^2 g}{\partial y^2}=(x^2+y^2)\dfrac{\partial^2 f}{\partial u^2}+\dfrac{\partial^2 f}{\partial v^2}=x^2+y^2$$
\end{solution}

\myspace{1}

3. $\int_{-\pi}^{\pi}\dfrac{x\sin x\arctan e^{x}}{1+\cos ^2 x}dx$
\myspace{1}
\begin{solution}
	
	区间再现
	
	$$I=\int_{-\pi}^{\pi}\dfrac{x\sin x(\dfrac{\pi}{2}-\arctan e^{x})}{1+\cos ^2 x}dx$$
	$$2I=\dfrac{\pi}{2}\int_{-\pi}^{\pi}\dfrac{x\sin x}{1+\cos ^2 x}dx=\pi\int_{0}^{\pi}\dfrac{x\sin x}{1+\cos ^2 x}dx$$
	
	令$J=\int_{0}^{\pi}\dfrac{x\sin x}{1+\cos ^2 x}dx$
	
	$$J=\int_{0}^{\pi}\dfrac{(\pi-x)\sin x}{1+\cos ^2 x}dx\Rightarrow 2J=\pi\int_{0}^{\pi}\dfrac{\sin x}{1+\cos ^2 x}dx=\dfrac{\pi^2}{2}$$
	
	$$J=\dfrac{\pi^2}{4},I=\dfrac{\pi}{2}J=\dfrac{\pi^3}{8}$$
\end{solution}

\myspace{1}

4. 求$\iint\limits_{D}\dfrac{1}{3x^2+y^2}dxdy,D\text{是由}x^2+y^2-xy=1,\ x^2+y^2-xy=2\text{和直线}y=\sqrt{3}x,\ y=0\text{围成}$
\myspace{1}
\begin{solution}
	
	原二重积分等价于: 
	$$I=\int_{0}^{\frac{\pi}{3}}d\theta\int_{r_{1}}^{r_{2}}\dfrac{r}{3r^2\cos ^2\theta+r^2\sin^2\theta}dr,\text{其中}\left\lbrace 
	\begin{array}{l}
		r_{1}=\sqrt{\dfrac{1}{1-\sin \theta\cos\theta}}\\
		r_{2}=\sqrt{\dfrac{2}{1-\sin \theta\cos\theta}}
	\end{array}
	\right. $$
	
	$$I=\frac{\ln 2}{2}\int_{0}^{\frac{\pi}{3}}\dfrac{1}{\sin^2\theta+3\cos^2\theta}d\theta=\frac{\ln 2}{2}\int_{0}^{\frac{\pi}{3}}\dfrac{1}{3+\tan^2\theta}d\tan \theta$$
	$$I=\frac{\ln 2}{2\sqrt{3}}\arctan(\frac{\tan\theta}{\sqrt{3}})|_{0}^{\sqrt{3}}=\frac{\pi \ln 2}{8\sqrt{3}}$$
\end{solution}

\myspace{1}
