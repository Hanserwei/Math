\chapterimage{chap31.jpg}
\chapter{September}
\section{Week \Rmnum{1}}
\hl{\textbf{\textit{September 1}}}

1. 设$f(x)$在$[0,1]$上二阶可导,且$\lim\limits_{x\rightarrow 0^{+}}\dfrac{f(x)}{x}=2$,$\int_{0}^{1}f(x)dx=1$.

(1). 证明:  $\exists \xi\in(0,1),\ s.t. f'(\xi)=f(\xi)-2\xi+2$

(2). 证明:  $\exists \eta\in(0,1),\ s.t. f''(\eta)=0$

(3). 证明:  $\exists \zeta\in(0,1),\ s.t. \int_{0}^{\zeta}f(t)dt+\zeta f(\zeta)=2\zeta$

(4). 证明:  $\exists \mu\in(0,1),\ s.t. \mu f(\mu)=2\int_{0}^{\mu}f(t)dt$
\myspace{1}
\begin{solution}

	我们由题意可得:  
	$$\left\lbrace
	\begin{array}{l}
		\lim\limits_{x\rightarrow 0^{+}}\dfrac{f(x)}{x}=2\\
		\int_{0}^{1}f(x)dx=1
	\end{array}
	\right. \Rightarrow \left\lbrace
	\begin{array}{l}
		f(0)=0\\
		f'(0)=2\\
		\exists c\in(0,1),\ s.t. f(c)=1
	\end{array}
	\right. $$
	
	(1)我们构造辅助函数:  $F(x)=\dfrac{f(x)-2x}{e^x}$,我们可以得到:  
	$$\left\lbrace
	\begin{array}{l}
		F(0)=0\\
		\int_{0}^{1}[f(x)-2x]=0\\
		F'(x)=e^{-x}[f'(x)-2-f(x)+2x]
	\end{array}
	\right. \Rightarrow \left\lbrace
	\begin{array}{l}
		F(0)=0\\
		\exists c\in(0,1),\ s.t. F(c)=0(\text{积分中值定理})
	\end{array}
	\right. $$
	
	我们对$F(x)$在$(0,c)$上使用罗尔定理可以得到:  
	$$\exists \xi\in(0,c)\subset(0,1),\ s.t. F'(\xi)=0\Rightarrow f'(\xi)=f(\xi)-2\xi+2$$
	
	(2). 我们构造辅助函数:  $G(x)=f(x)-2x$,我们可以得到:  
	$$\left\lbrace
	\begin{array}{l}
		G(0)=0\\
		G'(0)=0\\
		\int_{0}^{1}G(x)dx=0\Rightarrow \exists a\in(0,1),s.t. G(a)=0
	\end{array}
	\right. \Rightarrow G(0)=G(d)=0$$
	
	我们对$G(x)$在$(0,a)$上使用罗尔定理得到:  
	$$\exists b\in(0,a),s.t. G'(b)=0$$
	
	我们对$G'(x)$在$(0,b)$上使用罗尔定理得到:  
	$$\exists \eta\in(0,b),s.t. G''(\eta)=0\Rightarrow f''(\eta)=0$$
	
	(3). 我们构造辅助函数:  $H(x)=x\int_{0}^{x}f(t)dt-x^2$,我们有:  
	$$\left\lbrace
	\begin{array}{l}
		H(0)=H(1)=0\\
		H'(x)=\int_{0}^{x}f(t)dt+xf(x)-2x
	\end{array}
	\right. $$
	
	我们对$H(x)$在$(0,1)$上使用罗尔定理可以得到:  
	$$\exists \zeta\in(0,1),\ s.t. H'(\zeta)=0\Rightarrow \int_{0}^{\zeta}f(t)dt+\zeta f(\zeta)=2\zeta$$
	
	(4). 我们构造辅助函数:  $P(x)=\left\lbrace
	\begin{array}{l}
		\dfrac{\int_{0}^{x}f(t)dt}{x^2},\ x\in(0,1]\\
		1,x=0
	\end{array}
	\right. \Rightarrow \lim\limits_{x\rightarrow 0^{+}}P(x)=1$
	
	我们可以得到:  $P(x)$在$[0,1]$上连续,$(0,1)$上可导,我们还可以得到:  
	$$\left\lbrace
	\begin{array}{l}
		P(0)=P(1)=1\\
		P'(x)=\dfrac{xf(x)-2\int_{0}^{x}f(t)dt}{x^3}
	\end{array}
	\right. $$
	
	我们对$P(x)$在$(0,1)$上使用罗尔定理可以得到:  
	$$\exists \mu\in(0,1),\ s.t. P'(\mu)=0\Rightarrow \mu f(\mu)=2\int_{0}^{\mu}f(t)dt$$
\end{solution}
\myspace{1}

2. 设$I=\int_{0}^{\sqrt{\pi}}x\cos x^2dx$,$J=\int_{0}^{\sqrt{\pi}}\cos x^2dx$,$K=\int_{0}^{\pi}\sqrt{x}\cos xdx$,比较$I,J,K$的大小
\myspace{1}
\begin{solution}

	我们有:  $I=\int_{0}^{\sqrt{\pi}}x\cos x^2dx=\dfrac{\sin x^2}{2}|_{0}^{\sqrt{\pi}}=0$
	
	$$J=\int_{0}^{\sqrt{\pi}}\cos x^2dx=\int_{0}^{\pi}\dfrac{\cos t}{2\sqrt{t}}dt>0$$
	
	$$K=\int_{0}^{\pi}\sqrt{x}\cos xdx<0$$
	
	综上所述,我们可以得到:  $K<I<J$
\end{solution}
\myspace{1}

3. 已知曲线$L: y=x^2-1(-1\leq x\leq 2)$,方向从$A(-1,0)$到$B(2,3)$,求曲线积分$\int_{L}\dfrac{xdy-ydx}{x^2+y^2}$
\myspace{1}
\begin{solution}

	原曲线积分可化为:  
	\begin{eqnarray*}
		\int_{L}\dfrac{xdy-ydx}{x^2+y^2}&=&\int_{-1}^{2}\dfrac{x^2+1}{x^2+(x^2-1)^2}dx\\
		&=&\int_{-1}^{2}\dfrac{x^2+1}{x^4-x^2+1}dx\\
		&=&\int_{-1}^{2}\dfrac{1+\frac{1}{x^2}}{x^2+\frac{1}{x^2}-1}dx\\
		&=&\int_{-1}^{2}\dfrac{1}{1+(x-\frac{1}{x})^2}d(x-\frac{1}{x})\\
		&=&\int_{-1}^{0^{-}}\dfrac{1}{1+(x-\frac{1}{x})^2}d(x-\frac{1}{x})+\int_{0^{+}}^{2}\dfrac{1}{1+(x-\frac{1}{x})^2}d(x-\frac{1}{x})\\
		&=&\arctan(x-\frac{1}{x})|_{x=-1}^{x=0^{-}}+\arctan(x-\frac{1}{x})|_{x=0^{+}}^{x=2}\\
		&=&\arctan\dfrac{3}{2}+\pi
	\end{eqnarray*}
\end{solution}
\myspace{1}

4. 设$f(x)=\lim\limits_{n\rightarrow+\infty}\sqrt[n]{1+x^n+\left( \dfrac{x^2}{2}\right) ^n}(x>0)$,求$\int f(x)dx$
\myspace{1}

\begin{solution}

	我们可以得到:  
	$$f(x)=\left\lbrace
	\begin{array}{l}
		1,x\in(0,1]\\
		x,x\in(1,2]\\
		\dfrac{x^2}{2},x\in(2,+\infty)
	\end{array}
	\right. \Rightarrow \int f(x)=\left\lbrace
	\begin{array}{l}
		x+C,\ x\in(0,1]\\
		\dfrac{x^2+1}{2}+C,\ x\in(1,2]\\
		\dfrac{x^3+7}{6}+C,\ x\in(2,+\infty)
	\end{array}
	\right. $$
\end{solution}
\myspace{1}

\hl{\textbf{\textit{September 2}}}

1. 设$f(x)$连续,$f(x+2)-f(x)=\sin x$,$\int_{0}^{2}f(x)dx=0$,求$\int_{1}^{3}f(x)dx$
\myspace{1}
\begin{solution}

	我们可以得到:  
	\begin{eqnarray*}
		\int_{1}^{3}f(x)dx&=&\int_{1}^{3}f(x)dx-\int_{0}^{2}f(x)dx\\
		&=&\int_{2}^{3}f(x)dx-\int_{0}^{1}f(x)dx\\
		&=&\int_{0}^{1}f(x+2)dx-\int_{0}^{1}f(x)dx\\
		&=&\int_{0}^{1}[f(x+2)-f(x)]dx\\
		&=&\int_{0}^{1}\sin xdx\\
		&=&1-\cos 1
	\end{eqnarray*}
	\begin{anymark}[注]
	我们构造辅助函数:  $F(x)=\int_{x}^{x+2}f(t)dt$,我们有:  
	$$\left\lbrace
	\begin{array}{l}
		F'(x)=f(x+2)-f(x)=\sin x\\
		F(0)=0
	\end{array}
	\right. \Rightarrow \left\lbrace
	\begin{array}{l}
		F(x)=-\cos x+1\\
		F(1)=\int_{1}^{3}f(x)dx=1-\cos 1
	\end{array}
	\right.$$
	\end{anymark}
\end{solution}
\myspace{1}

2. 设$f(x)$在$[0,1]$上连续,在$(0,1)$内可导,且$f(0)=1,f(1)=0$

(1). 证明:  $\exists \xi_{1},\xi_{2}\in(0,1)(\xi_{1}\neq \xi_{2}),\ s.t.\ f'(\xi_{1})+f'(\xi_{2})=-2$

(2). 证明:  $\exists \eta,\zeta\in(0,1)(\eta\neq \zeta),\ s.t.\ f'(\eta)f'(\zeta)=1$
\myspace{1}
\begin{solution}

	(1). 我们取$c=f(\frac{1}{2})$,我们在$(0,\dfrac{1}{2})$和$(\dfrac{1}{2},1)$上分别对$f(x)$使用拉格朗日中值定理可以得到:  
	$$\left\lbrace
	\begin{array}{l}
		\exists \xi_{1}\in(0,\frac{1}{2}),\ s.t. \dfrac{c-1}{\frac{1}{2}}=f'(\xi_{1})\\
		\exists \xi_{2}\in(\frac{1}{2},1),\ s.t. \dfrac{-c}{\frac{1}{2}}=f'(\xi_{2})
	\end{array}
	\right.\Rightarrow f(\xi_{1})+f(\xi_{2})=-2$$
	
	(2). 我们构造辅助函数:  $F(x)=f(x)-x$,我们有:  
	$$\left\lbrace
	\begin{array}{l}
		F(0)=f(0)=1>0\\
		F(1)=f(1)-1=-1<0
	\end{array}
	\right. \Rightarrow \text{根据零点定理:  }\exists c\in(0,1),\ s.t. F(c)=f(c)-c=0$$
	
	我们对$f(x)$分别在$(0,c)$和$(c,1)$上使用拉格朗日中值定理得到:  
	$$\left\lbrace
	\begin{array}{l}
		\exists\eta\in(0,c),s.t. \dfrac{f(c)-1}{c}=f'(\eta)\\
		\exists\zeta\in(c,1),s.t. \dfrac{-f(c)}{1-c}=f'(\zeta)
	\end{array}
	\right. \Rightarrow f'(\eta)f'(\zeta)=\dfrac{c-1}{c}\cdot\dfrac{-c}{1-c}=1$$
\end{solution}
\myspace{1}

3. 设$f(x)=\int_{-1}^{x}t\cos tdt,x\in(-\frac{\pi}{2},\frac{\pi}{2})$,则曲线$y=f(x)$与$x$轴所围成的图形面积为:  
\begin{itemize}
	\item A. $2\int_{0}^{1}x\sin xdx$
	\item B. $2\int_{0}^{1}x^2\sin xdx$
	\item C. $2\int_{0}^{1}x\cos xdx$
	\item \hl{D}. $2\int_{0}^{1}x^2\cos xdx$
\end{itemize}
\myspace{1}
\begin{solution}

	我们可以得到:  $f(x)$为偶函数,且$f'(x)=x\cos x$,我们可以得到:  
	$$\left\lbrace
	\begin{array}{l}
		x\in(-\frac{\pi}{2}),f'(x)<0\\
		x\in(0,\frac{\pi}{2}),f'(x)>0
	\end{array}
	\right. \Rightarrow f(x)\text{在}(-\frac{\pi}{2},0)\text{上单调递减},\text{在}(0,\frac{\pi}{2})\text{上单调递增}$$
	
	我们有:  $f(1)=f(-1)=0$,我们得到:  
	\begin{eqnarray*}
		S&=&-2\int_{0}^{1}f(x)dx\\
		&=&2\int_{0}^{1}xf'(x)dx\\
		&=&2\int_{0}^{1}x^2\cos xdx
	\end{eqnarray*}
\end{solution}
\myspace{1}

4. 设$\varGamma$是柱面$x^2+y^2=1$与平面$z=x+y$的交线,从$z$轴正向往负向看去为逆时针,计算积分$\oint_{\varGamma}xzdx+xdy+\frac{y^2}{2}dz$
\myspace{1}
\begin{solution}

	原曲线积分可以化为:  
	\begin{eqnarray*}
		I&=&\int_{0}^{2\pi}\left[ -\sin\theta\cos\theta(\cos\theta+\sin\theta)+\cos^2\theta+\dfrac{\sin^2 \theta}{2}(\cos\theta-\sin\theta)\right]d\theta\\
		&=&\int_{0}^{2\pi}\cos^2\theta d\theta+\dfrac{1}{2}\int_{0}^{2\pi}\sin^2\theta\cos\theta d\theta-\dfrac{1}{2}\int_{0}^{2\pi}\sin^3\theta d\theta\\
		&=&\pi
	\end{eqnarray*}
\end{solution}
\myspace{1}

5. $\int\dfrac{1}{\sin^3 x+\cos^3 x}dx$
\myspace{1}
\begin{solution}

	原不定积分可以化为:  
	\begin{eqnarray*}
		I&=&\int \dfrac{1}{(\sin x+\cos x)(\sin^2 x+\cos^2 x-\sin x\cos x)}dx\\
		&=&\int \dfrac{d(\sin x-\cos x)}{(1+2\sin x+\cos x)(\sin^2 x+\cos^2 x-\sin x\cos x)}\\
		&=&\int \dfrac{2d(\sin x-\cos x)}{[2-(\sin x-\cos x)^2][1+(\sin x-\cos x)^2]}\\
		&=&\int \dfrac{2du}{(2-u^2)(1+u^2)}\\
		&=&\dfrac{2}{3}[\int\dfrac{1}{2-u^2}du+\int\dfrac{1}{1+u^2}du]\\
		&=&\dfrac{2}{3}[\dfrac{1}{2\sqrt{2}}\ln|\dfrac{\sqrt{2}+u}{\sqrt{2}-u}|+\arctan u]\\
		&=&\dfrac{1}{3\sqrt{2}}\ln|\dfrac{1+\sin(x-\frac{\pi}{4})}{1-\sin(x-\frac{\pi}{4})}|+\dfrac{2}{3}\arctan(\sin x-\cos x)+C
	\end{eqnarray*}
\end{solution}
\myspace{1}

6. 设$F(x)=\int_{x}^{x+2\pi}e^{\sin t}\sin tdt$,求$F(x)$
\myspace{1}
\begin{solution}

	我们不妨设$f(x)=e^{\sin x}\sin x$,$f(x)$为周期函数,我们可以得到:  
	\begin{eqnarray*}
		F(x)&=&\int_{x}^{x+2\pi}e^{\sin t}\sin tdt\\
		&=&\int_{0}^{2\pi}e^{\sin x}\sin xdx\\
		&=&\int_{0}^{\pi}e^{\sin x}\sin xdx+\int_{\pi}^{2\pi}e^{\sin x}\sin xdx\\
		&=&\int_{0}^{\pi}e^{\sin x}\sin xdx-\int_{0}^{\pi}e^{-\sin x}\sin xdx\\
		&=&\int_{0}^{\pi}(e^{\sin x}-e^{-\sin x})\sin xdx>0\\
	\end{eqnarray*}

	我们可以得到:  $F(x)=C>0$
\end{solution}
\myspace{1}

\hl{\textbf{\textit{September 3}}}

1. 设$f(x)$在$[0,1]$上连续,且$I=\int_{0}^{1}f(x)dx\neq 0$,证明:  $\exists \xi,\eta\in(0,1)(\xi\neq \eta),s.t. \dfrac{1}{f(\xi)}+\dfrac{1}{f(\eta)}=\dfrac{2}{I}$
\myspace{1}
\begin{solution}

	我们构造辅助函数:  $F(x)=\int_{0}^{x}f(t)dt$,$F(0)=0,F(1)=I\neq 0$
	
	我们由介值定理可以得到:  
	$$\exists c\in(0,1),s.t. \ F(c)=\dfrac{I}{2}$$
	
	我们对$F(x)$在区间$(0,c)$和$(c,1)$上使用拉格朗日中值定理得到:  
	$$\left\lbrace
	\begin{array}{l}
		\exists\xi\in(0,c),\ s.t. \dfrac{F(c)}{c}=F'(\xi)=f(\xi)\\
		\exists\eta\in(c,1).\ s.t. \dfrac{F(1)-F(c)}{1-c}=F'(\eta)=f(\eta)
	\end{array}
	\right. \Rightarrow \dfrac{1}{f(\xi)}+\dfrac{1}{f(\eta)}=\dfrac{2c}{I}+\dfrac{2(1-c)}{I}=\dfrac{I}{2}$$
	
	综上所述,我们得到:  $\exists \xi,\eta\in(0,1)(\xi\neq \eta),s.t. \dfrac{1}{f(\xi)}+\dfrac{1}{f(\eta)}=\dfrac{2}{I}$
\end{solution}
\myspace{1}

2. 设函数$f(x)$在$(-\infty,+\infty)$内有定义,在$(-\infty,0)\cup(0,+\infty)$内可导,$\lim\limits_{x\rightarrow 0}f'(x)$存在,$f(x)$在$x=0$处可导的充分条件为:  
\begin{itemize}
	\item A $\lim\limits_{x\rightarrow 0}\dfrac{f(x)}{x}$存在
	\item B $\lim\limits_{x\rightarrow 0}\dfrac{f'(x)}{x}$存在
	\item \hl{C} $f(x)$在点$x=0$处连续
	\item D $\int_{0}^{x}f(t)dt$在点$x=0$处可导
\end{itemize}
\myspace{1}
\begin{solution}

	我们可以得到:  
	$$\left\lbrace
	\begin{array}{l}
		f(x)\text{在}(-\infty,0)\cup(0,+\infty)\text{上可导}\\
		\lim\limits_{x\rightarrow 0}f'(x)\text{存在}
	\end{array}
	\right. $$
	
	(1). $\lim\limits_{x\rightarrow 0}\dfrac{f(x)}{x}$存在$\Rightarrow \lim\limits_{x\rightarrow 0}f(x)$存在
	
	(2). $\lim\limits_{x\rightarrow 0}\dfrac{f'(x)}{x}$存在$\Rightarrow \lim\limits_{x\rightarrow 0}f'(x)$存在
	
	(3). $f(x)$在点$x=0$处连续$\Rightarrow \lim\limits_{x\rightarrow 0}f(x)=f(0)$
	$$\left\lbrace
	\begin{array}{l}
		\lim\limits_{x\rightarrow 0}f'(x)=k\\
		\lim\limits_{x\rightarrow 0}\dfrac{f(x)-f(0)}{x}=\lim\limits_{x\rightarrow 0}f'(x)
	\end{array}
	\right. \Rightarrow f'(0)=\lim\limits_{x\rightarrow 0}f'(x)=k$$
	
	(4). $\int_{0}^{x}f(t)dt$在点$x=0$处可导$\Rightarrow f(0)$存在
\end{solution}
\myspace{1}

\hl{\textbf{\textit{September 4}}}

1. 求不定积分$\int x\arctan x\cdot \ln(1+x^2)dx$
\myspace{1}
\begin{solution}

	原不定积分可化为:  
	\begin{eqnarray*}
		I&=&\int x\arctan x\cdot \ln(1+x^2)dx\\
		&=&\dfrac{1}{2}\int \arctan x\cdot \ln(1+x^2)d(x^2+1)\\
		&=&\dfrac{1}{2}(x^2+1)\arctan x\ln(x^2+1)-\dfrac{1}{2}\int \ln(x^2+1)dx-\int x\arctan xdx\\
		&=&\dfrac{1}{2}(x^2+1)\arctan x\ln(x^2+1)-\dfrac{1}{2}x\ln(x^2+1)+x-\arctan x-\dfrac{1}{2}x^2\arctan x+\dfrac{1}{2}x-\dfrac{1}{2}\arctan x\\
		&=&\dfrac{(x^2+1)\ln(x^2+1)-x^2-3}{2}\arctan x-\dfrac{x\ln(x^2+1)}{2}+\dfrac{3}{2}x+C
	\end{eqnarray*}
\end{solution}
\myspace{1}

2. 设$\varGamma:=\left\lbrace
\begin{array}{l}
	x=2\sqrt{1-y^2}\\
	z=x+y
\end{array}
\right. $,从$z$轴正向往负向看去为逆时针,计算积分$\int\limits_{\varGamma}\dfrac{ydx+zdy+xdz}{x^2+y^2+z^2}$
\myspace{1}
\begin{solution}

	原曲线积分可化为:  
	\begin{eqnarray*}
		I&=&\int_{-\frac{\pi}{2}}^{\frac{\pi}{2}}\dfrac{4\cos^2\theta-2\sin^2\theta-3\cos\theta\sin\theta}{2\sin^2\theta+8\cos^2\theta+4\cos\theta\sin\theta}d\theta\\
		&=&\int_{-\frac{\pi}{2}}^{\frac{\pi}{2}}\dfrac{1+3\cos 2\theta-\frac{3}{2}\sin 2\theta}{3\cos 2\theta+2\sin\theta+5}d\theta\\
		&=&\dfrac{1}{2}\int_{-\pi}^{\pi}\dfrac{1+3\cos x-\frac{3}{2}\sin x}{3\cos x+2\sin x+5}dx\\
		&=&\dfrac{1}{2}\left[ \int_{-\pi}^{\pi}\dfrac{6}{13}dx+\dfrac{11}{26}\int_{-\pi}^{\pi}\dfrac{2\cos x-3\sin x}{3\cos x+2\sin x+5}-\dfrac{17}{13}\int_{-\pi}^{\pi}\dfrac{1}{3\cos x+2\sin x+5}dx\right]\\
		&=&\dfrac{6}{13}\pi+0-\dfrac{17\sqrt{3}\pi}{39}\\
		&=&\dfrac{(36-17\sqrt{3})\pi}{78}
	\end{eqnarray*}
\end{solution}
\myspace{1}

3. 设函数$f(x)$在$(-\infty,+\infty)$上是以$T$为最小正周期的连续奇函数,下列函数中不是周期函数的个数:  
\begin{itemize}
	\item (1). $\int_{a}^{x}f(t)dt$
	\item (2). $\int_{-x}^{a}f(t)dt$
	\item (3). $\int_{-x}^{x}tf(t)dt$
	\item (4). $\int_{-x}^{x}t^2f(t)dt$
\end{itemize}
\myspace{1}
\begin{solution}

	$f(x)$是周期函数,且为奇函数$\Rightarrow \int_{-\frac{T}{2}}^{\frac{T}{2}}f(x)dx=0$
	
	(1). 我们不妨设$F(x)=\int_{a}^{x}f(t)dt$,我们有:  
	$$\left\lbrace
	\begin{array}{l}
		F(x+T)=\int_{a}^{x+T}f(t)dt\\
		F(x+T)-F(x)=\int_{x}^{x+T}f(t)dt=\int_{-\frac{T}{2}}^{\frac{T}{2}}f(x)dx=0
	\end{array}
	\right. \Rightarrow F(x+T)=F(x)\Rightarrow F(x)\text{为周期函数}$$
	
	(2). 我们不妨设$F(x)=\int_{-x}^{a}f(t)dt$,我们有:  
	$$\left\lbrace
	\begin{array}{l}
		F(x+T)=\int_{-x-T}^{a}f(t)dt\\
		F(x+T)-F(x)=\int_{-x-T}^{-x}f(t)dt=\int_{-\frac{T}{2}}^{\frac{T}{2}}f(x)dx=0
	\end{array}
	\right. \Rightarrow F(x+T)=F(x)\Rightarrow F(x)\text{为周期函数}$$
	
	(3). 我们不妨设$F(x)=\int_{-x}^{x}tf(t)dt$,我们有:  
	$$\left\lbrace
	\begin{array}{l}
		F(x+T)=\int_{-x-T}^{x+T}tf(t)dt\\
		F(x+T)-F(x)=2\int_{x}^{x+T}tf(t)dt\neq 0
	\end{array}
	\right. \Rightarrow F(x+T)\neq F(x)\Rightarrow F(x)\text{不是周期函数}$$
	
	(4). $\int_{-x}^{x}t^2f(t)dt\equiv 0\Rightarrow F(x)\text{是周期函数}$
	
	综上所述,上述函数只有$(3)$不是周期函数,$(1)(2)(4)$均为周期函数.
\end{solution}
\myspace{1}

4. 设随机变量$(X,Y)$服从二维正态分布,$X\sim N(1,3^2),Y\sim N(0,4^2)$,且满足$\rho_{XY}=-\dfrac{1}{2}$,$Z=\dfrac{X}{3}+\dfrac{Y}{2}$

(1).求$E(Z)$与$D(Z)$

(2).求$\rho_{XZ}$

(3).证明$X$与$Z$是否独立
\myspace{1}
\begin{solution}
	
\end{solution}
\myspace{1}

\hl{\textbf{\textit{September 5}}}

1.计算二重积分:  $\iint\limits_{D}\dfrac{1}{\sqrt{xy}}dxdy$,$D:\{(x,y)|(\dfrac{x}{2}+\dfrac{y}{4})^2\leq \dfrac{x}{6},x,y\geq 0\}$
\myspace{1}
\begin{solution}
	原二重积分可化为:  
	\begin{eqnarray*}
		I&=&\int_{0}^{\frac{2}{3}}dx\int_{0}^{4\sqrt{\frac{x}{6}}-2x}\dfrac{1}{\sqrt{xy}}dy\\
		&=&2\int_{0}^{\frac{2}{3}}\dfrac{1}{\sqrt{x}}\sqrt{4\sqrt{\frac{x}{6}}-2x}dx\\
		&=&8\sqrt{6}\int_{0}^{\frac{1}{3}}\sqrt{t-3t^2}dt\\
		&=&\dfrac{24\sqrt{2}}{36}\int_{-\frac{\pi}{2}}^{\frac{\pi}{2}}\cos^2\theta d\theta\\
		&=&\dfrac{\sqrt{2}\pi}{3}
	\end{eqnarray*}
\end{solution}
\begin{anymark}[注]
	我们不妨设$\sqrt{x}=m,\sqrt{y}=n$,我们有:  $\dfrac{(m-\frac{1}{\sqrt{6}})^2}{2}+\dfrac{n^2}{4}=\dfrac{1}{12}(m,n>0)$.
	
	我们根据雅可比行列式,得到:  
	$$dxdy=4mndmdn\Rightarrow \iint\limits_{D}\dfrac{1}{\sqrt{xy}}dxdy=\iint\limits_{D'}4dmdn$$
	
	我们得到原二重积分与椭圆面积之间的关系:  $I=2S_{D'}=2ab\pi=2\pi\times\dfrac{1}{\sqrt{6}}\times\dfrac{1}{\sqrt{3}}=\dfrac{\sqrt{2}\pi}{3}$
\end{anymark}
\myspace{1}

2. $\int \dfrac{\sqrt{x-1}\arctan \sqrt{x-1}}{x}dx$
\myspace{1}
\begin{solution}

	我们令$\sqrt{x-1}=t,x=t^2+1,dx=2tdt$,原不定积分可化为:  
	\begin{eqnarray*}
		I&=&\int \dfrac{2t^2\arctan t}{t^2+1}dt\\
		&=&2\int \arctan tdt-2\int\dfrac{\arctan t}{1+t^2}dt\\
		&=&2t\arctan t-\ln|1+t^2|-(\arctan t)^2\\
		&=&2\sqrt{x-1}\arctan \sqrt{x-1}-\ln|x|-(\arctan\sqrt{x-1})^2+C
	\end{eqnarray*}
\end{solution}
\myspace{1}

3. 设函数$f(x)$在$(-\infty,+\infty)$内满足$f(x)=f(x-\pi)+\sin x$,且$f(x)=x,x\in[0,\pi)$,求$\int_{\pi}^{3\pi}f(x)dx$
\myspace{1}
\begin{solution}

	我们由:  $\left\lbrace
	\begin{array}{l}
		f(x)=f(x-\pi)+\sin x\\
		f(x)=x,x\in[0,\pi)
	\end{array}
	\right. $可以得到:  
	$$f(x)=x+\sin x-\pi,x\in[\pi,2\pi)$$
	
	我们有:  
	\begin{eqnarray*}
		\int_{\pi}^{3\pi}f(x)dx&=&\int_{\pi}^{3\pi}[f(x-\pi)+\sin x]dx\\
		&=&\int_{\pi}^{3\pi}f(x-\pi)dx\\
		&=&\int_{0}^{2\pi}f(x)dx\\
		&=&\int_{0}^{\pi}xdx+\int_{\pi}^{2\pi}[x+\sin x-\pi]dx\\
		&=&\pi^2-2
	\end{eqnarray*}
\end{solution}
\myspace{1}

\hl{\textbf{\textit{September 6}}}

1. $\int \dfrac{\cos^3 x-2\cos x}{1+\sin^2 x+\sin^4 x}dx$
\myspace{1}
\begin{solution}

	我们令$t=\sin x$,我们可以得到原不定积分为:  
	\begin{eqnarray*}
		I&=&-\int \dfrac{t^2+1}{1+t^2+t^4}dt\\
		&=&-\int \dfrac{1+\frac{1}{t^2}}{1+t^2+\frac{1}{t^2}}dt\\
		&=&-\int \dfrac{d(t-\frac{1}{t})}{(t-\frac{1}{t})^2+3}\\
		&=&-\dfrac{1}{\sqrt{3}}\arctan(\dfrac{t-\frac{1}{t}}{\sqrt{3}})+C\\
		&=&\dfrac{1}{\sqrt{3}}\arctan(\dfrac{\cos^2 x}{\sqrt{3}\sin x})dx+C
	\end{eqnarray*}
\end{solution}
\myspace{1}

2. 下列积分中,与积分$I=\int_{0}^{1}\dfrac{1}{2}xe^{-\sqrt{x}}dx$值最接近的是:  
\begin{itemize}
	\item A. $\int_{0}^{1}\sqrt{x}e^{-x}dx$
	\item B. $\int_{0}^{1}xe^{-x}dx$
	\item C. $\int_{0}^{1}x^2e^{-x}dx$
	\item \hl{D}. $\int_{0}^{1}x^4e^{-x}dx$
\end{itemize}
\myspace{1}
\begin{solution}

	我们可以得到:  
	\begin{eqnarray*}
		I&=&\int_{0}^{1}\dfrac{1}{2}xe^{-\sqrt{x}}dx\\
		&=&\int_{0}^{1}t^3e^{-t}dt\\
		&=&\int_{0}^{1}x^3e^{-x}dx
	\end{eqnarray*}
	
	我们可以得到:  
	$$\int_{0}^{1}\sqrt{x}e^{-x}dx>\int_{0}^{1}xe^{-x}dx>\int_{0}^{1}x^2e^{-x}dx>\int_{0}^{1}x^3e^{-x}dx>\int_{0}^{1}x^4e^{-x}dx$$
	
	我们只需要比较$x^2-x^3$和$x^3-x^4$的大小,我们有:  $x^2-x^3>x^3-x^4$,我们得到最接近$I$的是$\int_{0}^{1}x^4e^{-x}dx$
\end{solution}
\myspace{1}

3. $f(x)=\dfrac{\left(\sqrt[n]{x}-1\right) ^n}{x+1}$,求$f^{(n)}(1)(n\geq 2)$
\myspace{1}
\begin{solution}

	我们有:  $$\lim\limits_{x\rightarrow 1}\dfrac{f(x)}{(x-1)^n}=\dfrac{1}{2}\lim\limits_{x\rightarrow 1}\dfrac{(\sqrt[n]{x}-1)^n}{(x-1)^n}=\dfrac{1}{2n^{n}}$$
	
	我们利用泰勒展开式:  
	$$\lim\limits_{x\rightarrow 1}\dfrac{f(x)}{(x-1)^n}=\lim\limits_{x\rightarrow 1}\dfrac{\sum\limits_{k=1}^{n}\frac{f^{(k)}(x)}{k!}(x-1)^k+o[(x-1)^n]}{(x-1)^n}$$
	
	上述极限存在,我们可以得到:  $f^{(k)}=0(k=1,2,\cdots,n-1)$
	$$\lim\limits_{x\rightarrow 1}\dfrac{f(x)}{(x-1)^n}=\dfrac{f^{(n)}(1)}{n!}\Rightarrow f^{(n)}(1)=\dfrac{n!}{2n^{n}}$$
\end{solution}
\myspace{1}

4. 设$f(x)$在$[0,1]$上二阶导数连续,$f(1)=f'(1)=0$,$D=\{(x,y)|0\leq x\leq 1,0\leq y\leq 1\}$

(1).证明:  $\iint\limits_{D}f(x)dxdy=\iint\limits_{D}x^2yf''(x)dxdy$

(2).证明:  $\exists \xi,\eta\in(0,1),\ s.t. \xi^2f''(\xi)=2f'(\eta)(\xi-1)$
\myspace{1}
\begin{solution}

	(1). \begin{eqnarray*}
		\text{左边}&=&\iint\limits_{D}f(x)dxdy\\
		&=&\int_{0}^{1}f(x)dx\int_{0}^{1}dy\\
		&=&\int_{0}^{1}f(x)dx\\
		&=&xf(x)|_{x=0}^{x=1}-\int_{0}^{1}xf'(x)dx\\
		&=&-\int_{0}^{1}xf'(x)dx
	\end{eqnarray*}
	\begin{eqnarray*}
		\text{右边}&=&\iint\limits_{D}x^2yf''(x)dxdy\\
		&=&\int_{0}^{1}x^2f''(x)dx\int_{0}^{1}ydy\\
		&=&\int_{0}^{1}\dfrac{x^2}{2}df'(x)\\
		&=&\dfrac{x^2}{2}f'(x)|_{x=0}^{x=1}-\int_{0}^{1}xf'(x)dx\\
		&=&-\int_{0}^{1}xf'(x)dx
	\end{eqnarray*}
	
	综上所述,$\text{左边}=\text{右边}\Rightarrow \iint\limits_{D}f(x)dxdy=\iint\limits_{D}x^2yf''(x)dxdy$
	
	(2). 原命题等价于:  $\exists \xi,\eta\in(0,1),\ s.t. \xi^2f''(\xi)=2f'(\eta)(\xi-1)=2[f(\xi)-f(1)]=2f(\xi)$
	
	我们需要证明:  $\exists \xi\in(0,1),\ s.t. \xi^2f''(\xi)-2f(\xi)=0$
	
	我们由$(1)$得到:  $\int_{0}^{1}f(x)dx=\int_{0}^{1}\dfrac{x^2}{2}f''(x)dx\Rightarrow \int_{0}^{1}[x^2f''(x)-2f(x)]dx=0$
	
	我们由积分中值定理得到:  
	$$\exists \xi\in(0,1),\ s.t. \xi^2f''(\xi)-2f(\xi)=0$$
	
	我们由拉格朗日中值定理得到:  
	$$\exists\eta\in(\xi,1),\ s.t. f(\xi)-f(1)=f'(\eta)(\xi-1)$$
	
	综上所述,我们得到:  $\exists \xi,\eta\in(0,1),\ s.t. \xi^2f''(\xi)=2f'(\eta)(\xi-1)$
\end{solution}
\myspace{1}

\hl{\textbf{\textit{September 7}}}

1. $\int_{-\pi}^{\pi}\dfrac{x\sin x\left( \arctan e^x+\int_{0}^{x}e^{t^2}dt\right) }{1+\cos^2 x}dx$
\myspace{1}
\begin{solution}

	原定积分可以化为:  
	\begin{eqnarray*}
		I&=&\int_{-\pi}^{\pi}\dfrac{x\sin x\arctan e^x}{1+\cos^2 x}dx\\
		&=&\int_{-\pi}^{\pi}\dfrac{x\sin x\arctan e^{-x}}{1+\cos^2 x}dx\\
		2I&=&\pi\int_{0}^{\pi}\dfrac{x\sin x}{1+\cos^2 x}dx\\
		&=&\pi\int_{0}^{\pi}\dfrac{(\pi-x)\sin x}{1+\cos^2 x}dx\\
		4I&=&\pi^2\int_{0}^{\pi}\dfrac{\sin x}{1+\cos^2 x}dx\\
		4I&=&\dfrac{\pi^3}{2}\\
		I&=&\dfrac{\pi^3}{8}
	\end{eqnarray*}
\end{solution}
\myspace{1}

2. 若$\int_{0}^{+\infty}\dfrac{\arctan(ax)}{x^n}dx$收敛,求$n$的取值范围
\myspace{1}
\begin{solution}

	原积分可能的暇点为$x=0,x=+\infty$
	
	(1). $x=0$处,$\dfrac{\arctan (ax)}{x^n}\sim ax^{1-n}$收敛$\Rightarrow 1-n>-1\Rightarrow n<2$
	
	(2). $x=+\infty$,$\dfrac{\arctan (ax)}{x^n}\sim \frac{\pi}{2}x^{-n}$收敛$\Rightarrow -n<-1\Rightarrow n>1$
	
	综上所述,$n\in(1,2)$
\end{solution}
\myspace{1}

3. 设$0<a<1$,$I_{1}=\int_{0}^{\frac{\pi}{4}}\dfrac{\sin ax}{\sin x}dx$,$I_{2}=\int_{0}^{\frac{\pi}{4}}\dfrac{\tan ax}{\tan x}dx$,比较$I_{1},I_{2}$和$\dfrac{\pi a}{4}$的大小
\myspace{1}
\begin{solution}

	我们可以得到:  $\left\lbrace
	\begin{array}{l}
		x\in(0,\frac{\pi}{4})\\
		ax\in(0,\frac{\pi}{4})\\
		ax<x
	\end{array}
	\right.$和函数$\left\lbrace
	\begin{array}{l}
		y=\sin x\\
		y=\tan x
	\end{array}
	\right. $在$(0,\frac{\pi}{2})$内单调递增.
	
	我们得到:  $$\left\lbrace
	\begin{array}{l}
		\dfrac{\sin(ax)}{\sin x}<1\\
		\dfrac{\tan(ax)}{\tan x}<1\\
		\dfrac{\frac{\tan(ax)}{\tan x}}{\frac{\sin(ax)}{\sin x}}=\dfrac{\cos(ax)}{\cos x}>1
	\end{array}
	\right. \Rightarrow I_{1}<I_{2}<\dfrac{\pi a}{4}$$
\end{solution}
\myspace{1}

4. 求$\iint\limits_{D}\dfrac{\tan^3x+y}{\sqrt{x^2+y^2}}dxdy$,其中$D=\{(x,y)|y\geq |x|,(x^2+y^2)^3\leq y^4\}$

\myspace{1}
\begin{solution}

	原二重积分等价于:  
	\begin{eqnarray*}
		I&=&\iint\limits_{D}\dfrac{\tan^3x+y}{\sqrt{x^2+y^2}}dxdy\\
		&=&\iint\limits_{D}\dfrac{y}{\sqrt{x^2+y^2}}dxdy\\
		&=&\int_{\frac{\pi}{4}}^{\frac{3\pi}{4}}d\theta\int_{0}^{\sin^2\theta}r\sin\theta dr\\
		&=&\int_{\frac{\pi}{4}}^{\frac{3\pi}{4}}\dfrac{1}{2}\sin^5\theta d\theta\\
		&=&\int_{\frac{\pi}{4}}^{\frac{\pi}{2}}\sin^5\theta d\theta\\
		&=&\dfrac{43\sqrt{2}}{120}
	\end{eqnarray*}
\end{solution}
\myspace{1}

5. 已知$\int_{1}^{+\infty}\left( \dfrac{2x^2+bx+a}{2x^2+ax}-1\right)dx=0$,求$a,b$

\myspace{1}
\begin{solution}

	原积分可化为:  $\int_{1}^{+\infty}\dfrac{(b-a)x+a}{2x^2+ax}dx=0\text{收敛}$
	
	(1). $x=+\infty$,假设$a\neq b\Rightarrow \dfrac{(b-a)x+a}{2x^2+ax}\sim \dfrac{1}{x}\text{发散}$,我们得到:  $a=b$
	
	原积分等价于:  $$\int_{1}^{+\infty}\dfrac{a}{2x^2+ax}dx=\int_{1}^{+\infty}(\dfrac{1}{x}-\dfrac{2}{2x+a})dx=0\Rightarrow \ln|\dfrac{x}{2x+a}|_{x=1}^{x=+\infty}=0\Rightarrow a=b=0$$
\end{solution}
\myspace{1}

\section{Week \Rmnum{2}}
\hl{\textbf{\textit{September 8}}}

1. 求$\lim\limits_{n\rightarrow +\infty}\left[\sum\limits_{k=1}^{n}\dfrac{\ln(n+k)}{n+\frac{1}{k}}-\ln n\right] $ 

\myspace{1}
\begin{solution}

	我们利用夹逼准则:  $$\sum\limits_{k=1}^{n}\dfrac{\ln(n+k)}{n+1}-\ln n<I<\sum\limits_{k=1}^{n}\dfrac{\ln(n+k)}{n}-\ln n$$
	\begin{eqnarray*}
		\text{左边}&=&\lim\limits_{n\rightarrow +\infty}\sum\limits_{k=1}^{n}\dfrac{\ln(n+k)}{n+1}-\ln n\\
		&=&\lim\limits_{n\rightarrow +\infty}\left\lbrace \dfrac{n}{n+1}\dfrac{1}{n}\sum\limits_{k=1}^{n}[\ln(n+k)-\ln n]-\dfrac{\ln n}{n+1}\right\rbrace \\
		&=&\int_{0}^{1}\ln(1+x)dx\\
		&=&2\ln2-1
	\end{eqnarray*}
	
	\begin{eqnarray*}
		\text{右边}&=&\lim\limits_{n\rightarrow +\infty}\sum\limits_{k=1}^{n}\dfrac{\ln(n+k)}{n}-\ln n\\
		&=&\lim\limits_{n\rightarrow +\infty}\dfrac{1}{n}\sum\limits_{k=1}^{n}[\ln(n+k)-\ln n]\\
		&=&\lim\limits_{n\rightarrow +\infty}\dfrac{1}{n}\sum\limits_{k=1}^{n}\ln(1+\frac{k}{n})\\
		&=&\int_{0}^{1}\ln(1+x)dx\\
		&=&2\ln2-1
	\end{eqnarray*}
	
	综上所述,我们可以得到原极限$I=2\ln2-1$
\end{solution}
\myspace{1}

2. $\int_{0}^{+\infty}\dfrac{dx}{(1+x^2)(1+x^4)}$

\myspace{1}
\begin{solution}

	原定积分等价于:  
	\begin{eqnarray*}
		I&=&\int_{0}^{+\infty}\dfrac{x^4}{(1+x^2)(1+x^4)}dx\\
		2I&=&\int_{0}^{+\infty}\dfrac{1}{1+x^2}dx\\
		&=&\dfrac{\pi}{2}\\
		I&=&\dfrac{\pi}{4}
	\end{eqnarray*}
\end{solution}
\myspace{1}

3. 设随机变量$Y=min\{|X|,1\}$,其中$X$为随机变量,且密度函数为$f(x)=\dfrac{k}{1+x^2}(k\text{为常数},-\infty<x<+\infty)$,下列说法不正确的是:  
\begin{itemize}
	\item A. $k=\dfrac{1}{\pi}$
	\item B. $E(X)=0$
	\item C. $Y$没有概率密度
	\item D. $E(Y)=\dfrac{\ln(2e^{\frac{\pi}{2}})}{\pi}$
\end{itemize}

\myspace{1}
\begin{solution}
	
\end{solution}
\myspace{1}

4. 求$y=e^{-x}\sqrt{\sin x}(0\leq x<+\infty)$绕$x$轴旋转一周的旋转体体积

\myspace{1}
\begin{solution}

	我们可以得到:  
	\begin{eqnarray*}
		V&=&\int_{0}^{+\infty}\pi y^2dx\\
		&=&\lim\limits_{n\rightarrow+\infty}\sum\limits_{k=0}^{n}\int_{2k\pi}^{(2k+1)\pi}\pi e^{-2x}\sin xdx\\
		&=&\lim\limits_{n\rightarrow+\infty}A(1+e^{-4\pi}+e^{-8\pi}+\cdots+e^{-4n\pi})\\
		&=&\dfrac{A\pi}{1-e^{-4\pi}}\\
		A&=&=\int_{0}^{\pi}e^{-2x}\sin xdx=\dfrac{e^{-2\pi}+1}{5}\\
		V&=&\dfrac{e^{-2\pi}+1}{5(1-e^{-4\pi})}=\dfrac{e^{2\pi}\pi}{5(e^{2\pi}-1)}
	\end{eqnarray*}
\end{solution}
\myspace{1}

\hl{\textbf{\textit{September 9}}}

1. 已知$\int_{0}^{+\infty}\dfrac{\ln(1+x)}{x^{\alpha}}dx$收敛,求$\alpha$的取值范围

\myspace{1}
\begin{solution}

	原积分可能存在的暇点为$x=0,x=+\infty$
	
	(1). $x=0$时,$\dfrac{\ln(1+x)}{x^{\alpha}}\sim x^{1-\alpha}$收敛$\Rightarrow 1-\alpha>-1\Rightarrow \alpha<2$
	
	(2). $x=+\infty$时,$\dfrac{\ln(1+x)}{x^{\alpha}}\sim x^{-\alpha}$收敛$\Rightarrow -\alpha<-1\Rightarrow \alpha>1$
	
	我们得到:  $\alpha\in(1,2)$
\end{solution}
\myspace{1}

2. 设随机变量$X,Y$相互独立,且$X\sim N(0,1)$,$Y\sim B(n,p)$,$0<p<1$,且$X+Y$的分布函数:  
\begin{itemize}
	\item A. 是连续函数
	\item B. 恰有$n+1$个间断点
	\item C. 恰有$1$个间断点
	\item D. 有无穷个间断点
\end{itemize}
\myspace{1}
\begin{solution}
	
\end{solution}
\myspace{1}

3. 已知微分方程$\cos^4 x\dfrac{d^2y}{dx^2}+2\cos^2 x(1-\sin x\cos x)\dfrac{dy}{dx}+y=e^{-\tan x}$,求该微分方程在$t=\tan x$变换下所得的$y$对$t$的微分方程,并求出其通解

\myspace{1}
\begin{solution}

	我们可以得到:  $$\left\lbrace
	\begin{array}{l}
		t=\tan x\\
		\sin x=\dfrac{t}{\sqrt{1+t^2}}\\
		\cos x=\dfrac{1}{\sqrt{1+t^2}}\\
		\dfrac{dy}{dx}=\dfrac{dy}{dt}\dfrac{dt}{dx}=\dfrac{(1+t^2)dy}{dt}\\
		\dfrac{d^2y}{dx^2}=\dfrac{\dfrac{dy}{dx}}{dt}\dfrac{dt}{dx}=\dfrac{(1+t^2)^2d^2y}{dt^2}-\dfrac{2t(1+t^2)dy}{dt}
	\end{array}
	\right. $$
	
	我们可以得到原微分方程等价于:  
	$$\dfrac{d^2y}{(1+t^2)^2dx^2}+\dfrac{2dy}{(1+t^2)dx}-\dfrac{2tdy}{(1+t^2)^2dx}+y=e^{-t}$$
	
	我们可以得到:  
	$$\dfrac{d^2y}{dt^2}+2\dfrac{dy}{dt}+y=e^{-t}\Rightarrow y''(t)+2y'(t)+y(t)=e^{-t}$$
	
	我们得到特征方程:  $\lambda^2+2\lambda+1=0\Rightarrow \lambda_{1}=\lambda_{2}=-1$
	
	我们可以得到:  $y=(C_{1}x+C_{2})e^{-x}+y^{*}\Rightarrow y^{*}=Ax^2e^{-x}\Rightarrow y^{*}=\dfrac{1}{2}x^2e^{-x}$
	
	综上所述,微分方程的通解为:  $y=(C_{1}x+C_{2})e^{-x}+\dfrac{1}{2}x^2e^{-x}$
\end{solution}
\myspace{1}

4. 求二重积分$I=\iint\limits_{D}\arcsin(2\sqrt{x-x^2})dxdy$,$D=\{(x,y)|0\leq x\leq 1,0\leq y\leq x\}$

\myspace{1}
\begin{solution}

	原二重积分可化为:  
	\begin{eqnarray*}
		I&=&\int_{0}^{1}x\arcsin(2\sqrt{x-x^2})dx\\
		&=&\int_{-\frac{\pi}{2}}^{\frac{\pi}{2}}\dfrac{\arcsin(\cos\theta) \cos\theta(\sin\theta+1)}{4}d\theta\\
		&=&\int_{-\frac{\pi}{2}}^{\frac{\pi}{2}}\dfrac{\arcsin(\cos\theta)\cos\theta}{4}d\theta\\
		&=&\dfrac{1}{2}\int_{0}^{\frac{\pi}{2}}(\frac{\pi}{2}-\theta)\cos\theta d\theta\\
		&=&\dfrac{\pi}{4}-\dfrac{1}{2}\int_{0}^{\frac{\pi}{2}}\theta d\sin\theta\\
		&=&\dfrac{1}{2}
	\end{eqnarray*}
\end{solution}
\myspace{1}

\hl{\textbf{\textit{September 10}}}

1. 设$f(x)=\lim\limits_{n\rightarrow+\infty}\dfrac{x^{n+1}-x^2}{x^n+1}$,$F(x)=\int_{0}^{x}f(t)dt$,下列说法正确的是:  
\begin{itemize}
	\item A. $f(x)$有$1$个间断点,$F(x)$有$1$个不可导点
	\item B. $f(x)$有$1$个间断点,$F(x)$有$2$个不可导点
	\item \hl{C}. $f(x)$有$2$个间断点,$F(x)$有$1$个不可导点
	\item D. $f(x)$有$2$个间断点,$F(x)$有$2$个不可导点
\end{itemize}
\myspace{1}
\begin{solution}

	我们可以得到:  $f(x)=\left\lbrace
	\begin{array}{l}
		x,\ x<-1\\
		-x^2,\ -1<x<1\\
		0,\ x=1\\
		x,\ x>1
	\end{array}
	\right. $
	
	$f(x)$在$x=-1$处无定义,$x=-1$是可去间断点,$x=1$是跳跃间断点.
	
	$F(x)=\int_{0}^{x}f(t)dt$处处连续,仅在$x=1$处不可导.
\end{solution}
\myspace{1}

2. 设当$x\geq 0$时,连续函数$f(x)$的原函数$F(x)$非负,且满足方程$\int_{0}^{x^2}f(x^2)f(t)dt=\dfrac{1}{2}(\sqrt{1+x^2}-1)$,$F(0)=0$,求$f(x)$

\myspace{1}
\begin{solution}

	我们设$F(x)=\int_{0}^{x}f(t)dt$,我们令$x^2=u$,我们得到:  
	$$f(u)\int_{0}^{u}f(t)dt=\dfrac{1}{2}(\sqrt{1+u}-1)\Rightarrow F'(x)F(x)=\dfrac{1}{2}(\sqrt{1+x}-1)\Rightarrow \dfrac{1}{2}[F^2(x)]'=\dfrac{1}{2}(\sqrt{1+x}-1)$$
	
	我们得到:  
	$$F^2(x)=\dfrac{2}{3}(1+x)^{\frac{3}{2}}-x+C,F(0)=0\Rightarrow \left\lbrace
	\begin{array}{l}
		F(x)=\sqrt{\dfrac{2}{3}(1+x)^{\frac{3}{2}}-x-\dfrac{2}{3}}\\
		f'(x)=\dfrac{\sqrt{1+x}-1}{2\sqrt{\dfrac{2}{3}(1+x)^{\frac{3}{2}}-x-\dfrac{2}{3}}}
	\end{array}
	\right. $$
\end{solution}
\myspace{1}

3. 证明:  $\sum\limits_{n=1}^{+\infty}\dfrac{1}{n^2}=\dfrac{\pi^2}{6}$

\myspace{1}
\begin{solution}

	我们有:  
	$$\sin x=x-\dfrac{x^3}{3!}+\dfrac{x^5}{5!}+\cdots+\dfrac{(-1)^{n}x^{2n+1}}{(2n+1)}+\cdots$$
	
	我们有:  $f(x)=\dfrac{\sin x}{x}=1-\dfrac{x^2}{3!}+\dfrac{x^4}{5!}+\cdots+\dfrac{(-1)^{n}x^{2n}}{(2n+1)}+\cdots$
	
	我们知道$f(x)$有零点$x=\pm\pi,\pm 2\pi,\cdots\Rightarrow f(x)=A(x-\pi)(x+\pi)(x-2\pi)(x+2\pi)\cdots$
	
	
	我们有:  
	$$f(x)=A(x^2-\pi^2)(x^2-4\pi^2)\cdots,\text{令}x=0,\text{我们有:  }A(-\pi^2)(-4\pi^2)\cdots=1\Rightarrow A=\dfrac{1}{(-\pi^2)(-4\pi^2)\cdots}$$
	
	我们有:  $$f(x)=\left\lbrace
	\begin{array}{l}
		1-\dfrac{x^2}{3!}+\dfrac{x^4}{5!}+\cdots+\dfrac{(-1)^{n}x^{2n}}{(2n+1)}+\cdots\\
		(1-\dfrac{x^2}{\pi^2})(1-\dfrac{x^2}{4\pi^2})\cdots
	\end{array}
	\right. \Rightarrow x^2\text{项系数相等}\Rightarrow -\dfrac{1}{6}=\sum\limits_{n=1}^{+\infty}(-\dfrac{1}{n^2\pi^2})$$
	
	我们可以得到:  $\sum\limits_{n=1}^{+\infty}\dfrac{1}{n^2}=\dfrac{\pi^2}{6}$
\end{solution}
\myspace{1}

\hl{\textbf{\textit{September 11}}}

1. 设$f(x)$在$[1,+\infty)$上有连续的一阶导数,$f'(x)=\dfrac{1}{1+f^{2}(x)}\left[\sqrt{\dfrac{1}{x}}-\sqrt{\ln(1+\dfrac{1}{x})} \right] $,证明:  $\lim\limits_{x\rightarrow +\infty}f(x)$存在

\myspace{1}
\begin{solution}

	我们由:  $\lim\limits_{x\rightarrow +\infty}f(x)-f(1)=\int_{1}^{+\infty}f'(x)dx\Rightarrow $我们只需要证明:  $\int_{1}^{+\infty}f'(x)dx$收敛
	
	我们有:  
	\begin{eqnarray*}
		\int_{1}^{+\infty}f'(x)dx&=&\int_{1}^{+\infty}\dfrac{1}{1+f^{2}(x)}\left[\sqrt{\dfrac{1}{x}}-\sqrt{\ln(1+\dfrac{1}{x})} \right]dx\\
		&\leq&\int_{1}^{+\infty}\left[\sqrt{\dfrac{1}{x}}-\sqrt{\ln(1+\dfrac{1}{x})} \right]dx
	\end{eqnarray*}
	
	我们由拉格朗日中值定理得到:  
	$$\lim\limits_{x\rightarrow+\infty}\left[\sqrt{\dfrac{1}{x}}-\sqrt{\ln(1+\dfrac{1}{x})} \right]=\lim\limits_{x\rightarrow+\infty}\dfrac{1}{2\sqrt{\xi}}\left(\dfrac{1}{x}-\ln(1+\dfrac{1}{x}) \right)=\lim\limits_{x\rightarrow+\infty}\dfrac{x^{-\frac{3}{2}}}{4}$$
	
	我们由比较判别法可以得到:  $\left\lbrace 
	\begin{array}{l}
		\int_{1}^{+\infty}\dfrac{1}{x^p}dx\text{收敛},p>1\\
		\int_{1}^{+\infty}\dfrac{1}{4x^{\frac{3}{2}}}dx\text{收敛}
	\end{array}
	\right. $
	
	综上所述,我们得到:  $\lim\limits_{x\rightarrow +\infty}f(x)$存在
\end{solution}
\myspace{1}

2. 函数$f(x)$在$[0,+\infty)$上可导,$f(0)=1$且满足等式:  
$$f'(x)+f(x)-\dfrac{1}{x+1}\int_{0}^{x}f(t)dt=0$$

(1).求导数$f'(x)$

(2).证明:  当$x\geq 0$时,不等式$e^{-x}\leq f(x)\leq 1$成立

\myspace{1}
\begin{solution}

	(1). 我们可以得到:  
	$$(x+1)f'(x)+(x+1)f(x)-\int_{0}^{x}f(t)dt=0$$
	
	我们对上式子求导可以得到:  
	$$(x+1)f''(x)+(x+2)f''(x)=0\Rightarrow \left\lbrace 
	\begin{array}{l}
		(x+1)e^{x}f'(x)=C\\
		f'(0)=-1
	\end{array}
	\right. \Rightarrow f'(x)=-\dfrac{1}{(1+x)e^{x}}$$
	
	(2). 我们有:  $$\left\lbrace 
	\begin{array}{l}
		F(x)=f(x)-1\\
		G(x)=f(x)-e^{-x}
	\end{array}
	\right.$$
	$$\downdownarrows$$ 
	
	$$\left\lbrace 
	\begin{array}{l}
		F'(x)=f'(x)=-\dfrac{1}{(1+x)e^{x}}<0\\
		F(0)=0\\
		G'(x)=f'(x)+e^{-x}=\dfrac{x}{(1+x)e^{x}}>0\\
		G(0)=0
	\end{array}
	\right.$$ 
	$$\downdownarrows$$ 
	$$\left\lbrace 
	\begin{array}{l}
		F(x)\text{单调递减}\\
		F(x)\leq F(0)=1\\
		G(x)\text{单调递增}\\
		G(x)\geq G(0)\Rightarrow f(x)\geq e^{-x}
	\end{array}
	\right. $$
\end{solution}
\myspace{1}

\hl{\textbf{\textit{September 12}}}

1. 设$f(x)=\int_{-1}^{x}(1-|t|)dt(x\geq -1)$,求曲线$y=f(x)$与$x$轴所围成的面积

\myspace{1}
\begin{solution}

	我们可以得到$f(x)$表达式:  
	$$f(x)=\left\lbrace 
	\begin{array}{l}
		\dfrac{(x+1)^2}{2},\ x\in[-1,0]\\
		\dfrac{2x-x^2+1}{2},\ x\in(1,+\infty)
	\end{array}
	\right. $$
	
	我们可以得到:  
	$$S=\int_{0}^{1}f(x)dx+\int_{0}^{1+\sqrt{2}}f(x)dx=\dfrac{1}{6}+\dfrac{5}{6}+\dfrac{2\sqrt{2}}{3}=1+\dfrac{2\sqrt{2}}{3}$$
\end{solution}
\myspace{1}

2. 设函数$f(x)$连续,$f'(0)$存在,并对于任意$x,y\in\mathbb{R}$,$f(x+y)=\dfrac{f(x)+f(y)}{1-4f(x)f(y)}$,且$f'(0)=\dfrac{1}{2}$,求$f(x)$

\myspace{1}
\begin{solution}

	我们有:  $f(0)=\dfrac{2f(0)}{1-4f^2(0)}\Rightarrow f(0)=0$
	
	我们利用导数定义:  
	\begin{eqnarray*}
		f'(x)&=&\lim\limits_{\Delta x\rightarrow 0}\dfrac{f(x+\Delta x)-f(x)}{\Delta x}\\
		&=&\lim\limits_{\Delta x\rightarrow 0}\dfrac{\dfrac{f(x)+f(\Delta x)}{1-4f(x)f(\Delta x)}-f(x)}{\Delta x}\\
		&=&(1+4f^{2}(x))\lim\limits_{\Delta x\rightarrow 0}\dfrac{f(\Delta x)}{\Delta}\\
		&=&\dfrac{1+4f^{2}(x)}{2}
	\end{eqnarray*}
	
	我们可以得到微分方程的解:  
	$$\arctan[2f(x)]=x+C,f(0)=0\rightarrow f(x)=\dfrac{1}{2}\tan x$$
\end{solution}
\myspace{1}

3. 设函数$f(x)$在$[0,+\infty)$上具有二阶连续导数,$f(0)=f'(0)=0,f''(0)=1$,对于任意的$x>0$,$u(x)$表示曲线$y=f(x)$在点$(x,f(x))$处的切线在$x$轴上的截距,当$x\rightarrow 0^{+}$时,下列等价无穷小不成立的是:  
\begin{itemize}
	\item A. $f(x)\sim \dfrac{x^2}{2}$
	\item B. $x\cdot f(u(x))\sim \dfrac{u(x)\cdot f(x)}{2}$
	\item C. $\int_{0}^{x}u(t)dt\sim \dfrac{x^2}{4}$
	\item \hl{D}. $\int_{0}^{f(x)}u(t)dt\sim \dfrac{x^4}{4}$
\end{itemize}
\myspace{1}
\begin{solution}

	我们求出$f(x)$在$(x,f(x))$处的切线方程:$y-f(x)=f'(x)(x'-x)$
	
	我们有:
	$$u(x)=x'=x-\dfrac{f(x)}{f'(x)}$$
	
	我们利用泰勒展开式,将$f(x)$展开:
	$$\left\lbrace 
	\begin{array}{l}
		f(x)=f(0)+f'(0)x+\dfrac{f''(0)}{2}x^2+o(x^2)\\
		f'(x)=f'(0)+f''(0)x+o(x)
	\end{array}
	\right. \rightarrow \left\lbrace 
	\begin{array}{l}
		f(x)=\dfrac{1}{2}x^2+o(x^2)\\
		f'(x)=x+o(x)
	\end{array}
	\right. $$
	
	当$x\rightarrow 0^{+}$时,$f(x)\sim \dfrac{1}{2}x^2,u(x)\sim \dfrac{1}{2}x$
	
	我们得到:
	$$\left\lbrace 
	\begin{array}{l}
		x\cdot f(u(x))\sim \dfrac{u(x)\cdot f(x)}{2}\\
		\int_{0}^{x}u(t)dt\sim \dfrac{x^2}{4}\\
		\int_{0}^{f(x)}u(t)dt\sim \dfrac{x^4}{16}
	\end{array}
	\right. $$
\end{solution}
\myspace{1}

\hl{\textbf{\textit{September 13}}}

1. 比较$\int_{0}^{\frac{\pi}{2}}\dfrac{\sin x}{1+x^2}dx$和$\int_{0}^{\frac{\pi}{2}}\dfrac{\cos x}{1+x^2}dx$的大小
\myspace{1}
\begin{solution}

	我们不妨设:$I=\int_{0}^{\frac{\pi}{2}}\dfrac{\sin x}{1+x^2}dx,J=\int_{0}^{\frac{\pi}{2}}\dfrac{\cos x}{1+x^2}dx$,我们有:
	\begin{eqnarray*}
		I-J&=&\int_{0}^{\frac{\pi}{2}}\dfrac{\sin x-\cos x}{1+x^2}dx\\
		&=&\int_{0}^{\frac{\pi}{2}}\dfrac{\sin(x-\frac{\pi}{4})}{1+x^2}dx\\
		&=&\int_{-\frac{\pi}{4}}^{\frac{\pi}{4}}\dfrac{\sin t}{1+(t+\frac{\pi}{4})^2}dt\\
		&=&\int_{-\frac{\pi}{4}}^{0}\dfrac{\sin t}{1+(t+\frac{\pi}{4})^2}dt+\int_{0}^{\frac{\pi}{4}}\dfrac{\sin t}{1+(t+\frac{\pi}{4})^2}dt\\
		&=&\int_{0}^{\frac{\pi}{4}}\sin t[\dfrac{1}{1+(t+\frac{\pi}{4})^2}-\dfrac{1}{1+(\frac{\pi}{4}-t)^2}]dt\\
		&=&\int_{0}^{\frac{\pi}{4}}\dfrac{-\pi t\sin t}{[1+(t+\frac{\pi}{4})^2][1+(\frac{\pi}{4}-t)^2]}dt<0
	\end{eqnarray*}
	
\end{solution}
\myspace{1}

2. 求曲线$y=\dfrac{x^2}{1+x^2}$与其渐近线所围区域绕该渐近线旋转所得旋转体体积
\myspace{1}
\begin{solution}

	我们可以得到$f(x)$的渐近线为$y=1$,因此我们有:
	\begin{eqnarray*}
		V&=&\pi\int_{-\infty}^{+\infty}\dfrac{1}{(1+x^2)^2}dx\\
		&=&2\pi\int_{0}^{\frac{\pi}{2}}\cos^2\theta d\theta\\
		&=&\dfrac{\pi^2}{2}
	\end{eqnarray*}
\end{solution}
\myspace{1}

3. 设$A$是$n$阶矩阵,$A$的第$i$行第$j$列元素为$a_{ij}$,满足$a_{ij}=i\cdot j$,下列命题正确的是:  
\begin{itemize}
	\item A. $r(A)=1$
	\item B. 矩阵$A$不可相似对角化
	\item C. 矩阵$A$的特征值之和为$\sum\limits_{k=1}^{n}k$
	\item \hl{D}. 矩阵$A$的特征值之和为$\sum\limits_{k=1}^{n}k^2$
\end{itemize}
\myspace{1}
\begin{solution}

	我们由:$a_{ij}=i\cdot j\rightarrow a_{ij}=a_{ji}=i\cdot j$,即矩阵$A$为实对称矩阵,$A$一定可以相似对角化
	
	我们可以得到:$|A|\neq 0\rightarrow r(A)=n$且矩阵$A$的特征值之和为$\sum\limits_{i=1}^{n}i^2$
\end{solution}
\myspace{1}

4. 求二重积分$\iint\limits_{D}\dfrac{1}{x^4+y^2}dxdy,\ D=\{(x,y)|y\geq x^2+1\}$
\myspace{1}
\begin{solution}

	原二重积分等价于:
	\begin{eqnarray*}
		I&=&\int_{-\infty}^{+\infty}dx\int_{x^2+1}^{+\infty}\dfrac{1}{x^4+y^2}dy\\
		&=&2\int_{0}^{+\infty}\dfrac{1}{x^2}\arctan(\dfrac{x^2}{x^2+1})dx\\
		&=&2\int_{0}^{+\infty}\arctan(\dfrac{1}{t^2+1})dt\\
		&=&
	\end{eqnarray*}
\end{solution}
\myspace{1}

5. 求二重积分$\iint\limits_{D}\dfrac{1}{x^4+y^2}dxdy,\ D=\{(x,y)|x\geq 1,y\geq x^2\}$
\myspace{1}
\begin{solution}

	原二重积分等价于:
	\begin{eqnarray*}
		I&=&\int_{1}^{+\infty}dx\int_{x^2}^{+\infty}\dfrac{1}{x^4+y^2}dy\\
		&=&\int_{1}^{+\infty}\dfrac{\pi}{4x^2}dx\\
		&=&\dfrac{\pi}{4}
	\end{eqnarray*}
\end{solution}
\myspace{1}

\hl{\textbf{\textit{September 14}}}

1. 曲线 $y=x^2$与直线 $y=mx(m>0)$在第一象限内所围成的图形绕该直线旋转所形成的旋转体的体积$V$
\myspace{1}
\begin{solution}

	我们设围成区域中任意一点$(x,y)$,我们有:$d=\dfrac{mx-y}{\sqrt{1+m^2}}$
	\begin{eqnarray*}
		V&=&\iint\limits_{S}2\pi ddxdy\\
		&=&\dfrac{2\pi}{\sqrt{1+m^2}}\int_{0}^{m}dx\int_{x^2}^{mx}(mx-y)dy\\
		&=&\dfrac{\pi m^5}{30\sqrt{1+m^2}}
	\end{eqnarray*}
\end{solution}
\myspace{1}

2. 设函数 $f(x,y)=\left\lbrace 
\begin{array}{l}
	(xy+a|x|+b\sqrt{|y|})\arctan \dfrac{1}{|x|+y^2},\ (x,y)\neq (0,0)\\
	0,\ (x,y)=(0,0)
\end{array}
\right. $,下列说法中正确的是:  
\begin{itemize}
	\item A. $f(x,y)$在$(0,0)$处的连续性和$a,b$的取值有关
	\item B. $f(x,y)$在$(0,0)$处偏导数存在的充要条件是$ab=0$
	\item C. $f(x,y)$在$(0,0)$处可微的充要条件是$f(x,y)$在$(0,0)$处偏导数存在
	\item D. $f(x,y)$在$(0,0)$处可微,则$(0,0)$是$f(x,y)$的极值点
\end{itemize}
\myspace{1}
\begin{solution}
	
\end{solution}
\myspace{1}

3. 设$f(x)$在$[0,+\infty)$上可导,$f'(x)+f^2(x)\geq 0,f(0)=1,f(x)\neq 0$,证明:  $f(x)\geq \dfrac{1}{x+1}$
\myspace{1}
\begin{solution}
	
\end{solution}
\myspace{1}

\section{Week \Rmnum{3}}
\hl{\textbf{\textit{September 15}}}

1. 设函数$f(x)$在$[0,1]$上二阶可导,且$\int_{0}^{1}f(x)dx=0$,则
\begin{itemize}
	\item A. 当$f'(x)<0$时,$f(\dfrac{1}{2})<0$
	\item B. 当$f''(x)<0$时,$f(\dfrac{1}{2})<0$
	\item C. 当$f'(x)>0$时,$f(\dfrac{1}{2})<0$
	\item D. 当$f''(x)>0$时,$f(\dfrac{1}{2})<0$
\end{itemize}
\myspace{1}
\begin{solution}
	
\end{solution}
\myspace{1}

2. 设$f(x)=\begin{pmatrix}
	1&x&x^2&x^3\\
	1&2&4&8\\
	1&-1&1&-1\\
	1&1&1&1
\end{pmatrix}$,则曲线$y=f(x)$在$(-1,2)$内存在水平切线的条数
\myspace{1}
\begin{solution}
	
\end{solution}
\myspace{1}

\hl{\textbf{\textit{September 16}}}

1. 已知正值连续函数$f(x)$在$[0,1]$上单调减少,对于任意的$a,b(0<a<b<1)$,下列结论不正确的是
\begin{itemize}
	\item A. $a\int_{0}^{b}f(x)dx>b\int_{0}^{a}f(x)dx$
	\item B. $b\int_{0}^{a}f(x)dx>a\int_{0}^{b}f(x)dx$
	\item C. $a\int_{0}^{b}\sqrt{f(x)}dx<b\int_{0}^{a}\sqrt{f(x)}dx$
	\item D. $b\int_{0}^{a}\sqrt{f(x)}dx<a\int_{0}^{b}\sqrt{f(x)}dx$
\end{itemize}
\myspace{1}
\begin{solution}
	
\end{solution}
\myspace{1}

2. 设函数$\varphi(x,y)$的全微分为$dz=(2x-y^2-2y)dx+(-2xy-2x+y^3+3y)dy$,$f(x,y)$连续,且$\lim\limits_{(x,y)\rightarrow (0,0)}\dfrac{f(x,y)}{\varphi(x,y)}=-1$
\begin{itemize}
	\item A. 点$(0,0)$是$f(x,y)$的极大值点
	\item B. 点$(0,0)$是$f(x,y)$的极小值点
	\item C. 点$(0,0)$不是$f(x,y)$的极值点
	\item D. 不能确定点$(0,0)$是否为$f(x,y)$的极值点
\end{itemize}
\myspace{1}
\begin{solution}
	
\end{solution}
\myspace{1}

3. 求$\int_{L}\dfrac{|y|}{x^2+y^2+z^2}ds$,其中$L:\left\lbrace 
\begin{array}{l}
	x^2+y^2+z^2=a^2\\
	x^2+y^2=2ax\\
	a\geq 0
\end{array}
\right. (a>0)$

\myspace{1}
\begin{solution}
	
\end{solution}
\myspace{1}

\hl{\textbf{\textit{September 17}}}

1. 设$I_{1}=\int_{0}^{\pi}\dfrac{x\sin^2 x}{1+e^{\cos^2 x}}dx$,$I_{2}=\int_{0}^{\pi}\dfrac{\sin^2 x}{1+e^{\cos^2 x}}dx$,$I_{3}=\int_{0}^{\frac{\pi}{2}}\dfrac{\cos^2 x}{1+e^{\sin^2 x}}dx$,比较$I_{1},I_{2},I_{3}$的大小
\myspace{1}
\begin{solution}
	
\end{solution}
\myspace{1}

2. 设$f(u,v)$有一阶偏导数,$f(x,1-x)=1,f_{1}^{'}(x,1-x)=x$

(1)设$z(t)=f(\cos t,\sin t)$,计算$z'(0)$

(2)证明:  $f(u,v)$在单位圆周上至少存在两个不同的点满足方程:  $v\dfrac{\partial f}{\partial u}=u\dfrac{\partial f}{\partial v}$
\myspace{1}
\begin{solution}
	
\end{solution}
\myspace{1}

\hl{\textbf{\textit{September 18}}}

1. 设$f(x)$在$[0,1]$连续可导,$\int_{0}^{1}f(x)dx=\dfrac{5}{2}$,$\int_{0}^{1}xf(x)dx=\dfrac{3}{2}$,证明:$\exists \xi\in(0,1),\ s.t. f'(\xi)=3$
\myspace{1}
\begin{solution}
	
\end{solution}
\myspace{1}

2. 求极限 $\lim\limits_{x\rightarrow 0 }\dfrac{(1-\cos^3 x)(1-\cos^{17} x)}{\frac{x^2}{2}-\int_{0}^{x}\sum\limits_{n=0}^{+\infty}\dfrac{(-1)^n}{3^n(2n+1)!}t^{2n+1}dt}$
\myspace{1}
\begin{solution}
	
\end{solution}
\myspace{1}

3. 设偶函数$f(t)$具有连续的导函数,且满足$f(t)=e^{4\pi t^2}+\iint\limits_{x^2+y^2\leq 4t^2}f(\dfrac{\sqrt{x^2+y^2}}{2})dxdy$,求方程$\int_{0}^{x}\sqrt{1+4\pi t^2}dt+\int_{\cos x}^{0}\dfrac{1+4\pi t^2}{f(t)}dt=0$在$(0,+\infty)$内根的个数
\myspace{1}
\begin{solution}
	
\end{solution}
\myspace{1}

4. 设二阶可导函数$f(x)$满足$f(0)=f(2)=0$,$f(1)=a>0$且$f''(x)<0$,则
\begin{itemize}
	\item A. $\int_{0}^{2}f(x)dx>a$
	\item B. $\int_{0}^{2}f(x)dx<a$
	\item C. $\int_{0}^{1}f(x)dx>\int_{1}^{2}f(x)dx$
	\item D. $\int_{0}^{1}f(x)dx<\int_{1}^{2}f(x)dx$
\end{itemize}
\myspace{1}
\begin{solution}
	
\end{solution}
\myspace{1}

\hl{\textbf{\textit{September 19}}}

1. 设$f(x,y)=\left\lbrace 
\begin{array}{l}
	\dfrac{y^2\sin x}{x^2+y^2},\ (x,y)\neq (0,0)\\
	0,\ (x,y)=(0,0)
\end{array}
\right. $,$f(x,y)$在$(0,0)$点处
\begin{itemize}
	\item A. 不连续
	\item B. 连续但不可导
	\item C. 可导但不可微
	\item D. 可微
\end{itemize}
\myspace{1}
\begin{solution}
	
\end{solution}
\myspace{1}

2. 求$\iint_{D}\dfrac{1-x^3y^2}{(y+2\sqrt{1-x^2})^2}dxdy$,其中$D:\{(x,y)|x^2+y^2\leq 1,-y\leq x\leq y\}$
\myspace{1}
\begin{solution}
	
\end{solution}
\myspace{1}

\hl{\textbf{\textit{September 20}}}

1. 求 $\iiint\limits_{\Omega}(mx+ly+nz)^2dv$,其中 $\Omega=\{(x,y,x)|x^2+y^2+z^2\leq a^2(a>0)\}$
\myspace{1}
\begin{solution}
	
\end{solution}
\myspace{1}

2. 设 $f(x)$在$[a,b]$上导函数连续,$f'(a)=f'(b)$,证明:  $\exists \xi\in(a,b),\ s.t.\ f'(\xi)=\dfrac{f(\xi)-f(a)}{\xi-a}$
\myspace{1}
\begin{solution}
	
\end{solution}
\myspace{1}

3. 下列函数在$(0,0)$点可微的是:
\begin{itemize}
	\item A.$f(x,y)=\sqrt{x^2+y^2}$
	\item B.$g(x,y)=\left\lbrace 
	\begin{array}{l}
		\dfrac{xy}{x^2+y^2},\ (x,y)\neq (0,0)\\
		0,\ (x,y)=(0,0)
	\end{array}
	\right. $
	\item C.$\varphi(x,y)=\left\lbrace 
	\begin{array}{l}
		\dfrac{xy}{\sqrt{x^2+y^2}},\ (x,y)\neq (0,0)\\
		0,\ (x,y)=(0,0)
	\end{array}
	\right. $
	\item D.$\psi(x,y)=\left\lbrace 
	\begin{array}{l}
		(x^2+y^2)\sin\frac{1}{x^2+y^2},\ (x,y)\neq (0,0)\\
		0,\ (x,y)=(0,0)
	\end{array}
	\right. $
\end{itemize}
\myspace{1}
\begin{solution}
	
\end{solution}
\myspace{1}

4. 已知$(b-a)(c-a)(d-a)(c-b)(d-c)=k$,求行列式
$$\begin{vmatrix}
	1&1&1&1\\
	a^2&b^2&c^2&d^2\\
	a^3&b^3&c^3&d^3\\
	a^4&b^4&c^4&d^4\\
\end{vmatrix}$$
\myspace{1}
\begin{solution}
	
\end{solution}
\myspace{1}

5. 求 $\int_{0}^{1}dx\int_{0}^{1-x}dy\int_{x+y}^{1}\dfrac{\sin z}{z}dz$
\myspace{1}
\begin{solution}
	
\end{solution}
\myspace{1}

\hl{\textbf{\textit{September 21}}}

1. 已知函数$f(x,y)$在$(0,0)$点的某邻域内有定义,则$\lim\limits_{x\rightarrow 0}f_{x}'(x,0)=f_{x}'(0,0),\lim\limits_{y\rightarrow 0}f_{y}'(0,y)=f_{y}'(0,0)$是$f(x,y)$在$(0,0)$处可微的什么条件?
\myspace{1}
\begin{solution}
	
\end{solution}
\myspace{1}

2. 设$n$阶可逆矩阵 $A=\begin{pmatrix}
	a&b&\cdots&b\\
	b&a&\cdots&b\\
	\vdots&\vdots&&\vdots\\
	b&b&\cdots&a
\end{pmatrix}$,其中$b\neq 0$,求$A^{-1}$
\myspace{1}
\begin{solution}
	
\end{solution}
\myspace{1}

3. 求微分方程$y''+(4x+e^{2y})(y')^3=0$的通解,其中$y'\neq 0$
\myspace{1}
\begin{solution}
	
\end{solution}
\myspace{1}

4. 设$f(x)$二阶可导,$f^{2}(x)-f^{2}(y)=f(x+y)-f(x-y)$

(1)证明:$f''(x)f(y)=f(x)f''(y)$

(2)若$f''(1)=f(1)=1$,求$f(x)$
\myspace{1}
\begin{solution}
	
\end{solution}
\myspace{1}

\section{Week \Rmnum{4}}
\hl{\textbf{\textit{September 22}}}

1. 设$z=\dfrac{x\cos(y-1)-(y-1)\cos x}{1+\sin x+\sin(y-1)}$,求$\dfrac{\partial z}{\partial y}|_{(0,1)}$
\myspace{1}
\begin{solution}
	
\end{solution}
\myspace{1}

2.已知$A$是正交矩阵,则$A^{*}=A^{T}$是$|A|=1$的什么条件?
\myspace{1}
\begin{solution}
	
\end{solution}
\myspace{1}

3. 求$\lim\limits_{n\rightarrow +\infty}\sqrt{n}\left( 1-\sum\limits_{i=1}^{n}\dfrac{1}{n+\sqrt{i}}\right) $
\myspace{1}
\begin{solution}
	
\end{solution}
\myspace{1}


\hl{\textbf{\textit{September 23}}}

1. 已知$f(x,y)=\left\lbrace 
\begin{array}{l}
	xy\dfrac{x^2-y^2}{x^2+y^2},\ (x,y)\neq(0,0)\\
	0,\ (x,y)=(0,0)
\end{array}
\right. $,求$f_{xy}''(0,0)\cdot f_{yx}''(0,0)$
\myspace{1}
\begin{solution}
	
\end{solution}
\myspace{1}

2. 设矩阵 $A=(a_{ij})$ 满足 $a_{ij}=A_{ij}$,$a_{11}=-1$,求 $Ax=\begin{pmatrix}
	1\\0\\0
\end{pmatrix}$ 的解
\myspace{1}
\begin{solution}
	
\end{solution}
\myspace{1}

3. 求 $\int_{0}^{+\infty}\dfrac{1-e^{-x}}{\sqrt{x^3}}dx$
\myspace{1}
\begin{solution}
	
\end{solution}
\myspace{1}

4. 方程 $x=e^{\sin^{n}x}(n=1,2,\cdots)$

(1). 证明方程在 $(\dfrac{\pi}{2},e)$ 内有唯一实数根

(2). 计算$\lim\limits_{n\rightarrow +\infty}\left(\dfrac{\pi}{2}\cdot \dfrac{\sin x_{n}}{x_{n}} \right)^{\dfrac{1}{x_{n}-\frac{\pi}{2}}}$
\myspace{1}
\begin{solution}
	
\end{solution}
\myspace{1}

\hl{\textbf{\textit{September 24}}}

1. 若 $\dfrac{\partial^2 z}{\partial x\partial y}=1$,且当 $x=0$ 时,$z=\sin y$, 当$y=0$ 时,$z=\sin x$,求$z(x,y)$
\myspace{1}
\begin{solution}
	
\end{solution}
\myspace{1}

2.设矩阵$\begin{bmatrix}
	a_{1}&b_{1}&c_{1}\\
	a_{2}&b_{2}&c_{2}\\
	a_{3}&b_{3}&c_{3}
\end{bmatrix}$满秩,则直线$l_{1}:\ \dfrac{x-a_{1}}{a_{1}-a_{2}}=\dfrac{y-b_{1}}{b_{1}-b_{2}}=\dfrac{z-c_{1}}{c_{1}-c_{2}}$与直线$l_{2}:\ \dfrac{x-a_{1}}{a_{2}-a_{3}}=\dfrac{y-b_{1}}{b_{2}-b_{3}}=\dfrac{z-c_{1}}{c_{2}-c_{3}}$关系
\myspace{1}
\begin{solution}
	
\end{solution}
\myspace{1}

3. 设 $A=\begin{bmatrix}
	-3&4&0&0\\
	-2&3&0&0\\
	0&0&1&1\\
	0&0&0&1
\end{bmatrix}$,求$A^{n}$
\myspace{1}
\begin{solution}
	
\end{solution}
\myspace{1}

4. 求$\sum\limits_{n=1}^{+\infty}\dfrac{(-1)^{n}n^{2}}{(n+1)!}x^{n}$的和函数$S(x)$
\myspace{1}
\begin{solution}
	
\end{solution}
\myspace{1}

\hl{\textbf{\textit{September 25}}}

1. 设可微函数 $f(x,y)$ 满足$\dfrac{\partial f}{\partial x}=-f(x,y)$,$f(0,\dfrac{\pi}{2})=1$,且$\lim\limits_{n\rightarrow +\infty}\left[\dfrac{f(0,y+\frac{1}{n})}{f(0,y)} \right]^{n}=e^{\cot y} $,求$f(x,y)$
\myspace{1}
\begin{solution}
	
\end{solution}
\myspace{1}

2. $f(x)$在$(-\infty,+\infty)$可微,$|f'(x)|<k\cdot f(x)(0<k<1)$

(1).$\exists \xi\in(-\infty),+\infty),\ s.t.\ \ln f(\xi)=\xi$

(2).设$a_{n}=\ln f(a_{n-1})(n=1,2,\cdots)$,证明$\{a_{n}\}$收敛
\myspace{1}
\begin{solution}
	
\end{solution}
\myspace{1}

\hl{\textbf{\textit{September 26}}}

1.设$f(x)$有连续一阶导数,$(xy-yf(x))dx+(f(x)+y^2)dy=du(x,y)$,且$f(0)=-1$,求$u(x,y)$
\myspace{1}
\begin{solution}
	
\end{solution}
\myspace{1}

2. 积分计算

(1).$\int_{0}^{\ln 2}dy\int_{e^{y}}^{2}\dfrac{e^{xy}}{x^{x}-1}dx$

(2).$\int_{0}^{1}dx\int_{x}^{1}ydy\int_{y}^{1}\sqrt{1+z^4}dz$
\myspace{1}
\begin{solution}
	
\end{solution}
\myspace{1}

\hl{\textbf{\textit{September 27}}}

1. $f(x)$在$x=0$处$n+1$阶可导,$f(0)=f'(0)=\cdots=f^{(n-1)}(0)=0$,$f^{(n)}(0)=a$,求$\lim\limits_{x\rightarrow 0}\dfrac{f(e^{x}-1)-f(x)}{x^{n+1}}$
\myspace{1}
\begin{solution}
	
\end{solution}
\myspace{1}

2. 若函数$z=z(x,y)$由方程$e^{x+2y+3z}+\dfrac{xyz}{\sqrt{1+x^2+y^2+z^2}}=1$确定,求$dz|_{(0,0)}$
\myspace{1}
\begin{solution}
	
\end{solution}
\myspace{1}

\hl{\textbf{\textit{September 28}}}

1.设$f(x)$在$[0,+\infty)$上有一阶连续导数,对半空间$x\geq 0$中任意光滑闭曲面$\varSigma$,我们有$\oiint\lim\limits_{\varSigma}e^{-x}f(x)dydz+y\sqrt{e^{x}-1}f^{2}(x)dzdx=0$,求$f(x)$
\myspace{1}
\begin{solution}
	
\end{solution}
\myspace{1}

2.设$f(t)$在$[t,+\infty)$上有连续二阶导数,且$f(1)=0,f'(1)=1$,$z=(x^2+y^2)f(x^2+y^2)$满足$\dfrac{\partial^2 z}{\partial x^2}+\dfrac{\partial^2 z}{\partial y^2}=0$,求$f(x)$在$[1,+\infty)$上的最大值
\myspace{1}
\begin{solution}
	
\end{solution}
\myspace{1}

\hl{\textbf{\textit{September 29}}}

1. 设$u=f(x,y,z)$,$z=z(x,y)$是由方程$\varphi(x+y,z)=1$所确定的隐函数,其中$f$和$\varphi$有二阶连续偏导数且$\varphi_{2}\neq 0$,求$\dfrac{\partial u}{\partial x},du,\dfrac{\partial^2 u}{\partial x\partial y}$
\myspace{1}
\begin{solution}
	
\end{solution}
\myspace{1}

\hl{\textbf{\textit{September 30}}}

1. 设函数$z=z(x,y)$的微分$dz=(2x+12y)dx+(12x+4y)dy$且$z(0,0)=0$,求函数$z=z(x,y)$在$4x^2+y^2\leq 25$上的最大值
\myspace{1}
\begin{solution}
	
\end{solution}
\myspace{1}
























