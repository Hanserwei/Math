\chapterimage{chap35.jpg}
\chapter{September}
\section{Week \Rmnum{1}}
\textcolor{orange}{September 1}

1. 求曲线$y=\dfrac{x^{1+x}}{(1+x)^x}(x>0)$的斜渐近线方程
\myspace{1}
\begin{solution}

	我们不妨设曲线的斜渐近线方程为:  $y=ax+b$,我们可以得到:  
	
	$$\left\lbrace
	\begin{array}{l}
		a=\lim\limits_{x\to +\infty}\dfrac{f(x)}{x}=e^{\lim\limits_{t\to 0}-\frac{\ln(t+1)}{t}}=e^{-1}\\
		b=\lim\limits_{x\to +\infty}[f(x)-ax]=\dfrac{1}{2e}
	\end{array}
	\right. $$
	
	综上所述,曲线的斜渐近线方程为:  $y=\frac{x}{e}+\frac{1}{2e}$
\end{solution}
\myspace{1}

2. $\int_{0}^{\frac{\pi}{2}}\ln(\sin x)dx$
\myspace{1}
\begin{solution}

	我们利用区间再现公式:  
	\begin{eqnarray*}
		I&=&\int_{0}^{\frac{\pi}{2}}\ln(\cos x)dx\\
		2I&=&\int_{0}^{\frac{\pi}{2}}\ln(\sin x\cos x)dx\\
		&=&\int_{0}^{\frac{\pi}{2}}\ln(\sin 2x)dx-\int_{0}^{\frac{\pi}{2}}\ln2dx\\
		&=&\dfrac{1}{2}\int_{0}^{\pi}\ln(\sin x)dx-\dfrac{\pi \ln 2}{2}\\
		&=&I-\dfrac{\pi \ln 2}{2}\\
		I&=&\dfrac{\pi \ln 2}{2}
	\end{eqnarray*}
\end{solution}
\myspace{1}

\textcolor{orange}{September 2}

1. $\int_{0}^{\frac{\pi}{2}}\dfrac{1}{1+\tan^{n}x}dx$
\myspace{1}
\begin{solution}

	原定积分等价于:  $\int_{0}^{+\infty}\dfrac{1}{(1+x^{n})(1+x^2)}dx$
	\begin{eqnarray*}
		I&=&\int_{0}^{+\infty}\dfrac{1}{(1+x^{n})(1+x^2)}dx\\
		&=&\int_{0}^{+\infty}\dfrac{x^n}{(1+x^{n})(1+x^2)}dx\\
		2I&=&\int_{0}^{+\infty}\dfrac{1}{1+x^2}dx\\
		&=&\dfrac{\pi}{2}\\
		I&=&\dfrac{\pi}{4}
	\end{eqnarray*}
\end{solution}
\myspace{1}

2. 设随即变量$(X,Y)$的联合概率密度为:  $f(x,y)=\left\lbrace
\begin{array}{l}
	\dfrac{21}{4}x^2y,\ x^2\leq y\leq 1\\
	0,\ \text{其他}
\end{array}
\right. $,求$P\{Y\geq 0.75|X=0.5\}$
\myspace{1}
\begin{solution}
	
\end{solution}
\myspace{1}

\textcolor{orange}{September 3}

1. 设$f(x)=\left\lbrace
\begin{array}{l}
	\dfrac{g(x)}{x},\ x\neq 0\\
	0,\ x=0
\end{array}
\right. $,其中$g(x)$在$x=0$的一个邻域内有二阶导数,且$g(0)=0$,$g'(0)=0$,则$f(x)$在点$x=0$处
\begin{itemize}
	\item A. 不连续
	\item B. 连续,但$f'(0)$不存在
	\item C. $f'(0)$存在,但$f'(x)$在$x=0$处不连续
	\item \hl{D}. $f'(x)$在$x=0$处连续
\end{itemize}
\myspace{1}
\begin{solution}

	(1). $f(x)$在$x=0$处连续性
	
	$$\left\lbrace
	\begin{array}{l}
		\lim\limits_{x\to 0}f(x)=\lim\limits_{x\to 0}\dfrac{g(x)}{x}=\lim\limits_{x\to 0}g'(x)=0\\
		f(0)=0
	\end{array}
	\right. \Rightarrow f(x)\text{在}x=0\text{处连续}$$
	
	(2). $f'(x)$在$x=0$处连续性
	
	$$f'(x)=\left\lbrace
	\begin{array}{l}
		\dfrac{xg'(x)-g(x)}{x^2},x\neq 0\\
		f'(0)=\lim\limits_{x\to 0}\dfrac{f(x)-f(0)}{x}=\lim\limits_{x\to 0}\dfrac{g(x)}{x^2}=\dfrac{g''(0)}{2}
	\end{array}
	\right.$$
	$$\lim\limits_{x\to 0}f'(x)=\lim\limits_{x\to 0}\dfrac{xg'(x)-g(x)}{x^2}=\dfrac{g''(0)}{2}=f'(0)$$
	
	综上所述,我们得到$f'(x)$在$x=0$处连续.
\end{solution}
\myspace{1}

2. 设$f(x)$在$[0,+\infty)$上连续,$\int_{0}^{+\infty}f^{2}(x)dx$收敛,$a_{n}=\int_{0}^{1}f(nx)dx$,证明:  $k>0\text{时},\sum\limits_{n=1}^{+\infty}\dfrac{a_{n}^2}{n^k}$收敛.
\myspace{1}
\begin{solution}

	我们由$a_{n}=\int_{0}^{1}f(nx)dx\Rightarrow a_{n}=\dfrac{1}{n}\int_{0}^{n}f(x)dx$,我们可以得到:  
	$$\left\lbrace
	\begin{array}{l}
		a_{n}^2=\dfrac{[\int_{0}^{n}f(x)dx]^2}{n^2}\leq \dfrac{\int_{0}^{n}[f(x)]^2dx}{n}(\text{积分形式柯西不等式})\\
		b=\lim\limits_{n\to+\infty}\int_{0}^{n}[f(x)]^2dx=\int_{0}^{+\infty}f^{2}(x)dx\text{收敛}
	\end{array}
	\right. $$
	
	我们得到:  $$\dfrac{a_{n}^2}{n^{k}}\leq \dfrac{\int_{0}^{n}[f(x)]^2dx}{n^{1+k}}\leq \dfrac{b}{n^{1+k}}$$
	
	我们已知级数$\sum\limits_{n=1}^{+\infty}\dfrac{b}{n^{1+k}}(k>0,b>0)$收敛,我们根据比较判别法可以得到:  级数$\sum\limits_{n=1}^{+\infty}\dfrac{a_{n}^2}{n^k}$收敛
\end{solution}
\myspace{1}

\textcolor{orange}{September 4}

1. 已知方程$x^5-5x+k=0$有三个不同的实数根,求$k$的取值范围
\myspace{1}
\begin{solution}

	我们令$f(x)=x^5-5x+k$,$f'(x)=5x^4-5=5(x^2+1)(x+1)(x-1)$
	
	$$\left\lbrace
	\begin{array}{l}
		x\in(-\infty,-1)\cup(1,+\infty),\ f'(x)>0\\
		x\in(-1,1),\ f'(x)<0
	\end{array}
	\right. $$
	
	我们得到:  $f(x)$在$(-\infty,-1)$上单调递增;$(-1,1)$上单调递减;$(1,+\infty)$上单调递增.
	
	我们有:  $\left\lbrace
	\begin{array}{l}
		\lim\limits_{x\to -\infty}f(x)=-\infty\\
		\lim\limits_{x\to +\infty}f(x)=+\infty
	\end{array}
	\right. $,$f(x)$有三个不同的实数根,我们需要满足:  
	$$\left\lbrace
	\begin{array}{l}
		f(-1)>0\\
		f(1)<0
	\end{array}
	\right. \Rightarrow \left\lbrace
	\begin{array}{l}
		k+4>0\\
		k-4>0
	\end{array}
	\right. \Rightarrow k\in(-4,4)$$
	
	综上所述,我们得到$k$的取值范围为$(-4,4)$.
\end{solution}
\myspace{1}

2. $\int_{0}^{+\infty}\dfrac{1}{(1+x^2)(1+x^n)}dx$
\myspace{1}
\begin{solution}

	原定积分可以化为:  
	\begin{eqnarray*}
		I&=&\int_{0}^{+\infty}\dfrac{x^n}{(1+x^2)(1+x^n)}dx\\
		2I&=&\int_{0}^{+\infty}\dfrac{1+x^n}{(1+x^2)(1+x^n)}dx\\
		&=&\int_{0}^{+\infty}\dfrac{1}{1+x^2}dx\\
		&=&\dfrac{\pi}{2}\\
		I&=&\dfrac{\pi}{4}
	\end{eqnarray*}
\end{solution}
\myspace{1}

\textcolor{orange}{September 5}

1. $\int_{0}^{1}\dfrac{\ln(1+x)}{1+x^2}dx$
\myspace{1}
\begin{solution}

	原定积分可以化为:  
	\begin{eqnarray*}
		I&=&\int_{0}^{\frac{\pi}{4}}\ln(1+\tan \theta)d\theta\\
		&=&\int_{0}^{\frac{\pi}{4}}\ln(1+\dfrac{1-\tan\theta}{1+\tan\theta})d\theta\\
		&=&\int_{0}^{\frac{\pi}{4}}\ln(\dfrac{2}{1+\tan\theta})d\theta\\
		2I&=&\int_{0}^{\frac{\pi}{4}}\ln2d\theta\\
		&=&\dfrac{\pi \ln 2}{4}\\
		I&=&\dfrac{\pi \ln 2}{8}
	\end{eqnarray*}
\end{solution}
\myspace{1}

2. 求极限$\lim\limits_{x\to 0}\int_{\ln(1+x)}^{x}\dfrac{(1-2t)^{\frac{1}{t}}}{t^2}dt$
\myspace{1}
\begin{solution}

	我们利用第二积分中值定理可以得到:  
	$$\lim\limits_{x\to 0}\dfrac{\xi}{x}=1$$
	\begin{eqnarray*}
		I&=&\lim\limits_{x\to 0}(x-\ln(1+x))\dfrac{(1-2\xi)^{\frac{1}{\xi}}{\xi^2}},\ \xi\in(\ln(1+x),x)\\
		 &=&\lim\limits_{x\to 0}\dfrac{x-\ln(1+x)}{\xi^2}\lim\limits_{\xi\to 0}(1-2\xi)^{\frac{1}{\xi}}\\
		 &=&\dfrac{e^{-2}}{2}
	\end{eqnarray*}
\end{solution}
\myspace{1}

\textcolor{orange}{September 6}

1. 已知$f(x)=[x]\sin \pi x$,其中$[x]$表示不超过$x$的最大整数,求$f'(x)$
\myspace{1}
\begin{solution}

	(1). 
	$$x\in(n,n+1),(n\in\mathbb{Z}),\ [x]=n$$
	
	我们得到:  $$f'(x)=n\pi\cos(\pi x),\ x\in(n,n+1), \ n\in\mathbb{Z}$$
	
	(2). 当$x=n,\ n\in\mathbb{Z}$时,我们利用导数定义可以得到:  
	$$\left\lbrace
	\begin{array}{l}
		\lim\limits_{x\to n^{+}}\dfrac{n\sin \pi x}{x-n}\\
		\lim\limits_{x\to n^{-}}\dfrac{(n-1)\sin \pi x}{x-n}\\
	\end{array}
	\right.\Rightarrow f'(n)\text{不存在}$$
	
	综上所述,我们得到:  $f'(x)=\left\lbrace
	\begin{array}{l}
		[x]\cos \pi x,\ x\neq n,\ n\in\mathbb{Z}\\
		\text{不存在},\ x=n,\ n\in\mathbb{Z}
	\end{array}
	\right. $
\end{solution}
\myspace{1}

2. 设函数$f(x)=ax-b\ln x(a>0)$有两个零点,求$\dfrac{b}{a}$的取值范围.
\myspace{1}
\begin{solution}

	我们首先得到$f(x)$定义域为$(0,+\infty)$,我们有:  $f'(x)=\dfrac{ax-b}{x}$.
	
	(1). 当$b\leq 0$时,$f'(x)>0$,$f(x)$在$(0,+\infty)$上单调递增,至多存在一个零点.
	
	(2). 当$b>0$时,令$f'(x)=0\Rightarrow x=\dfrac{b}{a}$.
	
	$$\left\lbrace
	\begin{array}{l}
		x\in(0,\frac{b}{a}),\ f'(x)<0\\
		x\in(\frac{b}{a},+\infty),\ f'(x)>0
	\end{array}
	\right. \Rightarrow f(x)_{min}=f(\dfrac{b}{a})$$
	
	我们有:  $$\left\lbrace
	\begin{array}{l}
		\lim\limits_{x\to 0}f(x)=+\infty\\
		\lim\limits_{x\to +\infty}f(x)=+\infty
	\end{array}
	\right. $$
	
	当且仅当$f(\dfrac{b}{a})<0$时,$f(x)$在$(0,+\infty)$上有两个零点.
	$$f(\dfrac{b}{a})=b(1-\ln\dfrac{b}{a})<0\Rightarrow \dfrac{b}{a}>e$$
	
	综上所述,我们得到$\dfrac{b}{a}$的取值范围为$(e,+\infty)$.
\end{solution}
\myspace{1}

\textcolor{orange}{September 7}

1. $\int_{0}^{1}\dfrac{x^2}{x+\sqrt{1-x^2}}dx$
\myspace{1}
\begin{solution}

	原定积分可以化为:  
	\begin{eqnarray*}
		I&=&\int_{0}^{\frac{\pi}{2}}\dfrac{\sin^2\theta\cos\theta}{\sin\theta+\cos\theta}d\theta\\
		&=&\int_{0}^{\frac{\pi}{2}}\dfrac{\cos^2\theta\sin\theta}{\sin\theta+\cos\theta}d\theta\\
		2I&=&\int_{0}^{\frac{\pi}{2}}\dfrac{\sin^2\theta\cos\theta+\cos^2\theta\sin\theta}{\sin\theta+\cos\theta}d\theta\\
		&=&\int_{0}^{\frac{\pi}{2}}\sin\theta\cos\theta d\theta\\
		&=&\dfrac{1}{2}\\
		I&=&\dfrac{1}{4}
	\end{eqnarray*}
\end{solution}
\myspace{1}

2. $\int_{0}^{1}\dfrac{1}{x+\sqrt{1-x^2}}dx$
\myspace{1}
\begin{solution}

	原定积分可以化为:  
	\begin{eqnarray*}
		I&=&\int_{0}^{\frac{\pi}{2}}\dfrac{\cos\theta}{\sin\theta+\cos\theta}d\theta\\
		&=&\int_{0}^{\frac{\pi}{2}}\dfrac{\sin\theta}{\sin\theta+\cos\theta}d\theta\\
		2I&=&\int_{0}^{\frac{\pi}{2}}1d\theta\\
		&=&\dfrac{\pi}{2}\\
		I&=&\dfrac{\pi}{4}
	\end{eqnarray*}
\end{solution}
\myspace{1}

\section{Week \Rmnum{2}}
\textcolor{blue}{September 8}

1. 求极限$\lim\limits_{n\to+\infty}\left[1+\sin\left(\pi\sqrt{1+4n^2} \right) \right]^n$
\myspace{1}
\begin{solution}

	原极限可以化为:  
	\begin{eqnarray*}
		I&=&\lim\limits_{n\to+\infty}\left[1+\sin\left(\dfrac{\pi}{\sqrt{1+4n^2}+2n}\right) \right]^n\\
		&=&e^{\lim\limits_{n\to+\infty}n\ln\left[1+\sin\left(\dfrac{\pi}{\sqrt{1+4n^2}+2n}\right)\right]}\\
		&=&e^{\lim\limits_{n\to+\infty}n\sin\left(\dfrac{\pi}{\sqrt{1+4n^2}+2n}\right)}\\
		&=&e^{\lim\limits_{n\to+\infty}\dfrac{\pi n}{\sqrt{1+4n^2}+2n}}\\
		&=&e^{\frac{\pi}{4}}
	\end{eqnarray*}
\end{solution}
\myspace{1}

2. 设$f_{n}(x)=\tan^n x(n=1,2,\cdots)$,且曲线$y=\tan^n x$在点$x=\dfrac{\pi}{4}$处的切线与$x$轴的交点为$(x_{n},0)$,求极限$\lim\limits_{n\to+\infty}f_{n}(x_{n})$
\myspace{1}
\begin{solution}

	我们先求切线方程:  
	$$\left\lbrace
	\begin{array}{l}
		f(x)=\tan^n x\\
		f(\frac{\pi}{4})=1\\
		f'(x)=n\tan^{n-1}x\sec^2 x
	\end{array}
	\right. \Rightarrow l:\ y-1=2n(x-\dfrac{\pi}{4})\Rightarrow x_{n}=\dfrac{\pi}{4}-\dfrac{1}{2n}$$
	
	原极限可以化为:  
	\begin{eqnarray*}
		I&=&\lim\limits_{n\to+\infty}\tan^{n}\left( \dfrac{\pi}{4}-\dfrac{1}{2n}\right) \\
		&=&e^{\lim\limits_{x\to 0}\dfrac{\ln(\tan(\frac{\pi}{4}-x))}{2x}}\\
		&=&e^{-1}
	\end{eqnarray*}

	综上所述,我们得到:  $\lim\limits_{n\to+\infty}f_{n}(x_{n})=e^{-1}$
\end{solution}
\myspace{1}

\textcolor{blue}{September 9}

1. 设$A$是$3$阶实对称矩阵,且不可逆,\ $\left(\begin{matrix}
	2&-2&1\\-2&-1&2
\end{matrix} \right)\cdot A=\left(\begin{matrix}
6&-6&3\\-12&-6&12
\end{matrix} \right) $,求$A$
\myspace{1}
\begin{solution}

	我们利用谱分解定理:  $A=\sum\limits_{i=1}^{3}\lambda_{i}G_{i},\ G_{i}=e_{i}e_{i}^{T}$
	
	我们将上述方程两边取转置得到:  
	$$A^{T}\cdot\left(\begin{matrix}
		2&-2\\-2&-1\\&1&2
	\end{matrix} \right)=\left(\begin{matrix}
	6&-12\\-6&-6\\3&12
\end{matrix} \right)\Rightarrow \left\lbrace
\begin{array}{l}
A^{T}\xi_{1}=3\xi_{1}\\
A^{T}\xi_{2}=6\xi_{2}
\end{array}
\right. $$

我们知道矩阵$A^{T}$的两个特征值为$3,6$,又因为矩阵$A$不可逆,矩阵$A^{T}$的三个特征值分别为$0,3,6$

我们可以得到:  $A^{T}=\sum\limits_{i=1}^{3}\lambda_{i}G_{i}=\left(\begin{matrix}
	4&0&-2\\0&2&-2\\-2&-2&3
\end{matrix} \right)$

我们有$A^{T}=A$,综上所述,我们得到:  $A=\left(\begin{matrix}
	4&0&-2\\0&2&-2\\-2&-2&3
\end{matrix} \right)$
\end{solution}
\myspace{1}


2. 已知方程$\dfrac{1}{\ln(1+x)}-\dfrac{1}{x}=k$在区间$(0,1)$有实数根,求$k$的取值范围
\myspace{1}
\begin{solution}

	我们构造辅助函数:  $f(x)=\dfrac{1}{\ln(1+x)}-\dfrac{1}{x}$,我们有:  
	$$f'(x)=\dfrac{(1+x)\ln^2(1+x)-x^2}{x^2(1+x)\ln^2(1+x)},\ x\in(0,1)$$
	
	我们构造辅助函数:  $g(x)=2\ln x-\dfrac{x^2-1}{x},\ x\in(1,2)$
	
	我们有:  
	$$\left\lbrace
	\begin{array}{l}
		g'(x)=\dfrac{-(x-1)^2}{x^2}\leq 0\\
		g(x)\text{单调递减}\\
		g(x)\leq g(1)=0
	\end{array}
	\right. \Rightarrow 2\ln t\leq \dfrac{t^2-1}{t},\ t\in(1,2)\Rightarrow \ln^2(1+x)\leq \dfrac{x^2}{1+x}$$
	
	我们得到:  $f'(x)\leq 0,\ x\in(0,1)\Rightarrow f(x)$在$(0,1)$上单调递减.
	
	我们有:  $$\left\lbrace
	\begin{array}{l}
		\lim\limits_{x\to 0}f(x)=\lim\limits_{x\to 0}\dfrac{x-\ln(1+x)}{x\ln(1+x)}=\dfrac{1}{2}\\
		\lim\limits_{x\to 1}f(x)=\dfrac{1}{\ln 2}-1
	\end{array}
	\right. $$
	
	综上所述,$k$的取值范围为:  $(\dfrac{1}{\ln2}-1,\dfrac{1}{2})$
\end{solution}
\myspace{1}

\textcolor{blue}{September 10}

1. 设$f(x)=\left\lbrace
\begin{array}{l}
	\dfrac{e^x-1}{x},\ x\neq 0\\
	1,\ x=0
\end{array}
\right. $,则下列关于$y=\dfrac{\int_{0}^{x}f(t)dt}{x}$的命题正确的个数
\begin{itemize}
	\item A. 有垂直渐近线
	\item \hl{B}. 有水平渐近线
	\item C. 有斜渐近线
	\item D. 是有界函数
\end{itemize}
\myspace{1}
\begin{solution}

	我们可以得到:  $f(x)$在$x=0$处连续,$f(x)$在$\mathbb{R}$上连续,我们令$g(x)=\dfrac{\int_{0}^{x}f(t)dt}{x}$
	
	首先$g(x)$在$x=0$处无定义,我们可以得到:  
	$$\left\lbrace
	\begin{array}{l}
		\lim\limits_{x\to 0^{+}}g(x)=1\\
		\lim\limits_{x\to 0^{-}}g(x)=1
	\end{array}
	\right. \Rightarrow g(x)\text{没有垂直渐近线}$$
	
	其次$g(x)$在$\infty$处水平渐近线:  
	$$\left\lbrace
	\begin{array}{l}
		\lim\limits_{x\to+\infty}g(x)=\lim\limits_{x\to+\infty}f(x)=+\infty\\
		\lim\limits_{x\to-\infty}g(x)=0
	\end{array}
	\right. \Rightarrow g(x)\text{在}x\Rightarrow -\infty\text{时}\text{有水平渐近线}y=0$$
	
	在$x\Rightarrow +\infty$时,$\lim\limits_{x\to+\infty}\dfrac{g(x)}{x^2}=\lim\limits_{x\to +\infty}\dfrac{e^x-1}{2x^2}=+\infty$,$g(x)$不存在斜渐近线,且$g(x)$为无界函数.
	
\end{solution}
\myspace{1}

2. $\int_{0}^{+\infty}\dfrac{\ln x}{1+x^2}dx$
\myspace{1}
\begin{solution}

	原定积分可以化为:  

	\begin{eqnarray*}
		I&=&-\int_{0}^{+\infty}\dfrac{\ln t}{1+t^2}dt\\
		&=&-I\\
		I&=&0
	\end{eqnarray*}
\end{solution}
\myspace{1}

\textcolor{blue}{September 11}

1. $\int_{0}^{+\infty}\dfrac{\ln(1-x^2+x^4)}{(1+x^2)\ln x}dx$
\myspace{1}
\begin{solution}

	原定积分可以化为:  
	\begin{eqnarray*}
		I&=&-\int_{0}^{+\infty}\dfrac{\ln (t^4-t^2+1)}{(1+t^2)\ln t}dt+\int_{0}^{+\infty}\dfrac{4}{1+t^2}dt\\
		&=&-I+2\pi\\
		I&=&\pi
	\end{eqnarray*}
\end{solution}
\myspace{1}

2. 已知函数$f(x)$三阶可导,则下列命题中不是$(0,f(0))$为曲线$y=f(x)$的拐点的必要条件的有:  
\begin{itemize}
	\item \hl{A}. 存在$\delta>0$,$f(x)$在$(-\delta,0)$内单调下降,$f(x)$在$(0,\delta)$内单调增加
	\item \hl{B}. 存在$\delta>0$,当$x\in(-\delta,\delta)$,有$f''(-x)=-f''(x)$
	\item \hl{C}. $f''(0)=0,\ f'''(0)\neq 0$
\end{itemize}
\myspace{1}
\begin{solution}

	拐点是函数凹凸性发生改变的函数点,由函数$f(x)$三阶可导,我们可以得到$f''(x_{0})=0$,且有:  
	$$\left\lbrace
	\begin{array}{l}
		f''(a)\cdot f''(b)<0,\ a\in(x_{0}-\xi,x_{0}),\ b\in(x_{0},x_{0}+\xi)\\
		f'''(x_{0})\text{符号不确定}\\
		f(x)=x^3
	\end{array}
	\right. $$
	
	综上所述,上述三个命题都不正确.
\end{solution}
\myspace{1}

\textcolor{blue}{September 12}

1. 下列命题正确的是:  
\begin{itemize}
	\item A. 若$\lim\limits_{x\to x_{0}}f'(x)=a$,则$f'(x_{0})=a$
	\item B. 若$\lim\limits_{x\to x_{0}}f'(x)$不存在,则$f(x)$在点$x_{0}$处不可导
	\item \hl{C}. 若$f'(x)$在$x=x_{0}$处连续,则在$x_{0}$的某邻域内$f'(x)$存在
	\item D. 若$f(x)$在$x=x_{0}$处可导,$g(x)$在$x=x_{0}$处不可导,则$f(x)\cdot g(x)$在$x=x_{0}$处不可导
\end{itemize}
\myspace{1}
\begin{solution}

	重要反例:  
	
	(1). 函数在某个点无定义,但是这个点处导数极限存在
	$$f(x)=\arctan\dfrac{1}{x},\ f'(x)=-\dfrac{1}{1+x^2}$$
	
	(2). 函数导函数在某点处可导,但是导函数在该点处极限不存在
	$$f(x)=\left\lbrace
	\begin{array}{l}
		x^2\sin\dfrac{1}{x},\ x\neq 0\\
		0
	\end{array}
	\right. ,f'(x)=\left\lbrace
	\begin{array}{l}
		2x\sin\dfrac{1}{x}-\cos\dfrac{1}{x},\ x\neq 0\\
		0
	\end{array}
	\right. $$
\end{solution}
\myspace{1}

2. 证明不等式和求极限

(1). 证明:  $\dfrac{1}{n}-\ln(1+\dfrac{1}{n})<\dfrac{1}{2}\cdot\dfrac{1}{n^2},\ (n\in\mathbb{N}^{*})$

(2). 求$\lim\limits_{n\to+\infty}\prod\limits_{i=1}^{n}\dfrac{n^2+i}{n^2}$
\myspace{1}
\begin{solution}

	(1). 我们构造辅助函数:  $f(x)=x-\ln(1+x)-\dfrac{x^2}{2},\ x\in(0,1]$
	
	我们有:  $f'(x)=1-x-\dfrac{1}{1+x}=\dfrac{-x^2}{1+x}<0$,$f(x)$在$(0,1]$上单调递减,我们得到:  $f(x)<f(x)_{max}=f(0)=0$
	
	我们得到:  $f(x)<0,\ x\in(0,1]\Rightarrow \dfrac{1}{n}-\ln(1+\dfrac{1}{n})<\dfrac{1}{2}\cdot\dfrac{1}{n^2},\ (n\in\mathbb{N}^{*})$
	
	(2). 我们由:  
	$$\left\lbrace
	\begin{array}{l}
		\ln(1+x)<x,\ x\in(0,1]\\
		\ln(1+x)>x-\dfrac{x^2}{2},\ x\in(0,1]
	\end{array}
	\right. \Rightarrow \left\lbrace
	\begin{array}{l}
		\ln(1+\dfrac{i}{n^2})<\dfrac{i}{n^2}\\
		\ln(1+\dfrac{i}{n^2})>\dfrac{i}{n^2}-\dfrac{i^2}{2n^4}
	\end{array}
	\right. $$
	
	我们可以得到:  
	\begin{eqnarray*}
		\lim\limits_{n\to+\infty}\prod\limits_{i=1}^{n}\dfrac{n^2+i}{n^2}=e^{\lim\limits_{n\to+\infty}\sum\limits_{i=1}^{+\infty}\ln(1+\frac{i}{n^2})}\\
		e^{\lim\limits_{n\to+\infty}\sum\limits_{i=1}^{+\infty}\dfrac{i}{n^2}-\dfrac{i^2}{2n^4}}\leq I\leq e^{\lim\limits_{n\to+\infty}\sum\limits_{i=1}^{+\infty}\dfrac{i}{n^2}}\\
		e^{\lim\limits_{n\to+\infty}\frac{n+1}{2n}-\frac{n(n+1)(2n+1)}{12n^4}}\leq I\leq e^{\lim\limits_{n\to+\infty}\frac{n+1}{2n}}
	\end{eqnarray*}
	$\text{左边}=\text{右边}=e^{\frac{1}{2}}$,
	我们由夹逼定理可以得到原式子$I=e^{\frac{1}{2}}$
\end{solution}
\myspace{1}

\textcolor{blue}{September 13}

1. 设区域$D_{1}:\{(x,y)| 0\leq x\leq 1,\ 0\leq y\leq 1\}$,$D_{2}:\{(x,y)| x^2+y^2\leq r^2,\ (r>0)\}$,比较$I=\iint \limits_{D_{1}}\left(\cos x^2+\sin y^2\right) dxdy$,\ $J=\iint\limits_{D_{1}}\sqrt{2}dxdy$,\ $K=\lim\limits_{r\to 0}\dfrac{1}{\pi r^2}\iint\limits_{D_{2}}e^{x^2-y^2}\cos(x+y)dxdy$大小
\myspace{1}
\begin{solution}

	我们首先可以得到:  $J=\sqrt{2}$
	
	我们根据轮换对称性:  
	\begin{eqnarray*}
		I&=&\iint \limits_{D_{1}}\left(\cos x^2+\sin y^2\right) dxdy\\
		&=&\iint \limits_{D_{1}}\left(\cos x^2+\sin x^2\right) dxdy\\
		&=&\iint \limits_{D_{1}}\left(\sqrt{2}\sin(x^2+\frac{\pi}{4})\right)dxdy\\
		&\in&(1,\sqrt{2})
	\end{eqnarray*}

	我们对于$K$,我们根据积分中值定理可以得到:  
	\begin{eqnarray*}
		K&=&\lim\limits_{r\to 0}\dfrac{1}{\pi r^2}e^{\xi^2-\eta^2}\cos(\xi+\eta)\iint\limits_{D_{2}}1dxdy\\
		&=&\lim\limits_{r\to 0}e^{\xi^2-\eta^2}\cos(\xi+\eta)\\
		&=&1
	\end{eqnarray*}

	综上所述,我们得到:  $\left\lbrace
	\begin{array}{l}
		I\in(1,\sqrt{2})\\
		J=\sqrt{2}\\
		K=1
	\end{array}
	\right.\Rightarrow K<I<J$
\end{solution}
\myspace{1}

2. 证明:  若在区间$I$上$f^{(n)}(x)\neq 0$,则方程$f(x)=0$在$I$上最多$n$个实数根
\myspace{1}
\begin{solution}

	我们利用反证法,来证明:  方程$f(x)=0$在区间$I$上有超过$n$个实数根,则$f^{(n)}(x)=0$在区间$I$上有解.
	
	我们不妨假设$\exists x_{1}<x_{2}<\cdots<x_{n}<x_{n+1}\in I$,且满足$f(x_{1})=f(x_{2})=\cdots=f(x_{n})=f(x_{n+1})=0$.
	
	我们对$f(x)$在区间$(x_{1},x_{2}),(x_{2},x_{3}),\cdots,(x_{n},x_{n+1})$上使用罗尔定理:  
	$$\left\lbrace
	\begin{array}{l}
		\exists \xi_{1}\in(x_{1},x_{2}),\ s.t. f'(\xi_{1})=0\\
		\exists \xi_{2}\in(x_{2},x_{3}),\ s.t. f'(\xi_{2})=0\\
		\cdots\cdots\\
		\exists \xi_{n}\in(x_{n},x_{n+1}),\ s.t. f'(\xi_{n})=0\\	
	\end{array}
	\right. $$
	
	我们以此类推,我们可以得到:  $\exists\eta_{1},\eta_{2}\in I,\ s.t. f^{(n-1)}(x)=0$,我们再次使用罗尔定理,可以得到:  
	$\exists \mu\in(\eta_{1},\eta_{2}),\ s.t. f^{(n)}(\mu)=0$
	
	综上所述,原命题的逆否命题成立,原命题成立.
\end{solution}
\myspace{1}

\textcolor{blue}{September 14}

1. 设函数$f(x)$在$[0,+\infty)$上连续,且$(0,+\infty)$上$f'(x)>0$,下列命题正确的是:  
\begin{itemize}
	\item A. 若$f(0)=0$,则$\lim\limits_{x\to 0^{+}}\dfrac{f(x)}{x}$存在
	\item B. 若$\lim\limits_{x\to+\infty}\dfrac{f(x)}{x}=2024$,则$\lim\limits_{x\to +\infty}f'(x)=2024$
	\item \hl{C}. 若$\lim\limits_{x\to+\infty}\dfrac{1}{x}\int_{0}^{x}f(t)dt=1$,则$\lim\limits_{x\to +\infty}f(x)=1$
	\item D. 若$\lim\limits_{x\to+\infty}\dfrac{1}{x}\int_{0}^{x}f(t)dt=1$且$\lim\limits_{x\to+\infty}\int_{0}^{x}f(t)dt=\infty$,则$\lim\limits_{x\to+\infty}f(x)$不存在
\end{itemize}
\myspace{1}
\begin{solution}

	(A). $f(x)=\sqrt{x}$
	
	(B). $f(x)=2024x+\sin x,\ f'(x)=2024+\cos x$,$\lim\limits_{x\to +\infty}f'(x)\text{不存在}$
	
	(C). $\lim\limits_{x\to +\infty}f(x)=\lim\limits_{x\to+\infty}\dfrac{\int_{0}^{x}f(x)dt}{x}$
	
	我们有:  $$\int_{0}^{x}f(t)dt<\int_{0}^{x}f(x)dt<\int_{x}^{2x}f(t)dt\Rightarrow \dfrac{\int_{0}^{x}f(t)dt}{x}<\dfrac{\int_{0}^{x}f(x)dt}{x}<\dfrac{\int_{x}^{2x}f(t)dt}{x}$$
	
	我们有:  $\left\lbrace
	\begin{array}{l}
		\lim\limits_{x\to+\infty}\dfrac{\int_{0}^{x}f(t)dt}{x}=1\\
		\lim\limits_{x\to+\infty}\dfrac{\int_{0}^{2x}f(t)dt}{x}=2
	\end{array}
	\right. $
	
	我们可以得到:  $\lim\limits_{x\to +\infty}f(x)=1$
	
	(D). 我们假设$\lim\limits_{x\to +\infty}f(x)$不存在,我们可以得到:  
	$$\lim\limits_{x\to +\infty}\dfrac{\int_{0}^{x}f(t)dt}{x}=\lim\limits_{x\to+\infty}f(x)\neq 1$$
	
\end{solution}
\myspace{1}

2. 讨论$\int_{0}^{+\infty}\dfrac{x^m|\ln x|^n}{1+x^k}dx$的敛散性
\myspace{1}
\begin{solution}

	我们可知函数可能的暇点为$x=0,\ x=1,\ x=+\infty$
	
	(1). $x=0$
	
	当$k\geq 0$时,$1+x^{k}$非零,原反常积分敛散性等价于:  $\int_{0}^{\frac{1}{2}}x^{m}|\ln x|^ndx$
	$$\left\lbrace
	\begin{array}{l}
		m>-1,\text{反常积分收敛}\\
		m=-1,n<-1\text{反常积分收敛}
	\end{array}
	\right. $$
	
	当$k<0$时,原反常积分敛散性等价于:  $\int_{0}^{\frac{1}{2}}x^{m-k}|\ln x|^ndx$
	$$\left\lbrace
	\begin{array}{l}
		m-k>-1,\text{反常积分收敛}\\
		m-k=-1,n<-1,\text{反常积分收敛}
	\end{array}
	\right. $$
	
	(2). $x=1$
	
	原反常积分敛散性等价于:  $\int_{\frac{1}{2}}^{1}|\ln x|^ndx$和$\int_{1}^{2}|\ln x|^ndx$收敛$\Rightarrow n>-1$
	
	(3).$x=+\infty$
	
	当$k\geq 0$时,原反常积分敛散性等价于:  $\int_{2}^{+\infty}x^{m-k}|\ln x|^ndx$
	$$\left\lbrace
	\begin{array}{l}
		m-k<-1,\text{反常积分收敛}\\
		m-k=-1,n<-1,\text{反常积分收敛}
	\end{array}
	\right. $$
	
	当$k<0$时,原反常积分敛散性等价于:  $\int_{2}^{+\infty}x^{m}|\ln x|^ndx$
	$$\left\lbrace
	\begin{array}{l}
		m<-1,\text{反常积分收敛}\\
		m=-1,n<-1,\text{反常积分收敛}
	\end{array}
	\right. $$
	
	综上所述,我们得到:  $\left\lbrace
	\begin{array}{l}
		n>-1\\
		k<0,k-1<m<-1\\
		k>0,-1<m<k-1
	\end{array}
	\right. $时,原反常积分收敛
\end{solution}
\myspace{1}

\section{Week \Rmnum{3}}
\textcolor{cyan}{September 15}

1. 对于$y=\left(1+x+\dfrac{x^2}{2!}+\cdots+\dfrac{x^n}{n!} \right)e^{-x}$而言,下列说法正确的是:  
\begin{itemize}
	\item A. 当$n$无论为偶数还是奇数时,函数都取极值
	\item B. 当$n$无论为偶数还是奇数时,函数都不取极值
	\item C. 当$n$为偶数时,函数取极值;当$n$为奇数时,函数不取极值
	\item \hl{D}. 当$n$为偶数时,函数不取极值;当$n$为奇数时,函数取极值
\end{itemize}
\myspace{1}
\begin{solution}

	我们对原函数求导,可以得到:  
	$$f'(x)=-\dfrac{x^{n}}{n!}e^{-x}$$
	
	(1). 当$n$为偶数时,$f'(x)\leq 0$,$f(x)$没有极值点
	
	(2). 当$n$为奇数时,$x<0$时,$f'(x)>0$;$x>0$时,$f'(x)<0$,$f(x)$在$x=0$处取极值.
\end{solution}
\myspace{1}

2. 设$f(x)$连续,$g(x)=f(x)\cdot\int_{0}^{x}f(t)dt$单调不增,证明$f(x)\equiv 0$
\myspace{1}
\begin{solution}

	我们构造辅助函数:  $F(x)=\dfrac{1}{2}\left(\int_{0}^{x}f(t)dt \right)^2$
	
	我们有:  $$\left\lbrace
	\begin{array}{l}
		F(0)=0\\
		F'(x)=f(x)\int_{0}^{x}f(t)dt\text{单调不增}\\
		F'(0)=0
	\end{array}
	\right. \Rightarrow \left\lbrace
	\begin{array}{l}
		x\in(-\infty,0),F'(x)\geq 0\\
		x\in(0,+\infty),F'(x)\leq 0
	\end{array}
	\right. \Rightarrow F(x)\leq F(0)=0$$
	
	我们有:  $F(x)=\dfrac{1}{2}\left(\int_{0}^{x}f(t)dt \right)^2\geq 0\Rightarrow F(x)=0\Rightarrow f(x)\equiv 0$
	
	综上所述,我们得到:  $f(x)\equiv 0$
\end{solution}
\myspace{1}

\textcolor{cyan}{September 16}

1. 设函数$f(x)$在$(-\infty,+\infty)$上二阶可导,且$f''(x)\neq 0$,$\lim\limits_{x\to \infty}\dfrac{\sqrt{x^2-x+1}}{xf'(x)}=\alpha>0$,且存在一点$f(x_{0})<0$,证明:  方程$f(x)=0$在$(-\infty,+\infty)$上恰有两个实数根
\myspace{1}
\begin{solution}

	我们可以得到:  $$\left\lbrace
	\begin{array}{l}
		\lim\limits_{x\to+\infty}f'(x)=\dfrac{1}{\alpha}>0\\
		\lim\limits_{x\to-\infty}f'(x)=-\dfrac{1}{\alpha}<0
	\end{array}
	\right. \Rightarrow\left\lbrace
	\begin{array}{l}
		\exists x_{1}\in(0,+\infty),f(x_{1})>0\\
		\exists x_{2}\in(-\infty,0),f(x_{2})>0
	\end{array}
	\right. $$
	
	我们由$f''(x)\neq 0$,且$\left\lbrace
	\begin{array}{l}
		\lim\limits_{x\to+\infty}f'(x)>0\\
		\lim\limits_{x\to-\infty}f'(x)<0
	\end{array}
	\right. \Rightarrow f''(x)>0$
	
	我们可以得到$f'(x)$单调递增,$\exists x_{3}\in(-\infty,+\infty),\ s.t. f'(x_{3})=0$.
	
	我们可以得到$f(x)$在$(-\infty,x_{3})$上单调递减,$(x_{3},+\infty)$上单调递增,且$\exists x_{0},s.t. f(x_{0})<0$.
	
	我们根据单调性可知$f(x)$至多两个零点,根据零点定理可知$(-\infty,x_{0})$和$(x_{0},+\infty)$上都有一个零点,我们得到$f(x)$有且仅有两个零点.
	
	综上所述,我们得到:  方程$f(x)=0$在$(-\infty,+\infty)$上恰有两个实数根
\end{solution}
\myspace{1}

2. 讨论$\int_{0}^{+\infty}\dfrac{(1+x^p)\ln|\ln x|}{x^q}dx$的敛散性
\myspace{1}
\begin{solution}

	我们可以找到函数可能的暇点为$x=0,x=1,x=+\infty$
	
	(1). $x=0$
	
	当$p\geq 0$时,$1+x^{p}$非零,原反常积分敛散性等价于:  $\int_{0}^{\frac{1}{2}}\dfrac{\ln|\ln x|}{x^q}dx$
	$$\left\lbrace
	\begin{array}{l}
		q<1,\text{反常积分收敛}\\
		q=1,n<-1\text{反常积分发散}
	\end{array}
	\right. $$
	
	当$p<0$时,原反常积分敛散性等价于:  $\int_{0}^{\frac{1}{2}}\dfrac{\ln|\ln x|}{x^{q-p}}dx$
	$$\left\lbrace
	\begin{array}{l}
		q-p<1,\text{反常积分收敛}\\
		q-p=1,\text{反常积分发散}
	\end{array}
	\right. $$
	
	(2). $x=1$
	
	原反常积分敛散性等价于:  $\int_{\frac{1}{2}}^{1}\ln|\ln x|dx$和$\int_{1}^{2}\ln|\ln x|dx$收敛
	
	(3).$x=+\infty$
	
	当$p\leq 0$时,原反常积分敛散性等价于:  $\int_{2}^{+\infty}\dfrac{\ln|\ln x|}{x^q}dx$
	$$\left\lbrace
	\begin{array}{l}
		q>1,\text{反常积分收敛}\\
		q=1,,\text{反常积分发散}
	\end{array}
	\right. $$
	
	当$p>0$时,原反常积分敛散性等价于:  $\int_{2}^{+\infty}\dfrac{\ln|\ln x|}{x^{q-p}}dx$
	$$\left\lbrace
	\begin{array}{l}
		q-p>1,\text{反常积分收敛}\\
		q-p=1,\text{反常积分发散}
	\end{array}
	\right. $$
	
	综上所述,我们得到,原反常积分一定发散.
	
\end{solution}
\myspace{1}

\textcolor{cyan}{September 17}

1. 设函数$f(x)$在$x=0$的某个邻域内可导,$g(x)$在$x=0$的某个邻域内连续,且$\lim\limits_{x\to 0}\dfrac{g(x)}{x}=0,\ f'(x)=\sin^2 x+\int_{0}^{x}g(x-t)dt$,则下列命题正确的是:  
\begin{itemize}
	\item A. $x=0$是$f(x)$的极大值点
	\item B. $x=0$是$f(x)$的极小值点
	\item \hl{C}. $(0,f(0))$是$y=f(x)$的拐点
	\item D. $x=0$是$f(x)$的极值点,但$(0,f(0))$不是$y=f(x)$的拐点
\end{itemize}
\myspace{1}
\begin{solution}

	我们首先可以得到:  $g(0)=0,g'(0)=0$,我们有:  
	$$\left\lbrace
	\begin{array}{l}
		f''(x)=2\sin x\cos x+g(x)\\
		f''(0)=0\\
		f'''(0)=\lim\limits_{x\to 0}\dfrac{f''(x)}{x}=\lim\limits_{x\to 0}\dfrac{2\sin x\cos x+g(x)}{x}=2
	\end{array}
	\right. $$
	
	我们可以得到:  $$\left\lbrace
	\begin{array}{l}
		x\in(-\xi,0),f''(x)<0\\
		x\in(0,\xi),f''(x)>0
	\end{array}
	\right. \Rightarrow f'(x)\text{在}(-\xi,0)\text{单调递减};f'(x)\text{在}(0,\xi)\text{单调递增}\Rightarrow f'(x)\geq f'(0)=0$$
	
	我们有:  $x=0$不是$f(x)$的极值点;$(0,f(0))$是$f(x)$的拐点.
\end{solution}
\myspace{1}

2. 设$X_{1},X_{2},\cdots,X_{9}$是来自正态总体$X$的简单随机样本,$Y_{1}=\dfrac{X_{1}+X_{2}+\cdots+X_{6}}{6}$,$Y_{2}=\dfrac{X_{7}+X_{8}+X_{9}}{3}$,$S^2=\dfrac{1}{2}\sum\limits_{i=1}^{9}(X_{i}-Y_{2})^2$,$Z=\dfrac{\sqrt{2}(Y_{1}-Y_{2})}{S}$,证明:  $Z\sim t(2)$
\myspace{1}
\begin{solution}
	
\end{solution}
\myspace{1}

\textcolor{cyan}{September 18}

1.设函数$f(x)$在区间$[-2,2]$上可导,且$f'(x)>f(x)>0$,下列命题正确的是:  
\begin{itemize}
	\item A. $\dfrac{f(-2)}{f(-1)}>1$
	\item \hl{B}. $\dfrac{f(0)}{f(-1)}>e$
	\item C. $\dfrac{f(1)}{f(-1)}<e^2$
	\item D. $\dfrac{f(2)}{f(-1)}<e^3$
\end{itemize}
\myspace{1}
\begin{solution}

	我们构造辅助函数:  $F(x)=\dfrac{f(x)}{e^x}$
	
	我们有:  $F'(x)=\dfrac{f'(x)-f(x)}{e^x}>0$,$F(x)$在$[-2,2]$上单调递增.
	$$\left\lbrace
	\begin{array}{l}
		F(-1)>F(-2)\\
		F(0)>F(-1)\\
		F(1)>F(-1)\\
		F(2)>F(-1)
	\end{array}
	\right. \Rightarrow \left\lbrace
	\begin{array}{l}
		\dfrac{f(-2)}{f(-1)}<\dfrac{1}{e}\\
		\dfrac{f(0)}{f(-1)}>e\\
		\dfrac{f(1)}{f(-1)}>e^2\\
		\dfrac{f(2)}{f(-1)}>e^3
	\end{array}
	\right. $$
\end{solution}
\myspace{1}

2. 设$D=\{(p,q)|\int_{0}^{+\infty}\dfrac{x^p|x-1|^q}{(x+1)\ln x\ln(x+2)}dx\text{收敛}\}$,求$D$绕$q$轴旋转一周扫过的体积
\myspace{1}
\begin{solution}

	我们可以找到函数可能的暇点为$x=0,x=1,x=+\infty$
	
	(1). $x=0$
	
	原反常积分敛散性等价于:  $\int_{0}^{\frac{1}{2}}\dfrac{x^p}{\ln x}dx$
	$$\left\lbrace
	\begin{array}{l}
		p>-1,\text{反常积分收敛}\\
		p\leq-1,\text{反常积分发散}
	\end{array}
	\right. $$
	
	(2). $x=1$
	
	原反常积分敛散性等价于:  $\int_{\frac{1}{2}}^{1}\dfrac{|x-1|^q}{\ln x}dx$和$\int_{1}^{2}\dfrac{|x-1|^q}{\ln x}dx$
	$$\left\lbrace
	\begin{array}{l}
		q-1>-1,\text{反常积分收敛}\\
		q-1\leq-1,\text{反常积分发散}
	\end{array}
	\right. $$
	
	(3).$x=+\infty$
	
	原反常积分敛散性等价于:  $\int_{2}^{+\infty}\dfrac{x^{p+q-1}}{\ln^2 x}dx$
	$$\left\lbrace
	\begin{array}{l}
		p+q-1<-1,\text{反常积分收敛}\\
		p+q-1=-1,\text{反常积分收敛}\\
		p+q-1>-1,\text{反常积分发散}
	\end{array}
	\right. $$
	
	综上所述,我们得到$\left\lbrace
	\begin{array}{l}
		p>-1\\
		q>0\\
		p+q\leq 0
	\end{array}
	\right. $时,原反常积分收敛.
	
	我们可以得到:  $D=\{(p,q)|-1<p<0,0<q<-p\}$,我们可以得到:  
	$$V=\dfrac{2\pi}{3}$$
\end{solution}
\myspace{1}

\textcolor{cyan}{September 19}

1.设$f(x)$是以$T$为周期的连续函数,下列哪些命题是$\int_{0}^{x}f(t)dt$是以$T$为周期的周期函数的充分条件:  
\begin{itemize}
	\item A. $\int_{0}^{T}\left[ f(t)-f(-t)\right]dt=0$
	\item \hl{B}. $f(-x)=f(x)$
	\item \hl{C}. $\int_{0}^{+\infty}f(t)dt$收敛
\end{itemize}
\myspace{1}
\begin{solution}

	我们设$F(x)=\int_{0}^{x}f(t)dt$,已知函数$f(x)$为周期函数,$F(x)$为在周期函数等价于:  
	$$F(x)\text{为周期函数}\Leftrightarrow \int_{0}^{T}f(x)dx=0$$
	
	(1). $f(x)=1$,$F(x)=x$
	
	(2). $f(-x)=-f(x)\Rightarrow f(x)\text{为奇函数}\Rightarrow \int_{-\frac{\pi}{2}}^{\frac{\pi}{2}}f(x)dx=\int_{0}^{T}f(x)dx=0$
	
	(3). $\int_{0}^{+\infty}f(t)dt=\lim\limits_{n\to+\infty}\sum\limits_{i=0}^{n}\int_{iT}^{(i+1)T}f(x)dx=\lim\limits_{n\to+\infty}nA\text{收敛}$
	
	我们得到:  $\int_{0}^{T}f(x)dx=A=0$
\end{solution}
\myspace{1}

2. 设随机变量$X$和$Y$独立同分布,$X$的概率密度为$f_{X}(x)=\left\lbrace
\begin{array}{l}
	\dfrac{2}{\sqrt{\pi}}e^{-x^2},\ x>0\\
	0,\ x<0
\end{array}
\right. $,令$Z=\sqrt{X^2+Y^2}$,求$D(Z)$
\myspace{1}
\begin{solution}
	
\end{solution}
\myspace{1}

\textcolor{cyan}{September 20}

1.已知常数$k\geq \ln2-1$,证明:  $(x-1)(x-\ln^2 x+2k\ln x-1)\geq 0$
\myspace{1}
\begin{solution}

	我们构造辅助函数:  $f(x)=x-\ln^2 x+2k\ln x-1,x>0$
	
	我们有:  $\left\lbrace
	\begin{array}{l}
		f'(x)=\dfrac{x-2\ln x+2k}{x}\\
		f(1)=0
	\end{array}
	\right. $
	
	我们不妨设$g(x)=x-2\ln x+2k$,$g(x)$与$f'(x)$同号,$g'(x)=\dfrac{x-2}{x}$
	
	我们可以得到:  $$\left\lbrace
	\begin{array}{l}
		x\in(0,2),g'(x)<0\\
		x\in(2,+\infty),g'(x)>0
	\end{array}
	\right. \Rightarrow g(x)_{min}=g(2)=2-2\ln2+2k\geq 0$$
	
	我们得到:  $g(x)\geq 0\Rightarrow f'(x)\geq 0\Rightarrow f(x)\text{在}(0,+\infty)\text{上单调递增}$
	
	我们得到:  
	$$\left\lbrace
	\begin{array}{l}
		x\in(0,1),f(x)<0,x-1<0\\
		x\in(1,+\infty),f(x)>0,x-1>0\\
		x=1,f(x)=0,x-1=0
	\end{array}
	\right. \Rightarrow (x-1)f(x)\geq 0$$
	
	综上所述,我们得到:  $(x-1)(x-\ln^2 x+2k\ln x-1)\geq 0$
\end{solution}
\myspace{1}

2. 求$\sum\limits_{n=1}^{+\infty}\dfrac{(n-1)^3}{n(n+1)}x^n$的和函数$S(x)$
\myspace{1}
\begin{solution}

	我们不妨设$a_{n}=\dfrac{(n-1)^3}{n(n+1)}$,我们可以得到:  
	$$\rho=\lim\limits_{n\to+\infty}|\dfrac{a_{n+1}}{a_{n}}|=1\Rightarrow R=\dfrac{1}{\rho}$$
	
	我们得到幂级数的收敛区间为$(-1,1)$,当$x=\pm 1$时,原级数发散,因此原级数的收敛域为$(-1,1)$
	
	我们有:  $a_{n}=\dfrac{(n-1)^3}{n(n+1)}=\dfrac{n^3-3n^3+3n-1}{n^2+n}=n-4+\dfrac{8}{n+1}-\dfrac{1}{n}$
	
	原级数可以等价于:  
	\begin{eqnarray*}
		\sum\limits_{n=1}^{+\infty}\dfrac{(n-1)^3}{n(n+1)}x^n&=&\sum\limits_{n=1}^{+\infty}a_{n}x^n\\
		&=&\sum\limits_{n=1}^{+\infty}(n+1)x^n-5\sum\limits_{n=1}^{+\infty}x^n+8\sum\limits_{n=1}^{+\infty}\dfrac{x^n}{n+1}-\sum\limits_{n=1}^{+\infty}\dfrac{x^n}{n}\\
		&=&(\dfrac{x^2}{1-x})'-\dfrac{5x}{1-x}+\dfrac{8}{x}[-\ln(1-x)-x]+\ln(1-x)\\
		&=&\dfrac{1}{(1-x)^2}-9-\dfrac{5x}{1-x}-\dfrac{8\ln(1-x)}{x}+\ln(1-x)
	\end{eqnarray*}

综上所述,$S(x)=\dfrac{1}{(1-x)^2}-9-\dfrac{5x}{1-x}-\dfrac{8\ln(1-x)}{x}+\ln(1-x), \ x\in(-1,1)$
\end{solution}
\myspace{1}

\textcolor{cyan}{September 21}

1. 设$f(x)=x[\dfrac{1}{x}]$,\ $[\dfrac{1}{x}]$表示不超过$\dfrac{1}{x}$的最大整数,判断$f(x)$的间断点及其类型
\myspace{1}
\begin{solution}

	我们发现$f(x)$可能的间断点在$x=0,\pm \dfrac{1}{n}$,且$x-1<[x]\leq x\Rightarrow \dfrac{1}{x}-1<[\dfrac{1}{x}]<\dfrac{1}{x}$
	
	(1). 当在$x=0$处附近,$f(x)$在$x=0$处无定义
	$$\left\lbrace
	\begin{array}{l}
		\lim\limits_{x\to 0^{+}}f(x)\in(1-x,1)\\
		\lim\limits_{x\to 0^{-}}f(x)\in(1,1-x)
	\end{array}
	\right. \Rightarrow \lim\limits_{x\to 0^{+}}f(x)=\lim\limits_{x\to 0^{-}}f(x)=1$$
	
	$x=0$是$f(x)$的可去间断点
	
	(2). 当$x=\dfrac{1}{n}$处附近:  
	$$\left\lbrace
	\begin{array}{l}
		\lim\limits_{x\to \frac{1}{n}^{+}}f(x)=1-\frac{1}{n}\\
		\lim\limits_{x\to \frac{1}{n}^{-}}f(x)=1
	\end{array}
	\right. \Rightarrow f(x)\text{在}x=\dfrac{1}{n}\text{跳跃间断}$$
	
	(3). 当$x=-\dfrac{1}{n}$处附近:  
	$$\left\lbrace
	\begin{array}{l}
		\lim\limits_{x\to -\frac{1}{n}^{+}}f(x)=1+\frac{1}{n}\\
		\lim\limits_{x\to -\frac{1}{n}^{-}}f(x)=1
	\end{array}
	\right. \Rightarrow f(x)\text{在}x=\dfrac{1}{n}\text{跳跃间断}$$
	
	综上所述,$f(x)$有且仅有一个可去间断点$x=0$,有无数多个跳跃间断点$x=\pm \dfrac{1}{n}$
\end{solution}
\myspace{1}

2. 设$f(x,y)$在全平面上有连续的偏导数,且$x\dfrac{\partial f}{\partial x}+y\dfrac{\partial f}{\partial y}=0$,证明:  $f(x,y)$为常数.
\myspace{1}
\begin{solution}

	我们构造辅助函数:  $g(t)=f(tx,ty)$,我们有:  
	$$\left\lbrace
	\begin{array}{l}
		g'(t)=xf_{1}(tx,ty)+yf_{2}(tx,ty)\\
		tg'(t)=txf_{1}(tx,ty)+tyf_{2}(tx,ty)\\
		x\dfrac{\partial f}{\partial x}+y\dfrac{\partial f}{\partial y}=0
	\end{array}
	\right. tg'(t)=0$$
	
	(1). 当$t\neq 0$,我们可以得到:  $g'(t)=0\Rightarrow g(t)=C$
	
	(2). 当$t\neq 0$时,我们由$f(x,y)$在全平面上有连续的偏导数,可以得到:  $g'(t)$连续$\Rightarrow g'(0)=0$
	
	综上所述,$g'(t)=0\Rightarrow g(t)=C\Rightarrow g(0)=g(1)\Rightarrow f(0,0)=f(x,y)=C$,$f(x,y)$为常数.
\end{solution}
\myspace{1}

\section{Week \Rmnum{4}}
\textcolor{purplea}{September 22}

1.设函数$f(x)$二阶可导,$f(0)=1,f'(0)=0$,且对任意$x\geq 0$,有$f''(x)-5f'(x)+6f(x)\geq 0$,证明不等式$f(x)\geq 3e^{2x}-2e^{3x},\ (x\geq 0)$
\myspace{1}
\begin{solution}

	我们构造辅助函数:  $F(x)=e^{-3x}(f'(x)-2f(x))$,我们有:  
	$$\left\lbrace
	\begin{array}{l}
		F'(x)=e^{-3x}(f''(x)-5f'(x)+6f(x))\geq 0,x\geq 0\\
		F(x)\text{在}(0,+\infty)\text{上单调递增}\\
		F(x)\geq F(0)=-2
	\end{array}
	\right. \Rightarrow f'(x)-2f(x)+2e^{3x}\geq 0$$
	
	我们构造辅助函数:  $G(x)=e^{-2x}(f(x)+2e^{3x}),x\geq 0$,我们有:  
	$$\left\lbrace
	\begin{array}{l}
		G'(x)=e^{-2x}(f'(x)-2f(x)+2e^{3x}),x\geq 0\\
		G'(x)\geq 0\\
		G(x)\geq G(0)=3\\
	\end{array}
	\right. \Rightarrow f(x)\geq 3e^{2x}-2e^{3x}$$
	
	综上所述,我们证明:  $f(x)\geq 3e^{2x}-2e^{3x},\ (x\geq 0)$
\end{solution}
\myspace{1}

2. 求$\sum\limits_{n=1}^{+\infty}\dfrac{n!}{(2n+1)!!(n+1)}$的和
\myspace{1}
\begin{solution}

	我们构造幂级数$\sum\limits_{n=1}^{+\infty}\dfrac{(2n)!!}{(2n+1)!!(n+1)}x^{2n+2}$,我们设该幂级数的和函数为$S(x)$
	
	我们可以得到原和式为$2S(\dfrac{1}{\sqrt{2}})$,我们需要求出幂级数的和函数$S(x)$
	
	我们不妨设$a_{n}=\dfrac{(2n)!!}{(2n+1)!!(n+1)}$,我们可以得到:  
	$$\rho=\lim\limits_{n\to +\infty}|\dfrac{a_{n+1}}{a_{n}}|=1\Rightarrow R^2=1$$
	
	我们可以得到幂级数的收敛区间为$(-1,1)$,我们对幂级数逐项求导可以得到:  
	$$S'(x)=2\sum\limits_{n=1}^{+\infty}\dfrac{(2n)!!}{(2n+1)!!}x^{2n+1}$$
	
	我们不妨设$f(x)=\sum\limits_{n=1}^{+\infty}\dfrac{(2n)!!}{(2n+1)!!}x^{2n+1}$,我们有:  $f'(x)=\sum\limits_{n=1}^{+\infty}\dfrac{(2n)!!}{(2n-1)!!}x^{2n}$
	
	我们对$f'(x)$稍作变形:  
	\begin{eqnarray*}
		f'(x)&=&\sum\limits_{n=1}^{+\infty}\dfrac{(2n)!!}{(2n-1)!!}x^{2n}\\
		&=&x\left[\sum\limits_{n=1}^{+\infty}\dfrac{(2n-2)!!}{(2n-1)!!}x^{2n} \right]'\\
		&=&x\left[ x\left(\sum\limits_{n=1}^{+\infty}\dfrac{(2n-2)!!}{(2n-1)!!}x^{2n-1} \right) \right]'\\
		&=&x\left[ xf(x)+x^2\right]'   
	\end{eqnarray*}

	我们得到:  $f'(x)-\dfrac{x}{1-x^2}f(x)=\dfrac{2x^2}{1-x^2}$
	
	我们可以解得:  $f(x)=\dfrac{\arcsin x}{\sqrt{1-x^2}}-x+C,f(0)=0\Rightarrow C=0$
	
	我们得到:  $S'(x)=\dfrac{2\arcsin x}{\sqrt{1-x^2}}-2x,S(0)=0\Rightarrow S(x)=(\arcsin x)^2-x^2$
	
	我们得到原和式为:  $I=2S(\dfrac{1}{\sqrt{2}})=\dfrac{\pi^2}{8}-1$
\end{solution}
\myspace{1}

\textcolor{purplea}{September 23}

1. 计算$\iint\limits_{D}\left[ x(1-y^3)+y(1+x^3)\right]dxdy$,其中$D=\{(x,y)|x^2+y^2\leq 2x+2y\}$
\myspace{1}
\begin{solution}

	我们根据轮换对称性,可以得到:
	\begin{eqnarray*}
		I&=&\iint\limits_{D}\left[ x(1-y^3)+y(1+x^3)\right]dxdy\\
		&=&\iint\limits_{D}\left[ y(1-x^3)+x(1+y^3)\right]dxdy\\
		&=&\iint\limits_{D}(x+y)dxdy\\
		&=&\iint\limits_{D}(x-1+y-1)dxdy+\iint\limits_{D}2dxdy\\
		&=&2S_{D}\\
		&=&4\pi
	\end{eqnarray*}
\end{solution}
\myspace{1}

2. 设$a_{n}=\int_{0}^{\frac{\pi}{2}}\cos^n x\sin nxdx$,证明:  $a_{n}=\dfrac{1}{2^{n+1}}\sum\limits_{k=1}^{n}\dfrac{2^k}{k},\ (n\geq 1)$
\myspace{1}
\begin{solution}

	我们可以得到:
	$$\left\lbrace
	\begin{array}{l}
		a_{n}=\int_{0}^{\frac{\pi}{2}}\cos^n x\sin nxdx\\
		a_{n-1}=\int_{0}^{\frac{\pi}{2}}\cos^{n-1} x\sin(n-1)xdx
	\end{array}
	\right. \Rightarrow a_{n-1}=a_{n}-\int_{0}^{\frac{\pi}{2}}\cos^{n-1}x\sin x\cos nxdx$$
	
	我们利用分部积分对$a_{n}$进行简化:  
	\begin{eqnarray*}
		a_{n}&=&\int_{0}^{\frac{\pi}{2}}\cos^n x\sin nxdx\\
		&=&-\dfrac{1}{n}\left(\int_{0}^{\frac{\pi}{2}}\cos^n xd\cos nx \right)\\
		&=&-\dfrac{1}{n}\left(\cos^n x\cos nx|_{x=0}^{x=\frac{\pi}{2}}+n\int_{0}^{\frac{\pi}{2}}\cos^{n-1}x\sin x\cos nxdx\right)\\
		&=&\dfrac{1}{n}-\int_{0}^{\frac{\pi}{2}}\cos^{n-1}x\sin x\cos nxdx
	\end{eqnarray*}

	我们可以得到:  $$a_{n}=\dfrac{1}{n}+a_{n-1}-a_{n}\Rightarrow a_{n}-\dfrac{a_{n-1}}{2}=\dfrac{1}{2n}$$
	
	我们有:  $a_{1}=\int_{0}^{\frac{\pi}{2}}\cos x\sin xdx=\dfrac{1}{2}$
	
	$$\left\lbrace
	\begin{array}{l}
		a_{1}=\dfrac{1}{2}\\
		a_{n}-\dfrac{a_{n-1}}{2}=\dfrac{1}{2n}
	\end{array}
	\right. \Rightarrow\text{数列}\{a_{n}\}\text{唯一}$$
	
	我们不妨假设:  $a_{n}=\dfrac{1}{2^{n+1}}\sum\limits_{k=1}^{n}\dfrac{2^k}{k}$,下面用数学归纳法证明:  
	
	(1). 当$n=1$时,$a_{1}=\dfrac{1}{2}$
	
	(2). 当$n=k$时,我们有:  $a_{k}=\dfrac{1}{2^{k+1}}\sum\limits_{i=1}^{k}\dfrac{2^i}{i}$
	$$a_{k+1}=\dfrac{1}{2(k+1)}+\dfrac{1}{2}a_{k}=\dfrac{2^{k+1}}{2^{k+2}(k+1)}+\dfrac{1}{2^{k+2}}\sum\limits_{i=1}^{k}\dfrac{2^i}{i}=\dfrac{1}{2^{k+2}}\sum\limits_{i=1}^{k+1}\dfrac{2^i}{i}$$
	
	综上所述,我们得到:  $a_{n}=\dfrac{1}{2^{n+1}}\sum\limits_{k=1}^{n}\dfrac{2^k}{k},\ (n\geq 1)$
\end{solution}
\myspace{1}

\textcolor{purplea}{September 24}

1. 设三元二次型$f=x^{T}Ax$正定,其中$x=(x_{1},x_{2},x_{3})^{T}$,$A$为实对称矩阵,则下列说法不正确的是:  
\begin{itemize}
	\item A. 仅在$x=0$处$f$取得最小值
	\item B. 齐次方程组$\left\lbrace
	\begin{array}{l}
		\dfrac{\partial f}{\partial x_{1}}=0\\
		\dfrac{\partial f}{\partial x_{2}}=0\\
		\dfrac{\partial f}{\partial x_{3}}=0
	\end{array}
	\right. $只有零解
	\item C. 二阶偏导数矩阵
	$$\left( \dfrac{\partial^2 f}{\partial x_{i}\partial x_{j}}\right)_{3\times 3}=\left( \begin{matrix}
		\dfrac{\partial^2 f}{\partial x_{1}^2}&\dfrac{\partial^2 f}{\partial x_{1}\partial x_{2}}&\dfrac{\partial^2 f}{\partial x_{1}\partial x_{3}}\\
		\dfrac{\partial^2 f}{\partial x_{2}\partial x_{1}}&\dfrac{\partial^2 f}{\partial x_{2}^2}&\dfrac{\partial^2 f}{\partial x_{2}\partial x_{3}}\\
		\dfrac{\partial^2 f}{\partial x_{3}\partial x_{1}}&\dfrac{\partial^2 f}{\partial x_{3}\partial x_{2}}&\dfrac{\partial^2 f}{\partial x_{3}^2}
	\end{matrix}\right) \text{正定}$$
	\item \hl{D}. 存在三维非零列向量$\alpha$,使得$A=\alpha\alpha^{T}$,从而$f=(\alpha^{T}x)^2$
\end{itemize}
\myspace{1}
\begin{solution}

	我们不妨假设$f=a_{11}x_{1}^2+a_{22}x_{2}^2+a_{33}x_{3}^2+2a_{12}x_{1}x_{2}+2a_{13}x_{1}x_{3}+2a_{23}x_{2}x_{3}$,我们可以得到:  
	$$A=\left[ \begin{matrix}
		a_{11}&a_{12}&a_{13}\\
		a_{11}&a_{22}&a_{23}\\
		a_{13}&a_{23}&a_{33}
	\end{matrix}\right] $$

	(1).我们由$f$正定,可以得到:  $f\geq 0$,当且仅当$\left(\begin{matrix}
		x_{1}\\x_{2}\\x_{3}
	\end{matrix} \right) =\left( \begin{matrix}
	0\\0\\0
\end{matrix}\right) $时等号成立.

(2)
$$\left\lbrace
\begin{array}{l}
	\dfrac{\partial f}{\partial x_{1}}=0\\
	\dfrac{\partial f}{\partial x_{2}}=0\\
	\dfrac{\partial f}{\partial x_{3}}=0
\end{array}
\right. \Rightarrow \left\lbrace
\begin{array}{l}
	2a_{11}x_{1}+2a_{12}x_{2}+2a_{13}x_{3}=0\\
	2a_{12}x_{1}+2a_{22}x_{2}+2a_{23}x_{3}=0\\
	2a_{13}x_{1}+2a_{23}x_{2}+2a_{33}x_{3}=0
\end{array}
\right. \Rightarrow \left( \begin{matrix}
	2a_{11}&2a_{12}&2a_{13}\\
	2a_{12}&2a_{22}&2a_{23}\\
	2a_{13}&2a_{23}&2a_{33}
\end{matrix}\right) \left(\begin{matrix}
x_{1}\\x_{2}\\x_{3}
\end{matrix} \right) =\left( \begin{matrix}
0\\0\\0
\end{matrix}\right) \Rightarrow 2Ax=0$$

矩阵$A$为正定矩阵$\Rightarrow |A|\neq 0\Rightarrow$原方程组只有零解.

(3).我们可以得到:  
$$\left( \dfrac{\partial^2 f}{\partial x_{i}\partial x_{j}}\right)_{3\times 3}=\left( \begin{matrix}
	\dfrac{\partial^2 f}{\partial x_{1}^2}&\dfrac{\partial^2 f}{\partial x_{1}\partial x_{2}}&\dfrac{\partial^2 f}{\partial x_{1}\partial x_{3}}\\
	\dfrac{\partial^2 f}{\partial x_{2}\partial x_{1}}&\dfrac{\partial^2 f}{\partial x_{2}^2}&\dfrac{\partial^2 f}{\partial x_{2}\partial x_{3}}\\
	\dfrac{\partial^2 f}{\partial x_{3}\partial x_{1}}&\dfrac{\partial^2 f}{\partial x_{3}\partial x_{2}}&\dfrac{\partial^2 f}{\partial x_{3}^2}
\end{matrix}\right)=\left( \begin{matrix}
2a_{11}&2a_{12}&2a_{13}\\
2a_{12}&2a_{22}&2a_{23}\\
2a_{13}&2a_{23}&2a_{33}
\end{matrix}\right)=2A\text{正定}$$

(4)$|A|\neq 0\Rightarrow rank(A)=3$,假设$A=\alpha\alpha^{T}$,$rank(A)=1$矛盾!!!
\end{solution}
\myspace{1}


2. 证明不等式:  $$|\dfrac{\sin x-\sin y}{x-y}-\cos y|\leq \dfrac{1}{2}|x-y|,\ (x\neq y)$$
\myspace{1}
\begin{solution}

	我们利用带拉格朗日余项的泰勒公式将$f(x)=\sin x$进行展开:  
	$$\sin x=\sin y+(x-y)\cos y+\dfrac{(x-y)^2}{2}\sin\xi, \xi\text{位于}x,y\text{之间}$$
	
	我们可以得到:  
	\begin{eqnarray*}
		|\dfrac{\sin x-\sin y}{x-y}-\cos y|&=&|\dfrac{\sin\xi}{2}(x-y)|\\
		&\leq&\dfrac{1}{2}|x-y|
	\end{eqnarray*}
\end{solution}
\myspace{1}

\textcolor{purplea}{September 25}

1.求$\sum\limits_{n=2}^{+\infty}\dfrac{n^2+3n+4}{2^n(n+1)!}x^{2n}$的和函数$S(x)$
\myspace{1}
\begin{solution}

	我们可以将原级数化为:  $\sum\limits_{n=2}^{+\infty}\dfrac{n^2+3n+4}{(n+1)!}(\dfrac{x^2}{2})^{n}$
	
	我们设$a_{n}=\dfrac{n^2+3n+4}{(n+1)!}$,我们可以得到:  
	$$a_{n}=\dfrac{1}{(n-1)!}+\dfrac{2}{n!}+\dfrac{2}{(n+1)!}$$
	
	我们先求出收敛半径:  
	$$\rho=\lim\limits_{n\to +\infty}|\dfrac{a_{n+1}}{a_{n}}|=0\Rightarrow R=\infty$$
	
	原级数可以化为:  
	\begin{eqnarray*}
		\sum\limits_{n=2}^{+\infty}\dfrac{n^2+3n+4}{2^n(n+1)!}x^{2n}&=&\sum\limits_{n=2}^{+\infty}\dfrac{1}{(n-1)!}(\dfrac{x^2}{2})^{n}+\sum\limits_{n=2}^{+\infty}\dfrac{2}{n!}(\dfrac{x^2}{2})^{n}+\sum\limits_{n=2}^{+\infty}\dfrac{2}{(n+1)!}(\dfrac{x^2}{2})^{n}\\
		&=&\dfrac{x^2}{2}\sum\limits_{n=1}^{+\infty}\dfrac{1}{n!}(\dfrac{x^2}{2})^{n}+2\sum\limits_{n=2}^{+\infty}\dfrac{1}{n!}(\dfrac{x^2}{2})^{n}+\dfrac{4}{x^2}\sum\limits_{n=3}^{+\infty}\dfrac{1}{n!}(\dfrac{x^2}{2})^{n}\\
		&=&\dfrac{x^2}{2}\left[e^{\frac{x^2}{2}}-1\right]+2\left[e^{\frac{x^2}{2}}-1-\dfrac{x^2}{2}\right]+\dfrac{4}{x^2}\left[ e^{\frac{x^2}{2}}-1-\dfrac{x^2}{2}-\dfrac{x^4}{8}\right]\\
		&=&\dfrac{x^2e^{\frac{x^2}{2}}}{2}+2e^{\frac{x^2}{2}}+\dfrac{4e^{\frac{x^2}{2}}}{x^2}-2x^2-4-\dfrac{4}{x^2}   
	\end{eqnarray*}

	综上所述,我们得到:  $S(x)=\left\lbrace
	\begin{array}{l}
		\dfrac{x^2e^{\frac{x^2}{2}}}{2}+2e^{\frac{x^2}{2}}+\dfrac{4e^{\frac{x^2}{2}}}{x^2}-2x^2-4-\dfrac{4}{x^2},\ x\neq 0\\
		0,\ x=0
	\end{array}
	\right. $
\end{solution}
\myspace{1}

2. 设线性方程组$Ax=\alpha$有解,$\left( \begin{matrix}
	A\\B
\end{matrix}\right)x=\left( \begin{matrix}
\alpha\\\beta
\end{matrix}\right) $无解,则下列结论正确的是:  
\begin{itemize}
	\item A. $r(B,\beta)=r(B)+1$
	\item B. $r\left(\begin{matrix}
		A&\alpha\\B&\beta
	\end{matrix}\right)<r\left( \begin{matrix}
	A\\B
\end{matrix}\right) +1$
	\item C. $r\left[B^{T}(B,\beta)\right]>r(B^{T}B)$ 
	\item \hl{D}. $r\left[(A^T,B^T)\left(\begin{matrix}
		A&\alpha\\B&\beta
	\end{matrix} \right) \right] =r\left[ (A^T,B^T)\left( \begin{matrix}
	A\\B
\end{matrix}\right) \right] $
\end{itemize}
\myspace{1}
\begin{solution}

	(1). 方程组$Ax=\alpha$与$Bx=\beta$无公共解,$r(B,\beta)$和$r(B)+1$可能相等,也可能不等
	
	(2). 方程组$\left( \begin{matrix}
		A\\B
	\end{matrix}\right)x=\left( \begin{matrix}
		\alpha\\\beta
	\end{matrix}\right)$无解$\Rightarrow r\left(\begin{matrix}
	A&\alpha\\B&\beta
\end{matrix}\right)=r\left( \begin{matrix}
A\\B
\end{matrix}\right) +1$

	(3). $\left\lbrace
	\begin{array}{l}
		r\left[B^{T}(B,\beta)\right]\leq r(B^{T})\\
		r(B^{T})=r(B)=r(BB^{T})=r(B^{T}B)\\
		r\left[B^{T}(B,\beta)\right]=r(B^{T}B,B^{T}\beta)\geq r(B^{T}B)
	\end{array}
	\right. \Rightarrow r\left[B^{T}(B,\beta)\right]=r(B^{T}B)$
	
	(4). $$\left\lbrace
	\begin{array}{l}
		r\left[(A^T,B^T)\left(\begin{matrix}
			A&\alpha\\B&\beta
		\end{matrix} \right) \right]\geq r\left[ (A^T,B^T)\left( \begin{matrix}
		A\\B
	\end{matrix}\right) \right] \\
		r\left[ (A^T,B^T)\left( \begin{matrix}
			A\\B
		\end{matrix}\right) \right] =r(A^{T},B^{T})\\
	r(A^{T},B^{T})\geq r\left[(A^T,B^T)\left(\begin{matrix}
		A&\alpha\\B&\beta
	\end{matrix} \right) \right]
	\end{array}
	\right. \Rightarrow r\left[(A^T,B^T)\left(\begin{matrix}
		A&\alpha\\B&\beta
	\end{matrix} \right) \right] =r\left[ (A^T,B^T)\left( \begin{matrix}
		A\\B
	\end{matrix}\right) \right] $$
\end{solution}
\myspace{1}

\textcolor{purplea}{September 26}

1. 已知$\lim\limits_{x\to 0^{+}}\dfrac{x^{\ln (3\cos x)}-3^{\ln x}}{x^k\ln x}=c,\ (c\neq 0)$,求$c$、$k$
\myspace{1}
\begin{solution}

	原极限可以化为:  
	\begin{eqnarray*}
		I&=&\lim\limits_{x\to 0^{+}}\dfrac{x^{\ln (3\cos x)}-3^{\ln x}}{x^k\ln x}\\
		&=&\lim\limits_{x\to 0^{+}}\dfrac{e^{\ln x\ln (3\cos x)}-e^{\ln3\ln x}}{x^k\ln x}\\
		&=&\lim\limits_{x\to 0^{+}}\dfrac{e^{\ln3\ln x}[e^{\ln(\cos x)\ln x}-1]}{x^k\ln x}\\
		&=&\lim\limits_{x\to 0^{+}}\dfrac{x^{\ln3}[\ln x\ln(\cos x)]}{x^k\ln x}\\
		&=&\lim\limits_{x\to 0^{+}}\dfrac{-x^{2+\ln3}}{2x^k}\\
		&=&c
	\end{eqnarray*}
	
	我们可以得到:  $\left\lbrace
	\begin{array}{l}
		k=2+\ln3\\
		c=-\dfrac{1}{2}
	\end{array}
	\right. $
\end{solution}
\myspace{1}

2. 设$f(x,y)$在$D=\{(x,y)|x^2+y^2\leq 1\}$内具有连续的偏导数,且在边界上取值为$0$,证明:  
$$f(0,0)=\lim\limits_{\xi\to 0^{+}}\dfrac{-1}{2\pi}\iint\limits_{D_{1}}\dfrac{x\dfrac{\partial f}{\partial x}+y\dfrac{\partial f}{\partial y}}{x^2+y^2}dxdy,\ (D_{1}=\{(x,y)|\xi^{2}\leq x^2+y^2\leq 1\})$$
\myspace{1}
\begin{solution}

	我们利用极坐标代换:  
	$$\left\lbrace
	\begin{array}{l}
		x=r\cos\theta\\
		y=r\sin\theta
	\end{array}
	\right. \Rightarrow f(x,y)=f(r\cos\theta,r\sin\theta)\Rightarrow \dfrac{\partial f}{\partial r}=\dfrac{\partial f}{\partial x}\cos\theta+\dfrac{\partial f}{\partial y}\sin\theta$$
	
	我们得到:  $r\dfrac{\partial f}{\partial r}=x\dfrac{\partial f}{\partial x}+y\dfrac{\partial f}{\partial y}$
	
	原极限可以化为:  
	\begin{eqnarray*}
		I&=&\lim\limits_{\xi\to 0^{+}}\dfrac{-1}{2\pi}\iint\limits_{D_{1}}\dfrac{\partial f}{\partial r}drd\theta\\
		&=&\lim\limits_{\xi\to 0^{+}}\dfrac{-1}{2\pi}\int_{0}^{2\pi}d\theta\int_{\xi}^{1}\dfrac{\partial f}{\partial r}dr\\
		&=&\lim\limits_{\xi\to 0^{+}}\dfrac{-1}{2\pi}\int_{0}^{2\pi}[f(\cos\theta,\sin\theta)-f(\xi\cos\theta,\xi\sin\theta)]d\theta\\
		&=&\lim\limits_{\xi\to 0^{+}}\dfrac{-1}{2\pi}(-2\pi)f(\xi\cos\alpha,\xi\sin\alpha),\alpha\in(0,2\pi)(\text{积分中值定理})\\
		&=&f(0,0)
	\end{eqnarray*}
\end{solution}
\myspace{1}

\textcolor{purplea}{September 27}

1. 设$f(x)$二阶可导,$f(x)>0$,$\lim\limits_{x\to 0}\dfrac{f(x)-1}{x}=1$,且$f(x)f''(x)>[f'(x)]^2$,下列命题一定成立的是:  
\begin{itemize}
	\item \hl{A}. $e^{-x}f(x)\geq 1$
	\item B. $e^{-x}f(x)<1$
	\item C. $\dfrac{\ln f(x)}{x}<1$
	\item D. $\dfrac{\ln f(x)}{x}>1$
\end{itemize}
\myspace{1}
\begin{solution}

	$\lim\limits_{x\to 0}\dfrac{f(x)-1}{x}=1\Rightarrow \left\lbrace
	\begin{array}{l}
		f(0)=1\\
		f'(0)=1
	\end{array}
	\right. $
	我们构造辅助函数:  $F(x)=\ln f(x)$,我们有:  
	$$\left\lbrace
	\begin{array}{l}
		F'(x)=\dfrac{f'(x)}{f(x)}\\
		F''(x)=\dfrac{f''(x)f(x)-[f'(x)]^2}{[f(x)]^2}>0\\
		F(0)=0\\
		F'(0)=1
	\end{array}
	\right. $$
	
	我们将$F(x)$在$x=0$处进行泰勒展开得到:  
	$$F(x)=F(0)+F'(0)x+\dfrac{F''(\xi)}{2}x^2,\ F''(x)>0\Rightarrow F(x)\geq x\Rightarrow e^{-x}f(x)\geq 1$$
\end{solution}
\myspace{1}

2. 求$\sum\limits_{n=1}^{+\infty}n^3x^n$的和函数$S(x)$
\myspace{1}
\begin{solution}

	首先求收敛域,不妨设$a_{n}=n^3$,我们有:  
	$$\rho=\lim\limits_{n\to +\infty}|\dfrac{a_{n+1}}{a_{n}}|=1\Rightarrow R=1$$
	
	当$x=\pm 1$时,原幂级数发散,因此幂级数的收敛域为$(-1,1)$
	
	我们已知幂级数:  $G(x)=\sum\limits_{n=1}^{+\infty}x^{n}=\dfrac{x}{1-x},x\in(-1,1)$
	
	我们可以得到:  
	$$\left\lbrace
	\begin{array}{l}
		G'(x)=\sum\limits_{n=1}^{+\infty}nx^{n-1}=\dfrac{1}{(x-1)^2}\\
		xG'(x)=\sum\limits_{n=1}^{+\infty}nx^{n}=\dfrac{x}{(x-1)^2}
	\end{array}
	\right.\Rightarrow \left\lbrace
	\begin{array}{l}
		[xG'(x)]'=\sum\limits_{n=1}^{+\infty}n^2x^{n-1}=\dfrac{1+x}{(1-x)^3}\\
		x[xG'(x)]'=\sum\limits_{n=1}^{+\infty}n^2x^{n}=\dfrac{x(1+x)}{(1-x)^3}\\
	\end{array}
	\right.\Rightarrow \left\lbrace
	\begin{array}{l}
		\{x[xG'(x)]'\}'=\sum\limits_{n=1}^{+\infty}n^3x^{n-1}=\dfrac{x^2+4x+1}{(x-1)^4}\\
		S(x)=x\{x[xG'(x)]'\}'=\dfrac{x^3+4x^2+x}{(x-1)^4}
	\end{array}
	\right. $$
	
	综上所述,原幂级数的和函数$S(x)=\dfrac{x^3+4x^2+x}{(x-1)^4}$
\end{solution}
\myspace{1}

\textcolor{purplea}{September 28}

1. 求极限$\lim\limits_{x\to +\infty}\left[(x+2)^{1+\frac{1}{x}}-x^{1+\frac{1}{x+\sin x}}\right]$
\myspace{1}
\begin{solution}

	原极限可化为:  
	\begin{eqnarray*}
		I&=&\lim\limits_{x\to +\infty}2(x+2)^{\frac{1}{x}}+\lim\limits_{x\to +\infty}x\left[(x+2)^{\frac{1}{x}}-x^{\frac{1}{x+\sin x}}\right]\\
		&=&2+\lim\limits_{x\to +\infty}x\left[e^{\frac{\ln(x+2)}{x}}-e^{\frac{\ln x}{x+\sin x}}\right]\\
		&=&2+\lim\limits_{x\to +\infty}e^{\xi}x[\frac{\ln(x+2)}{x}-\frac{\ln x}{x+\sin x}](\text{拉格朗日中值定理})\\
		&=&2
	\end{eqnarray*}
\end{solution}
\myspace{1}


2. 设$A$为$n$阶方阵,$\beta$为$n$维非零列向量,且$r\left( \begin{matrix}
	A^T\\\beta^T
\end{matrix}\right)<r\left( \begin{matrix}
A^T&0\\\beta^T&1
\end{matrix}\right)$,下列说法正确的是:  
\begin{itemize}
	\item A. $\left( \begin{matrix}
		A&\beta\\\beta^T&0
	\end{matrix}\right)\left( \begin{matrix}
	x\\y
\end{matrix}\right)=0$必有非零解
	\item B. $\left( \begin{matrix}
		A&\beta\\\beta^T&0
	\end{matrix}\right)\left( \begin{matrix}
		x\\y
	\end{matrix}\right)=0$只有零解
	\item C. 方程组$Ax=\beta$必无解
	\item \hl{D}. 方程组$Ax=\beta$必有解
\end{itemize}
\myspace{1}
\begin{solution}

	$$\left( \begin{matrix}
		A&\beta\\\beta^T&0
	\end{matrix}\right)\left( \begin{matrix}
		x\\y
	\end{matrix}\right)=0\Rightarrow (A,\beta)v=0$$
	我们由:  $r\left( \begin{matrix}
		A^T\\\beta^T
	\end{matrix}\right)<r\left( \begin{matrix}
		A^T&0\\\beta^T&1
	\end{matrix}\right)$得到:  

	(1). 当$r\left( \begin{matrix}
		A^T&0\\\beta^T&1
	\end{matrix}\right)=n+1$时,$(A,\beta)v=0$只有零解$\Rightarrow \left( \begin{matrix}
	A&\beta\\\beta^T&0
\end{matrix}\right)\left( \begin{matrix}
x\\y
\end{matrix}\right)=0$只有零解

	(2). 当$r\left( \begin{matrix}
		A^T&0\\\beta^T&1
	\end{matrix}\right)<n+1$时,$\left( \begin{matrix}
	A&\beta\\\beta^T&0
\end{matrix}\right)\left( \begin{matrix}
x\\y
\end{matrix}\right)=0$必有非零解

	以上两种情况我们都有:  $r\left( \begin{matrix}
		A^T\\\beta^T
	\end{matrix}\right)<r\left( \begin{matrix}
		A^T&0\\\beta^T&1
	\end{matrix}\right)<\left( \begin{matrix}
	A^T\\\beta^T
\end{matrix}\right)+1\Rightarrow r(A,\beta)=r(A)$,我们可以得到:  方程组$Ax=\beta$一定有解
\end{solution}
\myspace{1}

\textcolor{purplea}{September 29}

1. 设$X\sim N(0,1)$,求$E(X^4e^{X})$
\myspace{1}
\begin{solution}
	
\end{solution}
\myspace{1}

2.设$f(x)$为非负连续函数,且当$x>0$时,有$\int_{0}^{x}f(x)f(x-t)dt=x^3$,求$f(x)$ 
\myspace{1}
\begin{solution}

	我们不妨设$F(x)=\int_{0}^{x}f(t)dt,F(0)=0$,上述条件可化为:  
	$$\int_{0}^{x}f(x)f(x-t)dt=x^3\Rightarrow f(x)\int_{0}^{x}f(t)dt=x^3\Rightarrow F(x)F'(x)=x^3\Rightarrow F(x)=\dfrac{x^2}{\sqrt{2}}$$
	$$f(x)=F'(x)=\sqrt{2}x$$
\end{solution}
\myspace{1}

\textcolor{purplea}{September 30}

1. $a_{n+2}=\dfrac{2}{n+2}a_{n+1}+\dfrac{3}{(n+2)(n+1)}a_{n}(n\geq 0)$,$a_{0}=0,a_{1}=1$,已知$a_{3}>a_{4}>a_{5}$,求$\sum\limits_{n=0}^{+\infty}a_{n}x^n$的和函数$S(x)$
\myspace{1}
\begin{solution}

	我们利用数学归纳法证明$\{a_{n}\}(n\geq 3)$单调递减:  
	
	(1). 当$n=3,4,5$,$a_{3}>a_{4}>a_{5}$
	
	(2). 假设$n=k(k\geq 3)$时,$a_{k+1}<a_{k}$,我们有:  
	$$a_{k+2}=\dfrac{2}{k+2}a_{k+1}+\dfrac{3}{(k+2)(k+1)}a_{k}<\dfrac{2}{k+1}a_{k}+\dfrac{3}{(k+1)k}a_{k-1}=a_{k+1}$$
	
	我们可以得到$\{a_{n}\}(n\geq 3)$单调递减,$a_{n}>0$,$\lim\limits_{n\to+\infty}a_{n}$存在
	
	我们假设$\lim\limits_{n\to+\infty}\dfrac{a_{n+1}}{a_{n}}=A$存在,我们有:  
	$$A=A\lim\limits_{n\to+\infty}\dfrac{2}{n+2}+\lim\limits_{n\to+\infty}\dfrac{3}{(n+2)(n+1)}\Rightarrow A=0\Rightarrow \text{收敛域}(-\infty,+\infty)$$
	
	我们可以得到:  
	$$(n+2)(n+1)a_{n+2}=2(n+1)a_{n+1}+3a_{n}\Rightarrow (a_{n+2}x^{n+2})''=2(a_{n+1}x^{n+1})'+3a_{n}x^n$$
	
	我们得到:  $$\sum\limits_{n=0}^{+\infty}(a_{n+2}x^{n+2})''=2\sum\limits_{n=0}^{+\infty}(a_{n+1}x^{n+1})'+3\sum\limits_{n=0}^{+\infty}a_{n}x^n$$
	
	我们进一步得到:  
	$$[S(x)-a_{0}-a_{1}x]''=2[S(x)-a_{0}]'+3S(x)\Rightarrow S''(x)=2S'(x)+3S(x),\left\lbrace
	\begin{array}{l}
		S(0)=a_{0}=0\\
		S'(0)=a_{1}=1
	\end{array}
	\right. $$
	
	我们可以得到:  $S(x)=\dfrac{e^{3x}-e^{-x}}{4}$
\end{solution}
\myspace{1}

2. 设$f(x)$在$[0,1]$上二阶可导,且$\lim\limits_{x\to 0^{+}}\dfrac{f(x)}{x}=2$,$\int_{0}^{1}f(x)dx=1$.

(1). 证明:  $\exists \xi\in(0,1),\ s.t. f'(\xi)=f(\xi)-2\xi+2$

(2). 证明:  $\exists \eta\in(0,1),\ s.t. f''(\eta)=0$

(3). 证明:  $\exists \zeta\in(0,1),\ s.t. \int_{0}^{\zeta}f(t)dt+\zeta f(\zeta)=2\zeta$

(4). 证明:  $\exists \mu\in(0,1),\ s.t. \mu f(\mu)=2\int_{0}^{\mu}f(t)dt$
\myspace{1}
\begin{solution}

	我们由题意可得:  
	$$\left\lbrace
	\begin{array}{l}
		\lim\limits_{x\to 0^{+}}\dfrac{f(x)}{x}=2\\
		\int_{0}^{1}f(x)dx=1
	\end{array}
	\right. \Rightarrow \left\lbrace
	\begin{array}{l}
		f(0)=0\\
		f'(0)=2\\
		\exists c\in(0,1),\ s.t. f(c)=1
	\end{array}
	\right. $$
	
	(1)我们构造辅助函数:  $F(x)=\dfrac{f(x)-2x}{e^x}$,我们可以得到:  
	$$\left\lbrace
	\begin{array}{l}
		F(0)=0\\
		\int_{0}^{1}[f(x)-2x]=0\\
		F'(x)=e^{-x}[f'(x)-2-f(x)+2x]
	\end{array}
	\right. \Rightarrow \left\lbrace
	\begin{array}{l}
		F(0)=0\\
		\exists c\in(0,1),\ s.t. F(c)=0(\text{积分中值定理})
	\end{array}
	\right. $$
	
	我们对$F(x)$在$(0,c)$上使用罗尔定理可以得到:  
	$$\exists \xi\in(0,c)\subset(0,1),\ s.t. F'(\xi)=0\Rightarrow f'(\xi)=f(\xi)-2\xi+2$$
	
	(2). 我们构造辅助函数:  $G(x)=f(x)-2x$,我们可以得到:  
	$$\left\lbrace
	\begin{array}{l}
		G(0)=0\\
		G'(0)=0\\
		\int_{0}^{1}G(x)dx=0\Rightarrow \exists a\in(0,1),s.t. G(a)=0
	\end{array}
	\right. \Rightarrow G(0)=G(d)=0$$
	
	我们对$G(x)$在$(0,a)$上使用罗尔定理得到:  
	$$\exists b\in(0,a),s.t. G'(b)=0$$
	
	我们对$G'(x)$在$(0,b)$上使用罗尔定理得到:  
	$$\exists \eta\in(0,b),s.t. G''(\eta)=0\Rightarrow f''(\eta)=0$$
	
	(3). 我们构造辅助函数:  $H(x)=x\int_{0}^{x}f(t)dt-x^2$,我们有:  
	$$\left\lbrace
	\begin{array}{l}
		H(0)=H(1)=0\\
		H'(x)=\int_{0}^{x}f(t)dt+xf(x)-2x
	\end{array}
	\right. $$
	
	我们对$H(x)$在$(0,1)$上使用罗尔定理可以得到:  
	$$\exists \zeta\in(0,1),\ s.t. H'(\zeta)=0\Rightarrow \int_{0}^{\zeta}f(t)dt+\zeta f(\zeta)=2\zeta$$
	
	(4). 我们构造辅助函数:  $P(x)=\left\lbrace
	\begin{array}{l}
		\dfrac{\int_{0}^{x}f(t)dt}{x^2},\ x\in(0,1]\\
		1,x=0
	\end{array}
	\right. \Rightarrow \lim\limits_{x\to 0^{+}}P(x)=1$
	
	我们可以得到:  $P(x)$在$[0,1]$上连续,$(0,1)$上可导,我们还可以得到:  
	$$\left\lbrace
	\begin{array}{l}
		P(0)=P(1)=1\\
		P'(x)=\dfrac{xf(x)-2\int_{0}^{x}f(t)dt}{x^3}
	\end{array}
	\right. $$
	
	我们对$P(x)$在$(0,1)$上使用罗尔定理可以得到:  
	$$\exists \mu\in(0,1),\ s.t. P'(\mu)=0\Rightarrow \mu f(\mu)=2\int_{0}^{\mu}f(t)dt$$
\end{solution}
