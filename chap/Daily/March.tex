\chapterimage{chap29.jpg}
\chapter{March}
\section{Week \Rmnum{1}}
\hl{\textbf{\textit{March 1}}}

1. 求不定积分 $\int\dfrac{1}{\cos^{3} x}dx$
\myspace{1}

2. 求不定积分 $\int\dfrac{2+x}{(1+x^{2})^{2}}dx$
\myspace{1}
\hl{\textbf{\textit{March 2}}}

1. 求不定积分 $\int\dfrac{\arcsin\sqrt{x}+\ln x}{\sqrt{x}}dx$
\myspace{1}

2. 求不定积分 $\int e^{x}\arcsin\sqrt{1-e^{2x}}dx$
\myspace{1}
\hl{\textbf{\textit{March 3}}}

1. 已知 $f(\ln x)=\dfrac{\ln(1+x)}{x}$, 求 $\int f(x)dx$
\myspace{1}

2. 求定积分 $\int_{0}^{\frac{\pi}{2}}x\sin^{2}x\cos^{2}xdx$
\myspace{1}
\hl{\textbf{\textit{March 4}}}

1. 求定积分 $\int_{0}^{\pi^{2}}\sqrt{x}\cos\sqrt{x}dx$
\myspace{1}

2. 求定积分 $\int_{0}^{2}x\sqrt{2x-x^{2}}dx$
\myspace{1}
\hl{\textbf{\textit{March 5}}}

1. 求定积分 $\int_{0}^{1}x(1-x^{4})^{\frac{3}{2}}dx$
\myspace{1} 

2. 求定积分 $\int_{1}^{2}(x-1)^{2}(x-2)^{2}dx$
\myspace{1}
\hl{\textbf{\textit{March 6}}}

1. 求定积分 $\int_{-2}^{2}x\ln(1+e^{x})dx$
\myspace{1}

2. 设 $D$ 是由曲线 $xy+1=0$ 与直线 $y+x=0$ 及 $y=2$ 围成的有界区域,求 $D$ 的面积
\myspace{1}
\hl{\textbf{\textit{March 7}}}

1. 设 $D$ 是由曲线 $y=x^{2}$ 与 $y=x$ 围成的有界区域,求区域 $D$ 分别绕直线 $y=0,x=0,x=1,x=2$ 旋转所得旋转体的体积
\myspace{1}

2. 方程 $y''+4y'+4y=e^{-2x}$ 满足条件 $y(0)=1,y'(0)=0$ 的特解
\myspace{1}
\section{Week \Rmnum{2}}
\hl{\textbf{\textit{March 8}}}

1. 具有特解 $y_{1}=e^{-x},y_{2}=2xe^{-x},y_{3}=3e^{x}$ 的三阶常系数线性齐次方程为:
\begin{itemize}
	\item A. $y'''-y''-y'+y=0$
	\item B. $y'''+y''-y'-y=0$
	\item C. $y'''-6y''+11y'-6y=0$
	\item D. $y'''-2y''-y'+2y=0$
\end{itemize}
\myspace{1}

2. 方程 $y''-2y'=xe^{2x}$ 的特解形式为:
\begin{itemize}
	\item A. $y=axe^{2x}$
	\item B. $y=(ax+b)e^{2x}$
	\item C. $y=x(ax+b)e^{2x}$
	\item D. $y=x^{2}(ax+b)e^{2x}$
\end{itemize}
\myspace{1}
\hl{\textbf{\textit{March 9}}}

1. 方程 $y''+y=e^{x}+1+\sin x$ 的特解形式为:
\begin{itemize}
	\item A. $ae^{x}+b+c\sin x$
	\item B. $ae^{x}+b+c\cos x+d\sin x$
	\item C. $ae^{x}+b+x(c\cos x+d\sin x)$
	\item D. $y=ae^{x}+b+cx\sin x$
\end{itemize}
\myspace{1}

2. 设函数 $f(x)$ 具有一阶连续导数, 且满足 $f(x)=\int_{0}^{x}(x^{2}-t^{2})f'(t)dt+x^{2}$, 求 $f(x)$ 表达式
\myspace{1}
\hl{\textbf{\textit{March 10}}}

1. 设 $L$ 是一条平面曲线, 其上任意一点 $P(x,y)(x>0)$ 到坐标原点的距离恒等于该点处切线在 $y$ 轴上的截距, 且 $L$ 经过 $(\dfrac{1}{2},0)$, 求曲线 $L$ 的渐近线方程为
\myspace{1}

2. 二元函数 $f(x,y)=
\begin{cases}
	\dfrac{x^{2}y}{x^{2}+y^{2}},&(x,y)\neq (0,0)\\
	0,&(x,y)=(0,0)
\end{cases}$ 在点 $(0,0)$ 处:
\begin{itemize}
	\item A. 不连续
	\item B. 两个偏导数都不存在
	\item C. 偏导数存在但不可微
	\item D. 可微
\end{itemize}
\myspace{1}
\hl{\textbf{\textit{March 11}}}

1. 二元函数 $f(x)$ 在点 $(x_{0},y_{0})$ 处两个偏导数 $f'_{x}(x_{0},y_{0}),f'_{y}(x_{0},y_{0})$ 存在, 是 $f(x)$ 在点 $(x_{0},y_{0})$ 处连续的:
\begin{itemize}
	\item A. 充分不必要条件
	\item B. 必要不充分条件
	\item C. 充分必要条件
	\item D. 不充分不必要条件
\end{itemize}
\myspace{1}

2. 已知 $f(x,y)=\sin\sqrt{x^{4}+y^{4}}$, 则:
\begin{itemize}
	\item A. $f'_{x}(0,0),f'_{y}(0,0)$ 都存在
	\item B. $f'_{x}(0,0)$ 不存在, $f'_{y}(0,0)$ 存在
	\item C. $f'_{x}(0,0)$ 存在, $f'_{y}(0,0)$ 不存在
	\item D. $f'_{x}(0,0),f'_{y}(0,0)$ 都不存在
\end{itemize}
\hl{\textbf{\textit{March 12}}}

1. 设 $f(x,y)=\dfrac{2x+y^{2}}{1+y^{2}\sqrt{1+x^{2}+y^{2}}}$, 则 $d f(0,0)$
\myspace{1}

2. 已知 $dF(x,y)=xye^{x}dx+(f(x)+y^{2})dy$, 且 $f(x)$ 有连续一阶导数, $f(x)=0$, 求 $F(x,y)$
\myspace{1}
\hl{\textbf{\textit{March 13}}}

1. 设函数 $f(x,y)$ 可微, 且对于任意 $x,y$ 都有 $\dfrac{\partial f(x,y)}{\partial x}>0,\dfrac{\partial f(x,y)}{\partial y}<0$, 则下列结论正确的是:
\begin{itemize}
	\item A. $f(1,1)>f(0,0)$
	\item B. $f(-1,1)>f(0,0)$
	\item C. $f(-1,-1)>f(0,0)$
	\item D. $f(1,-1)>f(0,0)$
\end{itemize}
\myspace{1}

2. 设 $z=(x+e^{y})^{x}$, 求 $\dfrac{\partial z}{\partial x}_{(1,0)}$
\myspace{1}
\hl{\textbf{\textit{March 14}}}

1. 设函数 $z=z(x,y)$ 由方程 $(x+1)z+y\ln z-\arctan(2xy)=1$ 确定, 求$\dfrac{\partial z}{\partial x}_{(0,2)}$
\myspace{1}

2. 设 $f(x,y,z)=e^{x}+y^{2}z$, 其中 $z=z(x,y)$ 是由方程 $x+y+z+xyz=0$ 所确定的隐函数, 求 $f'_{x}(0,1,-1)$
\myspace{1}
\section{Week \Rmnum{3}}
\hl{\textbf{\textit{March 15}}}

1. 设 $z=xyf(\dfrac{y}{x})$, 其中 $f(u)$ 可导, 求 $xz'_{x}+yz'_{y}$
\myspace{1}

2. 设 $z=e^{xy}+f(x+y,xy)$, 求 $\dfrac{\partial^{2} z}{\partial x\partial y}$, 其中 $f(u,v)$ 有二阶连续偏导数
\myspace{1}
\hl{\textbf{\textit{March 16}}}

1. 已知 $z=f(x,y)$ 在 $(x_{0},y_{0})$ 处取得极小值, 则:
\begin{itemize}
	\item A. $f'_{x}(x_{0},y_{0})=f'_{y}(x_{0},y_{0})=0$
	\item B. $f''_{xx}(x_{0},y_{0})f''_{yy}(x_{0},y_{0})-(f''_{xy}(x_{0},y_{0}))^{2}>0$, 且 $f''_{xx}(x_{0},y_{0})>0$
	\item C. $f(x_{0},y)$ 在 $y_{0}$ 处取得极大值
	\item D. $f(x,y_{0})$ 在 $x_{0}$ 处取得极小值
\end{itemize}
\myspace{1}

2. 设函数 $f(x),g(x)$ 均有二阶连续导数,满足 $f(0)>0,g(0)<0$, 且 $f'(0)=g'(0)=0$, 则函数 $z=f(x)g(y)$ 在点 $(0,0)$ 处取得极小值的一个充分条件是:
\begin{itemize}
	\item A. $f''(0)<0,g''(0)>0$
	\item B. $f''(0)<0,g''(0)<0$
	\item C. $f''(0)>0,g''(0)>0$
	\item D. $f''(0)>0,g''(0)<0$
\end{itemize}
\myspace{1}
\hl{\textbf{\textit{March 17}}}

1. 已知函数 $z=f(x,y)$ 的全微分 $dz=(ay-x^{2})dx+(ax-y^{2})dy,(a>0)$, 则函数 $f(x,y)$:
\begin{itemize}
	\item A. 无极值点
	\item B. 点 $(a,a)$ 为极小值点
	\item C. 点 $(a,a)$ 为极大值点
	\item D. 是否有极值点与 $a$ 的取值有关
\end{itemize}
\myspace{1}

2. 设函数 $z=f(xy,yg(x))$, 其中 $f$ 函数具有二阶连续偏导数, 函数 $g(x)$ 可导且在 $x=1$ 处取得极值 $g(1)=1$, 求 $\dfrac{\partial^{2}z}{\partial x\partial y}|_{(1,1)}$
\myspace{1}
\hl{\textbf{\textit{March 18}}}

1. 求函数 $f(x,y)=xe^{-\frac{x^{2}+y^{2}}{2}}$ 的极值
\myspace{1}

2. 求 $f(x,y)=x^{2}-y^{2}+2$ 在椭圆域 $D=\left\lbrace (x,y)|x^{2}+\dfrac{y^{2}}{4}\leq 1\right\rbrace$ 上的最大值和最小值
\myspace{1}
\hl{\textbf{\textit{March 19}}}

1. 交换二次积分的积分次序($a>0$)

(1). $\int_{0}^{2}dx\int_{\frac{x^{2}}{4}}^{3-x}f(x,y)dy$

(2). $\int_{0}^{a}dy\int_{0}^{\sqrt{ay}}f(x,y)dx+\int_{a}^{2a}dy\int_{0}^{2a-y}dx$
\myspace{1}

2. 设 $f(x)$ 是连续函数, 则 $\int_{0}^{1}dy\int_{-\sqrt{1-y^{2}}}^{1-y}f(x,y)dx$ 等价于:
\begin{itemize}
	\item A. $\int_{0}^{1}dx\int_{0}^{x-1}f(x,y)dy+\int_{-1}^{0}dx\int_{0}^{\sqrt{1-x^{2}}}f(x,y)dy$
	\item B. $\int_{0}^{1}dx\int_{0}^{1-x}f(x,y)dy+\int_{-1}^{0}dx\int_{-\sqrt{1-x^{2}}}^{0}f(x,y)dy$
	\item C. $\int_{0}^{\frac{\pi}{2}}d\theta\int_{0}^{\frac{1}{\cos\theta+\sin\theta}}f(r\cos\theta,r\sin\theta)dr-\int_{\frac{\pi}{2}}^{\pi}d\theta\int_{0}^{1}f(r\cos\theta,r\sin\theta)dr$
	\item D. $\int_{0}^{\frac{\pi}{2}}d\theta\int_{0}^{\frac{1}{\cos\theta+\sin\theta}}f(r\cos\theta,r\sin\theta)rdr-\int_{\frac{\pi}{2}}^{\pi}d\theta\int_{0}^{1}f(r\cos\theta,r\sin\theta)rdr$
\end{itemize}
\myspace{1}
\hl{\textbf{\textit{March 20}}}

1. 设函数 $f(t)$ 连续,则二次积分 $\int_{0}^{\frac{\pi}{2}}d\theta\int_{2\cos\theta}^{2}f(r^{2})rdr$
\begin{itemize}
	\item A. $\int_{0}^{2}dx\int_{\sqrt{2x-x^{2}}}^{\sqrt{4-x^{2}}}\sqrt{x^{2}+y^{2}}f(x^{2}+y^{2})dy$
	\item B. $\int_{0}^{2}dx\int_{\sqrt{2x-x^{2}}}^{\sqrt{4-x^{2}}}f(x^{2}+y^{2})dy$
	\item C. $\int_{0}^{2}dy\int_{1+\sqrt{1-y^{2}}}^{\sqrt{4-y^{2}}}\sqrt{x^{2}+y^{2}}f(x^{2}+y^{2})dx$
	\item D. $\int_{0}^{2}dy\int_{1+\sqrt{1-y^{2}}}^{\sqrt{4-y^{2}}}f(x^{2}+y^{2})dx$
\end{itemize}
\myspace{1}

2. 计算二重积分

(1). $\iint\limits_{x^{2}+y^{2}\leq 1}(2x+3y)^{2}d\sigma$

(2). $\int_{\frac{1}{4}}^{\frac{1}{2}}dy\int_{\frac{1}{2}}^{\sqrt{y}}e^{\frac{y}{x}}dx+\int_{\frac{1}{2}}^{1}dy\int_{y}^{\sqrt{y}}e^{\frac{y}{x}}dx$

(3). $\int_{0}^{1}dy\int_{y}^{1}\sqrt{x^{2}-y^{2}}dx$

(4). $\int_{0}^{1}dy\int_{y}^{1}\left(\frac{e^{x^{2}}}{x}-e^{y^{3}}\right)dx$

(5). $\iint\limits_{D}\left(xy^{5}-1\right)dxdy,D=\left\{(x,y)|-\frac{\pi}{2}\leq x \leq \frac{\pi}{2},\sin x\leq y \leq 1 \right\}$

(6). $\iint\limits_{D}x^{2}ydxdy$, 其中 $D$ 是由双曲线 $x^{2}-y^{2}=1$以及直线 $y=0,y=1$ 所围成的平面区域

(7). $\iint\limits_{D}\sqrt{x^{2}+y^{2}}dxdy,D=\left\{(x,y)|0\leq y \leq x, x^{2}+y^{2}\leq 2x\right\}$

(8). $\iint\limits_{D}r^{2}\sin\theta\sqrt{1-r^{2}\cos 2\theta}drd\theta,D=\left\{(r,\theta)|0\leq r\leq \sec\theta,0\leq \theta \leq \frac{\pi}{4}\right\}$

\myspace{1}

\hl{\textbf{\textit{March 21}}}

1. $\lim\limits_{n\rightarrow +\infty}\int_{0}^{1}\dfrac{nx^n}{1+e^x}dx$
\myspace{1}
\begin{lemma}[第一积分中值定理]\label{lem: 第一积分中值定理}
	
	$$\lim\limits_{n\rightarrow +\infty}\int_{0}^{1}x^nf(x)dx$$
	
	其中$f(x)$在$[0,1]$上连续可导,则: $\lim\limits_{n\rightarrow +\infty}\int_{0}^{1}x^nf(x)dx=0$
	$f(x)$在$[0,1]$上连续可导,$|f(x)|\leq M$
	因此: 
	$$0\leq\int_{0}^{1}x^nf(x)dx\leq\int_{0}^{1}Mx^ndx=\frac{M}{n+1}$$
	
	由夹逼准则得到: 
	$$\lim\limits_{n\rightarrow +\infty}0=0\quad\quad \lim\limits_{n\rightarrow +\infty}\frac{M}{n+1}=0$$
	$$\lim\limits_{n\rightarrow +\infty}\int_{0}^{1}x^nf(x)dx=0$$
	
	利用不等式: $x>\sin x;x>\ln(1+x)$,可以将上面的式子进行一些变换
	
\end{lemma}
\begin{lemma}
	
	$$\lim\limits_{n\rightarrow +\infty}\int_{0}^{1}nx^nf(x)dx$$
	
	利用 $\mathbf{lem: }$ \ref{lem: 第一积分中值定理},原极限可以化为: 
	$$\lim\limits_{n\rightarrow +\infty}\int_{0}^{1}(n+1)x^nf(x)dx=\lim\limits_{n\rightarrow +\infty}\int_{0}^{1}f(x)dx^n=\lim\limits_{n\rightarrow+\infty}[(f(x)x^n)|_{0}^{1}-\int_{0}^{1}f'(x)x^ndx]=f(1)$$
\end{lemma}
\begin{solution}
	
	令: $f(x)=\dfrac{1}{1+e^x}$
	
	利用引理,我们得到: $$\lim\limits_{n\rightarrow +\infty}\int_{0}^{1}\dfrac{nx^n}{1+e^x}dx=f(1)=\dfrac{1}{1+e}$$
\end{solution}
\myspace{1}

2. $\lim\limits_{x\rightarrow 0}\left[ \dfrac{1}{\ln(x+\sqrt{1+x^2})}-\dfrac{1}{\ln(1+x)}\right] $
\myspace{1}
\begin{solution}
	
	两个等价无穷小:  $$x\rightarrow 0,\sqrt{1+x^{2}}-1\sim \dfrac{1}{2}x^{2}, x\sim \ln(x+\sqrt{1+x^2})$$
	
	设 $f(x)=\ln x,f'(x)=\frac{1}{x}$
	
	原极限可以化为: 
	$$\lim\limits_{x\rightarrow 0}-\frac{\ln(x+\sqrt{1+x^2})-\ln(1+x)}{\ln(x+\sqrt{1+x^2})\ln(1+x)}\stackrel{Lagrange}{\longrightarrow}\lim\limits_{x\rightarrow 0}-\frac{f'(\varphi)(\sqrt{1+x^2}-1)}{x^2}=\lim\limits_{x\rightarrow 0}-\frac{1}{2}f'(\varphi)$$
	$$1+x<\varphi<x+\sqrt{1+x^2},\lim\limits_{x\rightarrow 0}(x+1)=\lim\limits_{x\rightarrow 0}\sqrt{1+x^2}=1\stackrel{squeeze theorem}{\longrightarrow} f'(\varphi)=1$$
	
	原极限: $\lim\limits_{x\rightarrow 0}\left[ \dfrac{1}{\ln(x+\sqrt{1+x^2})}-\dfrac{1}{\ln(1+x)}\right] =-\dfrac{1}{2}$
	
\end{solution}
\myspace{1}

\section{Week \Rmnum{4}}

\hl{\textbf{\textit{March 22}}}

1. $\int\dfrac{\sin x}{\sqrt{2+\sin 2x}}dx$\label{problem: 组合积分法}
\myspace{1}
\begin{solution}
	
	我们有: $2+\sin 2x=1+(\sin x+\cos x)^2$
	
	不妨设 $I=A(\sin x+\cos x),J=B(\cos x-\sin x)$ 
	
	$ I+J=\sin x\Rightarrow A=\dfrac{1}{2},B=-\dfrac{1}{2}$
	
	原不定积分化为: 
	$$\int\frac{I+J}{\sqrt{1+(\sin x+\cos x)^2}}dx=\frac{1}{2}\left( \int\frac{\sin x+\cos x}{\sqrt{3-(\sin x-\cos x)^2}}dx-\int\frac{\cos x-\sin x}{\sqrt{1+(\sin x+\cos x)^2}}dx\right)$$
	
	原不定积分为: 
	$$\dfrac{1}{2}\left[ \arcsin \frac{\sin x-\cos x}{\sqrt{3}}-\ln(\sin x+\cos x+\sqrt{1+(\sin x+\cos x)^2} )\right] +C$$
\end{solution}
\myspace{1}

2. $\int_{\frac{1}{6}}^{+\infty}\dfrac{1}{x}\left[ \dfrac{1}{\sqrt{x}}\right]dx$
\myspace{1}
\begin{solution}
	
	$$\begin{cases}
		\left[ \dfrac{1}{\sqrt{x}}\right]=0, &x\geq 1\\
		\left[ \dfrac{1}{\sqrt{x}}\right]=1, &\dfrac{1}{4}\leq x\leq 1\\
		\left[ \dfrac{1}{\sqrt{x}}\right]=2, &\dfrac{1}{6}\leq x\leq \dfrac{1}{4}
	\end{cases}$$
	
	原积分为: 
	$$\int_{\frac{1}{6}}^{\frac{1}{4}}\dfrac{2}{x}dx+\int_{\frac{1}{4}}^{1}\dfrac{1}{x}dx=2\ln 3$$
\end{solution}
\myspace{1}

\hl{\textbf{\textit{March 23}}}

1. 已知函数 $f(x,y)=
\begin{cases}
	\dfrac{\sin x^2\cos y^2}{\sqrt{x^2+y^2}}, &(x,y)\neq (0,0)\\
	0, &(x,y)=(0,0)
\end{cases}$ 求 $f'_{x}(0,0),f'_{y}(0,0)$
\myspace{1}
\begin{solution}
	
	$$\lim\limits_{\Delta x\rightarrow 0}\frac{f(\Delta x,0)}{\Delta x}=\lim\limits_{\Delta x\rightarrow 0}\frac{\sin (\Delta x)^2}{(\Delta x)^2}|\Delta x|$$
	$$\lim\limits_{\Delta y\rightarrow 0}\frac{f(0,\Delta y)}{\Delta y}=0$$
	
	$f'_{x}(0,0)$ 不存在, $f'_{y}(0,0)=0$
\end{solution}
\myspace{1}

2. 设 $z=e^{xy}+f(x+y,xy)$, 求 $\dfrac{\partial^2 z}{\partial x\partial y}$,其中 $f(u,v)$ 有二阶连续偏导数
\myspace{1}
\begin{solution}
	
	$$\frac{\partial z}{\partial x}=ye^{xy}+f_{1}'(x+y,xy)+yf_{2}'(x+y,xy)$$
	$$\frac{\partial^{2} z}{\partial x\partial y}=(1+xy)e^{xy}+f_{11}'(x+y,xy)+xf_{12}''(x+y,xy)+f_{2}'(x+y,xy)+y(f_{21}''(x+y,xy)+xf_{22}''(x+y,xy))$$
	$$\frac{\partial^{2} z}{\partial x\partial y}=(1+xy)e^{xy}+f_{11}''(x+y,xy)+xyf_{22}''(x+y,xy)+(x+y)f_{12}''(x+y,xy)+f_{2}'(x+y,xy)$$
\end{solution}
\myspace{1}

\hl{\textbf{\textit{March 24}}}

1. $\lim\limits_{n\rightarrow +\infty}\dfrac{1}{n^4}\prod\limits_{k=1}^{2n}(n^2+k^2)^{\frac{1}{n}}$
\myspace{1}
\begin{solution}
	
	原极限等价于: 
	$$\lim\limits_{n\rightarrow +\infty}e^{\frac{1}{n}\ln(\prod\limits_{k=1}^{2n}(n^2+k^2))-4\ln n} $$
	$$\lim\limits_{n\rightarrow +\infty}e^{\frac{1}{n}\ln(\prod\limits_{k=1}^{2n}(1+(\frac{k}{n})^2))}=e^{\int_{0}^{2}\ln(1+x^2)dx}$$
	$$\int_{0}^{2}\ln(1+x^2)dx=x\ln(1+x^2)|_{x=0}^{x=2}-\int_{0}^{2}\dfrac{x^2}{1+x^2}dx=2\ln5 -2+\arctan 2$$
	
	原极限: $\lim\limits_{n\rightarrow
		+\infty}\dfrac{1}{n^4}\prod\limits_{k=1}^{2n}(n^2+k^2)^{\frac{1}{n}}=25e^{\arctan 2-2}$
\end{solution}
\myspace{1}

2. 设连续函数 $z=f(x,y)$ 满足$\lim\limits_{x\rightarrow 0,y\rightarrow 1}\dfrac{f(x,y)-2x+y-2}{\sqrt{x^2+(y-1)^2}}=0$,求 $dz|_{(0,1)}$
\myspace{1}
\begin{solution}
	
	$$dz|_{(0,1)}=2dx-dy$$
\end{solution}
\myspace{1}

\hl{\textbf{\textit{March 25}}}

1. $\int_{0}^{1}x^{a}(1-x)^{b}\ln xdx$
\myspace{1}
\begin{lemma}[特殊反常积分]\label{lem: 特殊反常积分}
	
	$$\int_{1}^{+\infty}\dfrac{1}{x^p}dx\left\lbrace 
	\begin{array}{l}
		p>1,\ \text{收敛}\\
		p\leq 1,\ \text{发散}
	\end{array}
	\right. $$
	$$\int_{0}^{1}\dfrac{1}{x^p}dx\left\lbrace 
	\begin{array}{l}
		0<p<1,\ \text{收敛}\\
		p\geq 1,\ \text{发散}
	\end{array}
	\right. $$
	$$\int_{0}^{1}\dfrac{\ln x}{x^p}dx\left\lbrace 
	\begin{array}{l}
		0<p<1,\ \text{收敛}\\
		p\geq 1,\ \text{发散}
	\end{array}
	\right. $$
\end{lemma}
\begin{solution}
	
	我们发现这个题可能的瑕点为$x=0,\ x=1$
	
	$$x\rightarrow 1^{-},f(x)=x^a(1-x)^b\ln x\sim -(1-x)^{b+1}\Rightarrow 0<-(b+1)<1\Rightarrow -2<b<-1$$
	$$x\rightarrow 0^{+},f(x)=x^a(1-x)^b\ln x\sim x^a\ln x=\dfrac{\ln x}{x^{-a}}\Rightarrow 0<-a<1\Rightarrow -1<a<0$$
	
	(1).$x=1,x=0$均为瑕点,我们得到: 
	$$\left\lbrace 
	\begin{array}{l}
		-2<b<-1\\
		-1<a<0
	\end{array}
	\right. $$
	
	(2).$x=1$是瑕点,$x=0$不是瑕点,我们有: 
	$$\left\lbrace 
	\begin{array}{l}
		a>0\\-2<b<-1
	\end{array}
	\right. $$
	
	(3).$x=0$是瑕点,$x=1$不是瑕点,我们有: 
	$$\left\lbrace 
	\begin{array}{l}
		-1<a<0\\b>-1
	\end{array}
	\right. $$
\end{solution}
\myspace{1}

2. $\int\dfrac{x^2}{(x\cos x-\sin x)(x\sin x+\cos x)}dx$
\myspace{1}
\begin{solution}
	
	$$\int\frac{x(x\cos x-x\sin x)+x\sin x(x\sin x+\cos x)}{(x\cos x-\sin x)(x\sin x+\cos x)}dx=\int\frac{x\cos x}{x\sin x+\cos x}dx+\int\frac{x\sin x}{x\cos x-\sin x}dx$$
	$$\int\frac{x\cos x}{x\sin x+\cos x}dx=\ln(x\sin x+\cos x)+C$$
	$$\int\frac{x\sin x}{x\cos x-\sin x}dx=\ln(x\cos x+\sin x)+C$$
	
	原不定积分为: 
	$$\int\frac{x^2}{(x\cos x-\sin x)(x\sin x+\cos x)}dx=\ln(x\sin x+\cos x)+\ln(x\cos x+\sin x)+C$$
\end{solution}
\myspace{1}

\hl{\textbf{\textit{March 26}}}

1. $\int_{0}^{+\infty}\dfrac{e^{-x^2}}{(x^2+\frac{1}{2})^2}dx$
\myspace{1}
\begin{lemma}[特殊积分]\label{lem: 特殊积分}
	
	$$I=\int_{0}^{+\infty}e^{-x^2}dx=\frac{\sqrt{\pi}}{2}$$
	$$I^2=\int_{0}^{+\infty}e^{-x^2}dx\int_{0}^{+\infty}e^{-y^2}dy=\iint_{D}e^{-x^2-y^2}dxdy$$
	$$I^2=\int_{0}^{\frac{\pi}{2}}d\theta\int_{0}^{+\infty}re^{-r^2}dr=\frac{\pi}{2}(-\frac{1}{2}e^{-r^2})|_{r=0}^{r=+\infty}=\frac{\pi}{4}$$
	$$I=\frac{\sqrt{\pi}}{2}$$
\end{lemma}

\begin{solution}
	$$\int_{0}^{+\infty}\frac{e^{-x^2}}{(x^2+\frac{1}{2})^2}dx=\int_{0}^{+\infty}\frac{e^{-x^2}}{2x}d(\frac{4x^2}{2x^2+1})=\frac{e^{-x^2}}{2x}\frac{4x^2}{2x^2+1}|_{x=0}^{x=+\infty}-\int_{0}^{+\infty}\frac{4x^2}{2x^2+1}\frac{e^{-x^2}(-4x^2-2)}{4x^2}dx$$
	$$\downdownarrows$$
	$$I=\frac{2xe^{-x^2}}{2x^2+1}|_{x=0}^{x=+\infty}+2\int_{0}^{+\infty}e^{-x^2}dx$$
	$$\lim\limits_{x\rightarrow +\infty}\frac{2xe^{-x^2}}{2x^2+1}=\frac{2e^{-x^2}}{2x+\frac{1}{x}}=0 \quad \lim\limits_{x\rightarrow 0}\frac{2xe^{-x^2}}{2x^2+1}=0$$
	
	因此,我们得到原定积分为: 
	$$I=2\int_{0}^{+\infty}e^{-x^2}dx=\sqrt{\pi}$$
\end{solution}
\myspace{1}

\hl{\textbf{\textit{March 27}}}

1. 已知级数 $\sum\limits_{n=1}^{\infty}(-1)^{n-1}a_{n}=2,\ \sum\limits_{n=1}^{\infty}a_{2n-1}=5$,则级数 $\sum\limits_{n=1}^{\infty}a_{n}=8$
\myspace{1}
\begin{solution}
	
	$$\sum\limits_{n=1}^{\infty}(-1)^{n-1}a_{n}=2,\sum\limits_{n=1}^{\infty}a_{2n-1}=5\Rightarrow \sum\limits_{n=1}^{\infty}a_{2n}=3$$
	$$\sum\limits_{n=1}^{\infty}a_{n}=\sum\limits_{n=1}^{\infty}a_{2n}+\sum\limits_{n=1}^{\infty}a_{2n-1}=8$$
\end{solution}
\myspace{1}

2. $\int_{0}^{+\infty}\dfrac{\arctan x}{x(1+\ln^2 x)}dx$
\myspace{1}
\begin{solution}
	
	令 $t=\dfrac{1}{x},x=\dfrac{1}{t},dx=-\dfrac{1}{t^2}dt$
	
	原反常积分等价于: 
	$$\int_{0}^{+\infty}\frac{t\arctan \frac{1}{t}}{1+\ln^2 t}\frac{1}{t^2}dt=\int_{0}^{+\infty}\frac{\arctan \frac{1}{x}}{x(1+\ln^2 x)}dx$$
	
	两式相加得到: 
	$$2I=\int_{0}^{+\infty}\frac{\frac{\pi}{2}}{x(1+\ln^2 x)}dx=\frac{\pi}{2}\arctan \ln x|_{0}^{+\infty}=\frac{\pi^2}{2}$$
	$$I=\frac{\pi^2}{4}$$
\end{solution}
\myspace{1}

\hl{\textbf{\textit{March 28}}}

1. $\int\dfrac{2x^4}{1+x^6}dx$
\myspace{1}
\begin{solution}
	
	$$\int\frac{2x^4}{1+x^6}dx=\int\frac{x^4-1}{1+x^6}dx+\int\frac{x^4+1}{1+x^6}dx=\int\frac{x^2+1}{1+x^4-x^2}dx+\int\frac{x^4-x^2+1+x^2}{1+x^6}dx$$
	$$\int\frac{x^2-1}{1+x^4-x^2}dx=\frac{1}{2\sqrt{3}}\ln|\frac{x+\frac{1}{x}-\sqrt{3}}{x+\frac{1}{x}+\sqrt{3}}|+C$$
	$$\int\frac{x^4-x^2+1+x^2}{1+x^6}dx=\arctan x+\frac{1}{3}\arctan x^3+C$$
\end{solution}
\myspace{1}

2. 级数 $\sum\limits_{n=1}^{+\infty}(-1)^n(1-\cos \frac{\alpha}{n}),\quad \alpha >0$ 绝对收敛
\myspace{1}
\begin{solution}
	
	$$n\rightarrow +\infty,1-\cos \frac{\alpha}{n}\sim \frac{\alpha^2}{2n^2}$$
	
	原级数和 $\sum\limits_{n=1}^{+\infty}(-1)^n\dfrac{\alpha^2}{2n^2}$ 同敛散性,后者绝对收敛.
\end{solution}
\myspace{1}

\hl{\textbf{\textit{March 29}}}

1. 设 $u_{n}=(-1)^{n}\ln(1+\dfrac{1}{\sqrt{n}})$,判断级数 $\sum\limits_{n=1}^{+\infty}u_{n}$和级数$\sum\limits_{n=1}^{+\infty}u^{2}_{n}$ 的敛散性
\myspace{1}
\begin{solution}
	
	比较判别法的极限形式: 
	$$\lim\limits_{n\rightarrow +\infty}\frac{\ln(1+\frac{1}{\sqrt{n}})}{\frac{1}{\sqrt{n}}}=1$$
	
	级数 $\sum\limits_{n=1}^{+\infty}\ln(1+\dfrac{1}{\sqrt{n}})$和级数 $\sum\limits_{n=1}^{+\infty}\dfrac{1}{\sqrt{n}}$同敛散性.
	
	判断交错级数 $u_{n}$ 敛散性,我们有: $\ln(1+\dfrac{1}{\sqrt{n}})$ 单调递减,且 $\lim\limits_{n\rightarrow+\infty}\ln(1+\dfrac{1}{\sqrt{n}})=0$
	
	我们有级数 $\sum\limits_{n=1}^{+\infty}u_{n}$ 收敛,级数 $\sum\limits_{n=1}^{+\infty}|u_{n}|$ 发散,级数 $\sum\limits_{n=1}^{+\infty}u_{n}$条件收敛.
	
	比较判别法的极限形式: 
	$$\lim\limits_{n\rightarrow +\infty}\frac{\ln^{2}(1+\frac{1}{\sqrt{n}})}{\frac{1}{n}}=1$$
	
	级数$\sum\limits_{n=1}^{+\infty}u^{2}_{n}$ 发散.
\end{solution}
\myspace{1}

2. 已知 $y^{2}(x-y)=x^2$,求 $\int\dfrac{1}{y^2}dx$ \label{problem: 隐函数转为参数方程}
\myspace{1}
\begin{solution}
	
	隐函数转为参数方程
	
	令 $\dfrac{y}{x}=t$,我们有 $xt^2(1-t)=1$,我们得到参数方程: 
	$$\left\lbrace\begin{array}{l}
		x=\dfrac{1}{t^2(1-t)}\\y=\dfrac{1}{t(1-t)}
	\end{array} \right. $$
	
	原不定积分:  
	$$ \int t^2(t-1)^2\frac{t(3t-2)}{t^4(1-t)^2}dt=\int(3-\frac{2}{t})dt=3t-2\ln t+C$$
\end{solution}
\myspace{1}

\hl{\textbf{\textit{March 30}}}

1. $\int_{-\frac{\pi}{2}}^{\frac{\pi}{2}}\dfrac{\cos^3 x}{1+\cos x-\sin x}dx$

\myspace{1}
\begin{lemma}[对称积分变换]\label{lem: 对称积分变换}
	$$\int_{-a}^{a}f(x)dx=\int_{0}^{a}(f(x)+f(-x))dx$$
\end{lemma}
\begin{solution}
	
	由 $\mathbf{lem: }$ \ref{lem: 对称积分变换} ,我们得到 : 
	$$f(x)=\frac{\cos^3 x}{1+\cos x-\sin x},f(-x)=\frac{\cos^3 x}{1+\cos x+\sin x}$$
	
	原定积分等价于: 
	$$\int_{0}^{\frac{\pi}{2}}(\frac{\cos^3 x}{1+\cos x-\sin x}+\frac{\cos^3 x}{1+\cos x+\sin x})dx=\int_{0}^{\frac{\pi}{2}}\frac{2(1+\cos x)\cos^3x}{2\cos x(1+\cos x)}dx=\int_{0}^{\frac{\pi}{2}}\cos^2 xdx=\frac{\pi}{4}$$
\end{solution}
\myspace{1}

2. 已知级数 $\sum\limits_{n=1}^{+\infty}(-1)^n\sqrt{n}\sin \dfrac{1}{n^{\alpha}}$ 绝对收敛,级数 $\sum\limits_{n=1}^{+\infty}\dfrac{(-1)^n}{n^{2-\alpha}}$ 条件收敛,求 $\alpha$ 取值范围.
\myspace{1}
\begin{solution}
	
	$p$ 级数敛散性 : 
	$$\left\lbrace \begin{array}{l}
		2-\alpha>0\\2-\alpha\leq 1
	\end{array}\right. \Rightarrow 1\leq\alpha<2$$
	
	比较判别法的极限形式: 
	$$\lim\limits_{n\rightarrow+\infty}\dfrac{\sqrt{n}\sin\frac{1}{n^{\alpha}}}{\frac{1}{n^{\alpha-\frac{1}{2}}}}=1\Rightarrow \alpha-\frac{1}{2}>1\Rightarrow \alpha >\frac{3}{2}$$
	
	我们得到 $\alpha$ 取值范围 $\dfrac{3}{2}<\alpha<2$
\end{solution}
\myspace{1}

\hl{\textbf{\textit{March 31}}}

1. $\sum\limits_{n=1}^{+\infty}\dfrac{1}{(2n+1)!}x^{2n+1}$
\myspace{1}
\begin{solution}
	
	$S(x)=\sum\limits_{n=1}^{+\infty}\dfrac{1}{(2n+1)!}x^{2n+1}$,$S'(x)=\sum\limits_{n=0}^{+\infty}\dfrac{1}{2n!}x^{2n}$.
	
	我们得到: $$S(x)+S'(x)=\sum\limits_{n=0}^{+\infty}\frac{1}{n!}x^{n}=e^{x}$$
	
	原问题转化为微分方程: $y'+y=e^x$ 的求解,且 $y(0)=0$
	
	一阶微分方程的求解公式: $$y'+p(x)y=q(x)\Rightarrow y=e^{-\int p(x)dx}(e^{\int p(x)dx}q(x)+C)$$
	
	我们得到:  $S(x)=\dfrac{1}{2}(e^{x}-e^{-x})$
\end{solution}
\myspace{1}

2. 求二重积分 $\iint\limits_{D}y^2dxdy$ 和 $\iint\limits_{D}(x+2y)dxdy$,其中 $D$ 是由参数方程 $\left\lbrace\begin{array}{l}
	x=a(t-\sin t)\\y=a(1-\cos t)
\end{array} \right. ,t\in [0,2\pi]$
\myspace{1}
\begin{solution}
	
	积分区域是摆线,\ $x\in [0,2\pi a]$,二重积分可以化为: 
	$$\iint\limits_{D}y^2dxdy=\int_{0}^{2\pi a}dx\int_{0}^{y(x)}y^2dy=\int_{0}^{2\pi a}\frac{1}{3}y^{3}(x)dx$$
	$$\int_{0}^{2\pi a}\frac{1}{3}y^{3}(x)dx=\int_{0}^{2\pi }\frac{1}{3}a^3(1-\cos t)^3(a-a\cos t)dt=\frac{2}{3}\int_{0}^{\pi}(2\sin^2 t)^4dt$$
	
	华里士公式:  $$\frac{2}{3}\int_{0}^{\pi}(2\sin^2 t)^4dt=\frac{2}{3}\times 16\times\frac{7}{8}\times\frac{5}{6}\times\frac{3}{4}\times\frac{1}{2}\times\frac{\pi}{2}=\frac{35\pi}{12}$$
	
	二重积分$\iint\limits_{D}(x+2y)dxdy$ 可化为: 
	$$\int_{0}^{2\pi a}dx\int_{0}^{y(x)}(x+2y)dy=\int_{0}^{2\pi a}(y^{2}(x)+xy(x))dx$$
	$$I=\int_{0}^{2\pi }[a^2(1-\cos t)^2+a^2(t-\sin t)(1-\cos t)](a-a\cos t)dt$$
	$$I=a^3\int_{0}^{2\pi }(1-\cos t)^3dt+a^3\int_{0}^{2\pi }(1-\cos t)^2(t-\sin t)dt$$
	$$I_{1}=2\int_{0}^{\pi}(2\sin^2 t)^3dt=5\pi$$
	$$I_{2}=\int_{-\pi}^{\pi}(1+\cos x)^2(x+\pi+\sin x)dx=2\pi\int_{0}^{\pi}(1+\cos x)^2dx=3\pi^2$$
\end{solution}
\myspace{1}

