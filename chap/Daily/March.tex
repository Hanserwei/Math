\chapterimage{chap29.jpg}
\chapter{March}
\section{Week \Rmnum{1}}
\hl{\textbf{\textit{March 1}}}

1. 求不定积分 $\int\dfrac{1}{\cos^{3} x}dx$
\myspace{1}

2. 求不定积分 $\int\dfrac{2+x}{(1+x^{2})^{2}}dx$
\myspace{1}
\hl{\textbf{\textit{March 2}}}

1. 求不定积分 $\int\dfrac{\arcsin\sqrt{x}+\ln x}{\sqrt{x}}dx$
\myspace{1}

2. 求不定积分 $\int e^{x}\arcsin\sqrt{1-e^{2x}}dx$
\myspace{1}
\hl{\textbf{\textit{March 3}}}

1. 已知 $f(\ln x)=\dfrac{\ln(1+x)}{x}$, 求 $\int f(x)dx$
\myspace{1}

2. 求定积分 $\int_{0}^{\frac{\pi}{2}}x\sin^{2}x\cos^{2}xdx$
\myspace{1}
\hl{\textbf{\textit{March 4}}}

1. 求定积分 $\int_{0}^{\pi^{2}}\sqrt{x}\cos\sqrt{x}dx$
\myspace{1}

2. 求定积分 $\int_{0}^{2}x\sqrt{2x-x^{2}}dx$
\myspace{1}
\hl{\textbf{\textit{March 5}}}

1. 求定积分 $\int_{0}^{1}x(1-x^{4})^{\frac{3}{2}}dx$
\myspace{1} 

2. 求定积分 $\int_{1}^{2}(x-1)^{2}(x-2)^{2}dx$
\myspace{1}
\hl{\textbf{\textit{March 6}}}

1. 求定积分 $\int_{-2}^{2}x\ln(1+e^{x})dx$
\myspace{1}

2. 设 $D$ 是由曲线 $xy+1=0$ 与直线 $y+x=0$ 及 $y=2$ 围成的有界区域,求 $D$ 的面积
\myspace{1}
\hl{\textbf{\textit{March 7}}}

1. 设 $D$ 是由曲线 $y=x^{2}$ 与 $y=x$ 围成的有界区域,求区域 $D$ 分别绕直线 $y=0,x=0,x=1,x=2$ 旋转所得旋转体的体积
\myspace{1}

2. 方程 $y''+4y'+4y=e^{-2x}$ 满足条件 $y(0)=1,y'(0)=0$ 的特解
\myspace{1}
\section{Week \Rmnum{2}}
\hl{\textbf{\textit{March 8}}}

1. 具有特解 $y_{1}=e^{-x},y_{2}=2xe^{-x},y_{3}=3e^{x}$ 的三阶常系数线性齐次方程为:
\begin{itemize}
	\item A. $y'''-y''-y'+y=0$
	\item B. $y'''+y''-y'-y=0$
	\item C. $y'''-6y''+11y'-6y=0$
	\item D. $y'''-2y''-y'+2y=0$
\end{itemize}
\myspace{1}

2. 方程 $y''-2y'=xe^{2x}$ 的特解形式为:
\begin{itemize}
	\item A. $y=axe^{2x}$
	\item B. $y=(ax+b)e^{2x}$
	\item C. $y=x(ax+b)e^{2x}$
	\item D. $y=x^{2}(ax+b)e^{2x}$
\end{itemize}
\myspace{1}
\hl{\textbf{\textit{March 9}}}

1. 方程 $y''+y=e^{x}+1+\sin x$ 的特解形式为:
\begin{itemize}
	\item A. $ae^{x}+b+c\sin x$
	\item B. $ae^{x}+b+c\cos x+d\sin x$
	\item C. $ae^{x}+b+x(c\cos x+d\sin x)$
	\item D. $y=ae^{x}+b+cx\sin x$
\end{itemize}
\myspace{1}

2. 设函数 $f(x)$ 具有一阶连续导数, 且满足 $f(x)=\int_{0}^{x}(x^{2}-t^{2})f'(t)dt+x^{2}$, 求 $f(x)$ 表达式
\myspace{1}
\hl{\textbf{\textit{March 10}}}

1. 设 $L$ 是一条平面曲线, 其上任意一点 $P(x,y)(x>0)$ 到坐标原点的距离恒等于该点处切线在 $y$ 轴上的截距, 且 $L$ 经过 $(\dfrac{1}{2},0)$, 求曲线 $L$ 的渐近线方程为
\myspace{1}

2. 二元函数 $f(x,y)=
\begin{cases}
	\dfrac{x^{2}y}{x^{2}+y^{2}},&(x,y)\neq (0,0)\\
	0,&(x,y)=(0,0)
\end{cases}$ 在点 $(0,0)$ 处:
\begin{itemize}
	\item A. 不连续
	\item B. 两个偏导数都不存在
	\item C. 偏导数存在但不可微
	\item D. 可微
\end{itemize}
\myspace{1}
\hl{\textbf{\textit{March 11}}}

1. 二元函数 $f(x)$ 在点 $(x_{0},y_{0})$ 处两个偏导数 $f'_{x}(x_{0},y_{0}),f'_{y}(x_{0},y_{0})$ 存在, 是 $f(x)$ 在点 $(x_{0},y_{0})$ 处连续的:
\begin{itemize}
	\item A. 充分不必要条件
	\item B. 必要不充分条件
	\item C. 充分必要条件
	\item D. 不充分不必要条件
\end{itemize}
\myspace{1}

2. 已知 $f(x,y)=\sin\sqrt{x^{4}+y^{4}}$, 则:
\begin{itemize}
	\item A. $f'_{x}(0,0),f'_{y}(0,0)$ 都存在
	\item B. $f'_{x}(0,0)$ 不存在, $f'_{y}(0,0)$ 存在
	\item C. $f'_{x}(0,0)$ 存在, $f'_{y}(0,0)$ 不存在
	\item D. $f'_{x}(0,0),f'_{y}(0,0)$ 都不存在
\end{itemize}
\hl{\textbf{\textit{March 12}}}

1. 设 $f(x,y)=\dfrac{2x+y^{2}}{1+y^{2}\sqrt{1+x^{2}+y^{2}}}$, 则 $d f(0,0)$
\myspace{1}

2. 已知 $dF(x,y)=xye^{x}dx+(f(x)+y^{2})dy$, 且 $f(x)$ 有连续一阶导数, $f(x)=0$, 求 $F(x,y)$
\myspace{1}
\hl{\textbf{\textit{March 13}}}

1. 设函数 $f(x,y)$ 可微, 且对于任意 $x,y$ 都有 $\dfrac{\partial f(x,y)}{\partial x}>0,\dfrac{\partial f(x,y)}{\partial y}<0$, 则下列结论正确的是:
\begin{itemize}
	\item A. $f(1,1)>f(0,0)$
	\item B. $f(-1,1)>f(0,0)$
	\item C. $f(-1,-1)>f(0,0)$
	\item D. $f(1,-1)>f(0,0)$
\end{itemize}
\myspace{1}

2. 设 $z=(x+e^{y})^{x}$, 求 $\dfrac{\partial z}{\partial x}_{(1,0)}$
\myspace{1}
\hl{\textbf{\textit{March 14}}}

1. 设函数 $z=z(x,y)$ 由方程 $(x+1)z+y\ln z-\arctan(2xy)=1$ 确定, 求$\dfrac{\partial z}{\partial x}_{(0,2)}$
\myspace{1}

2. 设 $f(x,y,z)=e^{x}+y^{2}z$, 其中 $z=z(x,y)$ 是由方程 $x+y+z+xyz=0$ 所确定的隐函数, 求 $f'_{x}(0,1,-1)$
\myspace{1}
\section{Week \Rmnum{3}}
\hl{\textbf{\textit{March 15}}}

1. 设 $z=xyf(\dfrac{y}{x})$, 其中 $f(u)$ 可导, 求 $xz'_{x}+yz'_{y}$
\myspace{1}

2. 设 $z=e^{xy}+f(x+y,xy)$, 求 $\dfrac{\partial^{2} z}{\partial x\partial y}$, 其中 $f(u,v)$ 有二阶连续偏导数
\myspace{1}
\hl{\textbf{\textit{March 16}}}

1. 已知 $z=f(x,y)$ 在 $(x_{0},y_{0})$ 处取得极小值, 则:
\begin{itemize}
	\item A. $f'_{x}(x_{0},y_{0})=f'_{y}(x_{0},y_{0})=0$
	\item B. $f''_{xx}(x_{0},y_{0})f''_{yy}(x_{0},y_{0})-(f''_{xy}(x_{0},y_{0}))^{2}>0$, 且 $f''_{xx}(x_{0},y_{0})>0$
	\item C. $f(x_{0},y)$ 在 $y_{0}$ 处取得极大值
	\item D. $f(x,y_{0})$ 在 $x_{0}$ 处取得极小值
\end{itemize}
\myspace{1}

2. 设函数 $f(x),g(x)$ 均有二阶连续导数,满足 $f(0)>0,g(0)<0$, 且 $f'(0)=g'(0)=0$, 则函数 $z=f(x)g(y)$ 在点 $(0,0)$ 处取得极小值的一个充分条件是:
\begin{itemize}
	\item A. $f''(0)<0,g''(0)>0$
	\item B. $f''(0)<0,g''(0)<0$
	\item C. $f''(0)>0,g''(0)>0$
	\item D. $f''(0)>0,g''(0)<0$
\end{itemize}
\myspace{1}
\hl{\textbf{\textit{March 17}}}

1. 已知函数 $z=f(x,y)$ 的全微分 $dz=(ay-x^{2})dx+(ax-y^{2})dy,(a>0)$, 则函数 $f(x,y)$:
\begin{itemize}
	\item A. 无极值点
	\item B. 点 $(a,a)$ 为极小值点
	\item C. 点 $(a,a)$ 为极大值点
	\item D. 是否有极值点与 $a$ 的取值有关
\end{itemize}
\myspace{1}

2. 设函数 $z=f(xy,yg(x))$, 其中 $f$ 函数具有二阶连续偏导数, 函数 $g(x)$ 可导且在 $x=1$ 处取得极值 $g(1)=1$, 求 $\dfrac{\partial^{2}z}{\partial x\partial y}|_{(1,1)}$
\myspace{1}
\hl{\textbf{\textit{March 18}}}

1. 求函数 $f(x,y)=xe^{-\frac{x^{2}+y^{2}}{2}}$ 的极值
\myspace{1}

2. 求 $f(x,y)=x^{2}-y^{2}+2$ 在椭圆域 $D=\left\lbrace (x,y)|x^{2}+\dfrac{y^{2}}{4}\leq 1\right\rbrace$ 上的最大值和最小值
\myspace{1}
\hl{\textbf{\textit{March 19}}}

1. 交换二次积分的积分次序($a>0$)

(1). $\int_{0}^{2}dx\int_{\frac{x^{2}}{4}}^{3-x}f(x,y)dy$

(2). $\int_{0}^{a}dy\int_{0}^{\sqrt{ay}}f(x,y)dx+\int_{a}^{2a}dy\int_{0}^{2a-y}dx$
\myspace{1}

2. 设 $f(x)$ 是连续函数, 则 $\int_{0}^{1}dy\int_{-\sqrt{1-y^{2}}}^{1-y}f(x,y)dx$ 等价于:
\begin{itemize}
	\item A. $\int_{0}^{1}dx\int_{0}^{x-1}f(x,y)dy+\int_{-1}^{0}dx\int_{0}^{\sqrt{1-x^{2}}}f(x,y)dy$
	\item B. $\int_{0}^{1}dx\int_{0}^{1-x}f(x,y)dy+\int_{-1}^{0}dx\int_{-\sqrt{1-x^{2}}}^{0}f(x,y)dy$
	\item C. $\int_{0}^{\frac{\pi}{2}}d\theta\int_{0}^{\frac{1}{\cos\theta+\sin\theta}}f(r\cos\theta,r\sin\theta)dr-\int_{\frac{\pi}{2}}^{\pi}d\theta\int_{0}^{1}f(r\cos\theta,r\sin\theta)dr$
	\item D. $\int_{0}^{\frac{\pi}{2}}d\theta\int_{0}^{\frac{1}{\cos\theta+\sin\theta}}f(r\cos\theta,r\sin\theta)rdr-\int_{\frac{\pi}{2}}^{\pi}d\theta\int_{0}^{1}f(r\cos\theta,r\sin\theta)rdr$
\end{itemize}
\myspace{1}
\hl{\textbf{\textit{March 20}}}

1. 设函数 $f(t)$ 连续,则二次积分 $\int_{0}^{\frac{\pi}{2}}d\theta\int_{2\cos\theta}^{2}f(r^{2})rdr$
\begin{itemize}
	\item A. $\int_{0}^{2}dx\int_{\sqrt{2x-x^{2}}}^{\sqrt{4-x^{2}}}\sqrt{x^{2}+y^{2}}f(x^{2}+y^{2})dy$
	\item B. $\int_{0}^{2}dx\int_{\sqrt{2x-x^{2}}}^{\sqrt{4-x^{2}}}f(x^{2}+y^{2})dy$
	\item C. $\int_{0}^{2}dy\int_{1+\sqrt{1-y^{2}}}^{\sqrt{4-y^{2}}}\sqrt{x^{2}+y^{2}}f(x^{2}+y^{2})dx$
	\item D. $\int_{0}^{2}dy\int_{1+\sqrt{1-y^{2}}}^{\sqrt{4-y^{2}}}f(x^{2}+y^{2})dx$
\end{itemize}
\myspace{1}

2. 计算二重积分

(1). $\iint\limits_{x^{2}+y^{2}\leq 1}(2x+3y)^{2}d\sigma$

(2). $\int_{\frac{1}{4}}^{\frac{1}{2}}dy\int_{\frac{1}{2}}^{\sqrt{y}}e^{\frac{y}{x}}dx+\int_{\frac{1}{2}}^{1}dy\int_{y}^{\sqrt{y}}e^{\frac{y}{x}}dx$

(3). $\int_{0}^{1}dy\int_{y}^{1}\sqrt{x^{2}-y^{2}}dx$

(4). $\int_{0}^{1}dy\int_{y}^{1}\left(\frac{e^{x^{2}}}{x}-e^{y^{3}}\right)dx$

(5). $\iint\limits_{D}\left(xy^{5}-1\right)dxdy,D=\left\{(x,y)|-\frac{\pi}{2}\leq x \leq \frac{\pi}{2},\sin x\leq y \leq 1 \right\}$

(6). $\iint\limits_{D}x^{2}ydxdy$, 其中 $D$ 是由双曲线 $x^{2}-y^{2}=1$以及直线 $y=0,y=1$ 所围成的平面区域

(7). $\iint\limits_{D}\sqrt{x^{2}+y^{2}}dxdy,D=\left\{(x,y)|0\leq y \leq x, x^{2}+y^{2}\leq 2x\right\}$

(8). $\iint\limits_{D}r^{2}\sin\theta\sqrt{1-r^{2}\cos 2\theta}drd\theta,D=\left\{(r,\theta)|0\leq r\leq \sec\theta,0\leq \theta \leq \frac{\pi}{4}\right\}$

\myspace{1}

\section{Week \Rmnum{4}}

