\chapterimage{chap30.jpg}
\chapter{April}
\section{Week \Rmnum{1}}

\hl{\textbf{\textit{April 1}}}

\hl{\textbf{\textit{April 2}}}

\hl{\textbf{\textit{April 3}}}

\hl{\textbf{\textit{April 4}}}

\hl{\textbf{\textit{April 5}}}

\hl{\textbf{\textit{April 6}}}

\hl{\textbf{\textit{April 7}}}

\section{Week \Rmnum{2}}

\hl{\textbf{\textit{April 8}}}

\hl{\textbf{\textit{April 9}}}

\hl{\textbf{\textit{April 10}}}

\hl{\textbf{\textit{April 11}}}

\hl{\textbf{\textit{April 12}}}

\hl{\textbf{\textit{April 13}}}

\hl{\textbf{\textit{April 14}}}

\section{Week \Rmnum{3}}

\hl{\textbf{\textit{April 15}}}

\hl{\textbf{\textit{April 16}}}

\hl{\textbf{\textit{April 17}}}

\hl{\textbf{\textit{April 18}}}

\hl{\textbf{\textit{April 19}}}

\hl{\textbf{\textit{April 20}}}

\hl{\textbf{\textit{April 21}}}

\section{Week \Rmnum{4}}

\hl{\textbf{\textit{April 22}}}

\hl{\textbf{\textit{April 23}}}

\hl{\textbf{\textit{April 24}}}



\myspace{1}

3. $\lim\limits_{n\rightarrow +\infty}\dfrac{1}{n^4}\prod\limits_{k=1}^{2n}(n^2+k^2)^{\frac{1}{n}}$
\myspace{1}
\begin{solution}
	
	原极限等价于: 
	$$\lim\limits_{n\rightarrow +\infty}e^{\frac{1}{n}\ln(\prod\limits_{k=1}^{2n}(n^2+k^2))-4\ln n} $$
	$$\lim\limits_{n\rightarrow +\infty}e^{\frac{1}{n}\ln(\prod\limits_{k=1}^{2n}(1+(\frac{k}{n})^2))}=e^{\int_{0}^{2}\ln(1+x^2)dx}$$
	$$\int_{0}^{2}\ln(1+x^2)dx=x\ln(1+x^2)|_{x=0}^{x=2}-\int_{0}^{2}\dfrac{x^2}{1+x^2}dx=2\ln5 -2+\arctan 2$$
	
	原极限: $\lim\limits_{n\rightarrow
		+\infty}\dfrac{1}{n^4}\prod\limits_{k=1}^{2n}(n^2+k^2)^{\frac{1}{n}}=25e^{\arctan 2-2}$
\end{solution}

\hl{\textbf{\textit{April 25}}}


2. 设连续函数 $z=f(x,y)$ 满足$\lim\limits_{x\rightarrow 0,y\rightarrow 1}\dfrac{f(x,y)-2x+y-2}{\sqrt{x^2+(y-1)^2}}=0$,求 $dz|_{(0,1)}$
\myspace{1}
\begin{solution}
	
	$$dz|_{(0,1)}=2dx-dy$$
\end{solution}

\myspace{1}

\hl{\textbf{\textit{April 26}}}

2. $\int_{0}^{1}x^{a}(1-x)^{b}\ln xdx$
\myspace{1}
\begin{lemma}[特殊反常积分]\label{lem: 特殊反常积分}
	
	$$\int_{1}^{+\infty}\dfrac{1}{x^p}dx\left\lbrace 
	\begin{array}{l}
		p>1,\ \text{收敛}\\
		p\leq 1,\ \text{发散}
	\end{array}
	\right. $$
	$$\int_{0}^{1}\dfrac{1}{x^p}dx\left\lbrace 
	\begin{array}{l}
		0<p<1,\ \text{收敛}\\
		p\geq 1,\ \text{发散}
	\end{array}
	\right. $$
	$$\int_{0}^{1}\dfrac{\ln x}{x^p}dx\left\lbrace 
	\begin{array}{l}
		0<p<1,\ \text{收敛}\\
		p\geq 1,\ \text{发散}
	\end{array}
	\right. $$
\end{lemma}
\begin{solution}
	
	我们发现这个题可能的瑕点为$x=0,\ x=1$
	
	$$x\rightarrow 1^{-},f(x)=x^a(1-x)^b\ln x\sim -(1-x)^{b+1}\Rightarrow 0<-(b+1)<1\Rightarrow -2<b<-1$$
	$$x\rightarrow 0^{+},f(x)=x^a(1-x)^b\ln x\sim x^a\ln x=\dfrac{\ln x}{x^{-a}}\Rightarrow 0<-a<1\Rightarrow -1<a<0$$
	
	(1).$x=1,x=0$均为瑕点,我们得到: 
	$$\left\lbrace 
	\begin{array}{l}
		-2<b<-1\\
		-1<a<0
	\end{array}
	\right. $$
	
	(2).$x=1$是瑕点,$x=0$不是瑕点,我们有: 
	$$\left\lbrace 
	\begin{array}{l}
		a>0\\-2<b<-1
	\end{array}
	\right. $$
	
	(3).$x=0$是瑕点,$x=1$不是瑕点,我们有: 
	$$\left\lbrace 
	\begin{array}{l}
		-1<a<0\\b>-1
	\end{array}
	\right. $$
\end{solution}

\myspace{1}

3. $\int\dfrac{x^2}{(x\cos x-\sin x)(x\sin x+\cos x)}dx$
\myspace{1}
\begin{solution}
	
	$$\int\frac{x(x\cos x-x\sin x)+x\sin x(x\sin x+\cos x)}{(x\cos x-\sin x)(x\sin x+\cos x)}dx=\int\frac{x\cos x}{x\sin x+\cos x}dx+\int\frac{x\sin x}{x\cos x-\sin x}dx$$
	$$\int\frac{x\cos x}{x\sin x+\cos x}dx=\ln(x\sin x+\cos x)+C$$
	$$\int\frac{x\sin x}{x\cos x-\sin x}dx=\ln(x\cos x+\sin x)+C$$
	
	原不定积分为: 
	$$\int\frac{x^2}{(x\cos x-\sin x)(x\sin x+\cos x)}dx=\ln(x\sin x+\cos x)+\ln(x\cos x+\sin x)+C$$
\end{solution}

\myspace{1}

4. $\int_{0}^{+\infty}\dfrac{e^{-x^2}}{(x^2+\frac{1}{2})^2}dx$
\myspace{1}
\begin{lemma}[特殊积分]\label{lem: 特殊积分}
	
	$$I=\int_{0}^{+\infty}e^{-x^2}dx=\frac{\sqrt{\pi}}{2}$$
	$$I^2=\int_{0}^{+\infty}e^{-x^2}dx\int_{0}^{+\infty}e^{-y^2}dy=\iint_{D}e^{-x^2-y^2}dxdy$$
	$$I^2=\int_{0}^{\frac{\pi}{2}}d\theta\int_{0}^{+\infty}re^{-r^2}dr=\frac{\pi}{2}(-\frac{1}{2}e^{-r^2})|_{r=0}^{r=+\infty}=\frac{\pi}{4}$$
	$$I=\frac{\sqrt{\pi}}{2}$$
\end{lemma}

\begin{solution}
	$$\int_{0}^{+\infty}\frac{e^{-x^2}}{(x^2+\frac{1}{2})^2}dx=\int_{0}^{+\infty}\frac{e^{-x^2}}{2x}d(\frac{4x^2}{2x^2+1})=\frac{e^{-x^2}}{2x}\frac{4x^2}{2x^2+1}|_{x=0}^{x=+\infty}-\int_{0}^{+\infty}\frac{4x^2}{2x^2+1}\frac{e^{-x^2}(-4x^2-2)}{4x^2}dx$$
	$$\downdownarrows$$
	$$I=\frac{2xe^{-x^2}}{2x^2+1}|_{x=0}^{x=+\infty}+2\int_{0}^{+\infty}e^{-x^2}dx$$
	$$\lim\limits_{x\rightarrow +\infty}\frac{2xe^{-x^2}}{2x^2+1}=\frac{2e^{-x^2}}{2x+\frac{1}{x}}=0 \quad \lim\limits_{x\rightarrow 0}\frac{2xe^{-x^2}}{2x^2+1}=0$$
	
	因此,我们得到原定积分为: 
	$$I=2\int_{0}^{+\infty}e^{-x^2}dx=\sqrt{\pi}$$
\end{solution}

\myspace{1}
\hl{\textbf{\textit{April 27}}}

1. $\int_{-\frac{\pi}{2}}^{\frac{\pi}{2}}\dfrac{\cos^3 x}{1+\cos x-\sin x}dx$

\myspace{1}
\begin{lemma}[对称积分变换]\label{lem: 对称积分变换}
	
	$$\int_{-a}^{a}f(x)dx=\int_{0}^{a}(f(x)+f(-x))dx$$
\end{lemma}
\begin{solution}
	
	由 $\mathbf{lem: }$ \ref{lem: 对称积分变换} ,我们得到 : 
	$$f(x)=\frac{\cos^3 x}{1+\cos x-\sin x},f(-x)=\frac{\cos^3 x}{1+\cos x+\sin x}$$
	
	原定积分等价于: 
	$$\int_{0}^{\frac{\pi}{2}}(\frac{\cos^3 x}{1+\cos x-\sin x}+\frac{\cos^3 x}{1+\cos x+\sin x})dx=\int_{0}^{\frac{\pi}{2}}\frac{2(1+\cos x)\cos^3x}{2\cos x(1+\cos x)}dx=\int_{0}^{\frac{\pi}{2}}\cos^2 xdx=\frac{\pi}{4}$$
\end{solution}

\myspace{1}

\hl{\textbf{\textit{April 28}}}

1. 已知级数 $\sum\limits_{n=1}^{\infty}(-1)^{n-1}a_{n}=2,\ \sum\limits_{n=1}^{\infty}a_{2n-1}=5$,则级数 $\sum\limits_{n=1}^{\infty}a_{n}=8$

\myspace{1}
\begin{solution}
	
	$$\sum\limits_{n=1}^{\infty}(-1)^{n-1}a_{n}=2,\sum\limits_{n=1}^{\infty}a_{2n-1}=5\Rightarrow \sum\limits_{n=1}^{\infty}a_{2n}=3$$
	$$\sum\limits_{n=1}^{\infty}a_{n}=\sum\limits_{n=1}^{\infty}a_{2n}+\sum\limits_{n=1}^{\infty}a_{2n-1}=8$$
\end{solution}

\myspace{1}

2. $\int_{0}^{+\infty}\dfrac{\arctan x}{x(1+\ln^2 x)}dx$
\myspace{1}
\begin{solution}
	
	令 $t=\dfrac{1}{x},x=\dfrac{1}{t},dx=-\dfrac{1}{t^2}dt$
	
	原反常积分等价于: 
	$$\int_{0}^{+\infty}\frac{t\arctan \frac{1}{t}}{1+\ln^2 t}\frac{1}{t^2}dt=\int_{0}^{+\infty}\frac{\arctan \frac{1}{x}}{x(1+\ln^2 x)}dx$$
	
	两式相加得到: 
	$$2I=\int_{0}^{+\infty}\frac{\frac{\pi}{2}}{x(1+\ln^2 x)}dx=\frac{\pi}{2}\arctan \ln x|_{0}^{+\infty}=\frac{\pi^2}{2}$$
	$$I=\frac{\pi^2}{4}$$
\end{solution}

\myspace{1}

3. $\int\dfrac{2x^4}{1+x^6}dx$
\myspace{1}
\begin{solution}
	
	$$\int\frac{2x^4}{1+x^6}dx=\int\frac{x^4-1}{1+x^6}dx+\int\frac{x^4+1}{1+x^6}dx=\int\frac{x^2+1}{1+x^4-x^2}dx+\int\frac{x^4-x^2+1+x^2}{1+x^6}dx$$
	$$\int\frac{x^2-1}{1+x^4-x^2}dx=\frac{1}{2\sqrt{3}}\ln|\frac{x+\frac{1}{x}-\sqrt{3}}{x+\frac{1}{x}+\sqrt{3}}|+C$$
	$$\int\frac{x^4-x^2+1+x^2}{1+x^6}dx=\arctan x+\frac{1}{3}\arctan x^3+C$$
\end{solution}

\myspace{1}

\hl{\textbf{\textit{April 29}}}

1. 级数 $\sum\limits_{n=1}^{+\infty}(-1)^n(1-\cos \frac{\alpha}{n}),\quad \alpha >0$ 绝对收敛

\myspace{1}
\begin{solution}
	
	$$n\rightarrow +\infty,1-\cos \frac{\alpha}{n}\sim \frac{\alpha^2}{2n^2}$$
	
	原级数和 $\sum\limits_{n=1}^{+\infty}(-1)^n\dfrac{\alpha^2}{2n^2}$ 同敛散性,后者绝对收敛.
\end{solution}

\myspace{1}

\hl{\textbf{\textit{April 30}}}

1. 设 $u_{n}=(-1)^{n}\ln(1+\dfrac{1}{\sqrt{n}})$,判断级数 $\sum\limits_{n=1}^{+\infty}u_{n}$和级数$\sum\limits_{n=1}^{+\infty}u^{2}_{n}$ 的敛散性
\myspace{1}
\begin{solution}
	
	比较判别法的极限形式: 
	$$\lim\limits_{n\rightarrow +\infty}\frac{\ln(1+\frac{1}{\sqrt{n}})}{\frac{1}{\sqrt{n}}}=1$$
	
	级数 $\sum\limits_{n=1}^{+\infty}\ln(1+\dfrac{1}{\sqrt{n}})$和级数 $\sum\limits_{n=1}^{+\infty}\dfrac{1}{\sqrt{n}}$同敛散性.
	
	判断交错级数 $u_{n}$ 敛散性,我们有: $\ln(1+\dfrac{1}{\sqrt{n}})$ 单调递减,且 $\lim\limits_{n\rightarrow+\infty}\ln(1+\dfrac{1}{\sqrt{n}})=0$
	
	我们有级数 $\sum\limits_{n=1}^{+\infty}u_{n}$ 收敛,级数 $\sum\limits_{n=1}^{+\infty}|u_{n}|$ 发散,级数 $\sum\limits_{n=1}^{+\infty}u_{n}$条件收敛.
	
	比较判别法的极限形式: 
	$$\lim\limits_{n\rightarrow +\infty}\frac{\ln^{2}(1+\frac{1}{\sqrt{n}})}{\frac{1}{n}}=1$$
	
	级数$\sum\limits_{n=1}^{+\infty}u^{2}_{n}$ 发散.
\end{solution}

\myspace{1}

2. 已知 $y^{2}(x-y)=x^2$,求 $\int\dfrac{1}{y^2}dx$ \label{problem: 隐函数转为参数方程}
\myspace{1}
\begin{solution}
	
	隐函数转为参数方程
	
	令 $\dfrac{y}{x}=t$,我们有 $xt^2(1-t)=1$,我们得到参数方程: 
	$$\left\lbrace\begin{array}{l}
		x=\dfrac{1}{t^2(1-t)}\\y=\dfrac{1}{t(1-t)}
	\end{array} \right. $$
	
	原不定积分:  
	$$ \int t^2(t-1)^2\frac{t(3t-2)}{t^4(1-t)^2}dt=\int(3-\frac{2}{t})dt=3t-2\ln t+C$$
\end{solution}

\myspace{1}
