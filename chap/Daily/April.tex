\chapterimage{chap30.jpg}
\chapter{April}
\section{Week \Rmnum{1}}
\hl{\textbf{\textit{April 1}}}

1. 级数 $\sum\limits_{n=1}^{+\infty}(\dfrac{1}{\sqrt{n}}-\dfrac{1}{\sqrt{n+1}}\sin(n+k))$ 敛散性
\myspace{1}
\begin{solution}
	
	设 $a_{n}=\dfrac{1}{\sqrt{n}}-\dfrac{1}{\sqrt{n+1}}$,$u_{n}<a_{n}$
	
	$\sum\limits_{n=1}^{+\infty}a_{n}$ 部分和 $S_{n}=1-\dfrac{1}{\sqrt{n+1}}$
	
	$$\lim\limits_{n\rightarrow +\infty}S_{n}=1$$
	
	原级数绝对收敛.
\end{solution}
\myspace{1}

2. $\lim\limits_{x\rightarrow 0^{+}}\dfrac{\sqrt{2(\sec x-1)}-\sqrt[3]{6(x-\sin x)}}{\int_{0}^{x^2}\arctan (e^{\sqrt{t}}-1)dt}$
\myspace{1}
\begin{solution}
	
	对于变上限积分,当 $x\rightarrow 0,g(x)\rightarrow 0,h(x)\rightarrow 0$,我们有: 
	$$\lim\limits_{x\rightarrow 0^{+}}\int_{0}^{g(x)}h(t)dt\Rightarrow \lim\limits_{x\rightarrow 0^{+}}\int_{0}^{G(x)}H(t)dt,x\rightarrow 0,f(x)\sim F(x),h(x)\sim H(x)$$
	
	我们得到原极限等价于: 
	$$\lim\limits_{x\rightarrow 0^{+}}\frac{\sqrt{2(\sec x-1)}-\sqrt[3]{6(x-\sin x)}}{\frac{2}{3}x^3}=\lim\limits_{x\rightarrow 0^{+}}\frac{\sqrt{2(\sec x-1)}-\sqrt{2\frac{1}{2}x^2}+\sqrt[3]{6\frac{1}{6}x^3}-\sqrt[3]{6(x-\sin x)}}{\frac{2}{3}x^3}$$
	
	前一个极限: $$\lim\limits_{x\rightarrow 0^{+}}\frac{\sqrt{2(\sec x-1)}-\sqrt{2\frac{1}{2}x^2}}{\frac{2}{3}x^3}=\lim\limits_{x\rightarrow 0^{+}}\frac{2(\sec x-1-\frac{1}{2}x^2)}{\frac{2}{3}x^3(\sqrt{2(\sec x-1)}+x)}$$
	
	我们有: $\lim\limits_{x\rightarrow 0^{+}}\dfrac{\sqrt{2(\sec x-1)}+x}{2x}=1$
	
	前一个极限: $$I_{1}=\frac{3}{2}\frac{1-\cos x-\frac{1}{2}x^2\cos x}{x^4\cos x}=\frac{5}{16}$$
	
	同理可得: $$I_{2}=-\lim\limits_{x\rightarrow 0^{+}}\frac{3}{2}\frac{\sqrt[3]{6\dfrac{x-\sin x}{x^3}}-1}{x^2}=-\lim\limits_{x\rightarrow 0^{+}}\frac{1}{2}\frac{6(x-\sin x)-x^3}{x^5}=\frac{1}{40}$$
	
	原极限为: $I=I_{1}+I_{2}=\dfrac{27}{80}$
\end{solution}
\myspace{1}

\hl{\textbf{\textit{April 2}}}

1. 级数 $\sum\limits_{n=2}^{+\infty}[sin \frac{1}{n}-k\ln(1-\frac{1}{n})]$ 收敛,求 $k$ 的值.
\myspace{1}
\begin{solution}
	
	泰勒公式和级数的比较判别法极限形式
	
	$$\lim\limits_{n\rightarrow+\infty}\dfrac{sin \frac{1}{n}-k\ln(1-\frac{1}{n})}{\frac{1}{n}}=\lim\limits_{x\rightarrow+0}\dfrac{sin x-k\ln(1-x)}{x}$$
	
	分子利用泰勒展开式得到: 
	$$x\rightarrow 0, \sin x-k\ln(1-x)\sim x-\dfrac{x^3}{6}+o(x^3)+kx+k\dfrac{x^2}{2}+o(x^2)\sim (1+k)x+o(x)$$
	
	我们得到: 
	$$\lim\limits_{n\rightarrow+\infty}\dfrac{sin \dfrac{1}{n}-k\ln(1-\dfrac{1}{n})}{\dfrac{1}{n}}=1+k$$
	
	当且仅当 $k=-1$ 时,级数收敛,因为当 $k\neq -1$时,原级数敛散性和$\sum\limits_{n=1}^{+\infty}\dfrac{1}{n}$一致,级数发散.
\end{solution}
\myspace{1}

2. $\int_{0}^{\frac{\pi}{4}}\left(\dfrac{\sin x-\cos x}{\sin x+\cos x} \right)^3dx $
\myspace{1}
\begin{solution}
	
	原定积分: 
	$$I=\int_{0}^{\frac{\pi}{4}}\left(\frac{-\sin x}{\cos x}\right)^3dx=-\int_{0}^{\frac{\pi}{4}}\tan^{3} xdx$$
	
	令 $tan x=t,x=\arctan t,dx=\dfrac{1}{1+t^2}dt,t\in[0,1]$,原定积分为: 
	$$I=-\frac{1}{2}\int_{0}^{1}\frac{t^2}{1+t^2}dt^2=-\frac{1}{2}\int_{0}^{1}\frac{t}{1+t}dt=\frac{\ln2-1}{2}$$
\end{solution}
\myspace{1}

\hl{\textbf{\textit{April 3}}}

1. 级数 $\sum\limits_{n=1}^{+\infty}a_{n}$ 收敛,判断级数 $\sum\limits_{n=1}^{+\infty}|a_{n}|$,$\sum\limits_{n=1}^{+\infty}(-1)^{n}a_{n}$,$\sum\limits_{n=1}^{+\infty}a_{n}a_{n+1}$,$\sum\limits_{n=1}^{+\infty}\dfrac{a_{n}+a_{n+1}}{2}$ 敛散性
\myspace{1}
\begin{solution}
	
	$$a_{n}=(-1)^n\frac{1}{n}\quad b_{n}=(-1)^n\frac{1}{n^{\frac{1}{2}}}$$
	
	$a_{n}$ 收敛,$|a_{n}|$发散;
	
	$a_{n}$ 收敛,$(-1)^na_{n}$发散;
	
	$b_{n}$ 收敛,$b_{n}b_{n+1}$发散;
\end{solution}
\myspace{1}

2. $\int_{0}^{\frac{\pi}{2}}\dfrac{1}{\sin x-\cos x}dx$
\myspace{1}
\begin{solution}
	
	原定积分内有瑕点,原积分等价于: 
	$$I=\int_{0}^{\frac{\pi}{4}}\frac{1}{\sin x-\cos x}dx+\int_{\frac{\pi}{4}}^{\frac{\pi}{2}}\frac{1}{\sin x-\cos x}dx$$
	
	我们有: $\int\dfrac{1}{\sin x-\cos x}dx=\ln|\dfrac{1-\cos(x-\frac{\pi}{4})}{\sin(x-\frac{\pi}{4})}|+C$
	
	我们可以得到原反常积分发散.
\end{solution}
\myspace{1}

\hl{\textbf{\textit{April 4}}}

1. $\lim\limits_{x\rightarrow 0}\left( -\dfrac{\cot x}{e^{-2x}}+\dfrac{1}{e^{-x}\sin^2 x}-\dfrac{1}{x^2}\right) $
\myspace{1}
\begin{solution}
	
	原极限等价于: 
	$$\lim\limits_{x\rightarrow 0}\frac{x^2e^{x}-\frac{1}{2}x^2\sin 2xe^{2x}-\sin^2 x}{x^4}=-\frac{7}{6}$$
\end{solution}
\myspace{1}

2. 判断下列命题是否正确 

(i). $\sum\limits_{n=0}^{+\infty}(u_{2n+1}+u_{2n})$ 收敛,$\sum\limits_{n=0}^{+\infty}u_{n}$ 收敛

(ii). $\lim\limits_{n\rightarrow +\infty}\dfrac{u_{n+1}}{u_{n}}>1$,则 $\sum\limits_{n=0}^{+\infty}u_{n}$发散.
\myspace{1}
\begin{solution}
	
	(i). $u_{n}=(-1)^{n}$, 第一个排除
	
	(ii). 正项级数比较判别法
\end{solution}
\myspace{1}

\hl{\textbf{\textit{April 5}}}

1. $(1+\sqrt{3})^{n}=a_{n}+b_{n}\sqrt{3}$,求$\lim\limits_{n\rightarrow +\infty}\dfrac{a_{n}}{b_{n}}$
\myspace{1}
\begin{solution}
	
	我们有: 
	$$(1+\sqrt{3})^{n+1}=(1+\sqrt{3})^{n}(1+\sqrt{3})=(a_{n}+b_{n}\sqrt{3})(1+\sqrt{3})$$
	
	我们得到: 
	$$(a_{n}+3b_{n})+(a_{n}+b_{n})\sqrt{3}=a_{n+1}+b_{n+1}\sqrt{3}\Rightarrow\left\lbrace\begin{array}{l}
		a_{n+1}=a_{n}+3b_{n}\\b_{n+1}=a_{n}+b_{n}
	\end{array} \right. $$
	
	我们化简得: 
	$$\frac{a_{n+1}}{b_{n+1}}=\frac{\frac{a_{n}}{b_{n}}+3}{\frac{a_{n}}{b_{n}}+1}$$
	
	不妨设 $x_{n}=\dfrac{a_{n}}{b_{n}}$,我们有$x_{n+1}=1+\dfrac{2}{x_{n}+1}$,$x_{1}=1$
	
	因为 $1<x_{1}<\sqrt{3},x_{n}>1$,我们得到: 
	$$0<|x_{n+1}-\sqrt{3}|=|1+\frac{2}{x_{n}+1}-(1+\frac{2}{\sqrt{3}+1})|=\frac{2}{(x_{n}+1)(\sqrt{3}+1)}|x_{n}-\sqrt{3}|$$
	
	化简得: 
	$$0<|x_{n+1}-\sqrt{3}|<\frac{1}{\sqrt{3}+1}|x_{n}-\sqrt{3}|=\frac{1}{(\sqrt{3}+1)^{n-1}}a_{1}$$
	
	夹逼定理得到: $\lim\limits_{n\rightarrow +\infty}|x_{n}-\sqrt{3}|=0\Rightarrow \lim\limits_{n\rightarrow +\infty}\dfrac{a_{n}}{b_{n}}=\sqrt{3}$
\end{solution}
\myspace{1}

2. $\int_{1}^{+\infty}\dfrac{dx}{x^2\sqrt{x^{2}-1}}$
\myspace{1}
\begin{solution}
	
	令 $x=\sec t,t\in [1,\frac{\pi}{2}], dx=\tan t\sec tdt$,我们有: 
	$$I=\int_{0}^{\frac{\pi}{2}}\frac{\sec t\tan t}{\sec^2 t\tan t}dt=\int_{0}^{\frac{\pi}{2}}\cos tdt=1$$	
\end{solution}
\myspace{1}

\hl{\textbf{\textit{April 6}}}

1. $\int_{0}^{1}\dfrac{xdx}{(2-x^2)\sqrt{1-x^2}}$
\myspace{1}
\begin{solution}
	
	令 $x=\sin t,t\in[0,\frac{\pi}{2}],dx=\cos tdt$,我们有: 
	$$I=\int_{0}^{\frac{\pi}{2}}\frac{\sin t\cos t}{(2-\sin^2 t)\cos t}dt
	=\int_{0}^{\frac{\pi}{2}}\frac{\sin t}{2-\sin^2 t}dt=
	-\int_{0}^{\frac{\pi}{2}}\frac{d\cos t}{\cos^2 t+1}=-\arctan(\cos t)|_{0}^{\frac{\pi}{2}}=\frac{\pi}{4}$$
\end{solution}
\myspace{1}

2. $y=x^2$ 与$y=mx$ 围成的部分绕着 $y=mx(m>0)$ 旋转一周得到的旋转体体积 $V$
\myspace{1}
\begin{solution}
	
	$$V=\int_{0}^{L}\pi r^2dl=\int_{0}^{m}\pi r^2\sqrt{1+m^2}dx=\frac{\pi}{\sqrt{1+m^2}}\int_{0}^{m}x^2(m-x)^2dx
	=\frac{m^5\pi}{30\sqrt{1+m^2}}$$
\end{solution}
\myspace{1}

\hl{\textbf{\textit{April 7}}}
1. $f(x)$ 在 $(2,4)$上二阶导数连续,$f(3)=0$,求证: $f''(\varepsilon)=3\int_{2}^{4}f(x)dx$
\myspace{1}
\begin{solution}
	
	设 $ F(x)=\int_{2}^{x}f(x)dx$,原命题转化为证明: $$\text{已知}F(2)=0,F'(3)=0,\text{求证}F^{'''}(\varepsilon)=3F(4)$$
	
	$F(x)$ 在$x=3$处的泰勒展开式为: 
	$$F(x)=F(3)+F'(3)(x-3)+\frac{F''(3)}{2}(x-3)^2+\frac{F^{'''}(\varepsilon_{1})}{6}(x-3)^{3}$$
	
	我们得到: 
	$$\left\lbrace 
	\begin{array}{l}
		F(2)=F(3)-F'(3)+\frac{F''(3)}{2}-\frac{F^{'''}(\varepsilon_{1})}{6},\varepsilon_{1}\in (2,3)\\
		F(4)=F(3)+F'(3)+\frac{F''(3)}{2}+\frac{F^{'''}(\varepsilon_{2})}{6},\varepsilon_{2}\in (3,4)
	\end{array}\right. $$

	我们得到: $F(4)=\dfrac{F^{'''}(\varepsilon_{1})+F^{'''}(\varepsilon_{2})}{6}$
	
	由平均值定理得到: $$\exists \varepsilon_{3}\in(\varepsilon_{1},\varepsilon_{2}),\ s.t\ F^{'''}(\varepsilon_{3})=\frac{F^{'''}(\varepsilon_{1})+F^{'''}(\varepsilon_{2})}{2}$$
	
	我们得到: $F(4)=\dfrac{F^{'''}(\varepsilon_{3})}{3}$,证毕
	
\end{solution}
\myspace{1}

2. $f(x)$在$[0,+\infty)$上可导,$f(0)=0$,其反函数为 $g(x)$,若 $\int_{0}^{f(x)}g(t)dt=x^2e^x$,求$f(x)$
\myspace{1}
\begin{solution}
	
	我们对 $\int_{0}^{f(x)}g(t)dt=x^2e^x$左右两边同时对 $x$ 求导: 
	$$g(f(x))f'(x)=(x^2+2x)e^x\Rightarrow f'(x)=(x+2)e^{x}$$
	
	我们得到: $f(x)=(x+1)e^x+C,f(0)=1+C=0,C=-1$
	
	$$f(x)=(x+1)e^{x}-1$$
\end{solution}
\myspace{1}

\section{Week \Rmnum{2}}
\hl{\textbf{\textit{April 8}}}

1. 求幂级数$\sum\limits_{n=1}^{+\infty}(-1)^{n}\dfrac{\ln n}{2^{n}}(x+1)^{2n+1}$ 收敛区间
\myspace{1}
\begin{solution}
	
	求解收敛半径的两种方法: 
	
	(i).$\rho=\lim\limits_{n\rightarrow +\infty}\frac{|a_{n+1}|}{|a_{n}|}$
	
	(ii).$\rho=\lim\limits_{n\rightarrow +\infty}\sqrt[n]{|a_{n}|}$
	
	本题中采用第二种方法: $\rho=\lim\limits_{n\rightarrow +\infty}\sqrt[n]{|a_{n}|}=\lim\limits_{n\rightarrow +\infty}\dfrac{1}{2}\sqrt[n]{\ln n}=\dfrac{1}{2}$
	
	此题中幂级数只有奇数项,收敛半径 $R=\sqrt{\dfrac{1}{\rho}}=\sqrt{2}$
	
	原幂级数收敛区间: $(-1-\sqrt{2},-1+\sqrt{2})$
\end{solution}
\myspace{1}

2. 证明: $\int_{0}^{\frac{\pi}{2}}\left( \dfrac{\sin nx}{\sin x}\right) ^2dx=\dfrac{n\pi}{2}$
\myspace{1}
\begin{solution}
	
	我们不妨设 $a_{n}=\int_{0}^{\frac{\pi}{2}}(\frac{\sin nx}{\sin x})^2dx$,我们有: 
	$$a_{1}=\frac{\pi}{2}$$ $$a_{n+1}-a_{n}=\int_{0}^{\frac{\pi}{2}}\frac{\sin ^2(n+1)x-\sin^2 nx}{\sin^2 x}dx=\int_{0}^{\frac{\pi}{2}}\frac{\sin (2n+1)x}{\sin x}dx$$
	
	我们令 $c_{n}=a_{n+1}-a_{n}$,$c_{0}=\frac{\pi}{2}$
	
	$$c_{n+1}-c_{n}=\int_{0}^{\frac{\pi}{2}}\frac{\sin (2n+3)x-\sin (2n+1)x}{\sin x}dx$$
	
	我们利用和差化积公式得到: 
	
	$$c_{n+1}-c_{n}=\int_{0}^{\frac{\pi}{2}}\frac{\sin [(2n+2)x+x]-\sin [(2n+2)x-x]}{\sin x}dx=\int_{0}^{\frac{\pi}{2}}\frac{2\sin x\cos(2n+2)x}{\sin x}dx=0\Rightarrow c_{n}=\frac{\pi}{2}$$
	
	我们得到: $a_{n}$是等差数列,$a_{n}=\dfrac{n\pi}{2}$
\end{solution}
\myspace{1}

\hl{\textbf{\textit{April 9}}}

1. 由方程$F(cx-az,cy-bz)=0$确立了函数$z=z(x,y)$,求 $a\dfrac{\partial z}{\partial x}+b\dfrac{\partial z}{\partial y}$
\myspace{1}
\begin{solution}
	
	隐函数求导法则: 
	$G(x,y,z)=F(u,v),\left\lbrace\begin{array}{l}
		u=cx-az\\v=cy-bz
	\end{array} \right. $,我们有: 
	$$\left\lbrace\begin{array}{l}
		\dfrac{\partial z}{\partial x}=-\dfrac{F_{x}'}{F_{z}'}=\dfrac{cF_{1}'}{aF_{1}'+bF_{2}'}\\
		\dfrac{\partial z}{\partial y}=-\dfrac{F_{y}'}{F_{z}'}=\dfrac{cF_{2}'}{aF_{1}'+bF_{2}'}
	\end{array} \right. $$
	$$a\dfrac{\partial z}{\partial x}+b\dfrac{\partial z}{\partial y}=c$$
\end{solution}
\myspace{1}

2. 设函数 $f,g$均可微,且$z=f[xy,\ln x+g(xy)]$,求$x\frac{\partial z}{\partial x}-y\frac{\partial z}{\partial y}$
\myspace{1}
\begin{solution}
	$$\left\lbrace 
	\begin{array}{l}
		\dfrac{\partial z}{\partial x}=yf_{1}'+(\frac{1}{x}+yg')f_{2}'\\
		\dfrac{\partial z}{\partial y}=xf_{1}'+xg'f_{2}'
	\end{array}\right. $$
	$$x\frac{\partial z}{\partial x}-y\frac{\partial z}{\partial y}=f_{2}'$$
\end{solution}
\myspace{1}

\hl{\textbf{\textit{April 10}}}

1. $f'(x)$连续,\quad$|f'(x)|\leq M,\ \int_{0}^{1}f(x)dx=0$,证明: $\forall a\in [0,1],\ |\int_{0}^{a}f(x)dx|\leq \dfrac{M}{8}$
\myspace{1}
\begin{solution}
	
	我们令: $F(x)=\int_{0}^{x}f(t)dt$,原命题等价于: 
	$$|F''(x)|\leq M, F(0)=F(1)=0, \forall a\in [0,1],|F(x)|\leq \frac{M}{8}$$
	
	利用泰勒反向展开: 
	$$\left\lbrace 
	\begin{array}{l}
		F(0)=F(x)+F'(x)(0-x)+\dfrac{F''(\varepsilon_{1})}{2}(0-x)^2 \qquad \circled{1} \\
		F(1)=F(x)+F'(x)(1-x)+\dfrac{F''(\varepsilon_{2})}{2}(1-x)^2 \qquad \circled{2}
	\end{array}\right. $$
	
	我们利用 $(1-x) \circled{1}+x \circled{2}$得到: 
	$$F(x)=-\frac{F''(\varepsilon_{1})}{2}x^2(1-x)-\frac{F''(\varepsilon_{2})}{2}x(1-x)^2$$
	$$|F(x)|\leq \frac{M}{2}[x^2(1-x)+x(1-x)^2]=\frac{M}{8}$$
\end{solution}
\myspace{1}

2. 幂级数 $\sum\limits_{n=1}^{+\infty}\frac{3^n+(-2)^n}{n}(x-1)^n$ 的收敛域
\myspace{1}
\begin{solution}
	
	先求幂级数收敛半径: 
	$$\rho=\lim\limits_{n\rightarrow +\infty}\sqrt[n]{\frac{3^n+(-2)^n}{n}}=3\Rightarrow r=\frac{1}{3}$$
	
	原幂级数中心点$x=1$,收敛区间为 $(\frac{2}{3},\frac{4}{3})$,我们验证端点值$x=\frac{2}{3},x=\frac{4}{3}$
	
	当 $x=d\frac{2}{3}$时,原幂级数为$\sum\limits_{n=1}^{+\infty}\dfrac{3^n+(-2)^{n}}{(-3)^{n}n}=\sum\limits_{n=1}^{+\infty}\dfrac{(-1)^n}{n}+\sum\limits_{n=1}^{+\infty}\dfrac{2^n}{n3^n}$,原级数收敛.
	
	当 $x=d\frac{4}{3}$时,原幂级数为$\sum\limits_{n=1}^{+\infty}\dfrac{1}{n}+\sum\limits_{n=1}^{+\infty}\dfrac{(-2)^n}{n3^n}$,原级数发散.
	
	幂级数收敛域为 $[\dfrac{2}{3},\dfrac{4}{3})$
\end{solution}
\myspace{1}

\hl{\textbf{\textit{April 11}}}

1. 设幂级数 $\sum\limits_{n=1}^{+\infty}a_{n}x^{n}$与 $\sum\limits_{n=1}^{+\infty}b_{n}x^{n}$的收敛半径分别为$\dfrac{\sqrt{5}}{3}$和 $\dfrac{1}{3}$,则幂级数 $\sum\limits_{n=1}^{+\infty}\dfrac{a_{n}^{2}}{b_{n}^{2}}x^{n}$收敛半径为
\myspace{1}
\begin{solution}
	由题意知: 
	$$\left\lbrace 
	\begin{array}{l}
		\lim\limits_{n\rightarrow+\infty}|\dfrac{a_{n+1}}{a_{n}}|=\frac{3}{\sqrt{5}}\\
		\lim\limits_{n\rightarrow+\infty}|\dfrac{b_{n+1}}{b_{n}}|=3
	\end{array}
	\right. $$
	
	后面幂级数收敛半径 $R=\dfrac{1}{\rho}$,我们有: 
	$$\rho=\lim\limits_{n\rightarrow+\infty}|\frac{a^{2}_{n+1}b_{n}^{2}}{a^{2}_{n}b_{n+1}^{2}}|=\frac{9}{5}\frac{1}{9}=\frac{1}{5}\Rightarrow R=5$$
	
	(\textbf{有些许问题})
\end{solution}
\myspace{1}

2. $f(x,y)=\left\lbrace
\begin{array}{l}
	xy\dfrac{x^2-y^2}{x^2+y^2},(x,y)\neq (0,0)\\
	0,(x,y)=(0,0)
\end{array} \right. $,计算 $f''_{xy}(0,0)$和 $f''_{yx}(0,0)$
\myspace{1}
\begin{solution}
	$$\left\lbrace
	\begin{array}{l}
		f'_{x}(0,0)=\lim\limits_{x\rightarrow 0}\dfrac{f(x,0)-f(0,0)}{x}=0\\
		f'_{y}(0,0)=\lim\limits_{y\rightarrow 0}\dfrac{f(0,y)-f(0,0)}{y}=0
	\end{array}
	\right. $$
	$$\left\lbrace 
	\begin{array}{l}
		f'_{x}=\dfrac{y(x^2-y^2)}{x^2+y^2}+\dfrac{4x^2y^3}{(x^2+y^2)^2}\\
		f'_{y}=\dfrac{x(x^2-y^2)}{x^2+y^2}-\dfrac{4x^2y^3}{(x^2+y^2)^2}
	\end{array}
	\right. $$
	$$\left\lbrace
	\begin{array}{l}
		f''_{xy}(0,0)=\lim\limits_{y\rightarrow 0}\dfrac{f'_{x}(0,y)-f'_{x}(0,0)}{y}=-1\\
		f''_{yx}(0,0)=\lim\limits_{x\rightarrow 0}\dfrac{f'_{y}(x,0)-f'_{y}(0,0)}{x}=1
	\end{array}
	\right. $$
\end{solution}
\myspace{1}

\hl{\textbf{\textit{April 12}}}

1. 设函数$f(x)$ 在$[0,1]$ 上二阶可导,且$f(0)=f(1)=0$,$\underset{x\in [0,1]}{min \{f(x)\}=-1}$,证明: $\exists \varepsilon \in(0,1)$,使得$f''(\varepsilon)\geq 8$
\myspace{1}
\begin{solution}
	
	$f(0)=f(1)=0$,$\underset{x\in [0,1]}{min \{f(x)\}=-1}$,由费马定理我们得到: 
	$$\exists x_{0}\in(0,1),f'(x_{0})=0$$
	
	我们利用泰勒展开,$f(x)$在$x=x_{0}$处的泰勒展开式: 
	$$f(x)=f(x_{0})+f'(x_{0})(x-x_{0})+\frac{f''(\eta)}{2}(x-x_{0})^2,\eta \in x\sim x_{0} $$
	
	(1). 当$x=0$时,$f(0)=-1+\dfrac{f''(\eta_{1})}{2}x_{0}^2=0\Rightarrow f''(\eta_{1})=\dfrac{2}{x_{0}^2}$
	
	(2). 当$x=0$时,$f(1)=-1+\dfrac{f''(\eta_{2})}{2}(1-x_{0})^2=0\Rightarrow f''(\eta_{1})=\dfrac{2}{(1-x_{0})^2}$
	
	我们不妨记$f''(\eta)=max(f''(\eta_{1}),f''(\eta_{2}))$,利用不等式的知识,我们得到: 
	$$f''(\eta)\geq \frac{2}{(\frac{1}{2})^2}=8$$
	
	我们得到: $\exists \varepsilon=\eta \in(0,1)$,使得$f''(\varepsilon)\geq 8$
\end{solution}
\myspace{1}

2. 设 $m,n$均是正整数,证明: $\int_{0}^{1}\frac{\sqrt[m]{\ln^{2}(1-x)}}{\sqrt[n]{x}}dx$ 收敛性与$m,n$无关
\myspace{1}
\begin{solution}
	
	令$f(x)=\dfrac{\sqrt[m]{\ln^{2}(1-x)}}{\sqrt[n]{x}}=x^{-\frac{1}{n}}[\ln(1-x)]^{-\frac{2}{m}}$
	
	我们需要讨论$x\rightarrow 0^{+}$和 $x\rightarrow 1^{-}$两个可能的瑕点
	
	(i).$\lim\limits_{x\rightarrow 0^{+}}f(x)=x^{\frac{2}{m}-\frac{1}{n}}=
	\left\lbrace
	\begin{array}{l}
		0,\frac{2}{m}>\frac{1}{n}\\
		1,\frac{2}{m}=\frac{1}{n}\\
		+\infty,\frac{2}{m}<\frac{1}{n}\\
	\end{array}
	\right. $
	
	(ii).$\lim\limits_{x\rightarrow 1^{-}}f(x)=-\infty$
	
	
	综合 (i)(ii),我们知道$x=1$一定是 $f(x)$的瑕点,$x=0$在一定情况下是$f(x)$的瑕点.
	
	$$\int_{0}^{1}f(x)dx=\int_{0}^{\frac{1}{2}}f(x)dx+\int_{\frac{1}{2}}^{1}f(x)dx=I_{1}+I_{2}$$
	
	(1).我们讨论 $I_{2}$的收敛性,我们比较 $f(x)$和$\dfrac{1}{\sqrt[m]{1-x}}$: 
	
	$$\lim\limits_{x\rightarrow 1^{-}}\frac{f(x)}{\frac{1}{\sqrt[m]{1-x}}}=\sqrt[m]{\frac{\ln^{2}(1-x)}{\frac{1}{1-x}}}\overset{t=1-x}{\Rightarrow}\lim\limits_{t\rightarrow 0^{+}}\sqrt[m]{t\ln^2 t}=0$$
	$$\int_{\frac{1}{2}}^{1}f(x)dx<\int_{\frac{1}{2}}^{1}\frac{1}{\sqrt[m]{1-x}}dx$$
	
	后面的定积分收敛,$I_{2}$收敛
	
	(2).我们讨论 $I_{1}$的收敛性,我们比较$f(x)$和 $\dfrac{1}{\sqrt[n]{x}}$
	$$\lim\limits_{x\rightarrow 0^{+}}\frac{f(x)}{\frac{1}{\sqrt[n]{x}}}=\lim\limits_{x\rightarrow 0^{+}}\sqrt[m]{\ln^{2}(1-x)}=0$$
	$$\int_{0}^{\frac{1}{2}}f(x)dx<\int_{0}^{\frac{1}{2}}\frac{1}{\sqrt[n]{x}}dx$$
	
	后面的定积分收敛,$I_{1}$收敛.
	
	$I$积分收敛,与$m,n$的取值无关.	
\end{solution}
\myspace{1}

\hl{\textbf{\textit{April 13}}}

1. 设数列${a_{n}}$单调减少,$\lim\limits_{n\rightarrow +\infty}a_{n}=0$,$S_{n}=\sum\limits_{k=1}^{n}a_{k}(n=1,2\dots)$无界,求幂级数 $\sum\limits_{n=1}^{+\infty}a_{n}(x-1)^{n}$收敛域.
\myspace{1}
\begin{solution}
	
	我们不难发现幂级数的中心点为$x=1$,数列${a_{n}}$单调减少,$\lim\limits_{n\rightarrow +\infty}a_{n}=0\Rightarrow a_{n}$是正项级数.
	
	(i).当 $x=0$时,我们得到幂级数为 $\sum\limits_{n=1}^{+\infty}(-1)^na_{n}$,莱布尼兹判别法得到级数收敛,由阿贝尔定理我们得到: $x\in(0,2)$,级数收敛.
	
	(ii).当 $x=2$时,我们得到幂级数为 $\sum\limits_{n=1}^{+\infty}a_{n}$,级数发散,由阿贝尔定理我们得到: $x\in (-\infty,0) \cup (2,+\infty)$,级数发散.
	
	综合 (i)(ii),我们的得到幂级数收敛域为$[0,2)$
\end{solution}
\myspace{1}

2.多元函数连续、偏导数、可微、一阶偏导数连续

验证函数$f(x,y)=\left\lbrace
\begin{array}{l}
	y\arctan\dfrac{1}{\sqrt{x^2+y^2}},(x,y)\neq 0\\
	0,(x,y)=(0,0)
\end{array}
\right. $在 $(0,0)$ 处是否连续,是否可微,一阶偏导数的值和是否连续.
\myspace{1}
\begin{solution}
	$$\lim\limits_{\substack{x\rightarrow 0\\ y\rightarrow 0}}f(x,y)=\lim\limits_{\substack{x\rightarrow 0\\ y\rightarrow 0}}y\frac{1}{\sqrt{x^2+y^2}}=\lim\limits_{\substack{x\rightarrow 0\\ y\rightarrow 0}}y\frac{\pi}{2}=0=f(0,0)$$
	
	(i).$f(x,y)$在$(0,0)$处连续.
	$$f'_{x}(0,0)=\lim\limits_{x\rightarrow 0}\frac{f(x,0)-f(0,0)}{x}=0$$
	$$f'_{y}(0,0)=\lim\limits_{y\rightarrow 0}\frac{f(0,y)-f(0,0)}{y}=\arctan\frac{1}{|y|}=\frac{\pi}{2}$$
	
	(ii).$f(x,y)$在$(0,0)$处偏导数为$f'_{x}(0,0)=0,f'_{y}(0,0)=\dfrac{\pi}{2}$
	
	$$\lim\limits_{\substack{x\rightarrow 0\\ y\rightarrow 0}}\frac{f(x,y)-f'_{x}(0,0)(x-0)-f'_{y}(0,0)(y-0)}{\sqrt{x^2+y^2}}=y\dfrac{\arctan\frac{1}{\sqrt{x^2+y^2}}-\dfrac{\pi}{2}}{\sqrt{x^2+y^2}}=0$$
	
	(iii).$f(x,y)$在 $(0,0)$处可微.
	
	公式法求$f(x,y)$偏导数: 
	$$\begin{array}{l}
		f'_{x}=y\dfrac{x^2+y^2}{1+x^2+y^2}(-x)(x^2+y^2)^{-\frac{3}{2}}=-xy\dfrac{(x^2+y^2)^{-\frac{1}{2}}}{1+x^2+y^2}\\
		f'_{y}=\arctan\dfrac{1}{\sqrt{x^2+y^2}}-y^2\dfrac{(x^2+y^2)^{-\frac{1}{2}}}{1+x^2+y^2}
	\end{array}$$
	
	我们得到一阶偏导数在$(0,0)$处的极限: 
	$$\left\lbrace \begin{array}{l}
		\lim\limits_{x\rightarrow0}f'_{x}(x,0)=0=f'_{x}(0,0)\\
		\lim\limits_{y\rightarrow0}f'_{y}(0,y)=\dfrac{\pi}{2}=f'_{y}(0,0)
	\end{array}\right. $$
	
	(iiii).原函数一阶偏导数在$(0,0)$处连续.
\end{solution}
\myspace{1}

\hl{\textbf{\textit{April 14}}}

1. 将 $f(x)=\dfrac{5x-12}{x^2+5x-6}$展开为$x$的幂级数.
\myspace{1}
\begin{solution}
	$$\frac{5x-12}{x^2+5x-6}=\frac{6}{x+6}+\frac{1}{1-x}=\dfrac{1}{1+\dfrac{x}{6}}+\frac{1}{1-x}$$
	
	我们由常见幂级数展开式得到: 
	$$\frac{1}{1-x}=\sum\limits_{n=0}^{+\infty}x^{n},\ -1<x<1$$
	$$\frac{1}{1+x}=\sum\limits_{n=0}^{+\infty}(-1)^{n}x^{n},\ -1<x<1$$
	
	我们可以得到: 
	$$\dfrac{1}{1+\dfrac{x}{6}}=\sum\limits_{n=0}^{+\infty}(-1)^{n}(\frac{x}{6})^{n},-1<\frac{x}{6}<1$$
	
	我们可以得到: 
	$$f(x)=\sum\limits_{n=0}^{+\infty}(-1)^{n}(\frac{x}{6})^{n}+\sum\limits_{n=0}^{+\infty}x^{n}=\sum\limits_{n=1}^{+\infty}(1+\frac{(-1)^n}{6^n})x^n,-1<x<1$$
	
\end{solution}
\myspace{1}

2. 证明: $f'_{x}(x,y)$和$f'_{y}(x,y)$在$(x_{0},y_{0})$处连续,$f(x,y)$在$(x_{0},y_{0})$处可微.
\myspace{1}
\begin{solution}
	
	由题意得: 
	$$\left\lbrace 
	\begin{array}{l}
		\lim\limits_{\substack{x\rightarrow x_{0}\\ y\rightarrow y_{0}}}\dfrac{f(x,y_{0})-f(x_{0},y_{0})}{x-x_{0}}=f'_{x}(x_{0},y_{0})\Rightarrow f(x,y_{0})-f(x_{0},y_{0})=f'_{x}(x_{0},y_{0})(x-x_{0})+\alpha(x,y)\\
		\lim\limits_{\substack{x\rightarrow x_{0}\\ y\rightarrow y_{0}}}\dfrac{f(x_{0},y)-f(x_{0},y_{0})}{y-y_{0}}=f'_{y}(x_{0},y_{0})\Rightarrow f(x_{0},y)-f(x_{0},y_{0})=f'_{y}(x_{0},y_{0})(y-y_{0})+\beta(x,y)
	\end{array}
	\right.$$
	
	我们要证明函数在 $(x_{0},y_{0})$ 处可微,我们只需要证明: 
	
	$$f(x,y)-f(x_{0},y_{0})=f'_{x}(x_{0},y_{0})(x-x_{0})+f'_{y}(x_{0},y_{0})(y-y_{0})+o(\sqrt{(x-x_{0})^2+(y-y_{0})^2})$$
	\begin{eqnarray*}
		f(x,y)-f(x_{0},y_{0})&=&(f(x,y)-f(x_{0},y))+(f(x_{0},y)-f(x_{0},y_{0}))\\
		&=&f'_{x}(\varepsilon_{1},y)(x-x_{0})+f'_{y}(x,\varepsilon_{2})(y-y_{0})\\
		&=&\left[f'_{x}(\varepsilon_{1},y)-f'_{x}(x_{0},y_{0})+f'_{x}(x_{0},y_{0}) \right](x-x_{0})\\ &+&\left[f'_{y}(x,\varepsilon_{2})-f'_{y}(x_{0},y_{0})+f'_{y}(x_{0},y_{0}) \right](y-y_{0})\\
		&=&dz+\alpha_{1}(x,y)+\beta_{1}(x,y)
	\end{eqnarray*}
	我们有: 
	$$\lim\limits_{\substack{x\rightarrow x_{0}\\ y\rightarrow y_{0}}}\alpha_{1}(x,y)=0\ \lim\limits_{\substack{x\rightarrow x_{0}\\ y\rightarrow y_{0}}}\beta_{1}(x,y)=0$$
	证毕.
\end{solution}
\myspace{1}

\section{Week \Rmnum{3}}
\hl{\textbf{\textit{April 15}}}

1. 将函数 $f(x)=\ln(1-x-2x^2)$展开为$x$的幂级数,并指出其收敛区间.
\myspace{1}
\begin{solution}
	$$\ln(1-x-2x^2)=\ln(1+x)+\ln(1-2x)$$
	
	我们根据: 
	$$\ln(1+x)=x-\frac{1}{2}x^2+\frac{1}{3}x^3+\dots+(-1)^{n-1}\frac{x^n}{n}=\sum\limits_{n=1}^{+\infty}(-1)^{n-1}\frac{x^n}{n},-1<x\leq 1$$
	
	得到上面式子的展开式: 
	$$\ln(1+x)=\sum\limits_{n=1}^{+\infty}(-1)^{n-1}\frac{x^n}{n},-1<x\leq 1$$
	$$\ln(1-2x)=-\sum\limits_{n=1}^{+\infty}\frac{(2x)^n}{n},-1<-2x\leq 1$$
	
	我们得到 $f(x)$ 的展开式为: 
	$$f(x)=\sum\limits_{n=1}^{+\infty}[\frac{(-1)^n-2^n}{n}]x^n,-\frac{1}{2}<x\leq \frac{1}{2}$$
	
	函数$f(x)$的收敛区间为 $(-\dfrac{1}{2},\dfrac{1}{2})$
\end{solution}
\myspace{1}

2. 设 $f(x),g(x)$在 $x\in[0,1]$ 上的导数连续,且 $f(0)=0,f'(x)\geq 0,g'(x)\geq 0$,证明: 
$\forall a\in[0,1]$,有 $\int_{0}^{a}g(x)f'(x)dx+\int_{0}^{1}f(x)g'(x)dx\geq f(a)g(1)$
\myspace{1}
\begin{solution}
	
	我们设 $F(x)=\int_{0}^{x}g(t)f'(t)dt+\int_{0}^{1}f(t)g'(t)dt-f(x)g(1)$.
	
	我们有: $$F'(x)=g(x)f'(x)-g(1)f'(x)=f'(x)[g(x)-g(1)]$$
	
	我们知道: $f'(x)\geq 0,g'(x)\geq 0\Rightarrow g(x)\leq g(1),x\in[0,1]$
	
	我们得到: $F'(x)\leq 0\Rightarrow F(x)\text{单调递减}$
	
	$$F(x)\geq F(1)=\int_{0}^{1}g(x)df(x)+\int_{0}^{1}f(x)g'(x)dx-f(1)g(1)=-f(0)g(0)=0$$
	
	原命题得证,证毕.
\end{solution}
\myspace{1}

\hl{\textbf{\textit{April 16}}}

1. 设 $f(x)$在$[0,1]$上连续且单调递减,证明: $\lambda\in (0,1),\int_{0}^{\lambda}f(x)dx>\lambda\int_{0}^{1}f(x)dx$
\myspace{1}
\begin{solution}
	
	我们构造: $F(x)=\dfrac{\int_{0}^{x}f(t)dt}{x}$
	
	我们对 $F(x)$求导得到: 
	$$F'(x)=\frac{xf(x)-\int_{0}^{x}f(t)dt}{x^2}$$
	
	我们令$G(x)=xf(x)-\int_{0}^{x}f(t)dt$,我们得到: 
	$$G'(x)=f(x)+xf'(x)-f(x)=xf'(x),\ x\in[0,1]$$
	
	我们已知 $f(x)\text{在}[0,1]\text{上连续且单调递减},\text{我们可以得到} f'(x)<0,\ x\in(0,1)$
	
	我们得出: $G'(x)<0,\ x\in(0,1)$
	
	$G(x)\text{在}(0,1)\text{上单调递减},G(x)<G(0)=0\Rightarrow F'(x)<0,\ x\in(0,1),F(x)\text{单调递减}$
	我们得到: 
	$$F(x)>F(1)\Rightarrow \frac{\int_{0}^{x}f(t)dt}{x}>\frac{\int_{0}^{1}f(x)dx}{1}$$
	
	即$\forall \lambda\in(0,1),\int_{0}^{\lambda}f(x)dx>\lambda\int_{0}^{1}f(x)dx$
\end{solution}
\myspace{1}

4. $z=(x^2+y^2)f(x^2+y^2),\dfrac{\partial^2 z}{\partial x^2}+\dfrac{\partial^2 z}{\partial y^2}=0,f(1)=0,f'(1)=1$,求$f(x)$表达式
\myspace{1}
\begin{solution}
	$$\left\lbrace 
	\begin{array}{l}
		\dfrac{\partial z}{\partial x}=2xf(x^2+y^2)+2x(x^2+y^2)f'(x^2+y^2)\\
		\dfrac{\partial z}{\partial y}=2yf(x^2+y^2)+2y(x^2+y^2)f'(x^2+y^2)
	\end{array}
	\right. $$
	$$\left\lbrace 
	\begin{array}{l}
		\dfrac{\partial^2 z}{\partial x^2}=2f(x^2+y^2)+4x^2f'(x^2+y^2)+(6x^2+2y^2)f'(x^2+y^2)+4x^2(x^2+y^2)f''(x^2+y^2)\\
		\dfrac{\partial^2 z}{\partial y^2}=2f(x^2+y^2)+4y^2f'(x^2+y^2)+(6y^2+2x^2)f'(x^2+y^2)+4y^2(x^2+y^2)f''(x^2+y^2)
	\end{array}
	\right. $$
	
	我们可以得到: 
	$$\frac{\partial^2 z}{\partial x^2}+\frac{\partial^2 z}{\partial y^2}=4f(u)+12uf'(u)+4u^2f''(u)=0,\ u=x^2+y^2$$
	
	问题转化为: 
	
	$$u^2f''(u)+3uf'(u)+f(u)=0\Rightarrow (u^2f''(u)+2uf'(u))+(uf'(u)+f(u))=[u^2f'(u)+uf(u)]'=0$$
	
	我们得到: $u^2f'(u)+uf(u)=C$,又因为 $f(1)=0,f'(1)=1\Rightarrow C=1$
	
	我们得到一个一阶线性微分方程: $uf'(u)+f(u)=\dfrac{1}{u}$
	
	利用公式法,我们得到: 
	$$(uf(u))'=\frac{1}{u}\Rightarrow uf(u)=\ln u+C_{2}\Rightarrow uf(u)=\ln u $$
	
	我们得到: $f(x)=\dfrac{\ln x }{x}$
\end{solution}
\myspace{1}

\hl{\textbf{\textit{April 17}}}

1. 将 $f(x)=\arctan\dfrac{1+x}{1-x}$展开为$x$的幂级数
\myspace{1}
\begin{solution}
	\begin{eqnarray*}
		f'(x)=\frac{1}{1+x^2}=\sum\limits_{n=0}^{+\infty}(-1)^nx^{2n}, -1<x^2<1
	\end{eqnarray*}

	我们有: 
	
	$$f(x)-f(0)=\int_{0}^{x}\frac{1}{1+x^2}dx=\int_{0}^{x}\sum\limits_{n=0}^{+\infty}(-1)^nx^{2n}dx$$
	$$\int_{0}^{x}\sum\limits_{n=0}^{+\infty}(-1)^nx^{2n}dx=\sum\limits_{n=0}^{+\infty}\int_{0}^{x}(-1)^{n}x^{2n}dx=\sum\limits_{n=0}^{+\infty}(-1)^n\dfrac{x^{2n+1}}{2n+1},$$
	
	$f(0)=\dfrac{\pi}{4}$,我们有: $f(x)=\dfrac{\pi}{4}+\sum\limits_{n=0}^{+\infty}(-1)^n\dfrac{x^{2n+1}}{2n+1},-1<x<1$
	
\end{solution}
\myspace{1}

2.微分方程$y'=\frac{y(1-x)}{x}$通解
\myspace{1}
\begin{solution}
	
	分离变量
	
	原微分方程可以化为: 
	\begin{eqnarray*}
		&(i) &y\neq 0\quad\frac{1}{y}dy=(\frac{1}{x}-1)dx\\
		&(ii)&y=0\quad y'(x)=0,\text{满足条件}
	\end{eqnarray*}
	
	针对(i),我们同时求不定积分得到: 
	$$\ln|y|=\ln|x|-x+C_{1}\Rightarrow \ln|y|=\ln|cxe^{-x}|\Rightarrow y=cxe^{-x}$$
	
	综合(i)(ii),我们得到微分方程通解: $y=Cxe^{-x},C\in \mathbb{R}$
\end{solution}
\myspace{1}

\hl{\textbf{\textit{April 18}}}

1. 隐函数渐近线 $x^3+y^3=3axy,a>0$,求$y=y(x)$的斜渐近线.
\myspace{1}
\begin{solution}
	
	令 $t=\dfrac{y}{x}\rightarrow y=tx$
	
	我们得到: $\left\lbrace 
	\begin{array}{l}
		x=\dfrac{3at}{1+t^3}\\
		y=\dfrac{3at^2}{1+t^3}
	\end{array}
	\right. $
	当 $x\rightarrow \infty,t\rightarrow -1$
	
	我们得到: 
	
	(i).
	\begin{eqnarray*}
		a&=&\lim\limits_{x\rightarrow +\infty}\frac{y}{x}=\lim\limits_{t\rightarrow -1^{-} }t=-1\\
		b&=&\lim\limits_{x\rightarrow +\infty}(y-ax)=\lim\limits_{t\rightarrow -1^{-}}(\frac{3at^2}{1+t^3}+\frac{3at}{1+t^3})=-a
	\end{eqnarray*}
	
	(ii).
	\begin{eqnarray*}
		a&=&\lim\limits_{x\rightarrow -\infty}\frac{y}{x}=\lim\limits_{t\rightarrow -1^{+} }t=-1\\
		b&=&\lim\limits_{x\rightarrow -\infty}(y-ax)=\lim\limits_{t\rightarrow -1^{+}}(\frac{3at^2}{1+t^3}+\frac{3at}{1+t^3})=-a
	\end{eqnarray*}
	
	斜渐近线为: $y=-x-a$
\end{solution}
\myspace{1}

2. 已知函数 $y=y(x)$在任意点$x$处的增量 $\Delta y=\dfrac{y\Delta x}{1+x^2}+\alpha$,当$\Delta x\rightarrow 0$,$\alpha$是$\Delta x$的高阶无穷小,$y(0)=\pi$,求$y(1)$
\myspace{1}
\begin{solution}
	
	我们由题意得到微分方程: 
	$$\frac{1}{y}dy=\frac{1}{1+x^2}dx\Rightarrow \ln|y|=\arctan x+C\Rightarrow y=Ce^{\arctan x}$$
	
	由 $y(0)=\pi,\Rightarrow C=\pi,\text{我们得到}y(x)\text{表达式}: y(x)=\pi e^{\arctan x}\Rightarrow y(1)=\pi e^{\frac{\pi}{4}}$
\end{solution}
\myspace{1}

\hl{\textbf{\textit{April 19}}}

1. 求幂级数 $\sum\limits_{n=0}^{+\infty}(2n+1)x^n$的收敛域,并求其和函数.
\myspace{1}
\begin{solution}
	
	(i).先求幂级数收敛半径: 
	$$\rho=\lim\limits_{n\rightarrow +\infty}\frac{a_{n+1}}{a_{n}}=\frac{2n+3}{2n+1}=1\Rightarrow R=\frac{1}{\rho}=1$$
	
	(ii).验证两个端点
	
	当 $x=\pm1$时,幂级数对应的级数发散.
	
	我们得到原幂级数的收敛域为 $(-1,1)$.
	
	$$S(x)=2\sum\limits_{n=0}^{+\infty}nx^n+\sum\limits_{n=0}^{+\infty}x^n=2x[\sum\limits_{n=1}^{+\infty}x^n]'+\frac{1}{1-x}=\frac{1+x}{(1-x)^2},-1<x<1$$
\end{solution}
\myspace{1}

2. 微分方程 $\dfrac{dy}{dx}=\dfrac{y}{x}-\dfrac{1}{2}(\dfrac{y}{x})^3$,满足$y|_{x=1}=1$的特解
\myspace{1}
\begin{solution}
	
	令 $z=\dfrac{y}{x},\text{我们得到}: y=xz\Rightarrow \dfrac{dy}{dx}=z+x\dfrac{dz}{dx}$
	
	原微分方程可化为: 
	$$(z+x\frac{dz}{dx})=z-\frac{1}{2}z^3\text{满足} z| _{x=1}=1$$
	
	我们可以得到: $-\dfrac{2}{z^3}dz=\dfrac{1}{x}dx\Rightarrow \dfrac{1}{z^2}=\ln x+c$
	
	$\text{我们有}z| _{x=1}=1,\text{得到}: 1+\ln x=\dfrac{1}{z^2}\Rightarrow y^2=\dfrac{x^2}{1+\ln x}$
	
	我们得到: $\dfrac{dy}{dx}|_{(1,1)}=\dfrac{1}{2}>0\Rightarrow y=\dfrac{x}{\sqrt{1+\ln x}}$
\end{solution}
\myspace{1}

\hl{\textbf{\textit{April 20}}}

1. 求微分方程 $(x+y)dx+(y-x)dy=0\text{满足条件}y(1)=-1\text{的特解}$
\myspace{1}
\begin{solution}
	
	我们对微分方程进行一些简单的变形: 
	$$\frac{dy}{dx}=\dfrac{1+\frac{y}{x}}{1-\frac{y}{x}}$$
	
	我们令 $z=\dfrac{y}{x}\Rightarrow \dfrac{dy}{dx}=z+x\dfrac{dz}{dx}$
	
	原微分方程可以化简为: 
	$$\frac{1}{x}dx=\frac{1-z}{1+z^2}dz\Rightarrow \ln |x|=\arctan z-\ln \sqrt{z^2+1}+C$$
	
	即: $\ln\sqrt{x^2+y^2}=\arctan \dfrac{y}{x}+C\quad ,\text{由}y(1)=-1\text{得到} C=\dfrac{1}{2}\ln 2+\dfrac{\pi}{4}$
	
	我们得到微分方程的特解为: $\ln\sqrt{x^2+y^2}-\arctan \dfrac{y}{x}=\dfrac{1}{2}\ln 2+\dfrac{\pi}{4}$
\end{solution}
\myspace{1}

2. $f(x,y)\text{连续}\lim\limits_{(x,y)\rightarrow (0,0)}\dfrac{f(x,y)-xy}{x^2+y^2}=1,f(0,0)\text{是极大值还是极小值?}$ 
\myspace{1}
\begin{solution}
	
	极小值点,理由如下: 
	$\lim\limits_{(x,y)\rightarrow (0,0)}\dfrac{f(x,y)-xy}{x^2+y^2}=1$,我们可以得到在$(0,0)$的一个去心邻域内,我们有: 
	\begin{eqnarray*}
		f(x,y)&=&xy+(x^2+y^2)(1+\alpha)=\frac{1}{2}(x+y)^2+(x^2+y^2)(\frac{1}{2}+\alpha)\\
		&\text{其中}&\lim\limits_{(x,y)\rightarrow (0,0)}\alpha=0
	\end{eqnarray*}

	我们得到在$(0,0)$的一个邻域内,$f(x,y)\geq 0,f(0,0)\text{是极小值}$
\end{solution}
\myspace{1}

\hl{\textbf{\textit{April 21}}}

1.求幂级数 $\sum\limits_{n=1}^{+\infty}(\dfrac{1}{2n+1}-1)x^{2n}\text{在区间}(-1,1)\text{内的和函数} S(x)$
\myspace{1}
\begin{solution}
	
	$$S(x)=\sum\limits_{n=1}^{+\infty}\frac{x^{2n}}{2n+1}-\sum\limits_{n=1}^{+\infty}x^{2n}=\sum\limits_{n=1}^{+\infty}\frac{x^{2n}}{2n+1}-\frac{x^2}{1-x^2}$$
	
	(i). 当 $x\neq 0$, $S(x)=\dfrac{1}{x}\sum\limits_{n=1}^{+\infty}\dfrac{x^{2n+1}}{2n+1}-\dfrac{x^2}{1-x^2}=\dfrac{1}{x}\int_{0}^{x}\sum\limits_{n=1}^{+\infty}x^{2n}-\dfrac{x^2}{1-x^2}$
	
	$$S(x)=\frac{\ln\frac{1-x}{1+x}}{2x}-\frac{1}{1-x^2}$$
	
	(ii). 当 $x=0$,$S(x)=0$.
	
	综上,$S(x)=\left\lbrace 
	\begin{array}{l}
		\frac{\ln\dfrac{1-x}{1+x}}{2x}-\dfrac{1}{1-x^2},0<|x|<1\\
		0,x=0
	\end{array}
	\right.$
\end{solution}
\myspace{1}

2. 微分方程 $(y+x^3)dx-2xdy=0\text{满足}y|_{x=1}=\dfrac{6}{5}\text{的特解为}$
\myspace{1}
\begin{solution}
	
	我们对微分方程化简: $y'-\dfrac{1}{2x}y=\dfrac{x}{2}\Rightarrow (\frac{y}{\sqrt{x}})'=\dfrac{x^{\frac{3}{2}}}{2}$
	
	我们得到: $y=\dfrac{x^3+C\sqrt{x}}{5},y|_{x=1}=\dfrac{6}{5}\Rightarrow C=5$
	
	原微分方程的解: $y=\dfrac{x^3+5\sqrt{x}}{5}$
\end{solution}
\myspace{1}

\section{Week \Rmnum{4}}
\hl{\textbf{\textit{April 22}}}

1. 微分方程 $xy'+2y=x\ln x\text{满足}y(1)=-\frac{1}{9}\text{的特解为}$
\myspace{1}
\begin{solution}
	
	原微分方程可化为: $y'+\dfrac{2}{x}y=\ln x\Rightarrow (x^2y)'=x^2\ln x$
	
	原微分方程的解为: $y=\dfrac{\int x^2\ln xdx+C}{x^2}\Rightarrow y=\dfrac{x(3\ln x-1)}{9}+\dfrac{C}{x^2}$
	
	我们由$y(1)=-\dfrac{1}{9}$ 得到: $y=\dfrac{x(3\ln x-1)}{9}$
\end{solution}
\myspace{1}

2.$f(x,y)=x^4+2y^2-3x^2y,f(0,0)\text{是极大值还是极小值?}$
\myspace{1}
\begin{solution}
	
	$f(0,0)$ 不是极值点,理由如下: 
	$$f(x,y)=(x^2-\frac{3}{2}y)^2-\frac{1}{4}y^2$$
	
	(i). 当$y=0,f(x,y)\geq 0$
	
	(ii).当 $x^2=\dfrac{3}{2}y,f(x,y)\leq 0$
\end{solution}
\myspace{1}

\hl{\textbf{\textit{April 23}}}

1.$\int_{0}^{+\infty}\left[ \dfrac{\sqrt{\pi}}{2}-\int_{0}^{x}e^{-t^2}dt\right]dx$
\myspace{1}
\begin{solution}
	
	方法 1:  分部积分法
	$$\int_{0}^{+\infty}\left[ \frac{\sqrt{\pi}}{2}-\int_{0}^{x}e^{-t^2}dt\right]dx=x\left[ \frac{\sqrt{\pi}}{2}-\int_{0}^{x}e^{-t^2}dt\right]_{0}^{+\infty}-\int_{0}^{+\infty}xe^{-x^2}dx$$
	$$\lim\limits_{x\rightarrow +\infty}\frac{\frac{\sqrt{\pi}}{2}-\int_{0}^{x}e^{-t^2}dt}{\frac{1}{x}}-\frac{1}{2}d(e^{-x^2})=\frac{1}{2}$$
	
	
	方法 2:  二重积分交换积分次序
	$$\int_{0}^{+\infty}\left[ \frac{\sqrt{\pi}}{2}-\int_{0}^{x}e^{-t^2}dt\right]dx=\int_{0}^{+\infty}\left[\int_{0}^{+\infty}e^{-t^2}dt-\int_{0}^{x}e^{-t^2}dt\right]dx$$
	
	我们得到: 
	$$\int_{0}^{+\infty}(\int_{x}^{+\infty}e^{-t^2}dt)dx=\int_{0}^{+\infty}dt\int_{0}^{t}e^{-t^2}dx=\int_{0}^{+\infty}te^{-t^2}dt=\frac{1}{2}$$
\end{solution}
\myspace{1}

2. 设 $F(x)=f(x)g(x)$,其中函数$f(x),g(x)$在$(-\infty,+\infty)$内满足以下条件: $f'(x)=g(x),g'(x)=f'(x)$,且 $f(0)=0$,$f(x)+g(x)=2e^x$

(i).求$F(x)$ 满足的一阶微分方程

(ii).求$F(x)$ 表达式
\myspace{1}
\begin{solution}
	
	(i).我们有: 
	$$F'(x)=f'(x)g(x)+g'(x)f(x)=f^2(x)+g^2(x)=(f(x)+g(x))^2-2f(x)g(x)$$
	
	我们得到$F(x)$ 满足的微分方程为: $F'(x)+2F(x)=4e^{2x}$
	
	(ii).我们利用一阶线性微分方程公式得到: 
	$$(e^{2x}F(x))'=4e^{4x}\Rightarrow F(x)=\frac{e^{4x}+C}{e^{2x}}$$
	
	由$f(0)=0\Rightarrow F(0)=1+C=0\Rightarrow C=-1$,$F(x)=e^{2x}-e^{-2x}$
\end{solution}
\myspace{1}

\hl{\textbf{\textit{April 24}}}

1. 求级数 $\sum\limits_{n=2}^{+\infty}\dfrac{1}{(n^2-1)2^n}$
\myspace{1}
\begin{solution}
	
	我们引入幂级数: 
	
	$$\sum\limits_{n=2}^{+\infty}\frac{x^n}{n^2-1}=\sum\limits_{n=2}^{+\infty}\frac{1}{2}(\frac{1}{n-1}-\frac{1}{n+1})x^n=\frac{1}{2}(\sum\limits_{n=2}^{+\infty}\frac{x^n}{n-1}-\sum\limits_{n=2}^{+\infty}\frac{x^n}{n+1})$$
	
	$$\sum\limits_{n=2}^{+\infty}\frac{x^n}{n-1}=x(\sum\limits_{n=2}^{+\infty}\int_{0}^{x}x^{n-2}dx)=x\int_{0}^{x}(\sum\limits_{n=2}^{+\infty}x^{n-2})dx=-x\ln(1-x)$$
	
	$$\sum\limits_{n=2}^{+\infty}\frac{x^n}{n+1}=\frac{\sum\limits_{n=2}^{+\infty}\frac{x^{n+1}}{n+1}}{x}=\frac{\sum\limits_{n=2}^{+\infty}\int_{0}^{x}x^ndx}{x}=\frac{\int_{0}^{x}(\sum\limits_{n=2}^{+\infty}x^n)dx}{x}=-\frac{x}{2}-1-\frac{\ln(1-x)}{x}$$
	
	原幂级数的和函数为 : 
	$$S(x)=\frac{1}{2}(-x\ln(1-x)+\frac{\ln(1-x)}{x}+\frac{x}{2}+1),-1<x<1$$
	
	我们得到: $S(\dfrac{1}{2})=\dfrac{5}{8}-\dfrac{3\ln 2}{4}$
\end{solution}
\myspace{1}

2. 已知 $f(x)=\lim\limits_{n\rightarrow  +\infty}\sqrt[n]{2+(2x)^{n}+x^{2n}}\quad (x\geq 0)$,$g(x)=\lim\limits_{n\rightarrow +\infty}\dfrac{1-x^{2n+1}}{1+x^{2n}}$,求$f(g(x))$
\myspace{1}
\begin{solution}
	
	我们易得到: $f(x)=\left\lbrace 
	\begin{array}{l}
		x^2,x\geq 2\\
		2x,\frac{1}{2}<x< 2\\
		1,0\leq x\leq \frac{1}{2}
	\end{array}
	\right. $,\quad $g(x)=\left\lbrace 
	\begin{array}{l}
		-x,|x|> 1\\
		0,x=1\\
		1,-1\leq x<1
	\end{array}
	\right. $.
	
	我们可以得到: 
	$$f(g(x))=\left\lbrace 
	\begin{array}{l}
		x^2,x\leq -2\\
		-2x,-2<x<-1\\
		2,-1\leq x<1\\
		1,x=1	
	\end{array}
	\right. $$
\end{solution}
\myspace{1}

\hl{\textbf{\textit{April 25}}}

1. 已知 $y_{1}=(1+x^2)^2-\sqrt{1+x^2}$,$y_{2}=(1+x^2)^2+\sqrt{1+x^2}$是微分方程$y'+p(x)y=q(x)$的两个解,求$q(x)$
\myspace{1}
\begin{solution}
	
	由题意得: 
	$$\left\lbrace 
	\begin{array}{l}
		y_{1}'+p(x)y_{1}=q(x)\\
		y_{2}'+p(x)y_{2}=q(x)
	\end{array}
	\right. \Rightarrow (y_{1}-y_{2})'+p(x)(y_{1}-y_{2})=0\Rightarrow p(x)=-\dfrac{(y_{1}-y_{2})'}{y_{1}-y_{2}}$$
	
	我们得到: $p(x)=-\dfrac{x}{1+x^2}$
	
	任意带入一个方程: $q(x)=y_{1}'+p(x)y_{1}=3x(1+x^2)$
\end{solution}
\myspace{1}

2. 设 $z=f(x,y)$具有二阶连续偏导数,$f'_{y}\neq 0$,证明: 对任意的常数 $k$,曲线$f(x,y)=k$ 是直线的充分必要条件为$(f'_{y})^2f''_{xx}-2f'_{x}f'_{y}f''_{xy}+(f'_{x})^2f''_{yy}=0$
\myspace{1}
\begin{solution}
	
	$f(x,y)=k\text{为直线}\Rightarrow f(x,y)=ax+by+c=k$
	
	(i).必要性
	
	$f(x,y)=k\text{为直线}\Rightarrow f''_{xx}=f''_{yy}=f''_{xy}=0$,我们可以得到: $$(f'_{y})^2f''_{xx}-2f'_{x}f'_{y}f''_{xy}+(f'_{x})^2f''_{yy}=0$$
	
	(ii).充分性
	
	我们记$\left\lbrace 
	\begin{array}{l}
		f'_{x}=f'_{1}\\
		f'_{y}=f'_{2}\\
		f''_{xx}=f''_{11}\\
		f''_{xy}=f''_{12}\\
		f''_{yx}=f''_{21}\\
		f''_{yy}=f''_{22}
	\end{array}
	\right. $,我们将$f(x,y)=k$对$x$求导: 
	$$f'_{1}+f'_{2}\frac{dy}{dx}=0$$
	
	再次对等式两边对$x$求导: 
	$$f''_{11}+f''_{12}\frac{dy}{dx}+(f''_{21}+f''_{22}\frac{dy}{dx})\frac{dy}{dx}+f'_{2}\frac{d^2y}{dx^2}=0$$
	
	若要证明$f(x,y)=k$是直线,我们只需要证明: 
	$$\frac{d^2y}{dx^2}=0\Rightarrow f''_{11}+f''_{12}\frac{dy}{dx}+(f''_{21}+f''_{22}\frac{dy}{dx})\frac{dy}{dx}=0$$
	
	由隐函数求导公式: $\dfrac{dy}{dx}=-\dfrac{f'_{1}}{f'_{2}}$,我们化简上式: 
	$$f''_{11}+f''_{12}(-\frac{f'_{1}}{f'_{2}})+(f''_{21}-f''_{22}\frac{f'_{1}}{f'_{2}})(-\frac{f'_{1}}{f'_{2}})=\frac{(f''_{2})^2f''_{11}-2f'_{1}f'_{2}f''_{12}+(f'_{1})^2f''_{22}}{(f''_{2})^2}=0$$
	
	令分子为$0$即可: 
	$$(f''_{2})^2f''_{11}-2f'_{1}f'_{2}f''_{12}+(f'_{1})^2f''_{22}=0\Rightarrow(f'_{y})^2f''_{xx}-2f'_{x}f'_{y}f''_{xy}+(f'_{x})^2f''_{yy}=0$$
\end{solution}
\myspace{1}

\hl{\textbf{\textit{April 26}}}

1.判断级数的敛散性 $\sum\limits_{n=1}^{+\infty}(\sqrt{n+2}-2\sqrt{n+1}+\sqrt{n})$
\myspace{1}
\begin{solution}
	
	我们不妨记$u_{n}=(\sqrt{n+2}-\sqrt{n+1})-(\sqrt{n+1}-\sqrt{n})$
	
	级数的部分和$S_{n}=u_{1}+u_{2}+\cdots+u_{n}$
	\begin{eqnarray*}
		S_{n}&=&(\sqrt{3}-\sqrt{2})-(\sqrt{2}-\sqrt{1})+(\sqrt{4}-\sqrt{3})-(\sqrt{3}-\sqrt{2})+\cdots\\
		&+&(\sqrt{n+2}-\sqrt{n+1})-(\sqrt{n+1}-\sqrt{n})\\
		&=&(\sqrt{n+2}-\sqrt{n+1})-(\sqrt{2}-\sqrt{1})
	\end{eqnarray*}
	
	我们得到: $\lim\limits_{n\rightarrow +\infty}S_{n}=1-\sqrt{2}$,原级数收敛
\end{solution}
\myspace{1}

2. 判断级数的敛散性 $\sum\limits_{n=1}^{+\infty}\dfrac{\sqrt{n+1}-\sqrt{n}}{n^{\alpha}}$
\myspace{1}
\begin{solution}
	我们有: 
	$$\sqrt{n+1}-\sqrt{n}=\dfrac{1}{\sqrt{n+1}+\sqrt{n}}$$
	
	$$n\rightarrow  +\infty,\quad \frac{1}{\sqrt{n+1}+\sqrt{n}}\sim\frac{1}{2\sqrt{n}}$$
	$$\sum\limits_{n=0}^{+\infty}\frac{\sqrt{n+1}-\sqrt{n}}{n^{\alpha}}\sim\sum\limits_{n=0}^{+\infty}\frac{1}{2}\frac{1}{n^{\alpha+\frac{1}{2}}}$$
	
	我们可以得到: 
	$$\sum\limits_{n=1}^{+\infty}\frac{\sqrt{n+1}-\sqrt{n}}{n^{\alpha}}\left\lbrace 
	\begin{array}{l}
		\text{收敛},\alpha>\dfrac{1}{2}\\
		\text{发散},\alpha \leq \dfrac{1}{2}
	\end{array}
	\right. $$
\end{solution}
\myspace{1}

\hl{\textbf{\textit{April 27}}}

1. 设连续函数 $f(x)$ 在 $(-\infty,+\infty)$ 上单调增加,下列说法正确的是: 
\begin{itemize}
	\item A. $\tan f(x)$ 在 $(-\infty,+\infty)$ 上单调增加
	\item B. $f'(x)>0,\quad x\in(-\infty,+\infty)$
	\item C. $\int_{-1}^{x}\dfrac{f(t)}{1+f^{2}(t)}dt$在$(-\infty,+\infty)$上单调增加
	\item \hl{\textbf{D}}. $\int_{-1}^{e^x}\dfrac{1}{1+f^{2}(t)}dt$在$(-\infty,+\infty)$上单调增加
\end{itemize}
\myspace{1}
\begin{solution}
	
(i). 对于$f(x)=x^3$: $f'(x)\geq 0$,$tan f(x)$在$(-\infty,+\infty)$ 上不单调,$A\text{、}B$错误
	
(ii).令$F(x)=\int_{-1}^{x}\frac{f(t)}{1+f^{2}(t)}dt,F'(x)=\frac{f(x)}{1+f^{2}(x)}$,$f(x)$正负性未知,无法判断函数单调性,$C$错误
	
(iii)令$F(x)=\int_{-1}^{x}\frac{1}{1+f^{2}(t)}dt,F'(x)=\frac{1}{1+f^{2}(x)}>0$,$F(x)$在$(-\infty,+\infty)$上单调递增.
	
正确答案: $D$
\end{solution}
\myspace{1}

2. 下列微分方程是以$y=C_{1}e^x+C_{2}\cos 2x+C_{3}\sin 2x,(C_{1},C_{2},C_{3}\in \mathbb{R})$ 为通解的微分方程为: 
\begin{itemize}
	\item A. $y'''+y''-4y'-4y=0$
	\item B. $y'''+y''+4y'+4y=0$
	\item C. $y'''-y''-4y'+4y=0$
	\item \hl{D}. $y'''-y''+4y'-4y=0$
\end{itemize}
\myspace{1}
\begin{solution}
	
	我们易得: $y=C_{1}e^{x}+C_{4}e^{0}(A\cos 2x+B\sin 2x)$
	
	我们可以得到原三阶微分方程对应的特征方程的三个根 $x_{1}=1,\ x_{2}=2i,\ x_{3}=-2i$,特征方程为: $(r-1)(r^2+4)=0\Rightarrow r^3-r^2+4r-4=0$
	
	我们得到微分方程表达式为: $y'''-y''+4y'-4y=0$,故答案选$D$
\end{solution}
\myspace{1}

\hl{\textbf{\textit{April 28}}}

1. 设 $f(x)$ 在 $[0,a]$连续可导,证明: $\int_{0}^{a}dx\int_{0}^{x}\dfrac{f'(y)}{\sqrt{(a-x)(x-y)}}dy=\pi[f(a)-f(0)]$
\myspace{1}
\begin{solution}
	
	原二重积分等价于: 
	\begin{eqnarray*}
		\int_{0}^{a}dy\int_{y}^{a}\frac{f'(y)}{\sqrt{(\frac{a-y}{2})^2-(x-\frac{a+y}{2})^2}}dx&=&\int_{0}^{a}f'(y)[\arcsin (\frac{2x-a-y}{a-y})]|_{y}^{a}dy\\
		&=&\pi\int_{0}^{a}f'(y)dy=\pi[f(a)-f(0)]
	\end{eqnarray*}
	
\end{solution}
\myspace{1}

2. $f(x)$ 连续且为奇函数,下列函数一定是偶函数的是: 
\begin{itemize}
	\item A. $\int_{0}^{x}du\int_{a}^{u}tf(t)dt$ 
	\item B. $\int_{a}^{x}du\int_{0}^{u}f(t)dt$ 
	\item C. $\int_{0}^{x}du\int_{a}^{u}f(t)dt$ 
	\item \hl{D}. $\int_{a}^{x}du\int_{0}^{u}tf(t)dt$ 
\end{itemize}
\myspace{1}
\begin{solution}
	
	(i).对于$A\text{、}C$,我们交换二重积分的积分次序: 
	$$A\text{: }\int_{0}^{x}du\int_{a}^{0}tf(t)dt+\int_{0}^{x}du\int_{0}^{u}tf(t)dt=x\int_{a}^{0}tf(t)dt+\int_{0}^{x}t^2f(t)dt,\text{前一个是奇函数,后一个是偶函数}$$
	$$C\text{: }\int_{0}^{x}du\int_{a}^{u}f(t)dt=\int_{0}^{x}du\int_{a}^{0}f(t)dt+\int_{0}^{x}du\int_{0}^{u}f(t)dt,\text{前一个是奇函数,后一个是奇函数}$$
	
	(ii).对于$B\text{、}D$,我们交换二重积分的积分次序: 
	
	$$B\text{: }\int_{a}^{0}du\int_{0}^{u}f(t)dt+\int_{0}^{x}du\int_{0}^{u}f(t)dt=\int_{a}^{x}tf(t)dt,\text{奇函数}$$
	$$D\text{: }\int_{a}^{0}du\int_{0}^{u}tf(t)dt+\int_{0}^{x}du\int_{0}^{u}tf(t)dt=\int_{a}^{x}t^2f(t)dt,\text{偶函数}$$
	
	此题答案为: $D$
\end{solution}
\myspace{1}

\hl{\textbf{\textit{April 29}}}

1. 证明:  $\int_{0}^{1}dx\int_{0}^{1}(xy)^{xy}dy=\int_{0}^{1}x^{x}dx$
\myspace{1}
\begin{solution}
	
	我们令$xy=t,y=\dfrac{1}{x}t$,我们得到原二重积分为: $\int_{0}^{1}dx\int_{0}^{x}\dfrac{1}{x}t^{t}dt$.
	
	我们交换二重积分的积分次序: 
	$$\int_{0}^{1}dt\int_{t}^{1}\frac{1}{x}t^{t}dx=-\int_{0}^{1}t^{t}\ln tdt=-\int_{0}^{1}x^{x}\ln xdx$$
	
	我们只需要证明: 
	$$-\int_{0}^{1}x^{x}\ln xdx=\int_{0}^{1}x^{x}dx\Rightarrow\int_{0}^{1}x^{x}(1+\ln x)dx=\int_{0}^{1}e^{x\ln x}d(x\ln x)=0$$
	
	证毕.
\end{solution}
\myspace{1}

2.微分方程 $y''-4y'+8y=e^{2x}(1+\cos 2x)$ 的特解可以设为哪种形式: 
\begin{itemize}
	\item A. $Ae^{2x}+e^{2x}(B\cos 2x+C\sin 2x)$ 
	\item B. $Axe^{2x}+e^{2x}(B\cos 2x+C\sin 2x)$ 
	\item \hl{C}. $Ae^{2x}+xe^{2x}(B\cos 2x+C\sin 2x)$ 
	\item D. $Axe^{2x}+xe^{2x}(B\cos 2x+C\sin 2x)$ 
\end{itemize}
\myspace{1}
\begin{solution}
	
	原微分方程对应的特征方程为: $r^2-4r+8=0\Rightarrow r_{1}=2+2i,\ r_{2}=2-2i$
	
	齐次微分方程的通解为: $e^{2x}(A\cos 2x+B\sin 2x)$
	
	对于方程: $y''-4y'+8y=e^{2x}$,特解为: $C_{1}e^{2x}$
	
	对于方程: $y''-4y'+8y=e^{2x}\cos 2x$,特解为: $xe^{2x}(C_{2}\cos 2x+C_{3}\sin 2x)$
	
	我们得到方程的特解形式为: $y=C_{1}e^{2x}+xe^{2x}(C_{2}\cos 2x+C_{3}\sin 2x)$,故此答案选$C$
\end{solution}
\myspace{1}

\hl{\textbf{\textit{April 30}}}

1. $f(x)$ 连续且为偶函数,下列函数一定是偶函数的是: 
\begin{itemize}
	\item A. $\int_{0}^{x}(x-t^2)f(t)dt$ 
	\item B. $\int_{a}^{x}f(x-t)dt$ 
	\item \hl{C}. $\int_{0}^{x}(x-2t)f(t)dt$ 
	\item D. $\int_{a}^{x}(x-2t)f(t)dt$ 
\end{itemize}
\myspace{1}
\begin{solution}
	
	令$f(x)=1$,我们得到: 
	
	(i).$\int_{0}^{x}(x-t^2)f(t)dt=\int_{0}^{x}(x-t^2)dt=x^2-\dfrac{x^3}{3}\quad \text{非奇非偶函数}$
	
	(ii).$\int_{a}^{x}f(x-t)dt=\int_{a}^{x}dt=x-a,\text{当}a=0\text{时为奇函数}$
	
	(iii).$\int_{0}^{x}(x-2t)f(t)dt=\int_{0}^{x}(x-2t)dt=-\dfrac{x^2}{2},\text{偶函数}$
	
	(iiii). $\int_{a}^{x}(x-2t)f(t)dt=\int_{a}^{x}(x-2t)dt=-\dfrac{x^2}{2}-\dfrac{ax}{2}+a^2,\text{当}a=0\text{时为偶函数}$
	
	故答案为: $C$
\end{solution}
\myspace{1}

2. $\lim\limits_{x\rightarrow +\infty}\dfrac{\int_{1}^{x}t^{-5}dt}{\int_{1}^{x}t^{-3}dt}$
\myspace{1}
\begin{solution}
	
	原极限为: $\lim\limits_{x\rightarrow +\infty}\dfrac{\frac{1}{4}-\frac{1}{4x^4}}{\frac{1}{2}-\frac{1}{2x^2}}=\lim\limits_{x\rightarrow +\infty}\dfrac{1}{2}+\dfrac{1}{2x^2}=\dfrac{1}{2}$
\end{solution}