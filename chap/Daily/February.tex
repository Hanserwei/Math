\chapterimage{chap28.jpg}
\chapter{February}
\section{Week \Rmnum{1}}
\textcolor{orange}{\textbf{February 1}}

\begin{example}[][Exam: 28.1.1]
	设函数 $f(x)$ 满足关系式 $f''(x)+[f'(x)]^{2}=x$ 且 $f'(0)=0$,则:
\begin{itemize}
	\item A. $f(0)$ 是 $f(x)$ 的极大值
	\item B. $f(0)$ 是 $f(x)$ 的极小值
	\item C. $(0,f(0))$ 是曲线 $y=f(x)$ 的拐点
	\item D. $f(0)$ 不是 $f(x)$ 的极值,$(0,f(0))$ 也不是曲线 $y=f(x)$ 的拐点
\end{itemize}
\end{example}

\begin{solution}

	$$f''(x) = x - [f'(x)]^{2}, f'(0) = 0\Rightarrow f''(0) = 0$$
	
	$$f^{(3)}(x) = 1 - 2f'(x)f''(x), f^{(3)}(0) = 1\neq 0$$

	综上所述: $x=0$ 不是 $f(x)$ 的极值点, 点 $(0,f(0))$ 是 曲线 $f(x)$ 的拐点
\end{solution}
\myspace{1}

\begin{example}[][Exam: 28.1.2]
	设函数 $f(x)$ 在 $(-\infty,+\infty)$ 上连续,其导函数图形如图所示,则:
\begin{tikzpicture}
    \draw[->] (-2,0) -- (3,0) node[right] {$x$};
    \draw[->] (0,-2) -- (0,2) node[above] {$y$};
    \draw[thick, domain=-1:0.9, smooth, variable=\x] plot ({\x},{2-(\x+1)*(\x+1)});
    \draw[thick, domain= 1.2:2.3, smooth, variable=\x] plot ({\x},{10*(\x-1.5)*(\x-2)*(\x-2)});
    \draw[dashed] (1,-2) -- (1,1.5);
    \node at (-0.2,-0.2) {O};
\end{tikzpicture}
\begin{itemize}
	\item A. 函数 $f(x)$ 有 $2$ 个极值点,曲线 $y=f(x)$ 有 $2$ 个拐点
	\item B. 函数 $f(x)$ 有 $2$ 个极值点,曲线 $y=f(x)$ 有 $3$ 个拐点
	\item C. 函数 $f(x)$ 有 $3$ 个极值点,曲线 $y=f(x)$ 有 $1$ 个拐点
	\item D. 函数 $f(x)$ 有 $3$ 个极值点,曲线 $y=f(x)$ 有 $2$ 个拐点
\end{itemize}
\end{example}

\begin{solution}

	$f'(x)$ 在 $(-\infty,+\infty)$ 上有 $3$ 个零点 $x_{1},x_{2},x_{3}$
	$$\begin{cases}
	x\in(-\infty,x_{1})\cup (x_{2},+\infty), f'(x) > 0\\
	x\in(x_{1},x_{2}), f'(x) < 0
	\end{cases}\Rightarrow 
	\begin{cases}
	x = x_{1}\text{是} f(x) \text{极大值点}\\
	x = x_{2}\text{是} f(x) \text{极小值点}
	\end{cases}$$

	观察函数 $f'(x)$ 的单调性, $f'(x)$ 在 $(-\infty,a)$ 上单调减少;$(a,b)$ 上单调增加, $(b,c)$ 上单调减少; $(c,+\infty)$ 上单调增加;
	$f(x)$ 有拐点 $(a,f(a))$ 和 $(b,f(b))$以及 $(c,f(c))$

	综上所述, $f(x)$ 有 $2$ 个极值点和 $3$ 个拐点
\end{solution}
\myspace{1}

\textcolor{orange}{\textbf{February 2}}

\begin{example}[][Exam: 28.1.3]
	曲线 $y=x\ln\left( e+\dfrac{1}{x}\right)(x>0) $ 的渐近线方程为:
\end{example}

\begin{solution}

	(i). 铅垂渐近线: $x\to 0, f(x)\to 0$, 无铅垂渐近线

	(ii). 水平渐近线和斜渐进线: $x\to +\infty, f(x)\to +\infty$, 无水平渐近线

	$$\begin{cases}
	a = \lim\limits_{x\to +\infty}\dfrac{f(x)}{x} = 1\\
	b = \lim\limits_{x\to +\infty}f(x)-ax = \dfrac{1}{e}
	\end{cases}$$

	综上所述, $f(x)$ 有且仅有一条斜渐近线: $x-y+\dfrac{1}{e} = 0$
\end{solution}
\myspace{1}

\begin{example}[][Exam: 28.1.4]
	曲线 $y=\dfrac{x^{2}+x}{x^{2}-1}$ 渐近线的条数为:
\begin{itemize}
	\item A. $0$
	\item B. $1$
	\item C. $2$
	\item D. $3$
\end{itemize}
\end{example}

\begin{solution}

	令 $f(x) = \dfrac{x(x+1)}{(x+1)(x-1)}=\dfrac{x}{x-1},x\neq \pm 1$, 我们有: 
	$\begin{cases}
	x\to -1, f(x)\to \dfrac{1}{2}\\
	x\to 1^{+}, f(x)\to +\infty\\
	x\to 1^{-}, f(x)\to -\infty
	\end{cases}$

	(i). 铅垂渐近线: $x = 1$

	(ii). 水平渐近线和斜渐近线: $x\to \infty, f(x)\to 1$, 无斜渐近线

	综上所述, $f(x)$ 有且仅有 $2$ 条渐近线, $1$ 条铅垂渐近线, $1$ 条水平渐近线
\end{solution}
\myspace{1}

\textcolor{orange}{\textbf{February 3}}

\begin{example}[][Exam: 28.1.5]
	设函数 $y=\dfrac{x^{3}+4}{x^{2}}$,求

(1). 函数的增减区间及极值

(2). 函数图像的凹凸区间及拐点

(3). 渐近线

(4). 作出其图形
\end{example}

\begin{solution}

	令 $f(x) = \dfrac{x^{3}+4}{x^{2}} = x+\dfrac{4}{x^{2}},x\neq 0$

	(1). $f'(x) = 1-\dfrac{8}{x^{3}}\Rightarrow 
	\begin{cases}
	x\in(-\infty,0)\cup (0,2), f'(x) < 0\\
	x\in(2,+\infty), f'(x) > 0
	\end{cases}$

	$f(x)$ 单调递增区间 $(2,+\infty)$, 单调递减区间 $(-\infty,0)$ 和 $(0,2)$, $f(x)$ 无极大值, 在 $x=2$ 处取得极小值 $f(2) = 3$

	(2). $f''(x) = \dfrac{24}{x^{4}} > 0$, $f(x)$ 在定义域上是凹函数, 无拐点

	(3). 
	
	(i). 铅垂渐近线: $x\to 0, f(x)\to +\infty$, $f(x)$ 有铅垂渐近线 $x=0$

	(ii). 水平渐近线和斜渐近线: $x\to \infty, f(x)\to \infty$, 无水平渐近线
	$$\begin{cases}
	a = \lim\limits_{x\to \infty} \dfrac{f(x)}{x} =1\\
	b = \lim\limits_{x\to \infty} f(x)-ax = 0
	\end{cases}\Rightarrow y = x$$

	综上所述, $f(x)$ 有 $2$ 条渐近线, $1$ 条铅垂渐近线 $x=0$, 一条斜渐近线 $y = x$

	(4). 如下图所示:

	\begin{tikzpicture}[scale=0.6]
		\draw[->] (-5,0) -- (5,0) node[right] {$x$};
		\draw[->] (0,-5) -- (0,5) node[above] {$y$};
		\draw[thick, domain=-3:-1, smooth, variable=\x] plot ({\x},{(\x+4/(\x*\x))});
		\draw[thick, domain=1:4, smooth, variable=\x] plot ({\x},{(\x+4/(\x*\x))});
		\draw[dashed, domain=-4:4, smooth, variable=\x] plot ({\x},{\x});
		\node at (-0.2,-0.2) {O};
	\end{tikzpicture}
\end{solution}
\myspace{1}

\begin{example}[][Exam: 28.1.6]
	在区间 $(-\infty,+\infty)$ 内,方程 $|x|^{\frac{1}{4}}+|x|^{\frac{1}{2}}-\cos x=0$: 
\begin{itemize}
	\item A. 无实根
	\item B. 有且仅有一个实根
	\item C. 有且仅有两个实根
	\item D. 有无穷多个实根
\end{itemize}
\end{example}

\begin{solution}

	令 $f(x) = |x|^{\frac{1}{4}}+|x|^{\frac{1}{2}}-\cos x$, $f(x)$ 是偶函数, 取 $x > 0$, $f(x) = x^{\frac{1}{4}}+x^{\frac{1}{2}}-\cos x$

	(i). 当 $x \geq 1$ 时, $f(x) > 0$

	(ii). 当 $x\in (0,1)$ 时, $f'(x) = \dfrac{1}{4}x^{-\frac{3}{4}}+\dfrac{1}{2}x^{-\frac{1}{2}}+\sin x > 0$
	
	$f(x)$ 在 $(0,1)$ 单调递增
	$$\begin{cases}f(0) = -1<0\\f(1)=2-\cos 1\\ f(0)\cdot f(1)<0 \end{cases}\Rightarrow f(x)\text{在}(0,1) \text{内有且仅有} 1 \text{个零点}$$

	综上所述, $f(x)$ 有且仅有 $2$ 个零点, $x_{1}\in(-1,0),x_{2}\in(0,1)$
\end{solution}
\myspace{1}

\textcolor{orange}{\textbf{February 4}}

\begin{example}[][Exam: 28.1.7]
	函数 $f(x)=\ln|(x-1)(x-2)(x-3)|$ 的驻点个数为:
\begin{itemize}
	\item A. $0$
	\item B. $1$
	\item C. $2$
	\item D. $3$
\end{itemize}
\end{example}

\begin{solution}

	令 $g(x) =|(x-1)(x-2)(x-3)|$, $f(x)$ 和 $g(x)$ 同单调性,令 $h(x) = (x-1)(x-2)(x-3)$

	罗尔定理: $\exists x_{1}\in(1,2), x_{2}\in (2,3),\ s.t.\ f'(x_{1}) =f'(x_{2}) =0$

	$$\begin{cases}
	x\in(-\infty,x_{1]})\cup (x_{2},+\infty), f'(x) > 0\\
	x\in(x_{1}, x_{2}), f'(x) < 0\\
	f(1) =f(2) =f(3) =0
	\end{cases}\Rightarrow |h(x)|\text{在} x=1,2,3\text{处均不可导}$$

	$f(x)$ 的驻点只有$2$个驻点, 分别是 $x=x_{1}$ 和 $x=x_{2}$ 
\end{solution}
\myspace{1}

\begin{example}[][Exam: 28.1.8]
	设 $f(x)=x^{2}(1-x)^{2}$,则方程 $f''(x)=0$在 $(0,1)$ 上:
\begin{itemize}
	\item A. 无实根
	\item B. 有且仅有一个实根
	\item C. 有且仅有两个实根
	\item D. 有且仅有三个实根
\end{itemize}
\end{example}

\begin{solution}

	$f'(x) = 2x(x-1)^{2}+2x^{2}(x-1) = 2x(x-1)(3x-1)$ 且 $f'(0) = f'(\dfrac{1}{3}) = f(1) = 0$

	罗尔定理: $\exists x_{1}\in (0,\dfrac{1}{3}), x_{1}\in(\dfrac{1}{3}),\ s.t.\ f''(x_{i}) = 0(i = 1,2)$

	$f''(x)$ 是一个二次多项式, 至多存在 $2$ 个实数根, 综上, $f''(x)$ 有且仅有 $2$ 个实数根.
\end{solution}
\myspace{1}

\textcolor{orange}{\textbf{February 5}}

\begin{example}[][Exam: 28.1.9]
	设常数 $k>0$,设函数 $f(x)=\ln x-\dfrac{x}{e}+k$ 在 $(0,+\infty)$ 内零点个数为:
\begin{itemize}
	\item A. $3$
	\item B. $2$
	\item C. $1$
	\item D. $0$
\end{itemize}
\end{example}

\begin{solution}

	$$f'(x) = \dfrac{1}{x}-\dfrac{1}{e} = \dfrac{e-x}{ex}, x > 0$$

	当 $x\in(0,e)$, $f'(x) > 0$; 当 $x\in(e,+\infty)$, $f'(x) < 0$; $f(x)$ 在 $x = e$ 处取得最大值 $f(e) = k > 0$

	且 $$\begin{cases}
		x\to 0, f(x)\to -\infty\\
		x\to +\infty, f(x)\to -\infty
	\end{cases}$$

	零点定理: $\exists x_{1}\in(0,e),x_{2}\in(e,+\infty),\ s.t.\ f(x_{i}) = 0(i = 1,2)$

	综上所述, $f(x)$ 在 $(0,+\infty)$ 内有且仅有 $2$ 个零点
\end{solution}
\myspace{1}

\begin{example}[][Exam: 28.1.10]
	证明:当 $x>0, \ln(1+\dfrac{1}{x})>\dfrac{1}{1+x}$
\end{example}

\begin{solution}
	
	构造辅助函数: $f(x) = \ln x, \ln(1+\dfrac{1}{x}) = \ln(1+x) -\ln x = f(x+1)-f(x)$

	拉格朗日中值定理: $$f(x+1)-f(x) = \dfrac{1}{\xi}, \xi\in(x,x+1)\Rightarrow \dfrac{1}{x+1}< f(x+1)-f(x) < \dfrac{1}{x}$$

	综上所述: $\dfrac{1}{1+x} < \ln(1+\dfrac{1}{x}) < \dfrac{1}{x}$
\end{solution}
\myspace{1}

\textcolor{orange}{\textbf{February 6}}

\begin{example}[][Exam: 28.1.11]
	证明: $x > 0,\arctan x+\dfrac{1}{x}>\dfrac{\pi}{2}$
\end{example}
\begin{solution}

	构造辅助函数: 
	$$f(x) = \arctan x +\dfrac{1}{x}, f'(x) = \dfrac{1}{1+x^{2}}-\dfrac{1}{x^{2}} = -\dfrac{1}{(1+x^{2})x^{2}} < 0, x>0$$

	$f(x)$ 在 $(0,+\infty)$ 上单调递减, $f(x) > \lim\limits_{x\to +\infty}f(x)$

	$$\lim\limits_{x\to +\infty}f(x) = \lim\limits_{x\to +\infty}(\arctan x +\dfrac{1}{x}) =\dfrac{\pi}{2}$$

	综上所述, $f(x) > \dfrac{\pi}{2}$
\end{solution}
\myspace{1}

\begin{example}[][Exam: 28.1.12]
	设 $p,q$ 是大于 $1$ 的常数,且 $\dfrac{1}{p}+\dfrac{1}{q}=1$,
	证明: $\forall x>0, \dfrac{1}{p}x^{p}+\dfrac{1}{q}\geq x$
\end{example}

\begin{solution}

	$\forall x > 0, \dfrac{1}{p}x^{p} -x +1-\dfrac{1}{p} \geq 0$
	
	构造辅助函数: $f(x) = \dfrac{1}{p}(x^{p}-1) - (x-1), x > 0$

	$$\begin{cases}
		f'(x) = x^{p-1}-1, p > 1\\
		f''(x) = (p-1)x^{p-2}
	\end{cases}\Rightarrow 
	\begin{cases}
		x\in (-\infty,1), f'(x) < 0\\
		x\in(1,+\infty), f'(x) > 0
	\end{cases}$$
	$$f(x)\text{在} x=1\text{取最小值} f(x)_{\min} = f(1) =0$$

	综上所述, $f(x)\geq 0\Rightarrow \forall x > 0, \dfrac{1}{p}x^{p}+\dfrac{1}{q}\geq x$
\end{solution}
\myspace{1}

\textcolor{orange}{\textbf{February 7}}

\begin{example}[][Exam: 28.1.13]
	设函数 $f(x)$ 在 $[0,3]$ 上连续,在 $(0,3)$内可导,且 $f(0)+f(1)+f(2)=3,f(3)=1$,证明: $$\exists \xi\in(0,3),\ s.t.\ f'(\xi)=0$$
\end{example}
 
\begin{solution}

	不妨设 $f(x)$ 在 $[0,2]$ 上最大值为 $M$, 最小值为 $m$, 介值定理:
	$$\begin{cases}
		m\leq f(0)\leq M\\
		m\leq f(1)\leq M\\
		m\leq f(2)\leq M
	\end{cases}$$
	$$m\leq \dfrac{f(0)+f(1)+f(2)}{3}\leq M
	\Rightarrow \exists \eta\in(0,2),\ s.t.\ f(\eta) = \dfrac{f(0)+f(1)+f(2)}{3}=1$$

	罗尔定理:
	$$\begin{cases} 
		f(\eta) = 1,\eta\in[0,2]\\
		f(3) =1
	\end{cases}\Rightarrow \exists \xi\in(\eta,3)\subset (0,3),\ s.t.\ f'(\xi) = 0$$

	综上所述, $\exists \xi\in(0,3),\ s.t.\ f'(\xi)=0$
\end{solution}
\myspace{1}

\begin{example}[][Exam: 28.1.14]
	设 $f(x)$在区间 $[a,b]$ 上具有二阶导数, 且 $f(a)=f(b)=0,f'(a)f'(b)>0$,证明:
	$$\exists \xi\in(a,b), \eta\in(a,b),\ s.t.\ f(\xi) = 0, f''(\eta) = 0$$
\end{example}

\begin{solution}

	极限保号性: 不妨设 $f'(a) > 0$
	$$\begin{cases}
	f'(a) = \lim\limits_{x\to a} \dfrac{f(x)-f(a)}{x-a}\\
	f'(b) = \lim\limits_{x\to a} \dfrac{f(x)-f(b)}{x-b}\\
	f'(a)\cdot f'(b) > 0
	\end{cases}\Rightarrow 
	\begin{cases}
	x\in(a,a+\delta), f(x) > f(a) =0, f(x_{1}) > 0, x_{1}\in(a,a+\delta)\\
	x\in(b-\delta,b), f(x) < f(b) =0, f(x_{2}) > 0, x_{2}\in(b-\delta,b)
	\end{cases}$$
	$$\begin{cases}
	f(x_{1}) > 0, x_{1}\in(a,a+\delta)\\
	f(x_{2}) > 0, x_{2}\in(b-\delta,b)
	\end{cases}\Rightarrow f(x_{1})\cdot f(x_{2}) < 0$$

	零点定理: $\exists \xi\in(a,b),\ s.t.\ f(\xi) = 0$

	罗尔定理: $f(a) = f(\xi) =f(b) =0$
	$$\begin{cases}
	\exists x_{3}\in (a,\xi),\ s.t.\ f'(x_{3}) = 0\\
	\exists x_{4}\in (\xi,b),\ s.t.\ f'(x_{4}) = 0
	\end{cases}\Rightarrow \exists \eta\in (x_{3},x_{4}),\ s.t.\ f''(\eta) = 0$$

	综上所述, 存在 $\xi\in(a,b)$ 和 $\eta\in(a,b)$,$s.t.\ f(\xi)=0$ 且 $f''(\eta)=0$
\end{solution}
\myspace{1}

\section{Week \Rmnum{2}}
\textcolor{blue}{\textbf{February 8}}

\begin{example}[][Exam: 28.2.1]
	设 $f(x)$ 在 $[a,b]$ 上连续,在 $(a,b)$ 内可导,且 $f(a)=f(b)=0$,证明:

(1). $\exists \xi\in(a,b),\ s.t.\ f'(\xi)+f(\xi)=0$

(2). $\exists \eta\in(a,b),\ s.t.\ f'(\eta)-f(\eta)=0$

(3). $\exists \zeta\in(a,b),\ s.t.\ f'(\zeta)+\lambda f(\zeta)=0$
\end{example}

\begin{solution}

	(1). 构造辅助函数: $g(x) = e^{x}f(x), g'(x) = e^{x}\left[f'(x) + f(x)\right]$, 且 $g(a) = g(b) = 0$

	罗尔定理: $\exists \xi\in(a,b),\ s.t.\ g'(\xi) = e^{\xi}\left[f'(\xi) + f(\xi)\right] = 0\Rightarrow f'(\xi) + f(\xi) = 0$

	

	(2). 构造辅助函数: $g(x) = e^{-x}f(x), g'(x) = e^{-x}\left[f'(x) - f(x)\right]$, 且 $g(a) = g(b) = 0$

	罗尔定理: $\exists \eta\in(a,b),\ s.t.\ g'(\eta) = e^{-\eta}\left[f'(\eta) - f(\eta)\right] = 0\Rightarrow f'(\eta) - f(\eta) = 0$

	

	(3). 构造辅助函数: $g(x) = e^{\lambda x}f(x), g'(x) = e^{\lambda x}\left[f'(x) + \lambda f(x)\right]$, 且 $g(a) = g(b) = 0$

	罗尔定理: $\exists \zeta\in(a,b),\ s.t.\ g'(\zeta) = e^{\lambda\zeta}\left[f'(\zeta) + \lambda f(\zeta)\right] = 0\Rightarrow f'(\zeta) + \lambda f(\zeta) = 0$
\end{solution}
\myspace{1}

\begin{example}[][Exam: 28.2.2]
	设 $f(x)$ 在 $[0,1]$ 上连续,在 $(0,1)$ 内可导,且 $f(1)=0$,证明:
	$$\exists \xi\in(0,1),\ s.t.\ \xi f'(\xi) + f(\xi) = 0$$
\end{example}

\begin{solution}

	构造辅助函数: $g(x) = xf(x), g'(x) = f(x) +xf'(x)$, 且 $g(1) = g(0) = 0$

	罗尔定理: $\exists \xi\in(0,1),\ s.t.\ g'(\xi) = 0\Rightarrow \xi f'(\xi) + f(\xi) = 0$

	综上所述, $\exists \xi\in(0,1),s.t.\ \xi f'(\xi) + f(\xi) = 0$
\end{solution}
\myspace{1}

\textcolor{blue}{\textbf{February 9}}

\begin{example}[][Exam: 28.2.3]
	设函数 $f(x)$ 在闭区间 $[0,1]$ 上连续,在开区间$(0,1)$ 内可导,且$f(0)=0,f(1)=\dfrac{1}{3}$,证明:
	$$\exists \xi\in\left(0,\dfrac{1}{2}\right),\eta\in\left( \dfrac{1}{2},1\right),\ s.t.\ f'(\xi)+f'(\eta)=\xi^{2}+\eta^{2}$$
\end{example}
\begin{solution}

	构造辅助函数: $g(x) = f(x) - \dfrac{1}{3}x^{3}$, $g(0) = g(1) = 0$, $g'(x) =f'(x) -x^{2}$

	拉格朗日中值定理:
	$$\begin{cases}
		2[g(\dfrac{1}{2})-g(0)] = g'(\xi), \xi\in(0,\dfrac{1}{2})\\
		2[g(1)-g(\dfrac{1}{2})] = g'(\eta), \eta\in(\dfrac{1}{2},1)\\
		g(1) = g(0) = 0
	\end{cases}\Rightarrow g'(\xi) + g'(\eta) = 0\Rightarrow f'(\xi) + f'(\eta) =\xi^{2} +\eta^{2}$$

	综上所述, $\exists \xi\in\left(0,\dfrac{1}{2}\right),\eta\in\left( \dfrac{1}{2},1\right),\ s.t.\ f'(\xi)+f'(\eta)=\xi^{2}+\eta^{2}$
\end{solution}
\myspace{1}

\begin{example}[][Exam: 28.2.4]
	设 $f(x)$ 在区间 $[a,b]$ 上连续,在 $(a,b)$ 内可导,且 $a,b$ 同号,证明:
	$$\exists \xi,\eta\in(a,b),\ s.t.\ abf'(\xi)=\eta^{2}f'(\eta)$$
\end{example}

\begin{solution}

	拉格朗日中值定理:
	$$\exists \xi\in (a,b),\ s.t.\ \dfrac{f(b)-f(a)}{b-a} = f'(\xi)$$

	柯西中值定理: $g(x) = \dfrac{1}{x}$
	$$\exists \eta\in(a,b),\ s.t.\ \dfrac{ab[f(b)-f(a)]}{a-b} = \eta^{2}f'(\eta)$$

	综上: $abf'(\xi) = \eta^{2}f'(\eta)$

	综上所述, $\exists \xi,\eta\in(a,b),\ s.t.\ abf'(\xi)=\eta^{2}f'(\eta)$
\end{solution}
\myspace{1}

\textcolor{blue}{\textbf{February 10}}

\begin{example}[][Exam: 28.2.5]
	求下列的不定积分

(1). $\displaystyle{\int \dfrac{1}{\cos x}dx}$

(2). $\displaystyle{\int \dfrac{1}{\sin x}dx}$
\end{example}

\begin{solution}

	(1). 
	\begin{eqnarray*}
		I & = & \int \dfrac{\cos x}{\cos ^{2}x}dx\\
		  & = & \int \dfrac{1}{1-\sin^{2}x}d(\sin x)\\
		  & = & \int \dfrac{1}{2}\left[\dfrac{1}{1+\sin x}+\dfrac{1}{1-\sin x}\right]d(\sin x)\\
		  & = & \dfrac{1}{2} \ln \dfrac{1+\sin x}{1-\sin x} +C\\
		  & = & \ln \big|\dfrac{1+\sin x}{\cos x}\big| +C\\
		  & = & \ln \big|\sec x + \tan x\big| +C 
	\end{eqnarray*}

	(2). 
	\begin{eqnarray*}
		I & = & \int \dfrac{\sin x}{\sin ^{2}x}dx\\
		  & = & -\int \dfrac{1}{1-\cos^{2}x}d(\cos x)\\
		  & = & -\int \dfrac{1}{2}\left[\dfrac{1}{1+\cos x}+\dfrac{1}{1-\cos x}\right]d(\cos x)\\
		  & = & -\dfrac{1}{2} \ln \dfrac{1+\cos x}{1-\cos x} +C\\
		  & = & -\ln \big|\dfrac{1+\cos x}{\sin x}\big| +C\\
		  & = & -\ln \big|\csc x + \cot x\big| +C 
	\end{eqnarray*}

\end{solution}
\myspace{1}

\begin{example}[][Exam: 28.2.6]
	求下列的不定积分

(1). $\displaystyle{\int \dfrac{x+1}{x(1+xe^{x})}dx}$

(2). $\displaystyle{\int (1+\ln x)(\ln x+\ln\ln x)dx}$
\end{example}

\begin{solution}

	(1). 
	\begin{eqnarray*}
		I & = & \int \dfrac{(x+1)e^{x}}{xe^{x}(1+xe^{x})}dx\\
		  & = & \int \dfrac{1}{xe^{x}(1+xe^{x})}d(xe^{x})\\
		  & = & \int \left[\dfrac{1}{xe^{x}}-\dfrac{1}{1+xe^{x}}\right]d(xe^{x})\\
		  & = & \ln \dfrac{xe^{x}}{1+xe^{x}} +C
	\end{eqnarray*}

	(2). 令 
	$$\begin{cases}
		f(x) = x\ln x \\ 
		g(x) = \ln x+\ln \ln x
	\end{cases}\Rightarrow 
	\begin{cases} 
		f'(x) = 1+\ln x\\
		g'(x) = \dfrac{1+\ln x}{x\ln x}
	\end{cases}$$
	
	\begin{eqnarray*}
		I & = & \int f'(x)g(x)dx\\
		  & = & f(x)g(x) - \int g'(x)f(x)dx\\
		  & = & x(\ln x)\ln(x\ln x) - \int (1+\ln x)dx\\
		  & = & x\ln x\left[\ln(x\ln x) - 1 \right] + C
	\end{eqnarray*}
\end{solution}
\myspace{1}

\textcolor{blue}{\textbf{February 11}}
\begin{example}[][Exam: 28.2.7]
	求下列的不定积分

(1). $\displaystyle{\int \dfrac{1+x}{1+x^{3}}dx}$

(2). $\displaystyle{\int \dfrac{1-x}{1+x^{3}}dx}$
\end{example}

\begin{solution}

	(1). 
	\begin{eqnarray*}
		I & = & \int \dfrac{1}{x^{2}-x+1}dx\\
		  & = & \int \dfrac{1}{(x-\frac{1}{2})^{2}+(\frac{\sqrt{3}}{2})^{2}}dx\\
		  & = & \dfrac{2}{\sqrt{3}}\arctan \dfrac{2x-1}{\sqrt{3}} +C
	\end{eqnarray*}

	(2). 
	\begin{eqnarray*}
		I & = & \int\dfrac{1-x+x^{2}}{1+x^{3}}dx + \int\dfrac{x^{2}}{1+x^{3}}dx\\
		  & = & \int\dfrac{1}{1+x}dx + \int\dfrac{1}{3(1+x^{3})}d(x^{3})\\
		  & = & \ln |1+x| + \dfrac{1}{3}\ln|1+x^{3}| + C
	\end{eqnarray*}
\end{solution}
\myspace{1}

\begin{example}[][Exam: 28.2.8]
	求下列的不定积分

(1). $\displaystyle{\int \dfrac{dx}{1+x^{3}}}$

(2). $\displaystyle{\int \dfrac{x}{1+x^{3}}dx}$
\end{example}
\myspace{1}
\begin{solution}

	(1). 
	\begin{eqnarray*}
		I & = & \int \dfrac{1}{1+x^{3}}dx\\
		  & = & \int \dfrac{1}{(x+1)(x^{2}-x+1)}dx\\
		  & = & \dfrac{1}{3}\int \left[\dfrac{1}{x+1}-\dfrac{x-2}{x^{2}-x+1}\right]dx\\
		  & = & \dfrac{1}{3}\int \dfrac{1}{1+x}dx - \dfrac{1}{6}\int \dfrac{2x-1}{1+x^{2}-x}dx + \dfrac{1}{2}\int \dfrac{1}{1+x^{2}-x}dx\\
		  & = & \dfrac{1}{3}\ln |1+x| - \dfrac{1}{6}\ln |1+x^{2}-x| + \dfrac{1}{\sqrt{3}}\arctan \dfrac{2x-1}{\sqrt{3}} + C\\
	\end{eqnarray*}

	(2). 
	\begin{eqnarray*}
		I & = & \int \dfrac{x}{1+x^{3}}dx\\
		  & = & \dfrac{1}{3}\int \left[\dfrac{x+1}{1+x^{2}-x}-\dfrac{1}{1+x}\right]dx\\
		  & = & \dfrac{1}{6}\int \dfrac{2x-1}{1+x^{2}-x}dx - \dfrac{1}{3}\int \dfrac{1}{1+x}dx + \dfrac{1}{2}\dfrac{1}{1+x^{2}-x}dx\\
		  & = & \dfrac{1}{6}\ln |1+x^{2}-x| - \dfrac{1}{3}\ln |1+x| + \dfrac{1}{\sqrt{3}}\arctan \dfrac{2x-1}{\sqrt{3}} + C
	\end{eqnarray*}
\end{solution}
\myspace{1}

\textcolor{blue}{\textbf{February 12}}

\begin{example}[][Exam: 28.2.9]
	已知 $f(x)$ 的一个原函数为 $\ln^{2}x$,求 $\int xf'(x)dx$
\end{example}
\begin{solution}
	\begin{eqnarray*}
		I & = & \int xf'(x)dx\\
		  & = & \int xdf(x)\\
		  & = & xf(x) - \int f(x)dx \\
		  & = & xf(x) - \ln^{2}x + C\\
		  & = & 2\ln x - \ln^{2}x + C
	\end{eqnarray*}
\end{solution}
\myspace{1}

\begin{example}[][Exam: 28.2.10]
	设 $f(\ln x)=\dfrac{\ln(1+x)}{x}$,求 $\int f(x)dx$
\end{example}
\begin{solution}

	$$f(\ln x) = \dfrac{\ln(1+x)}{x}\Rightarrow f(x) = \dfrac{\ln(1+e^{x})}{e^{x}}$$
	\begin{eqnarray*}
		\int f(x)dx & = & \int \dfrac{\ln(1+e^{x})}{e^{x}}dx\\
		  			& = & \int \ln(1+e^{x})d(-e^{-x})\\
		  			& = & -\dfrac{\ln(1+e^{x})}{e^{x}} + \int \dfrac{1}{1+e^{x}}dx\\
					& = & x - \dfrac{\ln(1+e^{x})}{e^{x}} - \ln(1+e^{x}) + C 
	\end{eqnarray*}
\end{solution}
\myspace{1}

\textcolor{blue}{\textbf{February 13}}

\begin{example}[][Exam: 28.2.11]
	$$\int \max(1,x^{2})dx$$
\end{example}
\begin{solution}
	
	$$\max(1,x^{2}) = 
	\begin{cases}
		x^{2} & x\in(-\infty,-1)\cup(1,+\infty)\\
		1 & x\in[-1,1]
	\end{cases}$$

	$$\int \max(1,x^{2}) dx = 
	\begin{cases}
		\dfrac{1}{3}x^{3} + C -\dfrac{2}{3}\\
		x + C\\
		\dfrac{1}{3}x^{3} + C +\dfrac{2}{3}
	\end{cases}$$
\end{solution}
\myspace{1}

\begin{example}[][Exam: 28.2.12]
	设 $M=\int_{-\frac{\pi}{2}}^{\frac{\pi}{2}}\dfrac{\sin x}{1+x^{2}}\cos^{4}xdx,N=\int_{-\frac{\pi}{2}}^{\frac{\pi}{2}}(\sin^{3}x+\cos^{4}x)dx,P=\int_{-\frac{\pi}{2}}^{\frac{\pi}{2}}(x^{2}\sin^{3}x-\cos^{4}x)dx$
\begin{itemize}
	\item A. $N<P<M$
	\item B. $M<P<N$
	\item C. $N<M<P$
	\item D. $P<M<N$
\end{itemize}
\end{example}
\begin{solution}
	$$\begin{cases}
	  M = 0\\
	  N = \int_{-\frac{\pi}{2}}^{\frac{\pi}{2}}cos^{4}xdx > 0\\
	  P = -\int_{-\frac{\pi}{2}}^{\frac{\pi}{2}}cos^{4}xdx < 0
	\end{cases}\Rightarrow P < M < N$$
\end{solution}
\myspace{1}

\textcolor{blue}{\textbf{February 14}}

\begin{example}[][Exam: 28.2.13]
	设 $M=\int_{-\frac{\pi}{2}}^{\frac{\pi}{2}}\dfrac{(1+x)^{2}}{1+x^{2}}dx,N=\int_{-\frac{\pi}{2}}^{\frac{\pi}{2}}\dfrac{1+x}{e^{x}}dx,K=\int_{-\frac{\pi}{2}}^{\frac{\pi}{2}}(1+\sqrt{\cos x})dx$
\begin{itemize}
	\item A. $M>N>K$
	\item B. $M>K>N$
	\item C. $K>M>N$
	\item D. $K>N>M$
\end{itemize}
\end{example}
\begin{solution}
	$$\begin{cases}
	  M = \int_{-\frac{\pi}{2}}^{\frac{\pi}{2}}\dfrac{1+2x+x^{2}}{1+x^{2}}dx = \pi\\
	  N < \int_{-\frac{\pi}{2}}^{\frac{\pi}{2}}1dx = \pi\\
	  K > \int_{-\frac{\pi}{2}}^{\frac{\pi}{2}}1dx = \pi
	\end{cases}\Rightarrow N < M < K$$
\end{solution}
\myspace{1}

\begin{example}[][Exam: 28.2.14]
	设 $I_{1}=\int_{0}^{\frac{\pi}{4}}\dfrac{\tan x}{x}dx,I_{2}=\int_{0}^{\frac{\pi}{4}}\dfrac{x}{\tan x}dx$
\begin{itemize}
	\item A. $I_{1}>I_{2}>1$
	\item B. $1>I_{1}>I_{2}$
	\item C. $I_{2}>I_{1}>1$
	\item D. $1>I_{2}>I_{1}$
\end{itemize}
\end{example}
\begin{solution}
	$$x \in (0,\dfrac{\pi}{4}), \tan x > x\Rightarrow I_{1} > I_{2}$$

	构造辅助函数: $f(x) = \dfrac{\tan x}{x}$

	$$\begin{cases}
	  f'(x) = \dfrac{x-\sin x}{x^{2}\cos^{2}x} > 0\\
	  f(x) < f(\dfrac{\pi}{4}) = \dfrac{4}{\pi}
	\end{cases}\Rightarrow I_{1} < \dfrac{\pi}{4}\times \dfrac{4}{\pi} = 1$$
\end{solution}
\myspace{1}

\section{Week \Rmnum{3}}
\textcolor{cyan}{\textbf{February 15}}

\begin{example}[][Exam: 28.3.1]
	$$\int_{-2}^{2}\left[\ln(x+\sqrt{1+x^{2}})+\sqrt{1-\dfrac{x^{2}}{4}}\right]dx$$
\end{example}
\begin{solution}
	\begin{eqnarray*}
		I & = & \int_{-2}^{2}\sqrt{1-\dfrac{x^{2}}{4}}dx\\
		  & = & 2\int_{-\frac{\pi}{2}}^{\frac{\pi}{2}}\cos^{2}xdx\\
		  & = & 4\int_{0}^{\frac{\pi}{2}}\cos^{2}xdx\\
		  & = & 4\times \dfrac{1}{2}\times \dfrac{\pi}{2}\\
		  & = & \pi 
	\end{eqnarray*}
\end{solution}
\myspace{1}

\begin{example}[][Exam: 28.3.2]
	$$\int_{-\pi}^{\pi}|x|[x^{3}+\sin^{2}x]\cos^{2}xdx$$
\end{example}
\begin{solution}
	\begin{eqnarray*}
		I & = & \int_{-\pi}^{\pi} |x|\sin^{2}x \cos^{2}xdx\\
		  & = & 2\int_{0}^{\pi}x\sin^{2}x\cos^{2}xdx\\
		  & = & 2\pi\int_{0}^{\frac{\pi}{2}}\sin^{2}x(1-\sin^{2}x)dx\\
		  & = & 2\pi\times \dfrac{1}{2}\times \dfrac{\pi}{2}\times \dfrac{1}{4}\\
		  & = & \dfrac{\pi^{2}}{8}
	\end{eqnarray*}
\end{solution}
\myspace{1}

\textcolor{cyan}{\textbf{February 16}}

\begin{example}[][Exam: 28.3.3]
	$$\int_{0}^{\pi}\sqrt{1-\sin x}dx$$
\end{example}
\begin{solution}
	\begin{eqnarray*}
		I & = & \int_{0}^{\pi}\sqrt{1-\sin x}dx\\
		  & = & \int_{0}^{\frac{\pi}{2}}\sqrt{1-\sin x}dx + \int_{\frac{\pi}{2}}^{\pi}\sqrt{1-\sin x}dx\\
		  & = & \int_{0}^{\frac{\pi}{2}}(\cos \frac{x}{2}-\sin \frac{x}{2})dx + \int_{\frac{\pi}{2}}^{\pi}(\sin \frac{x}{2}-\cos \frac{x}{2})dx\\
		  & = & (2\sin \frac{x}{2}+2\cos \frac{x}{2})\big|_{x=0}^{x=\frac{\pi}{2}} - (2\sin \frac{x}{2}+2\cos \frac{x}{2})\big|_{x=\frac{\pi}{2}}^{x=\pi}\\
		  & = & 4\sqrt{2}-4
	\end{eqnarray*}
\end{solution}
\myspace{1}

\begin{example}[][Exam: 28.3.4]
	$$\int_{\sqrt{e}}^{e^{\frac{3}{4}}}\dfrac{dx}{x\sqrt{\ln x(1-\ln x)}}$$
\end{example}
\begin{solution}
	\begin{eqnarray*}
		I & = & \int_{\sqrt{e}}^{e^{\frac{3}{4}}}\dfrac{d(\ln x)}{\sqrt{\ln x(1-\ln x)}}\\
		  & = & \int_{\frac{1}{2}}^{\frac{3}{4}}\dfrac{1}{\sqrt{(\frac{1}{2})^{2}-(t-\frac{1}{2})^{2}}}dt\\
		  & = & \arcsin(2t-1)\big|_{t=\frac{1}{2}}^{t=\frac{3}{4}}\\
		  & = & \dfrac{\pi}{6}
	\end{eqnarray*}
\end{solution}
\myspace{1}

\textcolor{cyan}{\textbf{February 17}}

\begin{example}[][Exam: 28.3.5]
	$$\int_{0}^{1}\dfrac{\arcsin \sqrt{x}}{\sqrt{x(1-x)}}dx$$
\end{example}
\begin{solution}
	\begin{eqnarray*}
		I & = & 2\int_{0}^{1}\dfrac{\arcsin \sqrt{x}}{2\sqrt{x(1-x)}}dx\\
		  & = & 2\int_{0}^{1}\arcsin \sqrt{x}d(\arcsin \sqrt{x})\\
		  & = & (\arcsin \sqrt{x})^{2}\big|_{x=0}^{x=1}\\
		  & = & \dfrac{\pi^{2}}{4}
	\end{eqnarray*}
\end{solution}
\myspace{1}

\textcolor{cyan}{\textbf{February 18}}

\begin{example}[][Exam: 28.3.6]
	$$\int_{0}^{1}\dfrac{\ln(1+x)}{(2-x)^{2}}dx$$
\end{example}
\begin{solution}
	\begin{eqnarray*}
		I & = & \int_{0}^{1}\dfrac{\ln(1+x)}{(2-x)^{2}}dx\\
		  & = & -\int_{0}^{1}\ln(1+x)d \dfrac{1}{x-2}\\
		  & = & -\dfrac{\ln(1+x)}{x-2}\big|_{x=0}^{x=1} + \int_{0}^{1}\dfrac{1}{(x+1)(x-2)}dx\\
		  & = & \ln 2 + \dfrac{1}{3}\ln\big|\dfrac{x-2}{x+1}\big|\big|_{x=0}^{x=1}\\
		  & = & \dfrac{\ln 2}{3}
	\end{eqnarray*}
\end{solution}
\myspace{1}
\begin{example}[][Exam: 28.3.7]
	已知函数 $f(x)=\int_{1}^{x}\sqrt{1+t^{4}}dt$,则 $\int_{0}^{1}x^{2}f(x)dx$
\end{example}
\begin{solution}
	\begin{eqnarray*}
		\int_{0}^{1}x^{2}f(x)dx & = & \dfrac{1}{3}\int_{0}^{1}f(x)d(x^{3})\\
		& = & \dfrac{1}{3}x^{3}f(x)\big|_{x=0}^{x=1} - \dfrac{1}{3}\int_{0}^{1}x^{3}df(x)\\
		& = & -\dfrac{1}{3}\int_{0}^{1}x^{3}\sqrt{1+x^{4}}dx\\
		& = & -\dfrac{1}{12}\int_{0}^{1}\sqrt{1+x^{4}}d(x^{4})\\
		& = & -\dfrac{1}{18}(1+x^{4})^{\frac{3}{2}}\big|_{x=0}^{x=1}\\
		& = & -\dfrac{\sqrt{2}}{9} 
	\end{eqnarray*}
\end{solution}
\myspace{1}

\textcolor{cyan}{\textbf{February 19}}

\begin{example}[][Exam: 28.3.8]
	已知 $f(x)$ 连续,$\int_{0}^{x}tf(x-t)dt=1-\cos x$,求 $\int_{0}^{\frac{\pi}{2}}f(x)dx$
\end{example}
\begin{solution}
	$$\int_{0}^{x}tf(x-t)dt = 1 - \cos x\Rightarrow \int_{0}^{x}(x-u)f(u)du = 1 - \cos x$$

	$f(x)$ 连续, $\int_{0}^{x}f(u)du$ 可导:
	$$\int_{0}^{x}f(u)du + xf(x) - xf(x) = \sin x\Rightarrow \int_{0}^{x}f(u)du = \sin x$$

	$$\int_{0}^{\frac{\pi}{2}}f(x)dx = \sin \dfrac{\pi}{2} = 1$$

\end{solution}
\myspace{1}

\begin{example}[][Exam: 28.3.9]
	设 $f(x)=\begin{cases}
		x^{2} & 0\leq x\leq 1\\
		2-x & 1<x\leq 2 
	\end{cases}$,记 $F(x)=\int_{0}^{x}f(t)dt(0\leq x\leq 2)$,则有:
	\begin{itemize}
		\item A. $F(x)=
		\begin{cases}
			\dfrac{x^{3}}{3} & 0\leq x\leq 1\\
			2x-\dfrac{x^{2}}{2} & 1<x\leq 2
		\end{cases}$
		\item B. $F(x)=
		\begin{cases}
			\dfrac{x^{3}}{3} & 0\leq x\leq 1\\
			\dfrac{1}{3}+2x-\dfrac{x^{2}}{2} & 1<x\leq 2
		\end{cases}$
		\item C. $F(x)=
		\begin{cases}
			\dfrac{x^{3}}{3} & 0\leq x\leq 1\\
			-\dfrac{7}{6}+2x-\dfrac{x^{2}}{2} & 1<x\leq 2
		\end{cases}$
		\item D. $F(x)=
		\begin{cases}
			\dfrac{x^{3}}{3} & 0\leq x\leq 1\\
			\dfrac{x^{3}}{3}+2x-\dfrac{x^{2}}{2} & 1<x\leq 2
		\end{cases}$
	\end{itemize}
\end{example}
\begin{solution}

	$f(x)$ 在 $[0,2]$ 上连续, $F(x) = \int_{0}^{x}f(t)dt$ 在 $(0,2)$ 上可导

	$$F(x) = 
	\begin{cases}
		\dfrac{1}{3}x^{3}  & 0 \leq x \leq 1\\
		2x - \dfrac{1}{2}x^{2} - \dfrac{7}{6} & 1 < x \leq 2
	\end{cases}$$
\end{solution}
\myspace{1}

\textcolor{cyan}{\textbf{February 20}}

\begin{example}[][Exam: 28.3.10]
	设 $x\geq -1$,求 $\int_{-1}^{x}(1-|t|)dt$
\end{example}
\begin{solution}

	$$\int_{-1}^{x}(1-|t|)dt = 
	\begin{cases}
		\dfrac{1}{2}x^{2} + x + \dfrac{1}{2} & -1 \leq x \leq 0\\
		-\dfrac{1}{2}x^{2} + x + \dfrac{1}{2} & x > 0
	\end{cases}$$
\end{solution}
\myspace{1}
\begin{example}[][Exam: 28.3.11]
	设 $x=x(t)$ 由方程 $\displaystyle{\sin t-\int_{1}^{x-t}e^{-u^{2}}du=0}$ 所确定,试求 $\dfrac{d^{2}x}{dt^{2}}\big|_{t=0}$
\end{example}
\begin{solution}
	$$\sin t -\int_{1}^{x-t}e^{-u^{2}}du = 0\Rightarrow \cos t = e^{-(x-t)^{2}}\left[\dfrac{dx}{dt}-1\right]$$

	当 $t = 0$ 时, $x = 1$ 且 $\dfrac{dx}{dt}\big|_{t = 0} = e + 1$

	$$-\sin t = -2(x-t)e^{-(x-t)^{2}}\left[\dfrac{dx}{dt}-1\right]^{2} + e^{-(x-t)^{2}}\dfrac{d^{2}x}{dt^{2}}$$

	且 $\begin{cases}
	  t = 0\\
	  x = 1\\
	  \frac{dx}{dt}\big|_{t = 0} = e + 1 
	\end{cases}$

	$$\dfrac{d^{2}x}{dt^{2}}\big|_{t = 0} = 2e^{2}$$
\end{solution}
\myspace{1}

\textcolor{cyan}{\textbf{February 21}}

\begin{example}[][Exam: 28.3.12]
	设函数 $f(x)=\int_{0}^{1}|t(t-x)|dt(0<x<1)$,求 $f(x)$ 的极值、单调区间及曲线 $y=f(x)$ 的凹凸区间
\end{example}
\begin{solution}
	\begin{eqnarray*}
		f(x) & = & \int_{0}^{x}t(x-t)dt + \int_{x}^{1}t(t-x)dt\\
			 & = & \dfrac{1}{3}x^{3} - \dfrac{1}{2}x + \dfrac{1}{3}, x\in (0,1)
	\end{eqnarray*}

	$$\begin{cases}
	  f'(x) = x^{2} - \dfrac{1}{2} & x\in(0,1)\\
	  f''(x) = 2x & x\in(0,1)
	\end{cases}\Rightarrow 
	\begin{cases}
		x\in (0,\dfrac{1}{\sqrt{2}}) & f'(x) < 0\\
		x\in (\dfrac{1}{\sqrt{2}},1) & f'(x) > 0
	\end{cases}$$

(1). $f(x)$ 在 $x = \dfrac{1}{\sqrt{2}}$ 处取极小值 $f(\dfrac{1}{\sqrt{2}}) = \dfrac{2-\sqrt{2}}{6}$

(2). $f(x)$ 单调递减区间 $(0,\dfrac{1}{\sqrt{2}})$, 单调递增区间 $(\dfrac{1}{\sqrt{2}},1)$

(3). $f(x)$ 的凹区间 $(0,1)$, 无凸区间

\end{solution}
\myspace{1}
\begin{example}[][Exam: 28.3.13]
	下列反常积分中发散的是:
\begin{itemize}
	\item A. $\int_{0}^{+\infty}xe^{-x}dx$
	\item B. $\int_{0}^{+\infty}xe^{-x^{2}}dx$
	\item C. $\int_{0}^{+\infty}\dfrac{\arctan x}{1+x^{2}}dx$
	\item D. $\int_{0}^{+\infty}\dfrac{x}{1+x^{2}}dx$
\end{itemize}
\end{example}
\begin{solution}
	$$\begin{cases}
	  I_{1} = 1 \\
	  I_{2} = \dfrac{1}{2} \\
	  I_{3} = \dfrac{\pi^{2}}{8} \\
	  I_{4} = \ln \sqrt{1+x^{2}}\big|_{x=0}^{x=+\infty}
	\end{cases}$$
\end{solution}
\myspace{1}

\section{Week \Rmnum{4}}
\textcolor{purplea}{\textbf{February 22}}

\begin{example}[][Exam: 28.4.1]
	$$\int_{5}^{+\infty}\dfrac{dx}{x^{2}-4x+3}$$
\end{example}
\begin{solution}
	\begin{eqnarray*}
		I & = & \int_{5}^{+\infty}\dfrac{dx}{x^{2}-4x+3}\\
		  & = & \int_{5}^{+\infty}\dfrac{dx}{(x-1)(x-3)}\\
		  & = & \dfrac{1}{2}\int_{5}^{+\infty}\left[\dfrac{1}{x-3}-\dfrac{1}{x-1}\right]dx\\
		  & = & \dfrac{1}{2}\ln\left|\dfrac{x-3}{x-1}\right|\big|_{x=5}^{x=+\infty}\\
		  & = & \dfrac{1}{2}\ln 2
	\end{eqnarray*}
\end{solution}
\myspace{1}

\begin{example}[][Exam: 28.4.2]
	$$\int_{1}^{+\infty}\dfrac{dx}{e^{1+x}+e^{3-x}}$$
\end{example}
\begin{solution}
	\begin{eqnarray*}
		I & = & \int_{1}^{+\infty}\dfrac{d(e^{x})}{e^{2x+1}+e^{3}}\\
		  & = & \int_{e}^{+\infty}\dfrac{1}{et^{2}+e^{3}}dt\\
		  & = & \dfrac{1}{e^{2}}\arctan \dfrac{t}{e}\big|_{t=e}^{t=+\infty}\\
		  & = & \dfrac{\pi}{4e^{2}}
	\end{eqnarray*}
\end{solution}
\myspace{1}

\textcolor{purplea}{\textbf{February 23}}

\begin{example}[][Exam: 28.4.3]
	$$\int_{0}^{+\infty}\dfrac{\ln(1+x)}{(1+x)^{2}}dx$$
\end{example}
\begin{solution}
	\begin{eqnarray*}
		I & = & \int_{0}^{+\infty}\dfrac{\ln(1+x)}{(1+x)^{2}}dx\\
		  & = & -\int_{0}^{+\infty}\ln(1+x)d\left(\dfrac{1}{1+x}\right)\\
		  & = & -\dfrac{\ln(1+x)}{1+x}\big|_{x=0}^{x=+\infty} + \int_{0}^{+\infty}\dfrac{1}{1+x}\dfrac{1}{1+x}dx\\
		  & = & -\dfrac{1}{1+x}\big|_{x=0}^{x=+\infty}\\
		  & = & 1
	\end{eqnarray*}
\end{solution}
\myspace{1}

\begin{example}[][Exam: 28.4.4]
	已知抛物线通过 $x$ 轴上的两点 $A(1,0),B(3,0)$

(1). 求证: 两坐标轴与该抛物线所围图形的面积等于 $x$ 轴与该抛物线所围图形的面积

(2). 计算上述两平面图形绕 $x$ 轴旋转一周所产生的两个旋转体体积之比
\end{example}
\begin{solution}

(1). 设抛物线方程为 $y=a(x-1)(x-3)$, 两坐标轴与抛物线围成的图形面积为 $S_{1}$, $x$ 轴与抛物线围成的图形面积为 $S_{2}$

$$\begin{cases}
  S_{1}= \big|\int_{0}^{1}a(x^{2}-4x+3)dx\big| = \dfrac{4}{3}|a|\\
  S_{2} = \big|\int_{1}^{3}a(x^{2}-4x+3)dx\big| = \dfrac{4}{3}|a|
\end{cases}\Rightarrow S_{1} = S_{2}$$

(2). 设 $S_{1}$, $S_{2}$ 绕 $x$ 轴旋转所成旋转体体积分别为 $V_{1}$, $V_{2}$

$$\begin{cases}
  V_{1} = \pi\int_{0}^{1}a^{2}(x^{2}-4x+3)^{2}dx = \dfrac{38\pi}{15}a^{2}\\
  V_{2} = \pi\int_{1}^{3}a^{2}(x^{2}-4x+3)^{2}dx = \dfrac{18\pi}{5}a^{2}
\end{cases}\Rightarrow \dfrac{V_{1}}{V_{2}} = \dfrac{19}{27}$$
\end{solution}
\myspace{1}

\textcolor{purplea}{\textbf{February 24}}

\begin{example}[][Exam: 28.4.5]
	求心形线 $r=a(1+\cos\theta)\ (a>0)$ 所围图形的面积
\end{example}
\begin{solution}
	\begin{eqnarray*}
		S & = & \dfrac{1}{2}\int_{0}^{2\pi}r^{2}d\theta\\
		  & = & \dfrac{1}{2}\int_{0}^{2\pi}a^{2}(1+\cos\theta)^{2}d\theta\\
		  & = & \dfrac{1}{2}\int_{0}^{2\pi}a^{2}(2+2\cos\theta+\cos^{2}\theta)d\theta\\
		  & = & \dfrac{1}{2}\int_{0}^{2\pi}a^{2}(3+2\cos\theta)d\theta\\
		  & = & \dfrac{1}{2}a^{2}(3\theta+2\sin\theta)\big|_{\theta=0}^{\theta=2\pi}\\
		  & = & 3\pi a^{2}
	\end{eqnarray*}	
\end{solution}
\myspace{1}
\begin{example}[][Exam: 28.4.6]
	已知平面区域 $D=\{(x,y)|0\leq y\leq \dfrac{1}{x\sqrt{1+x^{2}}},x\geq 1\}$
\myspace{1}
(1). 求 $D$ 的面积

(2). 求 $D$ 绕 $x$ 轴旋转所成旋转体的体积
\end{example}
\begin{solution}

(1). $D$ 的面积为:
	\begin{eqnarray*}
		S_{D} & = & \int_{1}^{+\infty}\dfrac{1}{x\sqrt{1+x^{2}}}dx\\
			  & = & \int_{\frac{\pi}{4}}^{\frac{\pi}{2}}\dfrac{1}{\sin x}dx\\
			  & = & \ln \dfrac{1-\cos x}{\sin x}\big|_{x=\frac{\pi}{4}}^{x=\frac{\pi}{2}}\\
			  & = & \ln (\sqrt{2} + 1)
	\end{eqnarray*}

(2). $D$ 绕 $x$ 轴旋转所成旋转体的体积为:
	\begin{eqnarray*}
		V & = & \pi\int_{1}^{+\infty}\dfrac{1}{x^{2}(1+x^{2})}dx\\
		  & = & \pi\int_{1}^{+\infty}\left[\dfrac{1}{x^{2}}-\dfrac{1}{1+x^{2}}\right]dx\\
		  & = & \pi\left[-\dfrac{1}{x}-\arctan x\right]\big|_{x=1}^{x=+\infty}\\
		  & = & \dfrac{4\pi-\pi^{2}}{4}
	\end{eqnarray*}
\end{solution}
\myspace{1}

\textcolor{purplea}{\textbf{February 25}}

\begin{example}[][Exam: 28.4.7]
	某水库的闸门形状为等腰梯形, 它的两条底边各长 $10m$ 和 $6m$, 高为 $20m$,较长的底边与水面相齐,求闸门的一侧所受水的压力
\end{example}
\begin{solution}
	\begin{eqnarray*}
		F & = & \rho g\int_{0}^{20}hL(h)dh\\
		  & = & \rho g\int_{0}^{20}h(10-\dfrac{h}{5})dh\\
		  & = & \rho g \left[5h^{2}-\dfrac{h^{3}}{15}\right]\big|_{h=0}^{h=20}\\
		  & = & \dfrac{4400}{3}\rho g
	\end{eqnarray*}
\end{solution}
\myspace{1}

\begin{example}[][Exam: 28.4.8]
	一个半径为 $R(m)$ 的球形贮水箱盛满了水,如果把箱中的水从顶部全部抽出,需要作的功
\end{example}
\begin{solution}
	\begin{eqnarray*}
		W & = & \rho g\int_{0}^{2R}h A(h)dh\\
		  & = & \rho g\int_{0}^{2R}h\pi (2Rh-h^{2})dh\\
		  & = & \pi \rho g \int_{0}^{2R}(2Rh^{2}-h^{3})dh\\
		  & = & \pi \rho g \left[\dfrac{2}{3}Rh^{3}-\dfrac{1}{4}h^{4}\right]\big|_{h=0}^{h=2R}\\
		  & = & \dfrac{4}{3}\pi \rho g R^{4}
	\end{eqnarray*}
\end{solution}
\myspace{1}

\textcolor{purplea}{\textbf{February 26}}

\begin{example}[][Exam: 28.4.9]
	方程 $(xy^{2}+x)dx+(y-x^{2}y)dy=0$ 的通解
\end{example}
\begin{solution}

\end{solution}
\myspace{1}
\begin{example}[][Exam: 28.4.10]
	方程 $y'=1+x+y^{2}+xy^{2}$ 的通解
\end{example}
\begin{solution}
\end{solution}
\myspace{1}

\textcolor{purplea}{\textbf{February 27}}

\begin{example}[][Exam: 28.4.11]
	方程 $(y+\sqrt{x^{2}+y^{2}})dx-xdy=0$ 满足条件 $y|_{x=1}=0$ 的特解
\end{example}

\begin{example}[][Exam: 28.4.12]
	方程 $\dfrac{dy}{dx}=\dfrac{y}{x+y^{4}}$ 的通解
\end{example}
\myspace{1}

\textcolor{purplea}{\textbf{February 28}}

\begin{example}[][Exam: 28.4.13]
	$$\int\dfrac{x\ln x+x\ln^{2}x}{2+x\ln x}dx$$
\end{example}
\begin{solution}
	\begin{eqnarray*}
		I & = & \int\dfrac{x\ln x(1+\ln x)}{2+x\ln x}dx\\
		  & = & \int\dfrac{x\ln x\left[x\ln x\right]'}{2+x\ln x}dx\\
		  & = & \int\dfrac{t}{2+t}dt\\
		  & = & t - 2\ln(t+2) + C\\
		  & = & x\ln x - 2\ln(x\ln x + 2) + C
	\end{eqnarray*}
\end{solution}
\myspace{1}
\begin{example}[][Exam: 28.4.14]
	$$\int\dfrac{\sin 2x\sin^{2}x}{2+\cos^{4} x}dx$$
\end{example}
\begin{solution}
	\begin{eqnarray*}
		I & = & \int\dfrac{2\cos x(\cos^{2}x-1)}{2+\cos^{4} x}d(\cos x)\\
		  & = & \int\dfrac{2\cos^{3} x - 2\cos x}{2+\cos^{4} x}d(\cos x)\\
		  & = & \int\dfrac{2t^{3}}{2+t^{4}}dt - \int \dfrac{1}{2+(t^{2})^{2}}d(t^{2})\\
		  & = & \dfrac{1}{2}\ln(2+t^{4}) - \dfrac{1}{\sqrt{2}}\arctan \dfrac{t^{2}}{\sqrt{2}}\\
		  & = & \dfrac{1}{2}\ln(2+\cos^{4} x) - \dfrac{1}{\sqrt{2}}\arctan \dfrac{\cos^{2} x}{\sqrt{2}} + C
	\end{eqnarray*}
\end{solution}
\myspace{1}

\textcolor{purplea}{\textbf{February 29}}
\begin{example}[][Exam: 28.4.15]
	$$\int\dfrac{\sin x}{\sin x+\cos x}dx$$
\end{example}
\begin{solution}
	\begin{eqnarray*}
		I & = & \int\dfrac{A(\sin x + \cos x) + B(\cos  x-\sin x)}{\sin x+\cos x}dx\\
		  & = & \dfrac{1}{2}\int\dfrac{\sin x+\cos x}{\sin x+\cos x}dx - \dfrac{1}{2}\int \dfrac{\cos x-\sin x}{\sin x+\cos x}dx\\
		  & = & \dfrac{1}{2}x - \dfrac{1}{2}\ln|\sin x+\cos x| + C
	\end{eqnarray*}
\end{solution}
\myspace{1}

\begin{example}[][Exam: 28.4.16]
	$$\int\dfrac{\cos 2x}{\cos^{2} x(1+\sin^{2} x)}dx$$
\end{example}
\begin{solution}
	\begin{eqnarray*}
		I & = & \int\dfrac{\cos^{2} x-\sin^{2} x}{\cos^{2} x+2\sin^{2} x}d(\tan x)\\
		  & = & \int\dfrac{1-\tan^{2}x}{1+2\tan^{2} x}d(\tan x)\\
		  & = & \dfrac{3}{2}\int \dfrac{1}{1+2t^{2}}dt - \dfrac{1}{2}\int dt\\
		  & = & \dfrac{3}{2\sqrt{2}}\arctan \sqrt{2}t - \dfrac{1}{2}t + C\\
		  & = & \dfrac{3}{2\sqrt{2}}\arctan \sqrt{2}\tan x - \dfrac{1}{2}\tan x + C
	\end{eqnarray*}
\end{solution}
\myspace{1}

