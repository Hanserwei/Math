\chapterimage{chap28.jpg}
\chapter{February}
\section{Week \Rmnum{1}}
\hl{\textbf{\textit{February 1}}}

1. 设函数 $f(x)$ 满足关系式 $f''(x)+[f'(x)]^{2}=x$ 且 $f'(0)=0$,则:
\begin{itemize}
	\item A. $f(0)$ 是 $f(x)$ 的极大值
	\item B. $f(0)$ 是 $f(x)$ 的极小值
	\item C. $(0,f(0))$ 是曲线 $y=f(x)$ 的拐点
	\item D. $f(0)$ 不是 $f(x)$ 的极值,$(0,f(0))$ 也不是曲线 $y=f(x)$ 的拐点
\end{itemize}
\myspace{1}

2. 设函数 $f(x)$ 在 $(-\infty,+\infty)$ 上连续,其导函数图形如图所示,则:
\begin{itemize}
	\item A. 函数 $f(x)$ 有 $2$ 个极值点,曲线 $y=f(x)$ 有 $2$ 个拐点
	\item B. 函数 $f(x)$ 有 $2$ 个极值点,曲线 $y=f(x)$ 有 $3$ 个拐点
	\item C. 函数 $f(x)$ 有 $3$ 个极值点,曲线 $y=f(x)$ 有 $1$ 个拐点
	\item D. 函数 $f(x)$ 有 $3$ 个极值点,曲线 $y=f(x)$ 有 $2$ 个拐点
\end{itemize}
\myspace{1}
\hl{\textbf{\textit{February 2}}}

1. 曲线 $y=x\ln\left( e+\dfrac{1}{x}\right)(x>0) $ 的渐近线方程为:
\myspace{1}

2. 曲线 $y=\dfrac{x^{2}+x}{x^{2}-1}$ 渐近线的条数为:
\begin{itemize}
	\item A. $0$
	\item B. $1$
	\item C. $2$
	\item D. $3$
\end{itemize}
\myspace{1}
\hl{\textbf{\textit{February 3}}}

1. 设函数 $y=\dfrac{x^{3}+4}{x^{2}}$,求

(1). 函数的增减区间及极值

(2). 函数图像的凹凸区间及拐点

(3). 渐近线

(4). 作出其图形
\myspace{1}

2. 在区间 $(-\infty,+\infty)$ 内,方程 $|x|^{\frac{1}{4}}+|x|^{\frac{1}{2}}-\cos x=0$: 
\begin{itemize}
	\item A. 无实根
	\item B. 有且仅有一个实根
	\item C. 有且仅有两个实根
	\item D. 有无穷多个实根
\end{itemize}
\myspace{1}
\hl{\textbf{\textit{February 4}}}

1. 函数 $f(x)=\ln|(x-1)(x-2)(x-3)|$ 的驻点个数为:
\begin{itemize}
	\item A. $0$
	\item B. $1$
	\item C. $2$
	\item D. $3$
\end{itemize}
\myspace{1}

2. 设 $f(x)=x^{2}(1-x)^{2}$,则方程 $f''(x)=0$在 $(0,1)$ 上:
\begin{itemize}
	\item A. 无实根
	\item B. 有且仅有一个实根
	\item C. 有且仅有两个实根
	\item D. 有且仅有三个实根
\end{itemize}
\myspace{1}
\hl{\textbf{\textit{February 5}}}

1. 设常数 $k>0$,设函数 $f(x)=\ln x-\dfrac{x}{e}+k$ 在 $(0,+\infty)$ 内零点个数为:
\begin{itemize}
	\item A. $3$
	\item B. $2$
	\item C. $1$
	\item D. $0$
\end{itemize}
\myspace{1}

2. 证明:当 $x>0$ 时,有不等式 $\ln(1+\dfrac{1}{x})>\dfrac{1}{1+x}$
\myspace{1}
\hl{\textbf{\textit{February 6}}}

1. 证明:当 $x>0$ 时,有不等式 $\arctan x+\dfrac{1}{x}>\dfrac{\pi}{2}$
\myspace{1}

2. 设 $p,q$ 是大于 $1$ 的常数,并且 $\dfrac{1}{p}+\dfrac{1}{q}=1$,证明:对于任意的 $x>0$,有 $\dfrac{1}{p}x^{p}+\dfrac{1}{q}\geq x$
\myspace{1}
\hl{\textbf{\textit{February 7}}}

1. 设函数 $f(x)$ 在 $[0,3]$ 上连续,在 $(0,3)$内可导,且 $f(0)+f(1)+f(2)=3,f(3)=1$,试证明:必存在$\xi\in(0,3),s.t.\ f'(\xi)=0$
\myspace{1}

2. 设 $f(x)$在区间 $[a,b]$ 上具有二阶导数,且 $f(a)=f(b)=0$,$f'(a)f'(b)>0$,试证明:存在 $\xi\in(a,b)$ 和 $\eta\in(a,b)$,$s.t.\ f(\xi)=0$ 且 $f''(\eta)=0$
\myspace{1}
\section{Week \Rmnum{2}}
\hl{\textbf{\textit{February 8}}}

1. 设 $f(x)$ 在 $[a,b]$ 上连续,在 $(a,b)$ 内可导,且 $f(a)=f(b)=0$,试证明:

(1). $\exists \xi\in(a,b),s.t.\ f'(\xi)+f(\xi)=0$

(2). $\exists \eta\in(a,b),s.t.\ f'(\eta)-f(\eta)=0$

(3). $\exists \zeta\in(a,b),s.t.\ f'(\zeta)+\lambda f(\zeta)=0$
\myspace{1}

2. 设 $f(x)$ 在 $[0,1]$ 上连续,在 $(0,1)$ 内可导,且 $f(1)=0$,试证明:$\exists \xi\in(0,1),s.t.\ \xi f'(\xi)=-f(\xi)$
\myspace{1}
\hl{\textbf{\textit{February 9}}}

1. 设函数 $f(x)$ 在闭区间 $[0,1]$ 上连续,在开区间$(0,1)$ 内可导,且$f(0)=0,f(1)=\dfrac{1}{3}$,证明:存在 $\xi\in\left(0,\dfrac{1}{2}\right),\eta\in\left( \dfrac{1}{2},1\right),s.t.\ f'(\xi)+f'(\eta)=\xi^{2}+\eta^{2}$
\myspace{1}

2. 设 $f(x)$ 在区间 $[a,b]$ 上连续,在 $(a,b)$ 内可导,且 $a,b$ 同号,证明:存在$\xi,\eta\in(a,b),s.t.\ abf'(\xi)=\eta^{2}f'(\eta)$
\myspace{1}
\hl{\textbf{\textit{February 10}}}

1. 求下列的不定积分

(1). $\int \dfrac{1}{\cos x}dx$

(2). $\int \dfrac{1}{\sin x}dx$
\myspace{1}

2. 求下列的不定积分

(1). $\int \dfrac{x+1}{x(1+xe^{x})}dx$

(2). $\int (1+\ln x)(\ln x+\ln\ln x)dx$
\myspace{1}
\hl{\textbf{\textit{February 11}}}

1. 求下列的不定积分

(1). $\int \dfrac{1+x}{1+x^{3}}dx$

(2). $\int \dfrac{1-x}{1+x^{3}}dx$
\myspace{1}

2. 求下列的不定积分

(1). $\int \dfrac{dx}{1+x^{3}}$

(2). $\int \dfrac{x}{1+x^{3}}dx$
\myspace{1}
\hl{\textbf{\textit{February 12}}}

1. 已知 $f(x)$ 的一个原函数为 $\ln^{2}x$,求 $\int xf'(x)dx$
\myspace{1}

2. 设 $f(\ln x)=\dfrac{\ln(1+x)}{x}$,求 $\int f(x)dx$
\myspace{1}
\hl{\textbf{\textit{February 13}}}

1. 计算不定积分 $\int \max(1,x^{2})dx$
\myspace{1}

2. 设 $M=\int_{-\frac{\pi}{2}}^{\frac{\pi}{2}}\dfrac{\sin x}{1+x^{2}}\cos^{4}xdx,N=\int_{-\frac{\pi}{2}}^{\frac{\pi}{2}}(\sin^{3}x+\cos^{4}x)dx,P=\int_{-\frac{\pi}{2}}^{\frac{\pi}{2}}(x^{2}\sin^{3}x-\cos^{4}x)dx$,则有: 
\begin{itemize}
	\item A. $N<P<M$
	\item B. $M<P<N$
	\item C. $N<M<P$
	\item D. $P<M<N$
\end{itemize}
\myspace{1}
\hl{\textbf{\textit{February 14}}}

1. 设 $M=\int_{-\frac{\pi}{2}}^{\frac{\pi}{2}}\dfrac{(1+x)^{2}}{1+x^{2}}dx,N=\int_{-\frac{\pi}{2}}^{\frac{\pi}{2}}\dfrac{1+x}{e^{x}}dx,K=\int_{-\frac{\pi}{2}}^{\frac{\pi}{2}}(1+\sqrt{\cos x})dx$,则有: 
\begin{itemize}
	\item A. $M>N>K$
	\item B. $M>K>N$
	\item C. $K>M>N$
	\item D. $K>N>M$
\end{itemize}
\myspace{1}

2. 设 $I_{1}=\int_{0}^{\frac{\pi}{4}}\dfrac{\tan x}{x}dx,I_{2}=\int_{0}^{\frac{\pi}{4}}\dfrac{x}{\tan x}dx$,则:
\begin{itemize}
	\item A. $I_{1}>I_{2}>1$
	\item B. $1>I_{1}>I_{2}$
	\item C. $I_{2}>I_{1}>1$
	\item D. $1>I_{2}>I_{1}$
\end{itemize}
\myspace{1}
\section{Week \Rmnum{3}}
\hl{\textbf{\textit{February 15}}}

1. 求定积分 $\int_{-2}^{2}[\ln(x+\sqrt{1+x^{2}})+\sqrt{1-\dfrac{x^{2}}{4}}]dx$
\myspace{1}

2. 求定积分 $\int_{-\pi}^{\pi}|x|[x^{3}+\sin^{2}x]\cos^{2}xdx$
\myspace{1}
\hl{\textbf{\textit{February 16}}}

1. 求定积分 $\int_{0}^{\pi}\sqrt{1-\sin x}dx$
\myspace{1}

2. 求定积分 $\int_{\sqrt{e}}^{e^{\frac{3}{4}}}\dfrac{dx}{x\sqrt{\ln x(1-\ln x)}}$
\myspace{1}
\hl{\textbf{\textit{February 17}}}

1. 求定积分 $\int_{0}^{1}\dfrac{\arcsin \sqrt{x}}{\sqrt{x(1-x)}}dx$
\myspace{1}
\hl{\textbf{\textit{February 18}}}

1. 求定积分 $\int_{0}^{1}\dfrac{\ln(1+x)}{(2-x)^{2}}dx$
\myspace{1}

2. 已知函数 $f(x)=\int_{1}^{x}\sqrt{1+t^{4}}dt$,则 $\int_{0}^{1}x^{2}f(x)dx$
\myspace{1}
\hl{\textbf{\textit{February 19}}}

1. 已知 $f(x)$ 连续,$\int_{0}^{x}tf(x-t)dt=1-\cos x$,求 $\int_{0}^{\frac{\pi}{2}}f(x)dx$
\myspace{1}

2. 设 $f(x)=\begin{cases}
	x^{2},0\leq x\leq 1\\2-x, 1<x\leq 2 
\end{cases}$,记 $F(x)=\int_{0}^{x}f(t)dt(0\leq x\leq 2)$,则有:
\begin{itemize}
	\item A. $F(x)=\begin{cases}
		\dfrac{x^{3}}{3},0\leq x\leq 1\\
		2x-\dfrac{x^{2}}{2}, 1<x\leq 2
	\end{cases}$
	\item B. $F(x)=\begin{cases}
		\dfrac{x^{3}}{3},0\leq x\leq 1\\
		\dfrac{1}{3}+2x-\dfrac{x^{2}}{2}, 1<x\leq 2
	\end{cases}$
	\item C. $F(x)=\begin{cases}
		\dfrac{x^{3}}{3},0\leq x\leq 1\\
		-\dfrac{7}{6}+2x-\dfrac{x^{2}}{2}, 1<x\leq 2
	\end{cases}$
	\item D. $F(x)=\begin{cases}
		\dfrac{x^{3}}{3},0\leq x\leq 1\\
		\dfrac{x^{3}}{3}+2x-\dfrac{x^{2}}{2}, 1<x\leq 2
	\end{cases}$
\end{itemize}
\myspace{1}
\hl{\textbf{\textit{February 20}}}

1. 设 $x\geq -1$,求 $\int_{-1}^{x}(1-|t|)dt$
\myspace{1}

2. 设 $x=x(t)$ 由方程 $\sin t-\int_{1}^{x-t}e^{-u^{2}}du=0$ 所确定,试求 $\dfrac{d^{2}x}{dt^{2}}|_{t=0}$
\myspace{1}
\hl{\textbf{\textit{February 21}}}

1. 设函数 $f(x)=\int_{0}^{1}|t(t-x)|dt(0<x<1)$,求 $f(x)$ 的极值、单调区间及曲线 $y=f(x)$ 的凹凸区间
\myspace{1}

2. 下列反常积分中发散的是:
\begin{itemize}
	\item A. $\int_{0}^{+\infty}xe^{-x}dx$
	\item B. $\int_{0}^{+\infty}xe^{-x^{2}}dx$
	\item C. $\int_{0}^{+\infty}\dfrac{\arctan x}{1+x^{2}}dx$
	\item D. $\int_{0}^{+\infty}\dfrac{x}{1+x^{2}}dx$
\end{itemize}
\myspace{1}
\section{Week \Rmnum{4}}
\hl{\textbf{\textit{February 22}}}

1. 求 $I=\int_{5}^{+\infty}\dfrac{dx}{x^{2}-4x+3}$
\myspace{1}

2. 求 $I=\int_{1}^{+\infty}\dfrac{dx}{e^{1+x}+e^{3-x}}$
\myspace{1}
\hl{\textbf{\textit{February 23}}}

1. 求 $\int_{0}^{+\infty}\dfrac{\ln(1+x)}{(1+x)^{2}}dx$
\myspace{1}

2. 已知抛物线通过 $x$ 轴上的两点 $A(1,0),B(3,0)$

(1). 求证: 两坐标轴与该抛物线所围图形的面积等于 $x$ 轴与该抛物线所围图形的面积

(2). 计算上述两平面图形绕 $x$ 轴旋转一周所产生的两个旋转体体积之比
\myspace{1}
\hl{\textbf{\textit{February 24}}}

1. 求心形线 $r=a(1+\cos\theta)\ (a>0)$ 所围图形的面积
\myspace{1}

2. 已知平面区域 $D=\{(x,y)|0\leq y\leq \dfrac{1}{x\sqrt{1+x^{2}}},x\geq 1\}$

(1). 求 $D$ 的面积
(2). 求 $D$ 绕 $x$ 轴旋转所成旋转体的体积
\myspace{1}
\hl{\textbf{\textit{February 25}}}

1. 某水库的闸门形状为等腰梯形, 它的两条底边各长 $10m$ 和 $6m$, 高为 $20m$,较长的底边与水面相齐,求闸门的一侧所受水的压力
\myspace{1}

2. 一个半径为 $R(m)$ 的球形贮水箱盛满了水,如果把箱中的水从顶部全部抽出,需要作的功
\myspace{1}
\hl{\textbf{\textit{February 26}}}

1. 方程 $(xy^{2}+x)dx+(y-x^{2}y)dy=0$ 的通解
\myspace{1}

2. 方程 $y'=1+x+y^{2}+xy^{2}$ 的通解
\myspace{1}
\hl{\textbf{\textit{February 27}}}

1. 方程 $(y+\sqrt{x^{2}+y^{2}})dx-xdy=0$ 满足条件 $y|_{x=1}=0$ 的特解
\myspace{1}

2. 方程 $\dfrac{dy}{dx}=\dfrac{y}{x+y^{4}}$ 的通解
\myspace{1}
\hl{\textbf{\textit{February 28}}}

1. 求不定积分 $\int\dfrac{x\ln x+x\ln^{2}x}{2+x\ln x}dx$
\myspace{1}

2. 求不定积分 $\int\dfrac{\sin 2x\sin^{2}x}{2+\cos^{4} x}dx$
\myspace{1}
\hl{\textbf{\textit{February 29}}}

1. 求不定积分 $\int\dfrac{\sin x}{\sin x+\cos x}dx$
\myspace{1}

2. 求不定积分 $\int\dfrac{\cos 2x}{\cos^{2} x(1+\sin^{2} x)}dx$
\myspace{1}
