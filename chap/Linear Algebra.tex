\chapterimage{chap15.jpg}

\chapter{行列式}
\section{行列式的定义和性质}
\begin{definition}[行列式]
	行列式是一个在方阵上按照一定法则计算得到的标量, 记作 $det(A)$ 或者 $|A|$

	$$A = \begin{bmatrix}
		a_{11} & a_{12} & \cdots & a_{1n}\\
		a_{21} & a_{22} & \cdots & a_{2n}\\
		\vdots & \vdots & \ddots & \vdots\\
		a_{n1} & a_{n2} & \cdots & a_{nn}
	\end{bmatrix}$$

	$$det(A) = |A| = \begin{vmatrix}
		a_{11} & a_{12} & \cdots & a_{1n}\\
		a_{21} & a_{22} & \cdots & a_{2n}\\
		\vdots & \vdots & \ddots & \vdots\\
		a_{n1} & a_{n2} & \cdots & a_{nn}
	\end{vmatrix}$$

1. \textcolor{cyan}{几何定义}

$n$ 阶行列式 $det(A_{n})$ 的几何意义是 $n$ 维空间中由 $n$ 阶行列式中的 $n$ 个向量围成的$n$ 维空间体的``体积''

$$ det(A_{2}) = |A_{2}| = 
\begin{vmatrix}
	a_{11} & a_{12}\\
	a_{21} & a_{22}
\end{vmatrix} = a_{11}a_{22} - a_{21}a_{12} = S_{D}$$

$$ det(A_{3}) =  |A_{3}| = 
\begin{vmatrix}
	a_{11} & a_{12} & a_{13}\\
	a_{21} & a_{22} & a_{23}\\
	a_{31} & a_{32} & a_{33}
\end{vmatrix} = V_{\Omega}$$

2. \textcolor{cyan}{逆序数法定义}


$$ det(A) = \begin{vmatrix}
	a_{11} & a_{12} & \cdots & a_{1n}\\
	a_{21} & a_{22} & \cdots & a_{2n}\\
	\vdots & \vdots & \ddots & \vdots\\
	a_{n1} & a_{n2} & \cdots & a_{nn}
\end{vmatrix} = 
\sum\limits_{j_{1}j_{2}\cdots j_{n}}(-1)^{\tau(j_{1}j_{2}\cdots j_{n})}a_{1 j_{1}}a_{2j_{2}}\cdots a_{n j_{n}}$$

其中 $\tau(j_{1}j_{2}\cdots j_{n})$ 是 $j_{1}j_{2}\cdots j_{n}$ 的逆序数
\end{definition}

\begin{corollary}[行列式的性质]
	\begin{itemize}
		\item \textcolor{purplea}{性质 1}
		
		行列式中\textcolor{red}{行列等价}, 行列互换, 行列式的值不变, $|A|=|A^{T}|$

		\item \textcolor{purplea}{性质 2} 
		
		行列式中某行或者某列元素全为 $0$, 行列式的值 $det(A) = 0$

		\item \textcolor{purplea}{性质 3}
		
		$$\begin{vmatrix}
			a_{11}  & a_{12}  & \cdots & a_{1n}\\
			\vdots  & \vdots  & \ddots & \vdots\\
			ka_{i1} & ka_{i2} & \cdots & ka_{in}\\
			\vdots  & \vdots  & \ddots & \vdots\\
			a_{n1}  & a_{n2}  & \cdots & a_{nn}
		\end{vmatrix} = k 
		\begin{vmatrix}
			a_{11} & a_{12} & \cdots & a_{1n}\\
			\vdots & \vdots & \ddots & \vdots\\
			a_{i1} & a_{i2} & \cdots & a_{in}\\
			\vdots & \vdots & \ddots & \vdots\\
			a_{n1} & a_{n2} & \cdots & a_{nn}
		\end{vmatrix}$$

		\item \textcolor{purplea}{性质 4}
		
		$$\begin{vmatrix}
			a_{11}          & a_{12}          & \cdots & a_{1n} \\
			\vdots          & \vdots          & \ddots & \vdots \\
			a_{i1} + b_{i1} & a_{i2} + b_{i2} & \cdots & a_{in} + b_{in} \\
			\vdots          & \vdots          & \ddots & \vdots \\
			a_{n1}          & a_{n2}          & \cdots & a_{nn}
		\end{vmatrix} = 
		\begin{vmatrix}
			a_{11} & a_{12} & \cdots & a_{1n}\\
			\vdots & \vdots & \ddots & \vdots\\
			a_{i1} & a_{i2} & \cdots & a_{in}\\
			\vdots & \vdots & \ddots & \vdots\\
			a_{n1} & a_{n2} & \cdots & a_{nn}
		\end{vmatrix} + 
		\begin{vmatrix}
			a_{11} & a_{12} & \cdots & a_{1n}\\
			\vdots & \vdots & \ddots & \vdots\\
			b_{i1} & b_{i2} & \cdots & b_{in}\\
			\vdots & \vdots & \ddots & \vdots\\
			a_{n1} & a_{n2} & \cdots & a_{nn}
		\end{vmatrix}$$

		\item \textcolor{purplea}{性质 5}
		
		行列式两行或者两列互换, 行列式的值相反

		\item \textcolor{purplea}{性质 6}
		
		行列式中两行或者两列成比例, 行列式的值为 $0$

		\item \textcolor{purplea}{性质 7}
		
		行列式中某一行加上另一行的 $k$ 倍, 行列式的值不变
	\end{itemize} 
\end{corollary}

\section{行列式展开定理}
\subsection{余子式}
\begin{definition}[余子式]
	行列式 $det(A)$ 去掉任意一项 $a_{ij}$ 所在行和列去掉后的 $n-1$ 阶行列式称为 $a_{ij}$ 的余子式 $M_{ij}$
\end{definition}

\subsection{代数余子式}

\begin{definition}[代数余子式]
	行列式中任意一项 $a_{ij}$ 的代数余子式 $A_{ij} = (-1)^{i+j}M_{ij}$
\end{definition}

\subsection{行列式展开定理}
\begin{theorem}[行列式展开定理]

	行列式 $det(A)$ 按照第 $i$ 行或者第 $j$ 列展开
	$$det(A) = \begin{vmatrix}
		a_{11} & a_{12} & \cdots & a_{1n}\\
		a_{21} & a_{22} & \cdots & a_{2n}\\
		\vdots & \vdots & \ddots & \vdots\\
		a_{n1} & a_{n2} & \cdots & a_{nn}
	\end{vmatrix} = 
	\begin{cases}
		\sum\limits_{k=1}^{n} a_{ik}A_{ik} \\
		\sum\limits_{k=1}^{n} a_{kj}A_{kj}
	\end{cases}$$
\end{theorem}

\section{几类特殊的行列式}

\subsection{上下三角行列式}

\begin{corollary}[上下三角行列式]
	$$\begin{vmatrix}
		a_{11} & a_{12} & \cdots & a_{1n}\\
		0      & a_{22} & \cdots & a_{2n}\\
		\vdots & \vdots & \ddots & \vdots\\
		0      & 0      & \cdots & a_{nn}
	\end{vmatrix} = 
	\begin{vmatrix}
		a_{11} & 0      & \cdots & 0\\
		a_{21} & a_{22} & \cdots & 0\\
		\vdots & \vdots & \ddots & \vdots\\
		a_{n1} & a_{n2} & \cdots & a_{nn}
	\end{vmatrix} = 
	\begin{vmatrix}
		a_{11} & 0      & \cdots & 0\\
		0      & a_{22} & \cdots & 0\\
		\vdots & \vdots & \ddots & \vdots\\
		0      & 0      & \cdots & a_{nn}
	\end{vmatrix} = \prod\limits_{i=1}^{n}a_{ii}$$
\end{corollary}
\subsection{副三角行列式}

\begin{corollary}[副三角行列式]
	$$\begin{vmatrix}
		a_{11} & \cdots & a_{1,n-1} & a_{1n}\\
		a_{21} & \cdots & a_{2,n-1} & 0\\
		\vdots & \ddots & \vdots    & \vdots\\
		a_{n1} & \cdots & 0         & 0
	\end{vmatrix} = (-1)^{\frac{n(n-1)}{2}} \prod\limits_{i = 1}^{n} a_{i,n-i+1}$$
	$$\begin{vmatrix}
		0      & \cdots & 0         & a_{1n}\\
		0      & \cdots & a_{2,n-1} & a_{2n}\\
		\vdots & \ddots & \vdots    & \vdots\\
		a_{n1} & \cdots & a_{n,n-1} & a_{nn}
	\end{vmatrix} = (-1)^{\frac{n(n-1)}{2}} \prod\limits_{i = 1}^{n} a_{i,n-i+1}$$
	$$\begin{vmatrix}
		0      & \cdots & 0          & a_{1n}\\
		0      & \cdots & a_{2,n-1}  & 0\\
		\vdots & \vdots & \ddots     & \vdots\\
		a_{n1} & \cdots & 0          & 0
	\end{vmatrix} = (-1)^{\frac{n(n-1)}{2}} \prod\limits_{i = 1}^{n} a_{i,n-i+1}$$
\end{corollary}
\subsection{拉普拉斯展开式}
\begin{corollary}[拉普拉斯展开式]
	$$\begin{vmatrix}
		A & O\\
		O & B
	\end{vmatrix} =  
	\begin{vmatrix}
		A & C\\
		O & B
	\end{vmatrix} =  
	\begin{vmatrix}
		A & O\\
		C & B
	\end{vmatrix} = \big|A\big|\big|B\big|$$

	$$\begin{vmatrix}
		O & A\\
		B & O
	\end{vmatrix} = 
	\begin{vmatrix}
		C & A\\
		B & O
	\end{vmatrix} =  
	\begin{vmatrix}
		O & A\\
		B & C
	\end{vmatrix} = (-1)^{mn}\big|A\big|\big|B\big|$$

	其中 $A$ 是 $m$ 阶矩阵, $B$ 是 $n$ 阶矩阵
\end{corollary}
\subsection{范德蒙行列式}
\begin{corollary}[范德蒙行列式]
	
	$$\begin{vmatrix}
		1           & 1           & \cdots & 1\\
		x_{1}       & x_{2}       & \cdots & x_{n}\\
		x_{1}^{2}   & x_{2}^{2}   & \cdots & x_{n}^{2}\\
		\vdots      & \vdots      & \ddots & \vdots\\
		x_{1}^{n-1} & x_{2}^{n-1} & \cdots & x_{n}^{n-1}
	\end{vmatrix} = \prod\limits_{1 \leq i < j \leq n}(x_{j}-x_{i})$$
\end{corollary}
\section{行列式计算}
\begin{proposition}[爪型行列式]
	$$det(A) = \begin{vmatrix}
		x_{1}  & z_{2}  & z_{3}  & \cdots & z_{n}\\
		y_{2}  & x_{2}  & 0      & \cdots & 0\\
		y_{3}  & 0      & x_{3}  & \cdots & 0\\
		\vdots & \vdots & \vdots & \ddots & \vdots\\
		y_{n}  & 0      & 0      & \cdots & x_{n}
	\end{vmatrix} =  \prod\limits_{i=2}^{n}x_{i}\left(x_{1} - \sum\limits_{j=2}^{n}\dfrac{z_{j}y_{j}}{x_{j}}\right)$$
\end{proposition}

\begin{proposition}[$X$ 型行列式]
	$$det(A) = \begin{vmatrix}
		a_{1} &        &         &         &  & b_{1}\\
		      & \ddots &         &         & \begin{rotate}{90}$\ddots$\end{rotate}  & \\
			  &        & a_{k}   & b_{k}   &  & \\	
			  &        & b_{k+1} & a_{k+1} &  & \\
			  & \begin{rotate}{90}$\ddots$\end{rotate} &         &         & \ddots & \\
		b_{n} &        &         &         &  & a_{n}					
	\end{vmatrix}$$
\end{proposition}

\begin{proposition}[两三角形行列式]
	$$det(A) = \begin{vmatrix}
		x_{1}      & b      & b      & \cdots & b\\
		c          & x_{2}  & b      & \cdots & b\\
		c          & c      & x_{3}  & \cdots & b\\
		\vdots     & \vdots & \vdots & \ddots & \vdots\\
		c          & c      & c      & \cdots & x_{n}
	\end{vmatrix}$$

	\begin{eqnarray*}
		det(A) & = & 
		\begin{vmatrix}
			x_{1}      & b      & b      & \cdots & b + 0\\
			c          & x_{2}  & b      & \cdots & b + 0\\
			c          & c      & x_{3}  & \cdots & b + 0\\
			\vdots     & \vdots & \vdots & \ddots & \vdots\\
			c          & c      & c      & \cdots & c + x_{n} - c
		\end{vmatrix}\\
			  & = & 
		\begin{vmatrix}
			x_{1}      & b      & b      & \cdots & b \\
			c          & x_{2}  & b      & \cdots & b \\
			c          & c      & x_{3}  & \cdots & b \\
			\vdots     & \vdots & \vdots & \ddots & \vdots\\
			c          & c      & c      & \cdots & c
		\end{vmatrix} + 
		\begin{vmatrix}
			x_{1}      & b      & b      & \cdots & 0\\
			c          & x_{2}  & b      & \cdots & 0\\
			c          & c      & x_{3}  & \cdots & 0\\
			\vdots     & \vdots & \vdots & \ddots & \vdots\\
			c          & c      & c      & \cdots & x_{n} - c
		\end{vmatrix}\\
			& = & 
		\begin{vmatrix}
			x_{1}-b      & 0      & 0      & \cdots & b \\
			c-b          & x_{2}-b  & 0      & \cdots & b \\
			c-b          & c-b      & x_{3}-b  & \cdots & b \\
			\vdots     & \vdots & \vdots & \ddots & \vdots\\
			0          & 0      & 0      & \cdots & c 
		\end{vmatrix} + (x_{n} - c)A_{n-1} \\
			& = & c\prod\limits_{i=1}^{n-1}(x_{i}-b) + (x_{n} - c)A_{n-1}
	\end{eqnarray*}

	$$\begin{cases}
	  A_{n} = c\prod\limits_{i=1}^{n-1}(x_{i}-b) + (x_{n} - c)A_{n-1}\\
	  A^{T}_{n} = b\prod\limits_{i=1}^{n-1}(x_{i}-c) + (x_{n} - b)A^{T}_{n-1}\\
	  A_{n} = A^{T}_{n}
	\end{cases}\Rightarrow 
	det(A) = \dfrac{b\prod\limits_{i=1}^{n}(x_{i}-c) - c\prod\limits_{j=1}^{n}(x_{j}-b)}{b-c}$$
\end{proposition}

\begin{proposition}
	$$det(A) = \begin{vmatrix}
		a      & b      & b      & \cdots & b\\
		c      & a      & b      & \cdots & b\\
		c      & c      & a      & \cdots & b\\
		\vdots & \vdots & \vdots & \ddots & \vdots\\
		c      & c      & c      & \cdots & a
	\end{vmatrix} = \dfrac{b(a-c)^{n} - c(a-b)^{n}}{b-c}$$

	当 $b = c$ 时 
	
	\begin{eqnarray*}
		det(A) & = & 
			\begin{vmatrix}
				a      & b      & b      & \cdots & b\\
				b      & a      & b      & \cdots & b\\
				b      & b      & a      & \cdots & b\\
				\vdots & \vdots & \vdots & \ddots & \vdots\\
				b      & b      & b      & \cdots & a
			\end{vmatrix}\\
			   & = & 
			\begin{vmatrix}
				a+(n-1)b & b      & b      & \cdots & b\\
				a+(n-1)b & a      & b      & \cdots & b\\
				a+(n-1)b & b      & a      & \cdots & b\\
				\vdots   & \vdots & \vdots & \ddots & \vdots\\
				a+(n-1)b & b      & b      & \cdots & a
			\end{vmatrix}\\
			   & = & [a+(n-1)b]
			\begin{vmatrix}
				1      & b      & b      & \cdots & b\\
				1      & a      & b      & \cdots & b\\
				1      & b      & a      & \cdots & b\\
				\vdots & \vdots & \vdots & \ddots & \vdots\\
				1      & b      & b      & \cdots & a
			\end{vmatrix}\\
			   & = & [a+(n-1)b]
			\begin{vmatrix}
				1      & b      & b      & \cdots & b\\
				0      & a-b    & 0      & \cdots & 0\\
				0      & 0      & a-b    & \cdots & 0\\
				\vdots & \vdots & \vdots & \ddots & \vdots\\
				0      & 0      & 0      & \cdots & a-b
			\end{vmatrix}\\
			   & = & [a+(n-1)b](a-b)^{n-1}
	\end{eqnarray*}
	$$det(A) = \begin{vmatrix}
		a      & b      & b      & \cdots & b\\
		b      & a      & b      & \cdots & b\\
		b      & b      & a      & \cdots & b\\
		\vdots & \vdots & \vdots & \ddots & \vdots\\
		b      & b      & b      & \cdots & a
	\end{vmatrix}$$

\end{proposition}

\begin{proposition}[三对角型行列式]
	$$det(A) = \begin{vmatrix}
		a      & b      & 0      & 0      & \cdots & 0      & 0\\
		c      & a      & b      & 0      & \cdots & 0      & 0\\
		0      & c      & a      & b      & \cdots & 0      & 0\\
		\vdots & \vdots & \vdots & \vdots & \ddots & \vdots & \vdots\\
		0      & 0      & 0      & 0      & \cdots & a      & b\\
		0      & 0      & 0      & 0      & \cdots & c      & a
	\end{vmatrix}$$

	递推式: $A_{n} = a A_{n-1} - bc A_{n-2}$

	特征方程: $x^{2} - ax + bc = 0\Rightarrow 
	\begin{cases}
		x_{1} = \dfrac{a + \sqrt{a^{2}-4bc}}{2}\\
		x_{2} = \dfrac{a - \sqrt{a^{2}-4bc}}{2}
	\end{cases}$

	$$det(A) = A_{n} = \dfrac{x_{1}^{n+1} - x_{2}^{n+1}}{x_{1}-x_{2}}$$
\end{proposition}

\section{Cramer\ Rule}
\begin{theorem}[Cramer\ Rule]
	对于 $n$ 个方程 $n$ 个未知数的非齐次线性方程组
	$$\begin{cases}
	  a_{11}x_{1} + a_{12}x_{2} + \cdots + a_{1n}x_{n} = b_{1}\\
	  a_{21}x_{1} + a_{22}x_{2} + \cdots + a_{2n}x_{n} = b_{2}\\
	  \qquad \qquad \cdots\cdots\\
	  a_{n1}x_{1} + a_{n2}x_{2} + \cdots + a_{nn}x_{n} = b_{n}
	\end{cases}$$

	若行列式 
	$$det(A) = \begin{vmatrix}
		a_{11} & a_{12} & \cdots & a_{1n}\\
		a_{21} & a_{22} & \cdots & a_{2n}\\
		\vdots & \vdots & \ddots & \vdots\\
		a_{n1} & a_{n2} & \cdots & a_{nn}
	\end{vmatrix}\neq 0$$

	方程组有唯一解 $x_{i} = \dfrac{det(A_{i})}{det(A)}$, 其中 $A_{i}$ 是讲 $det(A)$ 的第 $i$ 列换成 $b_{1}, b_{2}, \cdots, b_{n}$

	对于 $n$ 个方程 $n$ 个未知数的齐次线性方程组
	$$\begin{cases}
	  a_{11}x_{1} + a_{12}x_{2} + \cdots + a_{1n}x_{n} = 0\\
	  a_{21}x_{1} + a_{22}x_{2} + \cdots + a_{2n}x_{n} = 0\\
	  \qquad \qquad \cdots\cdots\\
	  a_{n1}x_{1} + a_{n2}x_{2} + \cdots + a_{nn}x_{n} = 0
	\end{cases}$$

	$det(A)\neq 0$, 方程组有唯一解 $x_{i} = 0$, $det(A) = 0$, 方程组有无穷多非零解

\end{theorem}


\chapterimage{chap16.jpg}
\chapter{矩阵}
\section{矩阵的定义和运算}
\subsection{矩阵定义和线性运算}
\begin{definition}[矩阵定义]
	由 $m\times n$ 个数 $a_{ij}$ 排成的 $m$ 行 $n$ 列的数表称为 $m$ 行 $n$ 列的矩阵, 记作:
	
	$$A = \begin{bmatrix}
		a_{11} & a_{12} & \cdots & a_{1n}\\
		a_{21} & a_{22} & \cdots & a_{2n}\\
		\vdots & \vdots & \ddots & \vdots\\
		a_{m1} & a_{m2} & \cdots & a_{mn}
	\end{bmatrix}$$

	$m\times n$ 个数称为矩阵 $A$ 的元素, 简称为元, 数 $a_{ij}$ 位于矩阵 $A$ 的第 $i$ 行第 $j$ 列, 称为矩阵 $A$ 的 $(i,j)$ 元, 
	$m\times n$ 矩阵乘法记作 $A_{mn}$ 或 $(a_{ij})_{m\times n}$
	\begin{itemize}
		\item  元素是实数的矩阵称为\textcolor{cyan}{实矩阵}, 元素是复数的矩阵称为\textcolor{cyan}{复矩阵}
		\item  行数和列数相等的矩阵称为\textcolor{cyan}{方阵}
	\end{itemize}
\end{definition}

\begin{definition}[矩阵的线性运算]
	\textcolor{purpleb}{1. 加法}

	$$C=A + B=(a_{ij})_{m\times n} + (b_{ij})_{m\times n}=(c_{ij})_{m\times n}$$
	
	其中,$c_{ij}=a_{ij} + b_{ij}$

	\textcolor{purpleb}{2. 标量乘法}

	$$kA=k(a_{ij})_{m\times n}=(ka_{ij})_{m\times n}$$
\end{definition}

\begin{definition}[重要矩阵]
	\begin{itemize}
		\item \textcolor{blue}{零矩阵}: 所有元素均为 $0$ 的矩阵, 记作 $O$
		\item \textcolor{blue}{单位矩阵}: 主对角线元素均为 $1$, 其余元素全为 $0$ 的 $n$ 阶方阵, 记作 $E$ 或 $I$
		\item \textcolor{blue}{数量矩阵}: 数 $k$ 和单位矩阵乘积得到的矩阵被称为数量矩阵
		\item \textcolor{blue}{对角矩阵}: 非主对角线元素均为 $0$ 的矩阵称为对角矩阵
		\item \textcolor{blue}{上(下)三角矩阵}: 当 $i>(<)j$, $a_{ij}=0$ 的矩阵称为上(下)三角矩阵
		\item \textcolor{blue}{对称矩阵}: 满足条件 $A^{T}=A$ 的矩阵称为对称矩阵
		\item \textcolor{blue}{反对称矩阵}: 满足条件 $A^{T}=-A$ 的矩阵称为对称矩阵
		\item \textcolor{blue}{幂等矩阵}: 满足条件 $A^{2}=A$ 的矩阵称为幂等矩阵
		\item \textcolor{blue}{正交矩阵}: $A$ 是 $n$ 阶方阵, 满足 $A^{T}A=E$ 的矩阵称为正交矩阵
		\item \textcolor{blue}{分块矩阵}
		$$A = \begin{bmatrix}
			a_{11} & a_{12} & \cdots & a_{1n}\\
			a_{21} & a_{22} & \cdots & a_{2n}\\
			\vdots & \vdots & \ddots & \vdots\\
			a_{m1} & a_{m2} & \cdots & a_{mn}
		\end{bmatrix} = 
		\begin{bmatrix}
			A_{1}\\
			A_{2}\\
			\vdots\\
			A_{m}
		\end{bmatrix}$$

		$$B = \begin{bmatrix}
			b_{11} & b_{12} & \cdots & b_{1n}\\
			b_{21} & b_{22} & \cdots & b_{2n}\\
			\vdots & \vdots & \ddots & \vdots\\
			b_{m1} & b_{m2} & \cdots & b_{mn}
		\end{bmatrix} = 
		\begin{bmatrix}
			B_{1} & B_{2} & \cdots & B_{n}
		\end{bmatrix}$$

		其中 $A_{i} = (a_{i1},a_{i2},\cdots,a_{in}), B_{j} = (b_{j1},b_{j2},\cdots,b_{jm})^{T}$

		$$\begin{bmatrix}
			A & B\\
			C & D
		\end{bmatrix}
		\begin{bmatrix}
			X & Y\\\
			Z & W
		\end{bmatrix} = 
		\begin{bmatrix}
			AX+BY & AY+BW\\
			CX+DY & CY+DW
		\end{bmatrix}$$

		$$\begin{bmatrix}
			A & O\\
			O & B
		\end{bmatrix}^{n} =
		\begin{bmatrix}
			A^{n} & O\\
			O & B^{n}
		\end{bmatrix}$$
	\end{itemize}
\end{definition}

\subsection{矩阵乘法和幂}

\begin{definition}[矩阵的乘法和幂]
	
	\textcolor{red}{1. 矩阵乘法}
	
	$A = (a_{ij})_{m\times s}, B = (b_{ij})_{s\times n}$, 矩阵 $A, B$ 可以相乘(左乘矩阵的列数和右乘矩阵的行数相等), 记 $C = A\times B = (c_{ij})_{m\times n}$
 
	$$c_{ij}=\sum\limits_{k=1}^{s}a_{ik}b_{kj}=a_{i1}b_{1j}+a_{i2}b_{2j}+\cdots+a_{is}b_{sj}$$
	
	\textcolor{red}{2. 矩阵转置}

	
	将矩阵 $A=(a_{ij})_{m\times n}$ 的行列互换得到矩阵$A$ 的转置矩阵,记作 $A^{T}$  

	$$A^{T} = \begin{bmatrix}
		a_{11} & a_{21} & \cdots & a_{m1}\\
		a_{12} & a_{22} & \cdots & a_{m2}\\
		\vdots & \vdots & \ddots & \vdots\\
		a_{1n} & a_{2n} & \cdots & a_{mn}
	\end{bmatrix}$$
	
	\begin{itemize}
		\item $(A^{T})^{T}=A$
		\item $(kA)^{T} = k(A)^{T}$
		\item $(A+B)^{T} = A^{T}+B^{T}$
		\item $(AB)^{T} = B^{T}A^{T}$
		\item 当 $m=n$ 时, $|A^{T}|=|A|$
	\end{itemize}
	
	\textcolor{red}{3. 矩阵的幂}
	
	$A$ 是 $n$ 阶方阵, $A^{m}=\overbrace{AA\cdots A}^{m\text{个}}$ 称为方阵 $A$ 的 $m$ 次幂
	  
	\begin{itemize}
		\item $(A\pm B)^2=A^2+B^2\pm AB \pm BA$
		\item $(A+B)(A-B)=A^2-AB+BA-B^2$
		\item $(AB)^m=\overbrace{(AB)(AB)\cdots(AB)}^{m\text{个}}$
		\item 当 $f(x)=a_{0}+a_{1}x+a_{2}x^2+\cdots+a_{n}x^{n}$ 时,$f(A)=a_{0}E+a_{1}A+a_{2}A^2+\cdots+a_{n}A^n$
	\end{itemize}
\end{definition}

\section{矩阵的逆和伴随矩阵}
\subsection{矩阵的逆}

\begin{definition}[逆矩阵]
	
	$A,B$ 是 $n$ 阶方阵, $E$ 是 $n$ 阶单位矩阵, 若 $AB=BA=E$, 则称 $A$ 是可逆矩阵, 并称 $B$ 是 $A$ 的逆矩阵, 且逆矩阵唯一, 将$A$的逆矩阵记作$A^{-1}$

	\begin{itemize}
		\item $(A^{-1})^{-1} = A$
		\item $AB$ 可逆, $(AB)^{-1} = B^{-1}A^{-1}$
		\item $k\neq 0,(kA)^{-1} = \dfrac{1}{k}A^{-1}$
		\item $|A^{-1}| = \dfrac{1}{|A|}$
		\item $A^{T}\text{可逆},(A^{T})^{-1}=(A^{-1})^{T}$
	\end{itemize}

\end{definition}

\begin{theorem}[逆矩阵存在充要条件]
	$A$ 是 $n$ 阶方阵, $A$ 可逆的充要条件:
	$$\big|A\big| \neq 0 $$
\end{theorem}

\subsection{伴随矩阵}

\begin{definition}[伴随矩阵]
	
	$A$ 是 $n$ 阶方阵, $A$ 的伴随矩阵 $A^{*}$ 是由 $A$ 的代数余子式构成的矩阵, 记作 $A^{*}=(A_{ij})_{n\times n}$, 其中 $A_{ij}$ 是 $a_{ji}$ 的代数余子式

	$$A^{*} = 
	\begin{bmatrix}
		A_{11} & A_{21} & \cdots & A_{n1}\\
		A_{12} & A_{22} & \cdots & A_{n2}\\
		\vdots & \vdots & \ddots & \vdots\\
		A_{1n} & A_{2n} & \cdots & A_{nn}
	\end{bmatrix}$$

	$$AA^{*}=A^{*}A=|A|E\Rightarrow A^{-1}=\frac{1}{|A|}A^{*}$$
	

	\begin{itemize}
		\item $\forall A_{n}, \big|A^{*}\big| = \big|A\big|^{n-1}$
		\item $|A|\neq 0, A^{*}=|A|A^{-1}, A=|A|(A^{*})^{-1}$
		\item $(kA)^{-1} = \dfrac{1}{k}A^{-1}$
		\item $(kA)^{*} = k^{n-1}A^{*}$
	\end{itemize}
\end{definition}

\section{初等变换和初等矩阵}

\begin{definition}[初等变换]
	\begin{itemize}
		\item 用一个非零常数乘以矩阵的某一行(列)
		\item 互换矩阵中的某两行(列)的位置
		\item 将矩阵的某一行(列)的 $k$ 倍加到另一行(列)
	\end{itemize}
\end{definition}
\begin{definition}[初等矩阵]
	由单位矩阵经过一次初等变换后得到的矩阵被称为初等矩阵
	\begin{itemize}
		\item $E_{i}(k)$ 表示 $E$ 的第 $i$ 行(列)乘 $k$ 倍
		$$E_{i}(k) = \begin{bmatrix}
			1 &        &   &   &   &        &   &\\
			  & \ddots &   &   &   &        &   &\\
			  &        & 1 &   &   &        &   &\\
			  &        &   & m &   &        &   &\\
			  &        &   &   & 1 &        &   &\\
			  &        &   &   &   & \ddots &   &\\
			  &        &   &   &   &        & 1 &\\
		\end{bmatrix}$$

		\item $E_{ij}$ 表示 $E$ 的第 $i$ 行(列)与第 $j$ 行(列)互换位置
		$$E_{ij} = \begin{bmatrix}
			1 &        &   &        &   &        &   &\\
			  & \ddots &   &        &   &        &   &\\
			  &        & 0 &        & 1 &        &   &\\
			  &        &   & \ddots &   &        &   &\\
			  &        & 1 &        & 0 &        &   &\\
			  &        &   &        &   & \ddots &   &\\
			  &        &   &        &   &        & 1 &\\
		\end{bmatrix}$$

		\item $E_{ij}(k)$ 表示 $E$ 的第 $i$ 行(列)的 $k$ 倍加到第 $j$ 行(列)
		$$E_{ij}(k) = \begin{bmatrix}
			1 &        &   &        &   &        &   &\\
			  & \ddots &   &        &   &        &   &\\
			  &        & 1 &        &   &        &   &\\
			  &        &   & \ddots &   &        &   &\\
			  &        & k &        & 1 &        &   &\\
			  &        &   &        &   & \ddots &   &\\
			  &        &   &        &   &        & 1 &\\
		\end{bmatrix}$$
	\end{itemize}
\end{definition}
\begin{corollary}[初等矩阵推论]
	$$\begin{cases}
	  E_{i}(k)^{T} = E_{i}(k)\\
	  E_{ij}^{T} = E_{ij}\\
	  E_{ij}(k)^{T} = E_{ji}(k)
	\end{cases}$$

	$$\begin{cases}
	  E_{i}(k)^{-1} = \dfrac{1}{k}E_{i}(k)\\
	  E_{ij}^{-1} = E_{ij}\\
	  E_{ij}(k)^{-1} = E_{ij}(-k)
	\end{cases}$$

	$$\begin{cases}
	  \big|E_{i}(k)\big| = k\\
	  \big|E_{ij}\big| = -1\\
	  \big|E_{ij}(k)\big| = 1
	\end{cases}$$

	\begin{itemize}
		\item \textcolor{purplec}{Gauss-Jordan Elimination}
	\end{itemize}
	
	$$\begin{bmatrix}
		A & E
	\end{bmatrix}\xrightarrow{\text{初等行变换}}
	\begin{bmatrix}
		E & A^{-1}
	\end{bmatrix}$$

	$$\begin{bmatrix}
		A\\E
	\end{bmatrix}\xrightarrow{\text{初等列变换}}
	\begin{bmatrix}
		E\\A^{-1}
	\end{bmatrix}$$
\end{corollary}

\begin{corollary}[矩阵变换]
	若 $A$ 是可逆矩阵, 存在有限个初等矩阵 $P_{1},P_{2},\cdots,P_{s}$, 满足 $A = P_{1}P_{2}\cdots P_{s}$, 且 $A^{-1} = P_{s}^{-1}\cdots P_{2}^{-1}P_{1}^{-1}$
\end{corollary}

\begin{definition}[行阶梯形矩阵]
	\textcolor{purplec}{行阶梯形矩阵}
	\begin{itemize}
		\item 零行全部位于非零行下方
		\item 非零行的首个非零元素(主元)在每一行中的列标号递增
	\end{itemize}

	\textcolor{purplec}{行最简阶梯形矩阵}
	\begin{itemize}
		\item 行阶梯形矩阵非零行的首个非零元素(主元)为 $1$
		\item 主元所在列的其他元素全为 $0$
	\end{itemize}
\end{definition}

\begin{corollary}[分块矩阵逆矩阵]
	
	$$\begin{bmatrix}
		A_{1} &       &        &\\
		      & A_{2} &        &\\
			  &	      & \ddots &\\
			  &       &        & A_{s}
	  \end{bmatrix}^{-1} = 
	  \begin{bmatrix}
		A_{1}^{-1} &            &        &\\
		           & A_{2}^{-1} &        &\\
				   &	        & \ddots &\\
				   &            &        & A_{s}^{-1}
	  \end{bmatrix}$$

	  $$\begin{bmatrix}
		      &        &       & A_{1}\\
		      &        & A_{2} &\\
			  &	\cdots &       &\\
		A_{s} &        &       &
	  \end{bmatrix}^{-1} = 
	  \begin{bmatrix}
			       &            &        & A_{s}^{-1}\\
			       &            & \cdots &\\
			       & A_{2}^{-1} &        &\\
		A_{1}^{-1} &            &        &
	\end{bmatrix}$$

	$$A = \begin{bmatrix}
		B_{s\times s} & O_{s\times r}\\
		D_{r\times s} & C_{r\times r}
	  \end{bmatrix}\to 
	  A^{-1} = \begin{bmatrix}
		B^{-1}          & O\\
		--C^{-1}DB^{-1} & C^{-1}
	  \end{bmatrix}$$

	$$A = \begin{bmatrix}
		B_{s\times s} & D_{s\times r}\\
		O_{r\times s} & C_{r\times r}
	  \end{bmatrix}\to 
	  A^{-1} = \begin{bmatrix}
		B^{-1} & -B^{-1}DC^{-1}\\
		O      & C^{-1}
	  \end{bmatrix}$$

	  $$A = \begin{bmatrix}
		O_{s\times r} & B_{s\times s}\\
		C_{r\times r} & D_{r\times s}
	  \end{bmatrix} \to 
	  A^{-1} = \begin{bmatrix}
		-C^{-1}DB^{-1} & C^{-1}\\
		B^{-1}         & O
	  \end{bmatrix}$$

	  $$A = \begin{bmatrix}
		D_{s\times r} & B_{s\times s}\\
		C_{r\times r} & O_{r\times s}
	  \end{bmatrix} \to 
	  A^{-1} = \begin{bmatrix}
		O      & C^{-1}\\
		B^{-1} & -B^{-1}DC^{-1}
	  \end{bmatrix}$$

\end{corollary}

\section{等价矩阵和矩阵的秩}

\begin{definition}[等价矩阵]
	
	设 $A,B$ 均为 $m\times n$ 矩阵, $\exists P_{m\times m},Q_{n\times n}(P,Q\text{可逆}),\ s.t.\ PAQ=B$, 
	称 $A,B$ 是等价矩阵, 记作$A\cong B$
	
	\begin{itemize}
		\item 同型矩阵 $A, B$ 等价的的充要条件: $r(A) = r(B)$
		\item $P,Q$ 均为可逆矩阵
		$$\forall A_{m\times n}, \exists P,Q,\ s.t. PAQ = \begin{bmatrix} E_{r} & O\\ O & O\end{bmatrix}$$

		其中 $P_{m\times m}, Q_{n\times n}$ 均为可逆矩阵, $E_{r}$ 是 $r$ 阶单位矩阵, 
		矩阵 $\begin{bmatrix} E_{r} & O\\ O & O\end{bmatrix}$ 是秩为 $r$ 矩阵的等价标准型
	\end{itemize}
\end{definition}

\begin{definition}[矩阵的秩]
		
	设 $A$ 是 $m\times n$ 矩阵, $A$ 中最高阶非零子式的阶数称为矩阵 $A$的秩, 记作 $r(A)$
	
	\begin{itemize}
		\item $r(A) = k$ 充要条件: $A$ 存在 $k$ 阶非零子式, 且所有 $k+1$ 阶子式全为 $0$
		\item $r(A) = k$ 充要条件: $A$ 的行(列)向量中存在 $k$ 个线性无关的向量, 且任意 $k+1$ 个行(列)向量线性相关
		\item $r(A_{n\times n}) = n\Leftrightarrow |A| \neq 0 \Leftrightarrow A$ 可逆
	\end{itemize}
\end{definition}

\begin{corollary}[秩的性质]	
	设$A$是$m\times n$矩阵,$B$是满足有关矩阵运算要求的矩阵,我们有
	\begin{itemize}
		\item $0\leq r(A) \leq \min\{m,n\}$
		\item $r(kA)=r(A), k\neq 0$
		\item $r(AB)\leq \min\{r(A),r(B)\}$
		\item $r(A+B)\leq r(A)+r(B)$
		\item $$ r(A^{*}) = 
		\begin{cases}
			n & r(A) = n\\
			1 & r(A) = n-1\\
			0 & r(A) < n-1	
		\end{cases}$$

		其中 $A$ 为 $n$ 阶方阵
	\end{itemize}
\end{corollary}

\begin{proposition}
	设 $A$ 是 $m\times n$ 矩阵, $B$ 是 $n\times s$ 矩阵, 则 $r(AB)\leq \min\{r(A),r(B)\}$
\end{proposition}
\begin{anymark}[证明]
	(1). $r(AB) \leq r(B)$

	令 
	$$A  = \begin{bmatrix}
		a_{11} & a_{12} & \cdots & a_{1n}\\
		a_{21} & a_{22} & \cdots & a_{2n}\\
		\vdots & \vdots & \ddots & \vdots\\
		a_{m1} & a_{m2} & \cdots & a_{mn}
	\end{bmatrix}, 
	B =  \begin{bmatrix}
		\beta_{1} \\ \beta_{2} \\ \vdots \\ \beta_{n}
	\end{bmatrix},
	AB = \begin{bmatrix}
		\gamma_{1} \\ \gamma_{2} \\ \vdots \\ \gamma_{m}
	\end{bmatrix}$$

	其中 $\beta_{i}, \gamma_{j}$ 代表行向量, $a_{ij}$ 代表元素

	$$AB = 
	\begin{bmatrix}
		a_{11} & a_{12} & \cdots & a_{1n}\\
		a_{21} & a_{22} & \cdots & a_{2n}\\
		\vdots & \vdots & \ddots & \vdots\\
		a_{m1} & a_{m2} & \cdots & a_{mn}
	\end{bmatrix}
	\begin{bmatrix}
		\beta_{1} \\ \beta_{2} \\ \vdots \\ \beta_{n}
	\end{bmatrix} = 
	\begin{bmatrix}
		a_{11}\beta_{1}+a_{12}\beta_{2}+\cdots+a_{1n}\beta_{n}\\
		a_{21}\beta_{1}+a_{22}\beta_{2}+\cdots+a_{2n}\beta_{n}\\
		\vdots\\
		a_{m1}\beta_{1}+a_{m2}\beta_{2}+\cdots+a_{mn}\beta_{n}
	\end{bmatrix} = 
	\begin{bmatrix}
		\gamma_{1} \\ \gamma_{2} \\ \vdots \\ \gamma_{m}
	\end{bmatrix}$$

	$AB$ 的行向量都可以由 $B$ 的行向量\textcolor{red}{线性表出} $\Rightarrow r(AB)\leq r(B)$

	(2). $r(AB) \leq r(A)$

	令
	$$A = \begin{bmatrix}
		\alpha_{1} & \alpha_{2} & \cdots & \alpha_{n}
	\end{bmatrix},
	B = \begin{bmatrix}
		b_{11} & b_{12} & \cdots & b_{1s}\\
		b_{21} & b_{22} & \cdots & b_{2s}\\
		\vdots & \vdots & \ddots & \vdots\\
		b_{n1} & b_{n2} & \cdots & b_{ns}
	\end{bmatrix},
	AB = \begin{bmatrix}
		\gamma_{1} & \gamma_{2} & \cdots & \gamma_{s}
	\end{bmatrix}$$

	其中 $\alpha_{i}, \gamma_{j}$ 代表列向量, $b_{ij}$ 代表元素

	$$AB =
	\begin{bmatrix}
		\alpha_{1} & \alpha_{2} & \cdots & \alpha_{n}
	\end{bmatrix}
	\begin{bmatrix}
		b_{11} & b_{12} & \cdots & b_{1s}\\
		b_{21} & b_{22} & \cdots & b_{2s}\\
		\vdots & \vdots & \ddots & \vdots\\
		b_{n1} & b_{n2} & \cdots & b_{ns}
	\end{bmatrix} =
	\begin{bmatrix}
		b_{11}\alpha_{1}+b_{21}\alpha_{2}+\cdots+b_{n1}\alpha_{n}\\
		b_{12}\alpha_{1}+b_{22}\alpha_{2}+\cdots+b_{n2}\alpha_{n}\\
		\vdots\\
		b_{1s}\alpha_{1}+b_{2s}\alpha_{2}+\cdots+b_{ns}\alpha_{n}
	\end{bmatrix}^{T} =
	\begin{bmatrix}
		\gamma_{1} & \gamma_{2} & \cdots & \gamma_{s}
	\end{bmatrix}$$

	$AB$ 的列向量都可以由 $A$ 的列向量\textcolor{red}{线性表出} $\Rightarrow r(AB)\leq r(A)$

	$$\begin{cases}
		r(AB)\leq r(B)\\
		r(AB)\leq r(A)
	\end{cases}\Rightarrow r(AB) \leq \min\{r(A),r(B)\}$$
\end{anymark}

\begin{proposition}
	设 $A,B$ 是 $m\times n$ 矩阵,  则 $r(A + B)\leq r(A) + r(B)$
\end{proposition}
\begin{anymark}[证明]

	令 $$
	\begin{cases}
		A = \begin{bmatrix}
			\alpha_{1} & \alpha_{2} & \cdots & \alpha_{n}
		\end{bmatrix}\\
		B = \begin{bmatrix}
			\beta_{1} & \beta_{2} & \cdots & \beta_{n}
		\end{bmatrix}\\
		[A,B] = \begin{bmatrix}
			\alpha_{1} & \alpha_{2} & \cdots & \alpha_{n} & \beta_{1} & \beta_{2} & \cdots & \beta_{n}
		\end{bmatrix}
	\end{cases}$$

	$$A + B = \begin{bmatrix}
		\alpha_{1}+\beta_{1} & \alpha_{2}+\beta_{2} & \cdots & \alpha_{n}+\beta_{n}
	\end{bmatrix}$$

	$A+B$ 的列向量都可以由 $[A,B]$ 的列向量\textcolor{red}{线性表出} $\Rightarrow r(A+B)\leq r([A,B])\Rightarrow r(A+B)\leq r(A)+r(B)$
\end{anymark}

\begin{proposition}
	设 $A$ 是 $m\times n$ 矩阵, $B$ 是 $n\times s$ 矩阵, 则 $r(A) + r(B) - n \leq r(AB)$
\end{proposition}
\begin{anymark}[证明]

	$$\begin{bmatrix}
		E_{m\times m} & -A_{m\times n}\\
		O_{n\times m} & E_{n\times n}
	\end{bmatrix} 
	\begin{bmatrix}
		A_{m\times n} & O_{m\times s}\\
		E_{n\times n} & B_{n\times s}
	\end{bmatrix}
	\begin{bmatrix}
		E_{n\times n} & -B_{n\times s}\\
		O_{n\times s} & E_{s\times s}
	\end{bmatrix} = 
	\begin{bmatrix}
		O_{m\times n} & -AB \\
		E_{n\times n} & O_{n\times s}
	\end{bmatrix}$$

	其中 $\begin{bmatrix}
		E_{m\times m} & -A_{m\times n}\\
		O_{n\times m} & E_{n\times n}
	\end{bmatrix}$ 和 $\begin{bmatrix}
		E_{n\times n} & -B_{n\times s}\\
		O_{n\times s} & E_{s\times s}
	\end{bmatrix}$ 均为可逆矩阵
	
	$$r(A) + r(B) \leq n + r(AB) \Rightarrow r(A) + r(B) -n \leq r(AB)$$

	特别的, 当 $AB = O$ 时, $r(AB) = 0\Rightarrow r(A) + r(B) \leq n$ 
\end{anymark}

\begin{proposition}
	设 $A$ 是 $n$ 阶方阵, 则 
	$$r(A^{*}) = 
	\begin{cases}
		n & r(A) = n\\
		1 & r(A) = n-1\\
		0 & r(A) < n-1	
	\end{cases}$$
\end{proposition}
\begin{anymark}[证明]

	(1). $r(A) = n$

	$$\begin{cases}
	  r(A) = n \\
	  AA^{*} = |A|E
	\end{cases}\Rightarrow
	\begin{cases}
	  |A| \neq 0\\
	  |A^{*}| = |A|^{n-1} \neq 0
	\end{cases}\Rightarrow r(A^{*}) = n$$

	(2). $r(A) = n-1$

	$$\begin{cases}
	  r(A) = n-1\\
	  AA^{*} = |A|E
	\end{cases}\Rightarrow 
	\begin{cases}
	  |A| = 0\\
	  AA^{*} = O
	\end{cases}\Rightarrow
	\begin{cases}
	  r(A) + r(A^{*}) \leq n\\
	  r(A^{*}) \leq 1
	\end{cases}$$

	$r(A) = n-1 \to A^{*}$ 至少存在一个元素不为 $0 \to r(A^{*}) \geq 1$

	$$\begin{cases}
	  r(A^{*}) \geq 1\\
	  r(A^{*}) \leq 1
	\end{cases}\Rightarrow r(A^{*}) = 1$$

	(3). $r(A) < n-1$

	$$\begin{cases}
	  r(A) < n-1\\
	  AA^{*} = |A|E
	\end{cases}\Rightarrow 
	\begin{cases}
	  |A| = 0\\
	  A^{*} = O
	\end{cases}\Rightarrow r(A^{*}) = 0$$
\end{anymark}


\chapterimage{chap17.jpg}
\chapter{向量组}

\section{向量和向量组的线性相关性}

\subsection{向量的定义和运算}
\begin{definition}[向量的定义]
	$n$ 个数构成的有序数组 $[a_{1},a_{2},\cdots,a_{n}]$ 称为一个 $n$ 维向量, 记作 
	$\alpha=[a_{1},a_{2},\cdots,a_{n}], \alpha$ 称为 $n$ 维行向量,
	$\alpha^{T}$ 称为 $n$ 维列向量, 其中 $a_{i}$ 称为向量的第 $i$ 个分量
\end{definition}
\begin{definition}[向量的线性运算]
	\textcolor{purpleb}{1. 加法}

	$$\alpha+\beta\overset{def}{\Longrightarrow}[a_{1}+b_{1},a_{2}+b_{2},\cdots,a_{n}+b_{n}]$$
	
	\textcolor{purpleb}{2. 标量乘法}

	$$k\alpha\overset{def}{\Longrightarrow}[ka_{1},ka_{2},\cdots,ka_{n}], k\in \mathbb{R}$$
	
\end{definition}
\begin{definition}[向量的内积和正交]
	\textcolor{blue}{内积}
	
	设 $\alpha=[a_{1},a_{2},\cdots,a_{n}]^{T}, \beta=[b_{1},b_{2},\cdots,b_{n}]^{T}$, 则称: 

	$$\alpha^{T}\beta=\sum\limits_{i=1}^{n}a_{i}b_{i}=a_{1}b_{1}+a_{2}b_{2}+\cdots+a_{n}b_{n}$$
	
	为向量 $\alpha,\beta$ 的内积, 记作$(\alpha, \beta)$
	
	\textcolor{blue}{模}

	$\|\alpha\| = \sqrt{\sum\limits_{i=1}^{n}a_{i}^2}$ 称为向量 $\alpha$ 的模, 特别的当 $\alpha$时, 称 $\alpha$ 为单位向量
	
	\textcolor{blue}{正交}
	
	当 $\alpha^{T}\beta=0$ 时, 称向量 $\alpha,\beta$ 是正交向量
	
	\textcolor{blue}{标准正交向量组}
	
	向量组 $\alpha_{1},\alpha_{2},\cdots,\alpha_{n}$ 满足:  
	$$\alpha_{i}^{T}\alpha_{j} =  
	\begin{cases}
		0 & i\neq j\\
		1 & i= j
	\end{cases}$$
	
	则称 $\alpha_{1},\alpha_{2},\cdots,\alpha_{n}$ 是标准或单位正交向量组

	\textcolor{blue}{正交矩阵}

	设 $A$ 是 $n$ 阶方阵, 若 $A^{T}A=E$, 则称 $A$ 为正交矩阵
	\begin{itemize}
		\item $A^{T}A=E\Rightarrow A^{T}=A^{-1}$
		\item $|A| = \pm 1$
		\item $A$ 的行(列)向量是标准正交向量组
	\end{itemize}
	
	\textcolor{blue}{施密特正交化}
	
	线性无关的向量组 $\alpha_{1},\alpha_{2},\cdots,\alpha_{n}$ 的标准正交化公式:
	
	$$\begin{cases}
		\beta_{1} = \alpha_{1}\\
		\beta_{2} = \alpha_{2}-\dfrac{(\alpha_{2},\beta_{1})}{(\beta_{1},\beta_{1})}\beta_{1}\\
		\cdots\\
		\beta_{n} = \alpha_{n}-\dfrac{(\alpha_{n},\beta_{1})}{(\beta_{1},\beta_{1})}\beta_{1}-\dfrac{(\alpha_{n},\beta_{2})}{(\beta_{2},\beta_{2})}\beta_{2}-\cdots-\dfrac{(\alpha_{n},\beta_{n-1})}{(\beta_{n-1},\beta_{n-1})}\beta_{n-1}
	\end{cases}$$

	将 $\beta_{1},\beta_{2},\cdots,\beta_{n}$ 单位化:
	$$\begin{cases}
		\eta_{1} = \dfrac{\beta_{1}}{\|\beta_{1}\|}\\
		\eta_{2} = \dfrac{\beta_{2}}{\|\beta_{2}\|}\\
		\cdots\\
		\eta_{n} = \dfrac{\beta_{n}}{\|\beta_{n}\|}
	\end{cases}$$  
	
	$\eta_{1},\eta_{2},\cdots,\eta_{n}$ 是一个标准正交向量组
\end{definition}

\subsection{向量组的线性相关性}

\begin{definition}[线性相关和线性表出]
	\textcolor{green}{1. 线性组合}
	
	设有 $m$ 个 $n$ 维向量 $\alpha_{1},\alpha_{2}\cdots,\alpha_{m}$ 和 $m$ 个数 $k_{1},k_{2},\cdots,k_{m}$, 向量
	$$k_{1}\alpha_{1}+k_{2}\alpha_{2}+\cdots+k_{m}\alpha_{m}$$
	称作向量组 $\alpha_{1},\alpha_{2}\cdots,\alpha_{m}$ 的\textcolor{blue}{线性组合}
	
	\textcolor{blue}{2. 线性表出}
	
	向量 $\beta$ 可以表示为向量组 $\alpha_{1},\alpha_{2}\cdots,\alpha_{m}$ 的线性组合
	
	$$\exists k_{i}(i = 1,2,\cdots,m)\in  \mathbb{R},\ s.t.\ \beta=k_{1}\alpha_{1}+k_{2}\alpha_{2}+\cdots+k_{m}\alpha_{m}$$

	向量$\beta$ 可以由向量组 $\alpha_{1},\alpha_{2}\cdots,\alpha_{m}$ \textcolor{blue}{线性表出}
	
	\textcolor{red}{3. 线性相关}
	
	向量组 $\alpha_{1},\alpha_{2}\cdots,\alpha_{m}$

	$$\exists k_{i}\neq 0(i=1,2,\cdots,m), \ s.t. \ k_{1}\alpha_{1}+k_{2}\alpha_{2}+\cdots+k_{m}\alpha_{m} = 0$$
	
	向量组$\alpha_{1},\alpha_{2}\cdots,\alpha_{m}$ \textcolor{blue}{线性相关}
	
	\textcolor{red}{4. 线性无关}
	
	向量组 $\alpha_{1},\alpha_{2}\cdots,\alpha_{m}$

	$$\exists k_{i}\in \mathbb{R},\ s.t.\ k_{1}\alpha_{1}+k_{2}\alpha_{2}+\cdots+k_{m}\alpha_{m} = 0$$

	当且仅当 $k_{1}=k_{2}=\cdots=k_{m}=0$ 时上式成立, 向量组$\alpha_{1},\alpha_{2}\cdots,\alpha_{m}$ \textcolor{blue}{线性无关}
\end{definition}

\begin{theorem}[判别线性相关性的七大定理]
	\textcolor{red}{定理\ 一}
	
	 向量组 $\alpha_{1},\alpha_{2}\cdots,\alpha_{n}$ 线性相关的充要条件: 至少有一个向量可以由其余的$n-1$个向量线性表出
	 
	 \begin{anymark}[证明]
	 	(i). 必要性
	 	
	 	向量组 $\alpha_{1},\alpha_{2}\cdots,\alpha_{n}$ 线性相关
	 	
		$$\exists k_{i}\neq 0(i=1,2,\cdots,n),\ s.t.\ k_{1}\alpha_{1}+k_{2}\alpha_{2}+\cdots+k_{n}\alpha_{n}=0$$
	 	
		不妨假设 $k_{m}\neq 0(1\leq m\leq n)$

	 	$$\alpha_{m}=-\dfrac{k_{1}}{k_{m}}\alpha_{1}-\cdots-\dfrac{k_{n}}{k_{m}}\alpha_{n}$$
	 	$\alpha_{m}$ 可以由其余 $n-1$ 个向量线性表出
	 	
	 	(ii). 充分性
	 	
	 	不妨假设 $\alpha_{m}$ 可以由其余 $n-1$ 个向量线性表出:  
	 	$$\alpha_{m}=k_{1}\alpha_{1}+k_{2}\alpha_{2}+\cdots+k_{n}\alpha_{n}$$
	 	
		$$\exists k_{i}\neq 0(k_{m} = 1),\ s.t. \ k_{1}\alpha_{1} + k_{2}\alpha_{2}+\cdots + 1\alpha_{m} + \cdots + k_{n}\alpha_{n}=0$$
	 	
		向量组 $\alpha_{1},\alpha_{2}\cdots,\alpha_{n}$ 线性相关
	 \end{anymark}

	\textcolor{red}{定理\ 二}
	
	若向量组 $\alpha_{1},\alpha_{2}\cdots,\alpha_{n}$ 线性无关, $\beta,\alpha_{1},\alpha_{2}\cdots,\alpha_{n}$ 线性相关, 
	$\beta$ 可以由 $\alpha_{1},\alpha_{2}\cdots,\alpha_{n}$ 线性表出, 且表示唯一
	\begin{anymark}[证明]
		(i). 存在性
		
		$\beta,\alpha_{1},\alpha_{2}\cdots,\alpha_{n}$ 线性相关

		$$\exists k_{i}\in\mathbb{R}(i = 1,2,\cdots,n,\beta),\ s.t. \ k_{\beta}\beta + k_{1}\alpha_{1} + k_{2}\alpha_{2} + \cdots+k_{n}\alpha_{n} = 0$$

		假设 $k_{\beta}=0$  
		
		$$k_{1}\alpha_{1}+k_{2}\alpha_{2}+\cdots+k_{n}\alpha_{n} = 0\Rightarrow \exists k_{i}\neq 0 (i\in \{1,2,\cdots,n\})$$
		
		向量组 $\alpha_{1},\alpha_{2}\cdots,\alpha_{n}$ 线性相关, 矛盾 $\Rightarrow k_{\beta} \neq 0$
		
		$$\beta = -\dfrac{k_{1]}}{k_{\beta}} \alpha_{1} - \dfrac{k_{2}}{k_{\beta}} \alpha_{2} - \cdots - \dfrac{k_{n}}{k_{\beta}}\alpha_{n}$$

		向量 $\beta$ 可以由 $\alpha_{1},\alpha_{2}\cdots,\alpha_{n}$ \textcolor{red}{线性表出}
		
		(ii). 唯一性
		
		假设向量组 $\alpha_{1},\alpha_{2}\cdots,\alpha_{n}$ 对 $\beta$ 存在两种不同的线性表出  
		
		$$\begin{cases}
			\beta=l_{1}\alpha_{1}+l_{2}\alpha_{2}+\cdots+l_{n}\alpha_{n}\\
			\beta=h_{1}\alpha_{1}+h_{2}\alpha_{2}+\cdots+h_{n}\alpha_{n}
		\end{cases}$$

		两式相减:  

		$$(l_{1}-h_{1})\alpha_{1}+(l_{2}-h_{2})\alpha_{2}+\cdots+(l_{n}-h_{n})\alpha_{n}=0$$

		$\exists l_{i}-h_{i}\neq 0(i\in\{1,2,\cdots,n\})$, 向量组 $\alpha_{1},\alpha_{2}\cdots,\alpha_{n}$ 线性相关, 矛盾 $\Rightarrow$ \textcolor{red}{线性表出唯一}
	\end{anymark}

	\textcolor{red}{定理\ 三}
	
	向量组 $\beta_{1},\beta_{2},\cdots,\beta_{t}$ 可以由 $\alpha_{1},\alpha_{2}\cdots,\alpha_{s}$ 线性表出, 且 $t > s$, 则 $\beta_{1},\beta_{2},\cdots,\beta_{t}$ 线性相关
	
	\begin{anymark}[证明]
		
		向量组 $\beta_{1},\beta_{2},\cdots,\beta_{t}$ 可以由 $\alpha_{1},\alpha_{2}\cdots,\alpha_{s}$ 线性表出  
		$$\begin{cases}
		    \beta_{1}=l_{11}\alpha_{1}+l_{12}\alpha_{2}+\cdots+l_{1s}\alpha_{s}\\
			\beta_{2}=l_{21}\alpha_{1}+l_{22}\alpha_{2}+\cdots+l_{2s}\alpha_{s}\\
			\cdots\\
			\beta_{t}=l_{t1}\alpha_{1}+l_{t2}\alpha_{2}+\cdots+l_{ts}\alpha_{s}
		\end{cases}$$

		假设 $\exists k_{i}\in \mathbb{R},\ s.t. \sum\limits_{i=1}^{s}k_{i}\beta_{i} = 0$
		
		$$\begin{cases}
		  k_{1}\beta_{1} = k_{1}(l_{11}\alpha_{1}+l_{12}\alpha_{2}+\cdots+l_{1s}\alpha_{s})\\
		  k_{2}\beta_{2} = k_{2}(l_{21}\alpha_{1}+l_{22}\alpha_{2}+\cdots+l_{2s}\alpha_{s})\\
		  \cdots\\
		  k_{t}\beta_{t} = k_{t}(l_{t1}\alpha_{1}+l_{t2}\alpha_{2}+\cdots+l_{ts}\alpha_{s})
		\end{cases}\Rightarrow 
		\sum_{i=1}^{t}k_{i}\beta_{i} = (\sum\limits_{i=1}^{t}k_{i}l_{i1})\alpha_{1}+(\sum\limits_{i=1}^{t}k_{i}l_{i2})\alpha_{2}+\cdots+(\sum\limits_{i=1}^{t}k_{i}l_{is})\alpha_{s}
		$$
		
		不妨令 $\sum\limits_{i=1}^{t}k_{i}l_{i1} = \sum\limits_{i=1}^{t}k_{i}l_{i2} = \cdots = \sum\limits_{i=1}^{t}k_{i}l_{is} = 0$

		$$\begin{cases}
		  k_{1}l_{11} + k_{2}l_{21} + \cdots + k_{t}l_{t1} = 0\\
		  k_{1}l_{12} + k_{2}l_{22} + \cdots + k_{t}l_{t2} = 0\\
		  \cdots\\
		  k_{1}l_{1s} + k_{2}l_{2s} + \cdots + k_{t}l_{ts} = 0
		\end{cases}\Rightarrow 
		LX = 0$$

		其中
		$$L = \begin{bmatrix}
			l_{11} & l_{21} & \cdots & l_{t1}\\
			l_{12} & l_{22} & \cdots & l_{t2}\\
			\vdots & \vdots & \ddots & \vdots\\
			l_{1s} & l_{2s} & \cdots & l_{ts}
		\end{bmatrix}, X = [k_{1},k_{2},\cdots,k_{t}]^{T}$$
		
		这是一个 $t$ 元 $s$ 个方程的齐次线性方程组, 且 $t > s$, 方程组一定有非零解 

		$$\exists k_{i}\neq 0(i\in\{1,2,\cdots,n\}), \ s.t. \ k_{1}\beta_{1}+k_{2}\beta_{2}+\cdots+k_{t}\beta_{t}=0$$
		
		$\beta_{1},\beta_{2},\cdots,\beta_{t}$ 线性相关
	\end{anymark}

	\textcolor{red}{定理\ 四}
	
	设 $m$ 个 $n$ 维向量 $\alpha_{1},\alpha_{2}\cdots,\alpha_{m}$, 其中
	
	$$\begin{cases}
		\alpha_{1}=[a_{11},a_{21},\cdots,a_{n1}]^{T}\\
		\alpha_{2}=[a_{12},a_{22},\cdots,a_{n2}]^{T}\\
		\cdots\\
		\alpha_{m}=[a_{1m},a_{2m},\cdots,a_{nm}]^{T}\\
	\end{cases}$$

	向量组 $\alpha_{1},\alpha_{2}\cdots,\alpha_{m}$ 线性相关的充要条件时齐次线性方程组 $AX = 0$ 有非零解, 也等价于零空间不为零

	$$A = \begin{bmatrix}
		a_{11} & a_{12} & \cdots & a_{1m}\\
		a_{21} & a_{22} & \cdots & a_{2m}\\
		\vdots & \vdots & \ddots & \vdots\\
		a_{n1} & a_{n2} & \cdots & a_{nm}
	\end{bmatrix}, X=[x_{1},x_{2},\cdots,x_{m}]^{T}$$
	
	\begin{anymark}[证明]
		(i). 必要性
		
		向量组 $\alpha_{1},\alpha_{2}\cdots,\alpha_{m}$ 线性相关  
		$$\exists x_{i}\neq 0(i\in \{1,2,\cdots,m\}),\ s.t.\ x_{1}\alpha_{1}+x_{2}\alpha_{2}+\cdots+x_{m}\alpha_{m}=0$$

		$$\begin{cases}
			a_{11}x_{1}+a_{12}x_{2}+\cdots+a_{1m}x_{m}=0\\
			a_{21}x_{1}+a_{22}x_{2}+\cdots+a_{2m}x_{m}=0\\
			\cdots\\
			a_{n1}x_{1}+a_{n2}x_{2}+\cdots+a_{nm}x_{m}=0\\
		\end{cases}$$
		有非零解
		
		(ii). 充分性
		
		方程组 $AX=0$ 有非零解

		$$\begin{cases}
			a_{11}x_{1}+a_{12}x_{2}+\cdots+a_{1m}x_{m}=0\\
			a_{21}x_{1}+a_{22}x_{2}+\cdots+a_{2m}x_{m}=0\\
			\cdots\\
			a_{n1}x_{1}+a_{n2}x_{2}+\cdots+a_{nm}x_{m}=0\\
		\end{cases}\Rightarrow
		x_{1}\alpha_{1}+x_{2}\alpha_{2}+\cdots+x_{m}\alpha_{m}=0$$
		
		向量组 $\alpha_{1},\alpha_{2}\cdots,\alpha_{m}$ 线性相关
	\end{anymark}
	
	\textcolor{red}{定理\ 五}
	
	向量 $\beta$ 可以由向量组 $\alpha_{1},\alpha_{2}\cdots,\alpha_{s}$ 线性表出, 等价于非齐次方程组方程 $AX=\beta$ 有解;
	向量 $\beta$ 不能由向量组 $\alpha_{1},\alpha_{2}\cdots,\alpha_{s}$ 线性表出, 等价于非齐次方程组方程$AX=\beta$无解
	\begin{anymark}[证明]

		向量 $\beta$ 可以由向量组 $\alpha_{1},\alpha_{2}\cdots,\alpha_{s}$ 线性表出
		$$\exists x_{i}\neq 0(i\in \{1,2,\cdots,s\}),\ s.t.\ x_{1}\alpha + x_{2}\alpha + \cdots + x_{s}\alpha_{s} = 0$$  
		$$[\alpha_{1},\alpha_{2}\cdots,\alpha_{s}]\begin{bmatrix}
			x_{1}\\
			x_{2}\\
			\cdots\\
			x_{s}
		\end{bmatrix} =\beta$$
		方程组 $AX=\beta$ 有非零解
	\end{anymark}
	
	\textcolor{red}{定理\ 六}
	
	如果向量组 $\alpha_{1},\alpha_{2}\cdots,\alpha_{n}$ 一部分向量线性相关, 那么整个向量组也线性相关
	
	\begin{anymark}[证明]
		不妨设 $\alpha_{1},\alpha_{2}\cdots,\alpha_{j}(j< n)$ 线性相关  
		
		$$\exists k_{i}\neq 0(i\in\{1,2,\cdots,j\}),\ s.t. \ k_{1}\alpha_{1}+k_{2}\alpha_{2}+\cdots+k_{j}\alpha_{j}=0$$
		 
		取 $k_{1},k_{2},\cdots,k_{j},k_{j+1} = k_{j+2} = \cdots = k_{n} = 0$

		$$k_{1}\alpha_{1}+k_{2}\alpha_{2}+\cdots+k_{j}\alpha_{j} + 0\alpha_{j+1} + \cdots + 0\alpha_{n}=0$$
		$k_{1},k_{2},\cdots,k_{j},k_{j+1}=k_{j+2}=\cdots=k_{n}=0$ 不全为 $0$, 整个向量组也线性相关
	\end{anymark}
	
	\textcolor{red}{定理\ 七}
	
	如果一组 $n$ 维向量 $\alpha_{1},\alpha_{2}\cdots,\alpha_{n}$ 线性无关,
	那么把这些向量各任意添加 $m$ 个分量得到的新向量组 $(n+m)$ 维 $\alpha_{1}^{*},\alpha_{2}^{*}\cdots,\alpha_{n}^{*}$ 线性无关;
	如果向量组 $\alpha_{1},\alpha_{2}\cdots,\alpha_{n}$ 线性相关, 那各去掉相同的若干个分量所得到的新向量组也线性相关
\end{theorem}

\section{极大线性无关组和向量组的秩}
\begin{definition}[极大线性无关组]
	在向量组 $\alpha_{1},\alpha_{2}\cdots,\alpha_{s}$, 如果存在部分向量组 $\alpha_{i_{1}},\alpha_{i_{2}}\cdots,\alpha_{i_{r}}$ 满足:  
	\begin{itemize}
		\item $\alpha_{i_{1}},\alpha_{i_{2}}\cdots,\alpha_{i_{r}}$ 线性无关
		\item 向量组中任意向量 $\alpha_{i}(i=1,2,\cdots,s)$ 都可以被向量组 $\alpha_{i_{1}},\alpha_{i_{2}}\cdots,\alpha_{i_{r}}$ 线性表出
	\end{itemize}
	则称向量组 $\alpha_{i_{1}},\alpha_{i_{2}}\cdots,\alpha_{i_{r}}$ 是原向量组的一个极大线性无关组
	
	一个向量组的极大线性无关组不唯一, 对于线性无关的向量组, 它的极大线性无关组是自身
\end{definition}

\begin{definition}[向量组的秩]
	向量组 $\alpha_{1},\alpha_{2}\cdots,\alpha_{s}$ 的极大线性无关组 $\alpha_{i_{1}},\alpha_{i_{2}}\cdots,\alpha_{i_{r}}$ 中所含向量的个数 $r$ 称为向量组的秩,记作:

	$$rank(\alpha_{1},\alpha_{2}\cdots,\alpha_{s}) = r(\alpha_{1},\alpha_{2}\cdots,\alpha_{s})=r$$

	\begin{itemize}
		\item $r(A) = r(col(A)) = r(row(A))$
		\item 初等行变换和初等列变换不改变矩阵的秩
		\item $A\overset{col}{\longrightarrow}B$, $A$ 的行向量与 $B$ 的行向量是等价向量组
		\item 设向量组 $\alpha_{1},\alpha_{2}\cdots,\alpha_{s}$ 及 $\beta_{1},\beta_{2}\cdots,\beta_{t}$, 若 $\beta_{i}(i=1,2,\cdots,t)$ 均可由 $\alpha_{1},\alpha_{2}\cdots,\alpha_{s}$ 线性表出,则:  
		$$r(\alpha_{1},\alpha_{2}\cdots,\alpha_{s})\leq r(\beta_{1},\beta_{2}\cdots,\beta_{t})$$
	\end{itemize}
\end{definition}
\section{等价向量组}
\begin{definition}[等价向量组]
	设两个向量组: $(\rm\Rmnum{1}) \alpha_{1},\alpha_{2}\cdots,\alpha_{s}, (\rm\Rmnum{2}) \beta_{1},\beta_{2}\cdots,\beta_{t}$,
	若 $(\rm\Rmnum{1})$ 中向量 $\alpha_{i}(i=1,2,\cdots,s)$ 均可由 $(\rm\Rmnum{2})$ 中向量线性表出, 则称向量组 $(\rm\Rmnum{1})$ 可由向量组 $(\rm\Rmnum{2})$ 线性表出;
	若向量组 $(\rm\Rmnum{1})$ 和向量组 $(\rm\Rmnum{2})$ 互相线性表出, 称向量组 $(\rm\Rmnum{1})$ 与向量组 $(\rm\Rmnum{2})$ 是等价向量组, 记作 $(\rm\Rmnum{1})\cong (\rm\Rmnum{2})$
	
	\begin{itemize}
		\item $(\rm\Rmnum{1})\cong (\rm\Rmnum{1})$
		\item $(\rm\Rmnum{1})\cong (\rm\Rmnum{2}), (\rm\Rmnum{2})\cong (\rm\Rmnum{1})$
		\item $(\rm\Rmnum{1})\cong (\rm\Rmnum{2}), (\rm\Rmnum{2})\cong (\rm\Rmnum{3}),\text{则} (\rm\Rmnum{1})\cong (\rm\Rmnum{3})$
		\item 向量组和它的极大线性无关组是等价向量组
	\end{itemize}
\end{definition}
\section{向量空间}
\begin{definition}[向量空间]
	设 $\xi_{1},\xi_{2},\cdots,\xi_{n}$ 是 $n$ 维向量空间 $\mathbb{R}^{n}$ 中线性无关的有序向量,
	$\forall \alpha\in \mathbb{R}^{n}$ 均可由向量组 $\xi_{1},\xi_{2},\cdots,\xi_{n}$ 线性表出, 且表出式:  
	
	$$\alpha=a_{1}\xi_{1}+a_{2}\xi_{2}+\cdots+a_{n}\xi_{n}$$
	
	称 $\xi_{1},\xi_{2},\cdots,\xi_{n}$ 是 $n$ 维向量空间 $\mathbb{R}^{n}$ 的一组基,
	基向量的个数 $n$ 称为向量空间的维度, $[a_{1},a_{2},\cdots,a_{n}]^{T}$ 是向量 $\alpha$ 在基向量 $\xi_{1},\xi_{2},\cdots,\xi_{n}$ 的坐标
\end{definition}

\begin{definition}[基变换]
	$\xi_{1},\xi_{2},\cdots,\xi_{n}$ 和 $\eta_{1},\eta_{2},\cdots,\eta_{n}$ 是向量空间 $\mathbb{R}^{n}$ 中的两个基, 其有关系:  
	
	$$[\eta_{1},\eta_{2},\cdots,\eta_{n}] = 
	[\xi_{1},\xi_{2},\cdots,\xi_{n}]
	\begin{bmatrix}
		c_{11} & c_{12} & \cdots & c_{1n}\\
		c_{21} & c_{22} & \cdots & c_{2n}\\
		\vdots & \vdots & \ddots & \vdots\\
		c_{n1} & c_{n2} & \cdots & c_{nn}
	\end{bmatrix} = [\xi_{1},\xi_{2},\cdots,\xi_{n}]C$$
	
	上式是基 $\xi_{1},\xi_{2},\cdots,\xi_{n}$ 到基 $\eta_{1},\eta_{2},\cdots,\eta_{n}$ 的基变换公式,
	矩阵 $C$ 是基 $\xi_{1},\xi_{2},\cdots,\xi_{n}$ 到基 $\eta_{1},\eta_{2},\cdots,\eta_{n}$ 的过渡矩阵,
	$C$ 的第 $i$ 列即是 $\eta_{i}$ 在基 $\xi_{1},\xi_{2},\cdots,\xi_{n}$ 下的坐标, 过渡矩阵 $C$ 为可逆矩阵
\end{definition}
\begin{definition}[坐标变换]
	设向量 $\alpha$ 在基 $\xi_{1},\xi_{2},\cdots,\xi_{n}$ 和基 $\eta_{1},\eta_{2},\cdots,\eta_{n}$ 下的坐标为别是 $\mathbf{x}=[x_{1},x_{2},\cdots,x_{n}]^{T},\mathbf{y}=[y_{1},y_{2},\cdots,y_{n}]^{T}$

	$$\alpha=[\xi_{1},\xi_{2},\cdots,\xi_{n}]\mathbf{x} = [\eta_{1},\eta_{2},\cdots,\eta_{n}]\mathbf{y}$$

	$$[\xi_{1},\xi_{2},\cdots,\xi_{n}] = [\eta_{1},\eta_{2},\cdots,\eta_{n}]C$$
	
	$$\mathbf{x} = C\mathbf{y} \Leftrightarrow \mathbf{y} = C^{-1}\mathbf{x}$$
\end{definition}



\chapterimage{chap18.jpg}
\chapter{线性方程组}
\section{具体型方程组}
\subsection{齐次方程组}
\begin{definition}[齐次方程组]

	方程组
	$$\begin{cases}
		a_{11}x_{1}+a_{12}x_{2}+\cdots+a_{1n}x_{n}=0\\
		a_{21}x_{1}+a_{22}x_{2}+\cdots+a_{2n}x_{n}=0\\
		\cdots\\
		a_{m1}x_{1}+a_{m2}x_{2}+\cdots+a_{mn}x_{n}=0\\
	\end{cases}$$

	称为 $m$ 个方程, $n$ 个未知数的齐次方程组
	
	向量形式:  
	
	$$x_{1}\alpha_{1}+x_{2}\alpha_{2}+\cdots+x_{n}\alpha_{n}=0$$
	
	其中:  
	$$\alpha_{i} = \begin{bmatrix}
		a_{1j}\\
		a_{2j}\\
		\vdots\\
		a_{mj}\\
	\end{bmatrix} (j = 1,2,\cdots,n)$$
	
	矩阵形式:  
	
	$$A_{m\times n}X=0$$
	
	$$A = 
	\begin{bmatrix}
		a_{11} & a_{12} & \cdots & a_{1n}\\
		a_{21} & a_{22} & \cdots & a_{2n}\\
		\vdots & \vdots & \ddots & \vdots\\
		a_{m1} & a_{m2} & \cdots & a_{mn}
	\end{bmatrix}, 
	X = \begin{bmatrix}
		x_{1}\\
		x_{2}\\
		\vdots\\
		x_{n}\\
	\end{bmatrix}$$
	
	2. 有解的条件
	
	(i).当$r(A)=n\text{时},(\alpha_{1},\alpha_{2},\cdots,\alpha_{n}\text{线性无关})$,方程组只有零解.
	
	(ii). 当$r(A)=r<n\text{时},(\alpha_{1},\alpha_{2},\cdots,\alpha_{n}\text{线性相关})$,方程组有非零解,且有$n-r$个线性无关解.
	
	3. 解的性质
	
	如果$A\xi_{1}=0,\ A\xi_{2}=0, \forall k_{1},k_{2}\in \mathbb{R},\ A(k_{1}\xi_{1}+k_{2}\xi_{2})=0 $
	
	4. 基础解系和解的结构
	
	(1). 基础解系
	
	设$\xi_{1},\xi_{2},\cdots,\xi_{n-r}$满足:  
	\begin{itemize}
		\item $\xi_{1},\xi_{2},\cdots,\xi_{n-r}\text{是方程组的解}$
		\item $\xi_{1},\xi_{2},\cdots,\xi_{n-r}\text{线性无关}$
		\item $\text{方程组}AX=0\text{的任意一个解均可以由}\xi_{1},\xi_{2},\cdots,\xi_{n-r}\text{线性表出}$
	\end{itemize}
	
	我们称$\xi_{1},\xi_{2},\cdots,\xi_{n-r}$为方程组$AX=0$的基础解系.
	
	(2). 通解
	
	设$\xi_{1},\xi_{2},\cdots,\xi_{n-r}$为方程组$AX=0$的基础解系,则$k_{1}\xi_{1}+k_{2}\xi_{2}+\cdots+k_{n-r}\xi_{n-r}$是方程组$AX=0$的通解,其中$k_{1},k_{2},\cdots,k_{n-r}\in \mathbb{R}$
\end{definition}
\subsection{非齐次方程组}
\begin{definition}[非齐次方程组]
	1. 形式
	
	方程组
	$$\left\lbrace 
	\begin{array}{l}
		a_{11}x_{1}+a_{12}x_{2}+\cdots+a_{1n}x_{n}=b_{1}\\
		a_{21}x_{1}+a_{22}x_{2}+\cdots+a_{2n}x_{n}=b_{2}\\
		\cdots\cdots\\
		a_{m1}x_{1}+a_{m2}x_{2}+\cdots+a_{mn}x_{n}=b_{m}\\
	\end{array}
	\right. $$
	称为$m$个方程,$n$个未知数的非齐次方程组.
	
	其向量形式为:  
	$$x_{1}\alpha_{1}+x_{2}\alpha_{2}+\cdots+x_{n}\alpha_{n}=\beta$$
	其中:  
	$$\alpha_{i}=\left[ \begin{matrix}
		a_{1j}\\
		a_{2j}\\
		\vdots\\
		a_{mj}
	\end{matrix}\right](j=1,2,\cdots,n) ,\ \beta=\left[ \begin{matrix}
	b_{1j}\\
	b_{2j}\\
	\vdots\\
	b_{mj}
	\end{matrix}\right]$$
	
	方程组的矩阵形式为:  
	$$A_{m\times n}X=\beta$$
	$$A=\left[
	\begin{matrix}
		a_{11}&a_{12}&\cdots&a_{1n}\\
		a_{21}&a_{22}&\cdots&a_{2n}\\
		\vdots&\vdots& &\vdots\\
		a_{m1}&a_{m2}&\cdots&a_{mn}
	\end{matrix}
	\right],\quad X=\left[ \begin{matrix}
		x_{1}\\
		x_{2}\\
		\vdots\\
		x_{n}\\
	\end{matrix}\right]$$
	矩阵$\left[ {\begin{array}{c:c}
			\begin{matrix}
				a_{11}&a_{12}&\cdots&a_{1n}\\
				a_{21}&a_{22}&\cdots&a_{2n}\\
				\vdots&\vdots& &\vdots\\
				a_{m1}&a_{m2}&\cdots&a_{mn}
			\end{matrix}&
			\begin{matrix}
				b_{1j}\\
				b_{2j}\\
				\vdots\\
				b_{mj}
			\end{matrix}
	\end{array}} \right]$记作为矩阵$A$的增广矩阵,简记为$\left[ {\begin{array}{c:c}
	\begin{matrix}
		A
	\end{matrix}&
	\begin{matrix}
	\beta
	\end{matrix}
\end{array}} \right]$

	2. 有解的条件
	
	(i).$r(A)\neq r([A,\beta]),\text{方程组无解}.(\beta\text{不能由}\alpha_{1},\alpha_{2},\cdots,\alpha_{n}\text{线性表出})$
	
	(ii). $r(A)=r([A,\beta])=n,\text{方程组有唯一解}$ 
	$$\alpha_{1},\alpha_{2},\cdots,\alpha_{n}\text{线性无关},\beta,\alpha_{1},\alpha_{2},\cdots,\alpha_{n}\text{线性相关}$$
	
	(iii).$r(A)=r([A,\beta])<n,\text{方程组有无穷多组解}.$
	
	3. 解的性质
	
	设$\eta_{1},\eta_{2},\eta_{3}$是非齐次方程组$AX=\beta$的解,$\xi$是对应齐次方程组$AX=0$的解,我们有:  
	\begin{itemize}
		\item $\eta_{1}-\eta_{2}\text{是}AX=0\text{的解}$
		\item $k\xi+\eta\text{是方程组}AX=\beta\text{的解}$
	\end{itemize}
	
	4. 解的结构
	
	(1). 特解
	
	$\eta\text{是非齐次性方程组}AX=\beta\text{的一个特解}$
	
	(2). 通解
	
	设$k_{1}\xi_{1}+k_{2}\xi_{2}+\cdots+k_{n-r}\xi_{n-r}$是方程组$AX=0$的通解,其中$k_{1},k_{2},\cdots,k_{n-r}\in \mathbb{R}$,我们可以得到非齐次性方程组的通解:  
	$$k_{1}\xi_{1}+k_{2}\xi_{2}+\cdots+k_{n-r}\xi_{n-r}+\eta$$
\end{definition}
\section{两个方程组的公共解}
\begin{definition}[两个方程组的公共解]
	(1).齐次性线性方程组$A_{m\times n}X=0$和$B_{m\times n}X=0$的公共解是满足方程组$\left[
	\begin{matrix}
		A\\B
	\end{matrix}
	\right]X=0$的解.
	
	(2). 非齐次性性线性方程组$A_{m\times n}X=\alpha$和$B_{m\times n}X=\beta$的公共解是满足方程组$\left[
	\begin{matrix}
		A\\B
	\end{matrix}
	\right]X=\left[
	\begin{matrix}
		\alpha\\\beta
	\end{matrix}
	\right]$的解.
	
	(3).给出方程组$A_{m\times n}X=0$的通解$k_{1}\xi_{1}+k_{2}\xi_{2}+\cdots+k_{s}\xi_{s}$,代入第二个方程组$B_{m\times n}X=0$得到$k_{i}(i=1,2,\cdots,s)$之间的关系,代回方程$A_{m\times n}X=0$
	
	(4).给出方程组$A_{m\times n}X=0$的基础解系$\xi_{1},\xi_{2},\cdots,\xi_{s}$和方程组$B_{m\times n}X=0$的基础解系$\eta_{1},\eta_{2},\cdots,\eta_{t}$,公共解为:  
	$$k_{1}\xi_{1}+k_{2}\xi_{2}+\cdots+k_{s}\xi_{s}=l_{1}\eta_{1}+l_{2}\eta_{2}+\cdots+l_{t}\eta_{t}$$
\end{definition}
\section{同解方程组}
\begin{definition}[同解方程组]
	如果两个方程组$A_{m\times n}X=0$和$B_{m\times n}X=0$有完全相同的解,则称它们为同解方程组.
	\begin{itemize}
		\item $AX=0\text{的解满足}BX=0\text{并且} BX=0\text{的解满足}AX=0$
		\item $r(A)=r(B)\text{并且}AX=0\text{的解满足}BX=0(BX=0\text{的解满足}AX=0)$
		\item $r(A)=r(B)=r(\left[
		\begin{matrix}
			A\\B
		\end{matrix}
		\right])$
	\end{itemize}
\end{definition}
\chapterimage{chap19.jpg}
\chapter{特征值和特征向量}
\section{特征值和特征向量定义}
\begin{definition}[特征值和特征向量]
	设$A$是$n$阶矩阵,$\lambda$为常数,存在非零列向量$\xi$,满足:  
	$$A\xi=\lambda\xi$$
	则称$\lambda$为$A$的特征值,$\xi$是$A$对应于特征值$\lambda$的特征向量
	\begin{anymark}[注]
		$$(\lambda E-A)\xi=O\Rightarrow |\lambda E-A|=0$$
		$$\left|
		\begin{matrix}
			\lambda-a_{11}&-a_{12}&\cdots&-a_{1n}\\
			-a_{21}&\lambda-a_{22}&\cdots&-a_{2n}\\
			\vdots&	\vdots& &	\vdots\\
			-a_{n1}&-a_{n2}&\cdots&\lambda-a_{nn}
		\end{matrix}
		\right|=(\lambda-\lambda_{1})(\lambda-\lambda_{2})\cdots(\lambda-\lambda_{n})=0$$
		上面右边是关于$\lambda$的特征多项式,也是$A$的特征方程:  
		$$\lambda^{n}-(\lambda_{1}+\lambda_{2}+\cdots+\lambda_{n})\lambda^{n-1}+\cdots+(-1)^{n}\prod\limits_{i=1}^{n}\lambda_{i}=0$$
		我们得到:  
		$$\left\lbrace 
		\begin{array}{l}
			\sum\limits_{i=1}^{n}\lambda_{i}=\sum\limits_{i=1}^{n}a_{ii}\\
			\prod\limits_{i=1}^{n}\lambda_{i}=|A|
		\end{array}
		\right. $$
	\end{anymark}
	\begin{corollary}[特征向量]
		\begin{itemize}
			\item $k\text{重特征值至多只有}k\text{个线性无关的特征向量}$
			\item $\text{若}\xi_{1},\xi_{2}\text{是}A\text{的属于不同特征值}\lambda_{1},\lambda_{2}\text{的特征向量},\lambda_{1},\lambda_{2}\text{线性无关}$
			\item $\text{若}\xi_{1},\xi_{2}\text{是}A\text{的属于同一特征值}\lambda\text{的特征向量}$
			
			$k_{1}\lambda_{1}+k_{2}\lambda_{2}(k_{1},k_{2}\text{不同时为}0)\text{仍然是}A\text{属于特征值}\lambda\text{的特征向量}$
		\end{itemize}
	\end{corollary}
\end{definition}
\begin{table}[h]
	\centering
	\caption{常用特征值和特征向量}
	\label{table: 常用特征值和特征向量}
	\begin{tblr}{
			hline{1,Z}={2pt},
			hline{2}={1pt},
			vline{2}={1pt},
			cells={c,$},
			cell{1-Z}{1} = {mode = text}
		}
		矩阵     & A      & kA        & A^{k}       & f(A)       & A^{-1}           &  A^{*}               & P^{-1}AP \\
		特征值   & \lambda & k\lambda & \lambda^{k} & f(\lambda) & \frac{1}{\lambda} & \frac{|A|}{\lambda} & \lambda   \\
		特征向量 & \xi     & \xi      & \xi         & \xi        & \xi               &  \xi                & P^{-1}\xi \\
	\end{tblr}
\end{table}
\section{相似}
\begin{definition}[矩阵的相似]
	设$A,B$是两个$n$阶方阵,若存在$n$阶可逆矩阵$P$,使得$P^{-1}AP=B$,则称$A$相似于$B$,记作$A\sim B$
	\begin{anymark}[注]
		(1). $A\sim A\quad \text{反身性}$
		
		(2). $A\sim B\Rightarrow B\sim A\quad \text{对称性}$
		
		(3). $A\sim B,\ B\sim C\Rightarrow A\sim C\quad \text{传递性}$
	\end{anymark}
	\begin{corollary}[相似矩阵]
		(1).$A\sim B$,我们得到:  
		\begin{itemize}
			\item $r(A)=r(B)$
			\item $|A|=|B|$
			\item $|\lambda A-E|=|\lambda B-E|$
			\item $A,B\text{具有相同的特征值}$
		\end{itemize}
		
		(2).$A\sim B$,我们得到:  
		\begin{itemize}
			\item $A^{m}\sim B^{m}$
			\item $f(A)\sim f(B)$
		\end{itemize}
		
		(3).$A\sim B\text{且}A\text{可逆}$,我们得到:  
		\begin{itemize}
			\item $A^{-1}\sim B^{-1}$
			\item $f(A^{-1})\sim f(B^{-1})$
		\end{itemize}
		
		(4).$A\sim B$,我们得到:  
		\begin{itemize}
			\item $A^{T}\sim B^{T}$
			\item $A^{*}\sim B^{*}$
		\end{itemize}
	\end{corollary}
\end{definition}
\subsection{矩阵的相似对角化}
\begin{definition}[相似对角化]
	设$A$是$n$阶方阵,若存在$n$阶可逆矩阵$P$,使得$P^{-1}AP=\varLambda$,其中$\varLambda$是对角矩阵,则称$A$可相似对角化,记作$A\sim \varLambda$,称$\varLambda$为$A$的相似标准型
	
	\begin{anymark}[注]
		$$P=[\xi_{1},\xi_{2},\cdots,\xi_{n}],\varLambda=\left[
		\begin{matrix}
			\lambda_{1}& & & \\
			&\lambda_{2}& & \\
			& &\ddots &\\
			& & &\lambda_{n}
		\end{matrix}
		\right]$$
		$$P^{-1}AP=\varLambda\Rightarrow AP=P\varLambda$$
		$$A[\xi_{1},\xi_{2},\cdots,\xi_{n}]=[\lambda_{1}\xi_{1},\lambda_{2}\xi_{2},\cdots,\lambda_{n}\xi_{n}]\Rightarrow A\xi_{i}=\lambda_{i}\xi_{i}(i=1,2,\cdots,n)$$
	\end{anymark}
	\begin{corollary}[对角化]
		\begin{itemize}
			\item $n\text{阶矩阵}A\text{可相似对角化}\Leftrightarrow A\text{有}n\text{个线性无关的特征向量}$
			\item $n\text{阶矩阵}A\text{可相似对角化}\Leftrightarrow \text{对于每个}k_{i}\text{重特征值都有}k_{i}\text{个特征向量}$
			\item $A\text{有}n\text{个特征值}\Rightarrow n\text{阶矩阵}A\text{可相似对角化}$
			\item $n\text{阶矩阵}A\text{为实对称矩阵}\Rightarrow A\text{可相似对角化}$
		\end{itemize}
	\end{corollary}
\end{definition}
\subsection{实对称矩阵的相似对角化}
\begin{definition}[实对称矩阵相似对角化]
	$A^{T}=A\text{且}A\text{中元素全为实数,我们把}A\text{称作实对称矩阵}$
	
	\begin{anymark}[性质]
		\begin{itemize}
			\item $\text{实对称矩阵必可相似对角化,特征值为实数,特征向量为实向量}$
			\item $\text{实对称矩阵属于不同特征值的特征向量互相正交}$
			\item $\exists \text{正交矩阵} Q,\text{s.t.}\ Q^{-1}AQ=Q^{T}AQ=\varLambda$
		\end{itemize}
	\end{anymark}
\end{definition}
\chapterimage{chap20.jpg}
\chapter{二次型}
\section{二次型定义}
\begin{definition}[二次型]
	$n$元变量$x_{1},x_{2},\cdots,x_{n}$的二次齐次多项式
	\begin{eqnarray*}
		f(x_{1},x_{2},\cdots,x_{n})=a_{11}x_{1}^{2}+2a_{12}x_{1}x_{2}+&\cdots&+2a_{1n}x_{1}x_{n}\\
		+a_{22}x_{2}^2+&\cdots&+2a_{2n}x_{2}x_{n}\\
		+&\cdots&\quad \\
		& &+a_{nn}x_{n}^{2}
	\end{eqnarray*}
	称为$n$元二次型,简称为二次型.
	
	我们令$a_{ij}=a_{ji}$,我们可以得到:  
	$$f(x_{1},x_{2},\cdots,x_{n})=a_{11}x_{1}^{2}+\cdots+a_{1n}x_{1}x_{n}+a_{21}x_{2}x_{1}+a_{22}x_{2}^2+\cdots+a_{2n}x_{2}x_{n}
	+\cdots+a_{n1}x_{n}x_{1}+\cdots+a_{nn}x_{n}^{2}$$
	$$f(x_{1},x_{2},\cdots,x_{n})=\sum\limits_{i=1}^{n}\sum\limits_{j=1}^{n}a_{ij}x_{i}x_{j}$$
	
	我们令$$A=\left[\begin{matrix}
		a_{11}&a_{12}&\dots&a_{1n}\\
		a_{21}&a_{22}&\dots&a_{2n}\\
		\vdots&\vdots&\quad&\vdots\\
		a_{n1}&a_{n2}&\dots&a_{nn}
	\end{matrix} \right],\ \mathtt{x}=\left[\begin{matrix}
	x_{1}\\x_{2}\\\vdots\\x_{n}
	\end{matrix} \right]$$
	$$\text{二次型可表示为:  }f(\mathtt{x})=\mathtt{x}^{T}A\mathtt{x},A\text{为二次型}f(\mathtt{x})\text{的矩阵}$$
\end{definition}
\section{二次型的标准型和规范型}
\begin{definition}[线性变换]
	对于$n$元二次型$f(x_{1},x_{2},\cdots,x_{n})$,令:  
	$$\left\lbrace 
	\begin{matrix}
		x_{1}=c_{11}y_{1}+c_{12}y_{2}+\cdots+c_{1n}y_{n}\\
		x_{2}=c_{21}y_{1}+c_{22}y_{2}+\cdots+c_{2n}y_{n}\\
		\cdots\cdots\\
		x_{n}=c_{n1}y_{1}+c_{n2}y_{2}+\cdots+c_{nn}y_{n}
	\end{matrix}
	\right. $$
	记$$\mathtt{x}=\left[\begin{matrix}
		x_{1}\\x_{2}\\\vdots\\x_{n}
	\end{matrix} \right],\ C=\left[ \begin{matrix}
	c_{11}&c_{12}&\cdots&c_{1n}\\
	c_{21}&c_{22}&\cdots&c_{2n}\\
	\vdots&\vdots& &\vdots\\
	c_{n1}&c_{n2}&\cdots&c_{nn}
	\end{matrix}\right],\ \mathtt{y}=\left[\begin{matrix}
	y_{1}\\y_{2}\\\vdots\\y_{n}
	\end{matrix} \right]$$
	上面的线性变化可写作:  $$\mathtt{x}=C\mathtt{y}$$
	我们把这种变换称为$x_{1},x_{2},\cdots,x_{n}$到$y_{1},y_{2},\cdots,y_{n}$的\textbf{线性变换},如果线性变换矩阵$C$可逆,$|C|\neq 0$,则称为\textbf{可逆线性变换}.
	
	我们有:  $$f(\mathtt{x})=\mathtt{x}^{T}A\mathtt{x},\ \mathtt{x}=C\mathtt{y}\Rightarrow f(\mathtt{x})=(C\mathtt{y})^{T}A(C\mathtt{y})=\mathtt{y}^{T}(C^{T}AC)\mathtt{y}$$
	如果我们记$B=C^{T}AC$,我们得到$$f(\mathtt{x})=y^{T}By=g(\mathtt{y})$$
	二次型$f(\mathtt{x})=\mathtt{x}^{T}A\mathtt{x}$通过线性变换$\mathtt{x}=C\mathtt{y}$得到了一个新二次型$g(\mathtt{y})=y^{T}By$
\end{definition}
\begin{definition}[矩阵合同]
	设$A,B$为$n$阶矩阵,若存在可逆矩阵$C$,使得:  
	$$B=C^{T}AC$$
	则称$A,B$合同,记作$A\simeq B$,其对应的二次型$f(\mathtt{x})$与$g(\mathtt{y})$为合同二次型.
	\begin{anymark}[注]
		(1). $A\simeq A\quad \text{反身性}$
		
		(2). $A\simeq B\Rightarrow B\simeq A\quad \text{对称性}$
		
		(3). $A\simeq B,\ B\simeq C\Rightarrow A\simeq C\quad \text{传递性}$
	\end{anymark}
\end{definition}
\begin{definition}[标准型和规范型]
	1. 二次型中只有平方项,而没有交叉项(所有交叉项系数全为0),形如:  
	$$d_{1}x_{1}^2+d_{2}x_{2}^2+\cdots+d_{n}x_{n}^2$$
	的二次型为标准二次型.
	
	2. 在标准二次型中,如果二次型的系数$d_{i}=\{0,1,-1\}$,这样的二次型称为规范型二次型.
	
	3. 二次型$f(\mathtt{x})=\mathtt{x}^{T}A\mathtt{x}$合同于标准型$d_{1}x_{1}^2+d_{2}x_{2}^2+\cdots+d_{n}x_{n}^2$,则称$d_{1}x_{1}^2+d_{2}x_{2}^2+\cdots+d_{n}x_{n}^2$为二次型$f(\mathtt{x})=\mathtt{x}^{T}A\mathtt{x}$的合同标准型.
	
	4. 二次型$f(\mathtt{x})=\mathtt{x}^{T}A\mathtt{x}$合同于规范型$x_{1}^2+\cdots++x_{p}^2-x_{p+1}^2-x_{q}^2$,则称$x_{1}^2+\cdots++x_{p}^2-x_{p+1}^2-x_{q}^2$为二次型$f(\mathtt{x})=\mathtt{x}^{T}A\mathtt{x}$的合同规范型.
	
	5. 任何二次型都可以通过可逆线性变换化为标准型或者规范型,对任意实对称矩阵$A$,必存在可逆矩阵$C$,使得$C^{T}AC=\varLambda$
	
	6. 任何二次型都可以通过正交变换化为标准型,对任意实对称矩阵$A$,必存在正交矩阵$Q$,使得$Q^{-1}AC=Q^{T}AQ=\varLambda$
\end{definition}
\begin{definition}[惯性定理]
	无论用什么样的可逆线性变换得到的二次型标准型或者规范型,标准型或者规范型中正项个数$p$,负项个数$q$都是不变的,$p$被称为正惯性指数,$q$被称为负惯性指数.
\end{definition}
\begin{anymark}[注]
	(1). $\text{二次型的秩为}r,r=p+q$
	
	(2). 两个二次型合同的充要条件为有相同的正、负惯性指数,或者相同的秩及正(负)惯性指数
\end{anymark}
\section{正定二次型}
\begin{definition}[正定矩阵]
	$n$元二次型$f(x_{1},x_{2},\cdots,x_{n})=\mathtt{x}^{T}A\mathtt{x}$,对于任意$\mathtt{x}=[x_{1},x_{2},\cdots,x_{n}]^{T}\neq 0$,都有$\mathtt{x}^{T}A\mathtt{x}>0$,则称$f$为正定二次型,二次型对应的矩阵$A$为正定矩阵.
	\begin{corollary}[二次型正定充要]
		\begin{itemize}
			\item $f\text{正定}\Leftrightarrow f\text{正惯性指数}p=n$
			\item $f\text{正定}\Leftrightarrow \exists \text{可逆矩阵}D,\text{s.t.}\ A=D^{T}D$
			\item $f\text{正定}\Leftrightarrow A\simeq E$
			\item $f\text{正定}\Leftrightarrow A\text{的所有特征值}\lambda_{i}>0$
			\item $f\text{正定}\Leftrightarrow A\text{的全部顺序主子式大于}0$
		\end{itemize}
	\end{corollary}
	\begin{corollary}[二次型正定必要]
		\begin{itemize}
			\item $f\text{正定}\Rightarrow a_{ii}>0$
			\item $f\text{正定}\Rightarrow |A|>0$
		\end{itemize}
	\end{corollary}
\end{definition}