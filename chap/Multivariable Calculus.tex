\chapterimage{chap7.jpg}
\chapter{多元函数微分}
\begin{introduction}
	\item 连续、偏导、可微、全微分
	\item 链式法则
	\item 隐函数存在定理
	\item 多元函数极值和最值
	\item 拉格朗日数乘法
\end{introduction}
\section{多元函数微分概念}
\begin{definition}[邻域]
	\begin{enumerate}
		\item $\delta$ 邻域: 设 $P_{0}(x_{0},y_{0})$ 是 $xOy$ 平面上的一个点, $U(P_{0},\delta)$ 表示以 $P_{0}$ 为中心, 
		半径为 $\delta$ 的圆盘, 即 $U(P_{0},\delta) = \{(x,y)|\sqrt{(x-x_{0})^{2}+(y-y_{0})^{2}}<\delta\}$
		\item 去心 $\delta$ 邻域: $\mathring{U}(P_{0},\delta) = \{(x,y)|0<\sqrt{(x-x_{0})^{2}+(y-y_{0})^{2}}<\delta\}$
	\end{enumerate}
\end{definition}
\begin{definition}[多元函数极限]
	设函数 $f(x,y)$ 在区域 $D$ 上有定义, $P_{0}(x_{0},y_{0})\in D$ 或为区域 $D$ 边界上的一点, 如果对于任意给定的正数 $\varepsilon$,
总存 $\delta>0$, 使得当点 $P(x,y)\in D$ 且 $0<\sqrt{(x-x_{0})^{2}+(y-y_{0})^{2}}<\delta$ 时, 对应的函数值 $f(x,y)$ 都满足不等式 
$|f(x,y)-A|<\varepsilon$, 那么称函数 $f(x,y)$ 当 $(x,y)\to(x_{0},y_{0})$ 时的极限为 $A$, 记为 $\lim\limits_{\substack{x\to x_{0}\\y\to y_{0}}}f(x,y)=A$
\end{definition}
\begin{definition}[连续]\label{def: 多元微分学概念: 连续、偏导、可微}
	设函数 $f(x,y)$ 在区域 $D$ 上有定义, $P_{0}(x_{0},y_{0})\in D$ 或为区域 $D$ 边界上的一点, 如果 $\lim\limits_{\substack{x\to x_{0}\\y\to y_{0}}}f(x,y)=f(x_{0},y_{0})$, 
那么称函数 $f(x,y)$ 在点 $(x_{0},y_{0})$ 处连续, 如果函数 $f(x,y)$ 在区域 $D$ 上每一点都连续, 那么称函数 $f(x,y)$ 在区域 $D$ 上连续
\end{definition}
\begin{definition}[偏导数]
	
	设函数 $f(x,y)$ 在 $(x_{0},y_{0})$ 邻域内有定义, 若极限 $\lim\limits_{\Delta x\rightarrow 0}\dfrac{f(x_{0}+\Delta x,y_{0})-f(x_{0},y_{0})}{\Delta x}$ 存在,
	我们将这记作 $f(x,y)$ 在 $(x_{0},y_{0})$ 处对 $x$ 的偏导数,记作: 
	$$\dfrac{\partial z}{\partial x}\big|_{\substack{x=x_{0}\\y=y_{0}}}\qquad\qquad 
	\dfrac{\partial f}{\partial x}\big|_{\substack{x=x_{0}\\y=y_{0}}}\qquad\qquad
	z_{x}'\big|_{\substack{x=x_{0}\\y=y_{0}}}$$

	类似地, 我们可以定义 $f(x,y)$ 在 $(x_{0},y_{0})$ 处对 $y$ 的偏导数, 记作:
	$$\dfrac{\partial z}{\partial y}\big|_{\substack{x=x_{0}\\y=y_{0}}}\qquad\qquad
	\dfrac{\partial f}{\partial y}\big|_{\substack{x=x_{0}\\y=y_{0}}}\qquad\qquad
	z_{y}'\big|_{\substack{x=x_{0}\\y=y_{0}}}$$

	$$f_{x}'(x_{0},y_{0}) = \lim\limits_{\Delta x\rightarrow 0}\dfrac{f(x_{0}+\Delta x,y_{0})-f(x_{0},y_{0})}{\Delta x}$$
	$$f_{y}'(x_{0},y_{0}) = \lim\limits_{\Delta y\rightarrow 0}\dfrac{f(x_{0},y_{0}+\Delta y)-f(x_{0},y_{0})}{\Delta y}$$
	对偏导数进一步求偏导数,我们可以得到高阶偏导数: $f''_{xx}(x,y),f''_{yy}(x,y),f''_{xy}(x,y),f''_{yx}(x,y)$
\end{definition}
\begin{definition}[可微]
	
	函数 $f(x,y)$ 在点 $(x,y)$ 处的全增量 $\Delta z=f(x+\Delta x,y+\Delta y)-f(x,y)$ 可表示为: 
	$$\Delta z=A\Delta x+B\Delta y+o(\rho)\qquad \rho=\sqrt{(\Delta x)^2+(\Delta y)^2}$$
	
	其中 $A,B$ 只与 $x,y$ 相关, 我们称 $f(x,y)$ 在点 $(x,y)$ 处可微,$A\Delta x+B\Delta y$ 是 $f(x,y)$ 在点 $(x,y)$ 处的全微分.
	$$dz=A\Delta x+B\Delta y=Adx+Bdy$$
	\begin{enumerate}
		\item 可微必要条件: $f(x,y)$ 在点 $(x,y)$ 处可微 $\Rightarrow$ $f(x,y)$ 在点 $(x,y)$ 处偏导数必定存在且 $\begin{cases} A = \dfrac{\partial z}{\partial x}  \\ B = \dfrac{\partial z}{\partial y}\end{cases}$
		\item 可微充分条件: $f(x,y)$ 在点 $(x,y)$ 处偏导数连续 $\Rightarrow$ $f(x,y)$ 在点 $(x,y)$ 处可微
	\end{enumerate}
\end{definition}
\begin{definition}[偏导数连续性]
	
	$$\begin{cases}
	f'_{x}(x_{0},y_{0}) = \lim\limits_{\Delta x\rightarrow 0}\dfrac{f(x_{0}+\Delta x,y_{0})-f(x_{0},y_{0})}{\Delta x} = \lim\limits_{x\to x_{0}} f'_{x}(x,y_{0})\\
	f'_{y}(x_{0},y_{0}) = \lim\limits_{\Delta y\rightarrow 0}\dfrac{f(x_{0},y_{0}+\Delta y)-f(x_{0},y_{0})}{\Delta y} = \lim\limits_{y\to y_{0}} f'_{y}(x_{0},y)
	\end{cases}$$
	
	如果这两个极限相等,我们就称偏导数在此点连续
\end{definition}
\section{链式法则}

\begin{definition}[链式法则]\label{def: 链式法则}
	$$z=f(u,v),u=\varphi(x,y),v=\phi(x,y)$$
	
	我们可以得到偏导数: 
	$$\dfrac{\partial z}{\partial x}=\dfrac{\partial z}{\partial u}\dfrac{\partial u}{\partial x}+\dfrac{\partial z}{\partial v}\dfrac{\partial v}{\partial x}$$
	$$\dfrac{\partial z}{\partial y}=\dfrac{\partial z}{\partial u}\dfrac{\partial u}{\partial y}+\dfrac{\partial z}{\partial v}\dfrac{\partial v}{\partial y}$$
	
	高阶偏导数: 
	$$\dfrac{\partial^2 z}{\partial x^2}=\left[\dfrac{\partial (\dfrac{\partial z}{\partial u})}{\partial u}\dfrac{\partial u}{\partial x}+\dfrac{\partial (\dfrac{\partial z}{\partial u})}{\partial v}\dfrac{\partial v}{\partial x}\right]\dfrac{\partial u}{\partial x}+
	\dfrac{\partial z}{\partial u}\frac{\partial ^2u}{\partial x^2}+
	\left[\dfrac{\partial (\dfrac{\partial z}{\partial v})}{\partial u}\dfrac{\partial u}{\partial x}+\dfrac{\partial (\dfrac{\partial z}{\partial v})}{\partial v}\dfrac{\partial v}{\partial x}\right]\dfrac{\partial v}{\partial x}+
	\dfrac{\partial z}{\partial v}\frac{\partial ^2v}{\partial x^2}$$
	
	$$\dfrac{\partial^2 z}{\partial y^2}=\left[\dfrac{\partial (\dfrac{\partial z}{\partial u})}{\partial u}\dfrac{\partial u}{\partial y}+\dfrac{\partial (\dfrac{\partial z}{\partial u})}{\partial v}\dfrac{\partial v}{\partial y}\right]\dfrac{\partial u}{\partial y}+
	\dfrac{\partial z}{\partial u}\dfrac{\partial ^2u}{\partial y^2}+
	\left[\dfrac{\partial (\dfrac{\partial z}{\partial v})}{\partial u}\dfrac{\partial u}{\partial y}+\dfrac{\partial (\dfrac{\partial z}{\partial v})}{\partial v}\dfrac{\partial v}{\partial y}\right]\dfrac{\partial v}{\partial y}+
	\dfrac{\partial z}{\partial v}\dfrac{\partial ^2v}{\partial y^2}$$
	
	$$\dfrac{\partial^2 z}{\partial x\partial y}=\left[\dfrac{\partial (\dfrac{\partial z}{\partial u})}{\partial u}\dfrac{\partial u}{\partial y}+\dfrac{\partial (\dfrac{\partial z}{\partial u})}{\partial v}\dfrac{\partial v}{\partial y}\right]\frac{\partial u}{\partial x}+
	\dfrac{\partial z}{\partial u}\dfrac{\partial ^2u}{\partial x\partial y}+
	\left[\dfrac{\partial (\dfrac{\partial z}{\partial v})}{\partial u}\dfrac{\partial u}{\partial y}+\dfrac{\partial (\dfrac{\partial z}{\partial v})}{\partial v}\dfrac{\partial v}{\partial y}\right]\dfrac{\partial v}{\partial x}+
	\dfrac{\partial z}{\partial v}\dfrac{\partial ^2v}{\partial x\partial y}$$

	$$\dfrac{\partial^2 z}{\partial y\partial x}=\left[\dfrac{\partial (\dfrac{\partial z}{\partial u})}{\partial u}\dfrac{\partial u}{\partial x}+\dfrac{\partial (\dfrac{\partial z}{\partial u})}{\partial v}\dfrac{\partial v}{\partial x}\right]\frac{\partial u}{\partial y}+
	\dfrac{\partial z}{\partial u}\dfrac{\partial ^2u}{\partial y\partial x}+
	\left[\dfrac{\partial (\dfrac{\partial z}{\partial v})}{\partial u}\dfrac{\partial u}{\partial x}+\dfrac{\partial (\dfrac{\partial z}{\partial v})}{\partial v}\dfrac{\partial v}{\partial x}\right]\dfrac{\partial v}{\partial y}+
	\dfrac{\partial z}{\partial v}\dfrac{\partial ^2v}{\partial y\partial x}$$

	综上所述, 我们有:
	$$\dfrac{\partial^2 z}{\partial x^2}= \dfrac{\partial ^2z}{\partial u^{2}}(\dfrac{\partial u}{\partial x})^{2}+\dfrac{\partial^{2} z}{\partial u\partial v}\frac{\partial v}{\partial x}\frac{\partial u}{\partial x}+\dfrac{\partial z}{\partial u}\frac{\partial ^2u}{\partial x^2}+
	\dfrac{\partial ^2z}{\partial v\partial u}\dfrac{\partial u}{\partial x}\dfrac{\partial v}{\partial x}+\dfrac{\partial^{2} z}{\partial v^{2}}(\frac{\partial v}{\partial x})^{2}+\dfrac{\partial z}{\partial v}\dfrac{\partial^{2} v}{\partial x^{2}}$$

	$$\dfrac{\partial^2 z}{\partial y^2}= \dfrac{\partial ^2z}{\partial u^{2}}(\dfrac{\partial u}{\partial y})^{2}+\dfrac{\partial^{2} z}{\partial u\partial v}\frac{\partial v}{\partial y}\frac{\partial u}{\partial y}+\dfrac{\partial z}{\partial u}\frac{\partial ^2u}{\partial y^2}+
	\dfrac{\partial ^2z}{\partial v\partial u}\dfrac{\partial u}{\partial y}\dfrac{\partial v}{\partial y}+\dfrac{\partial^{2} z}{\partial v^{2}}(\frac{\partial v}{\partial y})^{2}+\dfrac{\partial z}{\partial v}\dfrac{\partial^{2} v}{\partial y^{2}}$$
	
	$$\dfrac{\partial^2 z}{\partial x\partial y}= \dfrac{\partial ^2z}{\partial u^{2}}\dfrac{\partial u}{\partial y}\dfrac{\partial u}{\partial x}+\dfrac{\partial^{2} z}{\partial u\partial v}\frac{\partial v}{\partial y}\frac{\partial u}{\partial x}+\dfrac{\partial z}{\partial u}\frac{\partial ^2u}{\partial x\partial y}+
	\dfrac{\partial ^2z}{\partial v\partial u}\dfrac{\partial u}{\partial y}\dfrac{\partial v}{\partial x}+\dfrac{\partial^{2} z}{\partial v^{2}}\frac{\partial v}{\partial y}\frac{\partial v}{\partial x}+\dfrac{\partial z}{\partial v}\dfrac{\partial^{2} v}{\partial x\partial y}$$

	$$\dfrac{\partial^2 z}{\partial y\partial x}= \dfrac{\partial ^2z}{\partial u^{2}}\dfrac{\partial u}{\partial x}\dfrac{\partial u}{\partial y}+\dfrac{\partial^{2} z}{\partial u\partial v}\frac{\partial v}{\partial x}\frac{\partial u}{\partial y}+\dfrac{\partial z}{\partial u}\frac{\partial ^2u}{\partial y\partial x}+
	\dfrac{\partial ^2z}{\partial v\partial u}\dfrac{\partial u}{\partial x}\dfrac{\partial v}{\partial y}+\dfrac{\partial^{2} z}{\partial v^{2}}\frac{\partial v}{\partial x}\frac{\partial v}{\partial y}+\dfrac{\partial z}{\partial v}\dfrac{\partial^{2} v}{\partial y\partial x}$$
\end{definition}
\begin{definition}[全微分形式不变性]
	设 $z=f(x,y)$, $x=x(u,v),y=y(u,v)$, 如果 $f(u,v),u(x,y),v(x,y)$ 分别有连续偏导数, 则复合函数 $z=f(u,v)$ 在 $(x,y)$ 处的全微分:
	$$dz=\dfrac{\partial z}{\partial x}dx+\dfrac{\partial z}{\partial y}dy=\dfrac{\partial z}{\partial u}du+\dfrac{\partial z}{\partial v}dv$$
\end{definition}
\section{隐函数存在定理}

\begin{theorem}[隐函数存在定理 1]\label{the: 隐函数存在定理}
	如果函数 $F(x,y)$ 满足一下条件:
	\begin{enumerate}
		\item $F(x_{0},y_{0}) = 0$
		\item $F(x,y)$ 在 $(x_{0},y_{0})$ 的某一个邻域内具有连续偏导数
		\item $F'(x_{0},y_{0})\neq 0$
	\end{enumerate}
	那么方程 $F(x,y)=0$ 在点 $(x_{0},y_{0})$ 的某一个邻域内能够确定唯一的连续且具有连续导数的函数 $y=f(x)$, 满足 $F(x,f(x))=0$ 且 $y_{0} = y(x_{0})$, 且有
	$$\dfrac{dy}{dx} = -\dfrac{F'_{x}}{F'_{y}}$$
\end{theorem}
\begin{theorem}[隐函数存在定理 2]
	如果函数 $F(x,y,z)$ 满足一下条件:
	\begin{enumerate}
		\item $F(x_{0},y_{0},z_{0}) = 0$
		\item $F(x,y,z)$ 在 $(x_{0},y_{0},z_{0})$ 的某一个邻域内具有连续偏导数
		\item $F'(x_{0},y_{0},z_{0})\neq 0$
	\end{enumerate}
	那么方程 $F(x,y,z)=0$ 在点 $(x_{0},y_{0},z_{0})$ 的某一个邻域内能够确定唯一的连续且具有连续导数的函数 $z=f(x,y)$, 满足 $F(x,y,z(x,y))=0$ 且 $z_{0} = z(x_{0},y_{0})$, 且有
	$$\dfrac{\partial z}{\partial x} = -\dfrac{F'_{x}}{F'_{z}}\qquad\qquad \dfrac{\partial z}{\partial y} = -\dfrac{F'_{y}}{F'_{z}}$$
\end{theorem}
\section{多元函数极值和最值}
\begin{definition}[多元函数极值和最值]
	\textbf{极值}

	设函数 $f(x,y)$ 在点 $(x_{0},y_{0})$ 处有定义
	
	1. 如果存在邻域 $U(P_{0},\delta)$, 使得对于任意 $(x,y)\in U(P_{0},\delta)$, 都有 $f(x,y)\leq f(x_{0},y_{0})$, 那么称 $f(x_{0},y_{0})$ 是函数 $f(x,y)$ 的一个极大值
	
	2. 如果存在邻域 $U(P_{0},\delta)$, 使得对于任意 $(x,y)\in U(P_{0},\delta)$, 都有 $f(x,y)\geq f(x_{0},y_{0})$, 那么称 $f(x_{0},y_{0})$ 是函数 $f(x,y)$ 的一个极小值

	\textbf{最值}

	1. 如果对于区域 $D$ 上的任意 $(x,y)$, 都有 $f(x,y)\leq f(x_{0},y_{0})$, 那么称 $f(x_{0},y_{0})$ 是函数 $f(x,y)$ 的一个最大值

	2. 如果对于区域 $D$ 上的任意 $(x,y)$, 都有 $f(x,y)\geq f(x_{0},y_{0})$, 那么称 $f(x_{0},y_{0})$ 是函数 $f(x,y)$ 的一个最小值
\end{definition}
\textbf{无条件极值}
\begin{definition}[多元函数极值]
	二元函数 $f(x,y)$ 在点 $(x_{0},y_{0})$ 取极值的必要条件: 
	$$f'_{x}(x_{0},y_{0})=f'_{y}(x_{0},y_{0})=0$$
	
	二元函数 $f(x,y)$ 在点 $(x_{0},y_{0})$ 取极值的充分条件: 
	$$\begin{cases}
		f_{xx}'(x_{0},y_{0})=A\\
		f_{xy}'(x_{0},y_{0})=B\\
		f_{yy}'(x_{0},y_{0})=C
	\end{cases}  
	\quad \Delta=AC-B^2\Rightarrow
	\begin{cases}
		\Delta>0,\begin{cases} A>0, & \min\\A<0, & \max\end{cases}\\
		\Delta<0, \text{非极值}\\ 
		\Delta=0, \text{方法失效}
	\end{cases}
	$$
\end{definition}
\textbf{条件极值}
\begin{definition}[拉格朗日数乘法]
	
	求目标函数 $u=f(x,y,z)$ 在条件 $\begin{cases}g(x,y,z)=0\\h(x,y,z)=0\end{cases}$ 下的最值
	
	构造辅助函数:  $F(x,y,z,\lambda,\mu)=f(x,y,z)+\lambda g(x,y,z)+\mu h(x,y,z)$
	
	令 $$\begin{cases} F_{x}'=f_{x}'+\lambda g_{x}'+\mu h_{x}'=0\\
		F_{y}'=f_{y}'+\lambda g_{y}'+\mu h_{y}'=0\\
		F_{z}'=f_{z}'+\lambda g_{z}'+\mu h_{z}'=0\\
		F_{\lambda}'= g(x,y,z)=0\\
		F_{\mu}'= h(x,y,z)=0
	\end{cases}$$
	
	得到所有的备选点 $P_{i}$,计算 $f(P_{i})$ 得到最大值和最小值.
\end{definition}
\begin{anymark}[注]
	1. 在不封闭曲线上求最值,可以用拉格朗日数乘法,但是要注意边界条件

	2. 闭区域上多元函数的最值,分为两部分, 第一部分是在区域内部求最值, 第二部分是在区域边界求最值; 前者利用驻点, 后者利用拉格朗日数乘法, 两者结合求最值
\end{anymark}



\chapterimage{chap8.jpg}
\chapter{二重积分}
\begin{introduction}
	\item 二重积分定义
	\item 积分次序
	\item 极坐标和直角坐标下的二重积分
	\item 变量替换
\end{introduction}

\section{概念和性质}
\begin{definition}[二重积分]
	设 $f(x,y)$ 在有界闭区域 $D$ 上有界, $D$ 的面积为 $S_{D}$, $D$ 上的一个分割为 $D=\bigcup\limits_{i=1}^{n}D_{i}$, $\Delta \sigma_{i}$ 是 $D_{i}$ 的面积, 任取 $(\varepsilon_{i},\eta_{i})\in D_{i}$, 作乘积 $f(\varepsilon_{i},\eta_{i})\sigma_{i}$,
并求和 $\sum\limits_{i=1}^{n}f(\varepsilon_{i},\eta_{i})\sigma_{i}$, 如果当 $\max\{d_{i}|d_{i}\text{是}D_{i}\text{区域的直径}\}\to 0$,极限 $\lim\limits_{n\to +\infty}\sum\limits_{i=1}^{n}f(\varepsilon_{i},\eta_{i})\sigma_{i}$ 存在, 且与 $D$ 的分割方法和 $(\varepsilon_{i},\eta_{i})$ 的取法无关, 
那么称此极限为 $f(x,y)$ 在区域 $D$ 上的二重积分, 记作 $\iint\limits_{D}f(x,y)d\sigma$
\end{definition}
\begin{definition}[性质]
	1. 二重积分的几何意义: 二重积分 $\iint\limits_{D}f(x,y)d\sigma$ 表示区域 $D$ 上以 $f(x,y)$ 为曲顶的曲顶柱体的体积

	2. 二重积分的线性性质: $\iint\limits_{D}(\alpha f(x,y)+\beta g(x,y))d\sigma=\alpha\iint\limits_{D}f(x,y)d\sigma+\beta\iint\limits_{D}g(x,y)d\sigma$
	
	3. 类比于一元积分学,我们有: $$\int_{a}^{b}1dx=b-a\to \iint\limits_{D}1d\sigma=S_{D}$$
\end{definition}
\begin{theorem}[对称性]
	1. 普通对称性

	(i). 区域 $D$ 关于 $x = a$ 对称, 我们有:
	$$\iint\limits_{D} f(x,y)d\sigma = \begin{cases} 2\iint\limits_{D_{1}} f(x,y)d\sigma, & f(2a-x) = f(x) \\ 0, & f(2a-x) = -f(x)  \end{cases}$$
	特别的, 当 $a = 0$ 时, 我们有:
	$$\iint\limits_{D} f(x,y)d\sigma = \begin{cases} 2\iint\limits_{D_{1}} f(x,y)d\sigma, & f(-x) = f(x) \\ 0, & f(-x) = -f(x)  \end{cases}$$

	(ii). 区域 $D$ 关于 $y = b$ 对称, 我们有:
	$$\iint\limits_{D} f(x,y)d\sigma = \begin{cases} 2\iint\limits_{D_{1}} f(x,y)d\sigma, & f(x,2b-y) = f(x,y) \\ 0, & f(x,2b-y) = -f(x,y)  \end{cases}$$
	特别的, 当 $b = 0$ 时, 我们有:
	$$\iint\limits_{D} f(x,y)d\sigma = \begin{cases} 2\iint\limits_{D_{1}} f(x,y)d\sigma, & f(x,-y) = f(x,y) \\ 0, & f(x,-y) = -f(x,y)  \end{cases}$$

	2. 轮换对称性

	区域 $D$ 关于 $x = y$ 对称, 我们有:
	$$\iint\limits_{D} f(x,y)d\sigma = \iint\limits_{D} f(y,x)d\sigma= \dfrac{1}{2}\iint\limits_{D}\left[f(x,y)+f(y,x)\right]d\sigma$$
	$$\iint\limits_{D} f(x,y)d\sigma = \begin{cases} 2\iint\limits_{D_{1}} f(x,y)d\sigma, & f(x,y) = f(y,x) \\ 0, & f(x,y) = -f(y,x)  \end{cases}$$

	3. 区域 $D$ 关于原点对称, 我们有:
	$$\iint\limits_{D} f(x,y)d\sigma = \begin{cases} 2\iint\limits_{D_{1}} f(x,y)d\sigma, & f(-x,-y) = f(x,y) \\ 0, & f(-x,-y) = -f(x,y)  \end{cases}$$
\end{theorem}
\section{计算}

1. 直角坐标(重要的是积分次序)\label{def: 积分次序}

$$\iint\limits_{D}f(x,y)d\sigma=
\begin{cases} \int_{a}^{b}dx\int_{h(x)}^{g(x)}f(x,y)dy \\
	\int_{a}^{b}dy\int_{p(y)}^{q(y)}f(x,y)dx\end{cases}$$

2. 极坐标计算(二重积分变量替换公式)\label{def: 极坐标计算二重积分}
$$\iint\limits_{D}f(x,y)d\sigma=\iint\limits_{D'}rf(r\cos \theta,r\sin \theta)drd\theta$$

3.变量替换\label{def: 变量替换}

设$D$和$D^{'}$是平面上两个(有界)区域,$D$到$D^{'}$的对应$\varphi :(u,v)\rightarrow(x,y)$ (这里 $x=x(u,v),y=y(u,v)$ 连续可微),称为变量替换,要求 $\varphi$ 在一个面积为 $0$ 的集合外是 $1\sim 1$,我们有:

$$dxdy=J_{\varphi}(u,v)dudv$$ 
$$J_{\varphi}(u,v)=\dfrac{D(x,y)}{D(u,v)}=
\begin{vmatrix}
	\dfrac{\partial x}{\partial u} & \dfrac{\partial x}{\partial v} \\
	\dfrac{\partial y}{\partial u} & \dfrac{\partial y}{\partial v}
\end{vmatrix}
$$

\begin{anymark}[注]
	$$d\sigma_{1}=dudv \qquad d\sigma_{2}=|l\times m|$$
	$$\begin{cases}
		x(u,v+dv)-x(u,v)=x'_{v}dv \\
		x(u+du,v)-x(u,v)=x'_{u}du \\
		y(u,v+dv)-y(u,v)=x'_{v}dv \\
		y(u+du,v)-y(u,v)=y'_{u}du
	 \end{cases}\Rightarrow 
	 \begin{cases}
		l=(x'_{u}du,y'_{u}du) \\
		m=(x'_{v}dv,y'_{v}dv)  
	\end{cases}$$
	$$d\sigma_{2}=(x'_{u}y'_{v}-x'_{v}y'_{u})dvdu=\begin{vmatrix}
			x'_{u} & x'_{v} \\
			y'_{u} & y'_{v}
		\end{vmatrix}d\sigma_{1}$$
\end{anymark}
\section{二重积分解决一元积分}
几个比较经典的例子:

1. $\int_{0}^{+\infty}e^{-x^{2}}dx$
\begin{anymark}[注]
	我们有: $I = \int_{0}^{+\infty}e^{-x^{2}}dx = \int_{0}^{+\infty}e^{-y^{2}}dy$

	\begin{eqnarray*}
		I^{2} &=& \int_{0}^{+\infty}e^{-x^{2}}dx\int_{0}^{+\infty}e^{-y^{2}}dy\\
			  &=& \int_{0}^{+\infty}dx\int_{0}^{+\infty}e^{-(x^{2}+y^{2})}dy\\
	\end{eqnarray*}
\end{anymark}
2. $\int_{0}^{a}f(x)dx\int_{0}^{a}\frac{1}{f(x)}dx\geq a^{2}$

\chapterimage{chap9.jpg}
\chapter{常微分方程}
\begin{introduction}
	\item 微分方程概念、解和通解
	\item 一阶微分方程
	\item 伯努利方程
	\item 高阶线性微分方程
	\item (非)齐次二阶常系数线性微分方程
	\item 欧拉方程
\end{introduction}
\begin{definition}[微分方程及其阶]
	表示未知函数及其导数(或者微分)与自变量之间关系的方程称为微分方程, 一般写为:
	$$F(x,y,y',y'',\dots,y^{(n)})=0\text{或} y^{(n)} = f(x,y,y',\cdots,y^{(n-1)})$$
	
	微分方程中未知函数的最高阶导数的阶数称为\textbf{微分方程的阶}.
\end{definition}

\begin{definition}[常微分方程]
	未知函数是一元函数的微分方程称为\textbf{常微分方程}.
\end{definition}

\begin{definition}[线性微分方程]
	$$a_{n}(x)y^{(n)} + a_{n-1}(x)y^{(n-1)} + \cdots + a_{1}(x)y' + a_{0}(x) y = f(x)$$
	形如上述的微分方程称为 $n$ 阶\textbf{线性微分方程}, 其中 $a_{k}(x)(k=0,1,2,\cdots,n)$ 都是自变量 $x$ 的函数, $a_{k}(x)\not\equiv 0$, 当 $a_{k}(x)(k=0,1,2,\cdots,n)$ 都是常数时,
	又称方程为 $n$ 阶\textbf{常系数线性微分方程}; 若右端 $f(x)\equiv 0$, 则称方程为 $n$ 阶\textbf{齐次线性微分方程}, 否则称其为 $n$ 阶\textbf{非齐次线性微分方程}.
\end{definition}
\begin{definition}[微分方程的解和通解]
	\begin{itemize}
		\item 若将函数代入微分方程, 使方程成为恒等式, 则该函数称为\textbf{微分方程的解}, 微分方程解的图形称为积分曲线
		\item 若微分方程的解中含有的独立常数的个数等于微分方程的阶数, 则该解称为微分方程的\textbf{通解}.
	\end{itemize}
\end{definition}
\begin{definition}[初始条件和特解]
	确定通解中常数的条件就是\textbf{初始条件},如 $y(x_{0})=a_{0},y'(x_{0})=a_{1},\cdots,y^{(n-1)}(x_{0})=a_{n-1}$,
	其中 $a_{0},a_{1},\cdots,a_{n-1}$ 为 $n$ 个给定的数, 确定通解中的常数后, 解就成为\textbf{特解}.
\end{definition}
\section{一阶微分方程}

\subsection{可分离变量型微分方程}\label{def: 分离变量型一阶微分方程}
\subsubsection{直接可分离}
$$\dfrac{dy}{dx} = F(x,y)=f(x)g(y)\Rightarrow \int \dfrac{dy}{g(y)} = \int f(x)dx$$
\subsubsection{换元后可分离}

$$\dfrac{dy}{dx} = f(ax+by+c)\Rightarrow 
\begin{cases}
	u = ax +by +c\\
	\dfrac{du}{dx} = a + b\dfrac{dy}{dx}\\
	\dfrac{du}{dx} = a + bf(u)
\end{cases}\Rightarrow \int \dfrac{du}{a + bf(u)} = \int dx$$
\begin{anymark}[注]
	\begin{itemize}
		\item 在换元过程中, 可能会因为定义域问题漏掉某些解, 这些解称为奇解.
		\item 非线性微分方程的所有解等于通解和奇解的并集; 线性微分方程的所有解等于通解, 没有奇解.
	\end{itemize}
\end{anymark}

\subsection{齐次型微分方程}
$$\dfrac{dy}{dx} = \varphi(\dfrac{y}{x})\Rightarrow 
\begin{cases}
	u = \dfrac{y}{x}\\
	\dfrac{dy}{dx} = \dfrac{d(ux)}{x} = u + x\dfrac{du}{dx}\\
	u + x\dfrac{du}{dx} = \varphi(u)
\end{cases}\Rightarrow \int \dfrac{du}{\varphi(u) - u} =\int \dfrac{dx}{x} $$

\subsection{一阶线性微分方程}\label{def: 一阶线性微分方程公式}
\begin{definition}[一阶线性微分方程]
	$$y'+p(x)y=q(x), p(x)\text{和} q(x)\text{是已知的连续函数}$$
\end{definition}
\begin{theorem}[一阶线性微分方程解]
	$$y=e^{-\int p(x)dx}\left[\int e^{\int p(x)dx}q(x)dx+C\right]$$
	\begin{anymark}[注]
		\begin{eqnarray*}
			&\quad & e^{\int p(x)dx}y' + p(x)e^{\int p(x)dx} = q(x)\cdot e^{\int p(x)dx}\\
			&\quad & \left[e^{\int p(x)dx}y\right]' = q(x)\cdot e^{\int p(x)dx}\\
			&\quad & e^{\int p(x)dx}y = \int q(x)\cdot e^{\int p(x)dx} + C\\
			&\quad & y = e^{-\int p(x)dx}\left[\int q(x)\cdot e^{\int p(x)dx} + C\right]
		\end{eqnarray*}
	\end{anymark}
\end{theorem}

\subsection{伯努利方程}
\begin{definition}[伯努利方程]\label{def: 伯努利方程}
	$$\dfrac{dy}{dx}+p(x)y=q(x)y^{n}$$
\end{definition}
\begin{theorem}
	$$y^{-n}\dfrac{dy}{dx} +p(x)y^{1-n} = q(x)\Rightarrow
	\begin{cases}
		z = y^{1-n}\\
		\dfrac{dz}{dx} = \dfrac{1}{1-n}y^{-n}\dfrac{dy}{dx}
	\end{cases}\Rightarrow (1-n)\dfrac{dz}{dx} + p(x)z =q(x)$$

	我们可以得到: 
	$$\dfrac{dz}{dx} + (1-n)p(x)z = (1-n)q(x)\Rightarrow z = e^{-\int (1-n)p(x)dx}\left[ e^{\int (1-n)p(x)dx}\cdot q(x)+ C \right]$$
\end{theorem}

\subsection{二阶可降阶微分方程}
\begin{definition}[二阶可降阶微分方程]

	1. $y''=f(y,y')\Leftrightarrow F(y,y',y'') = 0$

	我们令:  $p=y'$,则 
	$$y''=\frac{dp}{dx}=\frac{dp}{dy}\frac{dy}{dx}=p'p \Rightarrow p\dfrac{dp}{dy} = f(y,p)$$

	2. $y''=f(x,y')\Leftrightarrow F(x,y',y'') = 0$

	我们令:  $p(x)=y'$,则
	$$y''=\frac{dp}{dx}\Rightarrow \dfrac{dp}{dx} = f(x,p)$$
\end{definition}
\section{高阶线性微分方程}
\subsection{二阶常系数线性微分方程}
\begin{definition}[二阶常系数线性微分方程]
	二阶常系数齐次微分方程:
	$$y''+py'+py=0$$
	二阶常系数非齐次微分方程:
	$$y''+py'+py=f(x)$$
\end{definition}
\begin{theorem}[二阶常系数齐次线性微分方程解]\label{the: 齐次二阶常系数线性微分方程}
	对于二阶常系数齐次x线性微分方程:

	特征方程:  $\lambda^{2}+p\lambda+q=0$

	\begin{itemize}
		\item 当方程有两个不同的实数根 $\lambda_{1},\lambda_{2}$ ,微分方程通解: $$y=C_{1}e^{\lambda_{1} x}+C_{2}e^{\lambda_{2}x}$$
		\item 当方程有两个相同的实根 $\lambda_{1}=\lambda_{2}=\lambda$ ,微分方程通解: $$y=C_{1}+C_{2}xe^{\lambda x}$$
		\item 当方程有两个不同的虚根 $\lambda_{1}=\alpha +i\beta,\lambda_{2}=\alpha-i\beta$ ,微分方程通解: $$y=e^{\alpha x}(C_{1}\cos \beta x+C_{2}\sin \beta x)$$
	\end{itemize}
\end{theorem}
\begin{theorem}[二阶常系数非齐次线性微分方程解]
	对于二阶常系数非齐次线性微分方程:
	$$y''+py'+py=f(x)$$
	\textbf{通解为二阶常系数齐次线性微分方程的通解加上特解}: $y_{0}=y^{*}+y$

	1. 当 $f(x)=e^{\alpha x}P_{n}(x)$时,特解 $y^{*}$:

	$$y^{*}=e^{\alpha x}x^{k}Q_{n}(x)$$
	\begin{itemize}
		\item 当 $\alpha$ 不是特征方程的根,$k=0$
		\item 当 $\alpha$ 是特征方程的一个根,$k=1$
		\item 当 $\alpha$ 是特征方程的重根,$k=2$
	\end{itemize}

	2. 当 $f(x)=e^{\alpha x}(P_{n}(x)\cos \beta x+P_{m}(x)\sin \beta x)$时,特解 $y^{*}$:

	$$y^{*}=e^{\alpha x}x^{k}(Q_{l}^{(1)}(x)\cos \beta x+Q_{l}^{(2)}(x)\sin \beta x),\quad l=max\{m,n\}$$
	\begin{itemize}
		\item 当 $\alpha\pm i\beta$ 不是特征方程的根,$k=0$
		\item 当 $\alpha\pm i\beta$ 是特征方程的根,$k=1$
	\end{itemize}
\end{theorem}

\subsection{欧拉方程}
\begin{definition}[欧拉方程]\label{def: 欧拉方程}
	形如以下形式的微分方程:
	$$x^{2}\dfrac{d^{2}y}{dx^2}+px\dfrac{dy}{dx}+qy=f(x)$$

	1. 当 $x>0$ 时,令 $x=e^t,t=\ln x;\dfrac{dt}{dx}=\dfrac{1}{x}$

	$$\dfrac{dy}{dx}=\dfrac{dy}{dt}\dfrac{dt}{dx}=\dfrac{1}{x}\dfrac{dy}{dt}$$
	$$\dfrac{d^{2}y}{dx^2}=\dfrac{d(\frac{dy}{dx})}{dt}\dfrac{dt}{dx}=\dfrac{1}{x^2}\dfrac{d^{2}y}{dt^2}$$

	原微分方程可化为:
	$$\dfrac{d^{2}y}{dt^2}+p\dfrac{dy}{dt}+qy=f(e^t)$$

	2. 当 $x<0$ 时,令 $x=-e^t,t=\ln(-x);\dfrac{dt}{dx}=\dfrac{1}{x}$

	$$\dfrac{dy}{dx}=\dfrac{dy}{dt}\dfrac{dt}{dx}=\dfrac{1}{x}\dfrac{dy}{dt}$$
	$$\dfrac{d^{2}y}{dx^2}=\dfrac{d(\frac{dy}{dx})}{dt}\dfrac{dt}{dx}=\dfrac{1}{x^2}\dfrac{d^{2}y}{dt^2}$$

	原微分方程可化为:
	$$\dfrac{d^{2}y}{dt^2}+p\dfrac{dy}{dt}+qy=f(-e^t)$$
\end{definition}
\subsection{高阶常系数齐次线性微分方程}

\chapterimage{chap10.jpg}	
\chapter{无穷级数}
\begin{introduction}
	\item 常数项级数
	\item 收敛半径和收敛域
	\item 幂级数
	\item 和函数
	\item 函数展开式
	\item 傅里叶级数
\end{introduction}
\begin{definition}
	给定一个无穷数列 $u_{1},u_{2},u_{3},\dots,u_{n},\dots$,将其各项相加得到 $\sum\limits_{n=1}^{+\infty}u_{n}$,即:
	$$u_{1}+u_{2}+u_{3}+\dots+u_{n}+\dots=\sum_{n=1}^{+\infty}u_{n}$$
	我们将 $\sum\limits_{n=1}^{+\infty}u_{n}$ 称为无穷级数,简称为级数,其中 $u_{n}$ 是无穷级数的通项,如果 $u_{n}$ 是常数项,则称为常数项级数;如果 $u_{n}$ 是函数,则称为函数项级数
\end{definition}
\begin{definition}[级数敛散性]
	级数 $\sum\limits_{n=1}^{+\infty}u_{n}$ 的敛散性研究:
	\myspace{1}
	引入 $S_{n}=\sum\limits_{i=1}^{n}u_{i}$,我们称 $S_{n}$ 是无穷级数的部分和,我们定义:
	\myspace{1}
	(1). 当 $\lim\limits_{n\rightarrow +\infty}S_{n}=S$ 时,我们称级数 $\sum\limits_{n=1}^{+\infty}u_{n}$ 收敛.
	\myspace{1}
	(2). 当 $\lim\limits_{n\rightarrow +\infty}S_{n}=\infty$ 或者不存在时,我们称级数 $\sum\limits_{n=1}^{+\infty}u_{n}$ 发散.
\end{definition}
\begin{corollary}
	(1). 当 $\sum\limits_{n=1}^{+\infty}u_{n}$ 收敛时,我们有:  $\lim\limits_{n\rightarrow +\infty}u_{n}=0$ (必要条件)
	\myspace{1}
	(2). 当 $\sum\limits_{n=1}^{+\infty}u_{n},\sum\limits_{n=1}^{+\infty}v_{n}$ 收敛时,且这两个级数的和分别为 $S,T$, $\forall \alpha ,\beta \in \mathbb{R} ,\sum\limits_{n=1}^{+\infty}(\alpha u_{n}+\beta v_{n})$ 收敛,且级数和为 $\alpha S+\beta T$
	\myspace{1}
	(3). 如果存在去掉 $m$ 项的级数 $\sum\limits_{n=m}^{+\infty}u_{n}$ 收敛,原级数收敛;反之亦然
\end{corollary}
\section{常数项级数}

常数项级数敛散性判别方法

1. 正项级数判别

(1). 定义法\label{定义法}
\begin{theorem}[收敛原则]\label{the: 正向级数敛散性的判别方法}
	$\sum\limits_{n=1}^{+\infty}u_{n}$ 收敛 $\Leftrightarrow$ $\lim\limits_{n\rightarrow +\infty}S_{n}$ 有界
\end{theorem}
\begin{anymark}[证明]
	$\sum\limits_{n=1}^{+\infty}u_{n}$ 是正项级数,$u_{n}>0$,$S_{n}$ 单调递增

	如果 $S_{n}$ 有界,$\lim\limits_{n\rightarrow+\infty}S_{n}$ 存在,原级数收敛;反之亦然
\end{anymark}
(2). 比较判别法
\begin{theorem}
	存在无穷级数 $\sum\limits_{n=1}^{+\infty}u_{n},\sum\limits_{n=1}^{+\infty}v_{n}$,若从某一项起满足 $u_{n}<v_{n}$,我们有下面的推论:
	\myspace{1}
	若 $\sum\limits_{n=1}^{+\infty}u_{n}$ 发散, $\sum\limits_{n=1}^{+\infty}v_{n}$ 发散
	\myspace{1}
	若 $\sum\limits_{n=1}^{+\infty}v_{n}$ 收敛, $\sum\limits_{n=1}^{+\infty}u_{n}$ 收敛
\end{theorem}
(3). 比较判别法的极限形式
\begin{theorem}\label{the: 比较判别法的极限形式}
	$\lim\limits_{n\rightarrow+\infty}\dfrac{u_{n}}{v_{n}}=A$
	\myspace{1}
	(i). $A=0$,若 $\sum\limits_{n=1}^{+\infty}v_{n}$ 收敛,$\sum\limits_{n=1}^{+\infty}u_{n}$ 收敛

	(ii). $0<A<+\infty$,$\sum\limits_{n=1}^{+\infty}u_{n}$ 和 $\sum\limits_{n=1}^{+\infty}v_{n}$ 有相同的敛散性

	(iii). $A=+\infty$,若 $\sum\limits_{n=1}^{+\infty}v_{n}$ 发散,$\sum\limits_{n=1}^{+\infty}u_{n}$ 发散
\end{theorem}
(4). 比值判别法
\begin{theorem}
	$\lim\limits_{n\rightarrow+\infty}\dfrac{u_{n+1}}{u_{n}}=\rho$
	\myspace{1}
	(i). $\rho<1$, $\sum\limits_{n=1}^{+\infty}u_{n}$ 收敛

	(ii). $\rho>1$, $\sum\limits_{n=1}^{+\infty}u_{n}$ 发散

	(iii). $\rho=1$, $\sum\limits_{n=1}^{+\infty}u_{n}$ 敛散性不确定
\end{theorem}
(5). 根值判别法(柯西判别法)
\begin{theorem}
	$\lim\limits_{n\rightarrow+\infty}\sqrt[n]{u_{n}}=\rho$
	\myspace{1}
	(i). $\rho<1$, $\sum\limits_{n=1}^{+\infty}u_{n}$ 收敛

	(ii). $\rho>1$, $\sum\limits_{n=1}^{+\infty}u_{n}$ 发散

	(iii). $\rho=1$, $\sum\limits_{n=1}^{+\infty}u_{n}$ 敛散性不确定
\end{theorem}
2. 交错级数判别
\begin{theorem}[莱布尼茨判别法]
	$u_{n}$ 单调不增且 $\lim\limits_{n\rightarrow +\infty}u_{n}=0$  $\Rightarrow\sum\limits_{n=1}^{+\infty}(-1)^{n-1}u_{n}$ 收敛
\end{theorem}
3. 任意项级数判别
\begin{definition}
	$\sum\limits_{n=1}^{+\infty}|u_{n}|$ 是原级数的绝对值级数
	\myspace{1}
	(i). 如果 $\sum\limits_{n=1}^{+\infty}|u_{n}|$ 收敛,称其\textbf{绝对收敛}
	\myspace{1}
	(ii). 如果 $\sum\limits_{n=1}^{+\infty}|u_{n}|$ 发散,$\sum\limits_{n=1}^{+\infty}u_{n}$ 收敛,称其\textbf{条件收敛}
\end{definition}
\begin{theorem}
	1. $\sum\limits_{n=1}^{+\infty}|u_{n}|$ 收敛 $\Rightarrow\sum\limits_{n=1}^{+\infty}u_{n}$ 收敛

	2. $\sum\limits_{n=1}^{+\infty}u_{n}(u_{n}>0)$ 收敛 $\Rightarrow\sum\limits_{n=1}^{+\infty}u_{n}^2$ 收敛
\end{theorem}
\begin{anymark}[证明]

	1. 我们构造级数$v_{n}=\sum\limits_{n=1}^{+\infty}\frac{1}{2}(u_{n}+|u_{n}|)$,我们发现当$u_{n}<0$时,$v_{n}=0$;当$u_{n}>0$时,$v_{n}=u_{n}$,我们得到:
	$$0\leq v_{n}\leq |u_{n}|$$

	我们得到$v_{n}=\sum\limits_{n=1}^{+\infty}\frac{1}{2}(u_{n}+|u_{n}|)$收敛,由收敛级数的可加性得到:

	级数$\sum\limits_{n=1}^{+\infty}(2v_{n}-|u_{n}|)$ 收敛

	综上,$\sum\limits_{n=1}^{+\infty}u_{n}$ 收敛


	2. 我们由$\sum\limits_{n=1}^{+\infty}u_{n}$ 收敛 可以得到:
	$$\exists M>0,\ s.t. |u_{n}|<M\Rightarrow 0<u_{n}^2<Mu_{n}$$

	我们得到$\sum\limits_{n=1}^{+\infty}u_{n}^2$收敛
\end{anymark}
\myspace{1}
\section{幂级数}
\begin{definition}[幂级数]
	$$\sum\limits_{n=1}^{+\infty}u_{n}(x)=u_{1}(x)+u_{2}(x)+u_{3}(x)+\dots+u_{n}(x)+\dots$$
	级数的每一项都是函数项,函数的定义域 $I$,当 $x=x_{0}$ 时, $\sum\limits_{n=1}^{+\infty}u_{n}(x_{0})$ 就是常数项级数.
	\myspace{1}
	$\sum\limits_{n=1}^{+\infty}u_{n}(x_{0})$ 收敛的 $x_{0}$ 点被称为\textbf{收敛点},所有收敛点的集合被称为\textbf{收敛域}
\end{definition}
\begin{definition}
	幂级数标准形式:
	$$\sum\limits_{n=0}^{+\infty}u_{n}(x)=a_{0}+a_{1}x+a_{2}x^{2}+\dots+a_{n}x^{n}+\dots$$
	幂级数的一般形式:
	$$\sum\limits_{n=0}^{+\infty}u_{n}(x)=a_{0}+a_{1}(x-x_{0})+a_{2}(x-x_{0})^{2}+\dots+a_{n}(x-x_{0})^{n}+\dots$$
\end{definition}


\begin{theorem}[阿贝尔定理]\label{the: 幂级数收敛区间(收敛半径)和收敛域}
	\textbf{幂级数收敛域判定\ (阿贝尔定理)}:
	\myspace{1}
	当幂级数 $\sum\limits_{n=0}^{+\infty}u_{n}(x)$ 在 $x=x_{1}$ 处收敛时, $\forall x<|x_{1}|$,幂级数 $\sum\limits_{n=0}^{+\infty}u_{n}(x)$ 都收敛

	当幂级数 $\sum\limits_{n=0}^{+\infty}u_{n}(x)$ 在 $x=x_{2}$ 处发散时, $\forall x>|x_{2}|$,幂级数 $\sum\limits_{n=0}^{+\infty}u_{n}(x)$ 都发散.
\end{theorem}
对于标准幂级数求收敛域,我们利用公式法:
\begin{theorem}
	$\lim\limits_{n\rightarrow +\infty}|\dfrac{a_{n+1}}{a_{n}}|=\rho$
	$$R=\left\lbrace \begin{matrix}
			\frac{1}{\rho},\rho \neq 0 \\
			0,\rho = \infty            \\
			\infty,\rho = 0
		\end{matrix}\right. $$
\end{theorem}
我们将 $(-R,R)$ 称为幂级数的收敛区间

幂级数的收敛域为$(-R,R) \ or \ [-R,R]\ or\ (-R,R] \ or\ [-R,R) $
\myspace{1}
\section{幂级数求和函数}


\begin{definition}[幂级数的和函数]\label{def: 幂级数求和函数}
	在幂级数收敛域上,我们称 $S(x)$ 是幂级数的和函数:
	$$S(x)=\sum\limits_{n=0}^{+\infty}u_{n}(x)$$
\end{definition}
\begin{theorem}[可积性与可导性]

	(i). 幂级数和函数 $S(x)=\sum\limits_{n=0}^{+\infty}u_{n}(x)$ 在收敛域上连续

	(ii). 幂级数在收敛域 $I$ 上可积,有逐项积分公式(收敛域 $I'\geq I$)
	$$\int_{0}^{x}S(t)dt=\int_{0}^{x}(\sum\limits_{n=0}^{+\infty}a_{n}t^{n})dt=\sum\limits_{n=0}^{+\infty}\frac{a_{n}}{n+1}x^{n+1},x\in I$$

	(iii). 幂级数在收敛域 $I$ 上可导,有逐项求导公式(收敛域 $I'\leq I$)
	$$S'(x)=(\sum\limits_{n=0}^{+\infty}a_{n}x^{n})'=\sum\limits_{n=0}^{+\infty}na_{n}x^{n-1},x\in I$$
\end{theorem}
\subsection{重要展开式}
\begin{theorem}\label{the: 重要幂级数展开式}
	$$e^{x}=1+x+\frac{x^2}{2!}+\dots+\frac{x^n}{n!}+\dots ,-\infty<x<+\infty$$
	$$\frac{1}{1-x}=1+x+x^2+\dots+x^{n}+\dots,-1<x<1$$
	$$\frac{1}{1+x}=1-x+x^2-x^3+\dots+(-1)^{n}x^{n}+\dots,-1<x<1$$
	$$\ln(1+x)=x-\frac{x^2}{2}+\frac{x^3}{3}-\frac{x^4}{4}+\dots+\frac{(-1)^nx^{n+1}}{n+1}+\dots=\sum\limits_{n=1}^{+\infty}(-1)^{n-1}\frac{x^{n}}{n},-1<x\leq 1$$
	$$\sin x=x-\frac{x^3}{3!}+\frac{x^5}{5!}-\dots+\frac{(-1)^{n}x^{2n+1}}{(2n+1)!}+\dots=\sum\limits_{n=0}^{+\infty}(-1)^{n}\frac{x^{2n+1}}{(2n+1)!},-\infty<x<+\infty$$
	$$\cos x=1-\frac{x^2}{2!}+\frac{x^4}{4!}+\dots+\frac{(-1)^nx^{2n}}{2n!}+\dots=\sum\limits_{n=0}^{+\infty}(-1)^{n}\frac{x^{2n}}{(2n)!},-\infty<x<+\infty$$
	$$(1+x)^{\alpha}=1+\alpha x+\frac{\alpha (\alpha-1)}{2!}x^2+\frac{\alpha (\alpha-1)(\alpha-2)}{3!}x^3+\dots+\frac{\alpha (\alpha-1)(\alpha-2)\dots(\alpha-n+1)}{n!}x^n+\dots$$
\end{theorem}
\section{函数展开成幂级数}

\begin{definition}\label{函数展开成幂级数}
	泰勒级数:  ( $f(x)$ 在点 $x=x_{0}$ 处存在任意阶导数 )
	$$f(x)=\sum\limits_{n=0}^{+\infty}\frac{f^{(n)}(x_{0})}{n!}(x-x_{0})^n$$
	麦克劳林级数:  ( $f(x)$ 在点 $x=0$ 处存在任意阶导数 )
	$$f(x)=\sum\limits_{n=0}^{+\infty}\frac{f^{(n)}(0)}{n!}x^n$$
\end{definition}
\section{傅里叶级数}
将满足特定条件的周期函数用一个序列的正弦函数叠加表示,这种表示我们称为傅里叶级或者三角级数
\begin{definition}[傅里叶级数]\label{def: 傅里叶级数}
	设 $f(x)$是周期函数且满足狄利克雷收敛定律

	$f(x)=A_{0}+\sum\limits_{n=1}^{+\infty}A_{n}\sin (n\omega t+\varphi_{n})$是函数的傅里叶展开,展开式是傅里叶级数.
	通过一些变量代换,可以得到:
	$$f(x)=A_{0}+\sum\limits_{n=1}^{+\infty}(a_{n}\cos nx+b_{n}\sin nx)$$
\end{definition}
\textbf{三角函数族的正交性}
\begin{definition}
	$\{1,\sin x,\cos x,\sin 2x,\cos 2x,\dots,\sin nx,\cos nx\dots\}$被称为三角函数族,满足任意两个不同的函数之积在 $[-\pi,\pi]$ 上的定积分 $\int_{-\pi}^{\pi}f(x)g(x)dx=0$
\end{definition}
利用三角函数族的正交性这一性质,我们可以求出傅里叶级数的傅里叶系数:
$$\int_{-\pi}^{\pi}f(x)\sin nxdx=\int_{-\pi}^{\pi}A_{0}\sin nxdx+\int_{-\pi}^{\pi}b_{n}\sin^{2}nxdx \Rightarrow b_{n}=\frac{1}{\pi}\int_{-\pi}^{\pi}f(x)\sin nxdx$$
$$\int_{-\pi}^{\pi}f(x)\cos nxdx=\int_{-\pi}^{\pi}A_{0}\cos nxdx+\int_{-\pi}^{\pi}a_{n}\cos^{2}nxdx \Rightarrow a_{n}=\frac{1}{\pi}\int_{-\pi}^{\pi}f(x)\cos nxdx$$
$$\int_{-\pi}^{\pi}f(x)=\int_{-\pi}^{\pi}A_{0}dx\Rightarrow A_{0}=\frac{1}{2\pi}\int_{-\pi}^{\pi}f(x)dx$$
\begin{theorem}
	$$f(x)~\frac{a_{0}}{2}+\sum\limits_{n=1}^{+\infty}(a_{n}\cos nx+b_{n}\sin nx)$$
	其中傅里叶系数 $a_{n},b_{n}$ 表达式:
	$$\left\lbrace \begin{array}{l}
			a_{n}=\dfrac{1}{\pi}\int_{-\pi}^{\pi}f(x)\cos nxdx,n=0,1,2,\cdots \\
			b_{n}=\dfrac{1}{\pi}\int_{-\pi}^{\pi}f(x)\sin nxdx,n=1,2,\cdots
		\end{array}\right. $$
\end{theorem}
\begin{theorem}
	$$S(x)=\frac{a_{0}}{2}+\sum\limits_{n=1}^{+\infty}(a_{n}\cos nx+b_{n}\sin nx)$$
	$$S(x)=\left\lbrace \begin{array}{l}
			f(x),x\ is\ contiue \\
			\\
			\dfrac{\lim\limits_{x\rightarrow x^{+}}f(x)+\lim\limits_{x\rightarrow x^{-}}f(x)}{2},x\ is\ uncontiue
			\\
			\\
			\dfrac{\lim\limits_{x\rightarrow x^{+}}f(x)+\lim\limits_{x\rightarrow x^{-}}f(x)}{2},x=\pm\pi
		\end{array}\right. $$
\end{theorem}
\textbf{任意对称区间中的傅里叶展开}
\begin{definition}
	设 $f(x)$ 定义域为 $[-l,l]$,我们令 $t=\frac{x\pi}{l},t\in[-\pi,\pi]$

	我们得到:
	$$g(t)=\frac{a_{0}}{2}+\sum\limits_{n=1}^{+\infty}(a_{n}\cos nt+b_{n}\sin nt)\rightarrow f(x)=\frac{a_{0}}{2}+\sum\limits_{n=1}^{+\infty}(a_{n}\cos \dfrac{\pi nx}{l}+b_{n}\sin \dfrac{\pi nx}{l})$$
	$$\left\lbrace \begin{array}{l}
			a_{n}=\dfrac{1}{\pi}\int_{-\pi}^{\pi}g(t)\cos ntdt,n=0,1,2,\cdots \\
			b_{n}=\dfrac{1}{\pi}\int_{-\pi}^{\pi}g(t)\sin ntdt,n=1,2,\cdots
		\end{array}\right. $$
	我们进行变量代换:
	$$\left\lbrace \begin{array}{l}
			a_{n}=\dfrac{1}{l}\int_{-l}^{l}f(x)\cos \dfrac{\pi nx}{l}dx,n=0,1,2,\cdots \\
			\\
			b_{n}=\dfrac{1}{l}\int_{-l}^{l}f(x)\sin \dfrac{\pi nx}{l}dx,n=1,2,\cdots
		\end{array}\right. $$
\end{definition}
\textbf{正弦级数和余弦级数}

当 $f(x)$ 有奇偶性时,$a_{n}=0 \ or\ b_{n}=0$;
$$f(x)=f(-x),b_{n}=0,a_{n}=\dfrac{2}{l}\int_{-l}{l}f(x)\cos \dfrac{n\pi}{l}xdx$$
$$f(x)=-f(-x),a_{n}=0,b_{n}=\dfrac{2}{l}\int_{-l}{l}f(x)\sin \dfrac{n\pi}{l}xdx$$
\begin{anymark}[总结]\label{mark: $p$级数}
	1. p级数 $\sum\limits_{n=1}^{+\infty}\dfrac{1}{n^{p}}$,当 $p>1$ 时,级数收敛;当 $p\leq 1$ 时,级数发散

	2. 级数 $\sum\limits_{n=1}^{+\infty}\dfrac{1}{n!}$ 收敛,$\lim\limits_{n\rightarrow+\infty}S_{n}=e$
\end{anymark}






\chapterimage{chap11.jpg}
\chapter{空间解析几何}
\section{向量代数}
\begin{definition}
	1. 方向角: 非零向量 $a$ 与$x,y,z$ 轴所成夹角 $\alpha,\beta,\gamma$ 称为向量 $a$ 的方向角. 
	
	2. 方向余弦: $\cos \alpha,\cos \beta,\cos\gamma$ 称为向量 $a$ 的方向余弦.
	$$\cos \alpha=\frac{a_{x}}{|a|},\quad \cos \beta=\frac{a_{y}}{|a|},\quad \cos \gamma=\frac{a_{z}}{|a|}$$
	
	3. 投影: 
	$$Prj_{b}a=\frac{\textbf{a}\bullet\textbf{b}}{|b|}=\frac{a_{x}b_{x}+a_{y}b_{y}+a_{z}b_{z}}{\sqrt{b_{x}^2+b_{y}^2+b_{z}^2}}$$
\end{definition}
\section{空间平面和直线}
\subsection{平面}
\begin{definition}[	平面方程]
	
	1. 一般式:  $Ax+By+Cz+D=0$
	\myspace{1}
	
	2. 点法式:  $A(x-x_{0})+B(y-y_{0}+C(z-z_{0})=0$
	\myspace{1}
	
	3. 截距式:  $\dfrac{x}{a}+\dfrac{y}{b}+\dfrac{z}{c}=1$
	\myspace{1}	
	
	4. 三点式 : $\left|\begin{array}{lll}
		x-x_{1}&y-y_{1}&z-z_{1}\\
		x-x_{2}&y-y_{2}&z-z_{2}\\
		x-x_{3}&y-y_{3}&z-z_{3}
	\end{array}\right|=0$,(平面过不共线的三点 $P(x_{i},y_{i},z_{i})$)
\end{definition}
\subsection{直线}
\begin{definition}[直线方程]
	
	1. 一般式:  $\left\lbrace \begin{array}{c}
		A_{1}x+B_{1}y+C_{1}z=0,n_{1}=(A_{1},B_{1},C_{1})\\A_{2}x+B_{2}y+C_{2}z=0,n_{2}=(A_{2},B_{2},C_{2})
	\end{array}\right. $,$n_{1},n_{2}$ 不平行
	\myspace{1}
	
	2. 点向式:  $\dfrac{x-x_{0}}{l}=\dfrac{y-y_{0}}{m}=\dfrac{z-z_{0}}{n}$,$(x_{0},y_{0},z_{0})$ 是直线上的点,$(l,m,n)$ 是直线的方向向量
	\myspace{1}
	
	3. 参数式:  $\left\lbrace\begin{array}{l}
		x=x_{0}+lt\\
		y=y_{0}+mt\\
		z=z_{0}+nt
	\end{array} \right. $,$(x_{0},y_{0},z_{0})$ 是直线上的点,$(l,m,n)$ 是直线的方向向量
	\myspace{1}	
	
	4.两点式: $\dfrac{x-x_{1}}{x_{1}-x_{2}}=\dfrac{y-y_{1}}{y_{1}-y_{2}}=\dfrac{z-z_{0}}{z_{1}-z_{2}}$,$(x_{1},y_{1},z_{1})$和$(x_{2},y_{2},z_{2})$ 是直线上的点
\end{definition}
\section{空间曲面和曲线}
\begin{definition}[空间曲面和曲线]
	1. 空间曲线: 
	
	(i). 一般式: $\left\lbrace\begin{array}{l}
		F(x,y,z)=0\\G(x,y,z)=0
	\end{array} \right. $, 是两个曲面的交线
	
	(ii). 参数方程式: $\left\lbrace\begin{array}{l}
		x=f(t)\\y=g(t)\\z=h(t)
	\end{array} \right. $,$t\in[\alpha,\beta]$ 为参数
	
	(iii). 空间曲线在坐标面的投影: 
	$$\left\lbrace\begin{array}{l}
		F(x,y,z)=0\\G(x,y,z)=0
	\end{array} \right. \Rightarrow \left\lbrace\begin{array}{l}
		H(x,y)=0\\z=0
	\end{array} \right.$$
	
	2. 空间曲面
	$$F(x,y,z)=0$$
\end{definition}
\subsection{空间曲线的切线和法平面}
\begin{definition}[曲线切线和法平面]
	
	(i).参数方程式: $\left\lbrace\begin{array}{l}
		x=f(t)\\y=g(t)\\z=h(t)
	\end{array} \right. $ 在 $t=t_{0}$ 时,点 $P_{0}(x_{0},y_{0},z_{0})$ 处
	\myspace{1}
	
	切线的方向向量: $\textbf{n}=(f'(t_{0}),g'(t_{0}),h'(t_{0}))$
	\myspace{1}
	
	切线方程为: $ \dfrac{x-x_{0}}{f'(t_{0})}=\dfrac{y-y_{0}}{g'(t_{0})}=\dfrac{z-z_{0}}{h'(t_{0})}$
	\myspace{1}
	
	曲线法平面: $f'(t_{0})(x-x_{0})+g'(t_{0})(y-y_{0})+h'(t_{0})(z-z_{0})=0 $
	\myspace{1}
	
	(ii). 一般式: $\left\lbrace\begin{array}{l}
		F(x,y,z)=0\\G(x,y,z)=0
	\end{array} \right. $
	\myspace{1}
	
	切线的方向向量: $(\left| \begin{array}{ll}
		F_{y}'&F_{z}'\\G_{y}'&G_{z}'
	\end{array}\right| ,\left|\begin{array}{ll}
		F_{z}'&F_{x}'\\G_{z}'&G_{x}'
	\end{array} \right|,\left|\begin{array}{ll}
		F_{x}'&F_{y}'\\G_{x}'&G_{y}'
	\end{array} \right|)$
	\myspace{1}
	
	切线方程为: $\dfrac{x-x_{0}}{\left| \begin{array}{ll}
			F_{y}'&F_{z}'\\G_{y}'&G_{z}'
		\end{array}\right|}=\dfrac{y-y_{0}}{\left|\begin{array}{ll}
			F_{z}'&F_{x}'\\G_{z}'&G_{x}'
		\end{array} \right|}=\dfrac{z-z_{0}}{\left|\begin{array}{ll}
			F_{x}'&F_{y}'\\G_{x}'&G_{y}'
		\end{array} \right|}$
	\myspace{1}
	
	曲线法平面: $\left| \begin{array}{ll}
		F_{y}'&F_{z}'\\G_{y}'&G_{z}'
	\end{array}\right|(x-x_{0})+\left|\begin{array}{ll}
		F_{z}'&F_{x}'\\G_{z}'&G_{x}'
	\end{array} \right|(y-y_{0})+\left|\begin{array}{ll}
		F_{x}'&F_{y}'\\G_{x}'&G_{y}'
	\end{array} \right|(z-z_{0})=0$
	
	
\end{definition}
\subsection{空间曲面的切平面和法线}
\begin{definition}
	曲面的切平面和法线
	
	1. 曲面方程:  $F(x,y,z)=0$
	\myspace{1}
	
	法向量: $\textbf{n}=(F_{x}'(x,y,z),F_{y}'(x,y,z),F_{z}'(x,y,z))$
	\myspace{1}
	
	切平面方程: $F_{x}'(x,y,z)(x-x_{0})+F_{y}'(x,y,z)(y-y_{0})+F_{z}'(x,y,z)(z-z_{0})=0$
	\myspace{1}
	
	法线方程: $\dfrac{x-x_{0}}{F_{x}'(x,y,z)}=\dfrac{y-y_{0}}{F_{y}'(x,y,z)}=\dfrac{z-z_{0}}{F_{z}'(x,y,z)}$
	\myspace{1}
	
	2. 曲面方程:  $z=f(x,y)$
	\myspace{1}
	
	法向量: $\textbf{n}=(f_{x}'(x,y),f_{y}'(x,y),-1)$
	\myspace{1}
	
	切平面方程: $f_{x}'(x,y)(x-x_{0})+f_{y}'(x,y)(y-y_{0})-(z-z_{0})=0$
	\myspace{1}
	
	法线方程: $\dfrac{x-x_{0}}{f_{x}'(x,y)}=\dfrac{y-y_{0}}{f_{y}'(x,y)}=\dfrac{z-z_{0}}{-1}$
\end{definition}
\section{场论初步}
\subsection{方向导数}
\begin{definition}[方向导数]
	
	设三元函数 $u=u(x,y,z)$ 在点 $P(x_{0},y_{0},z_{0})$ 的某空间邻域内 $U\subset R^3$ 有定义,$l$ 是从 $P_{0}$ 出发的一条射线,$P(x,y,z)$ 为 $l$ 上且在 $U$ 中的任意一点,我们有: 
	$$\left\lbrace \begin{array}{l}
		x-x_{0}=\Delta x=t\cos \alpha\\
		y-y_{0}=\Delta y=t\cos \beta\\
		z-z_{0}=\Delta z=t\cos \gamma
	\end{array}\right. $$ 
	
	$t=\sqrt{(\Delta x)^2+(\Delta y)^2+(\Delta z)^2}$ 表示$|PP_{0}|$,如果下面极限存在: 
	$$\lim\lim\limits_{t\rightarrow 0}\frac{u(P)-u(P_{0})}{t}=\lim\lim\limits_{t\rightarrow 0}\frac{u(x_{0}+t\cos \alpha,y_{0}+t\cos \beta,z_{0}+t\cos \gamma)-u(x_{0},y_{0},z_{0})}{t}$$
	
	我们将此极限称为 $u=f(x,y,z)$ 在$P_{0}$ 处沿着 $l$ 的方向导数,记作 $\frac{\partial u}{\partial l}|_{P_{0}}$
\end{definition}
\begin{theorem}[方向导数计算公式]
	$$\frac{\partial u}{\partial l}|_{P_{0}}=u_{x}'\cos \alpha+u_{y}'\cos \beta+u_{z}'\cos \gamma$$
	
	其中$\cos \alpha,\cos \beta,\cos \gamma$ 为方向 $l$ 的方向余弦.
\end{theorem}
\subsection{梯度}
\begin{definition}[梯度]
	
	设三元函数$u=u(x,y,z)$ 在点 $P(x_{0},y_{0},z_{0})$ 处具有一阶偏导数,定义下面为$u=u(x,y,z)$ 在 $ P_{0}(x_{0},y_{0},z_{0})$ 处的梯度: 
	$$\textbf{guad}\ u|_{P_{0}}=(u_{x}'(P_{0}),u_{y}'(P_{0}),u_{z}'(P_{0}))$$
	
	
	梯度和方向导数之间的关系: 
	$$\frac{\partial u}{\partial l}|_{P_{0}}=\textbf{guad}\ u|_{P_{0}}\bullet \textbf{l}=|\textbf{guad}\ u|_{P_{0}}|l|\cos \theta$$
\end{definition}
\subsection{散度和旋度}
\begin{definition}[散度和旋度]
	设向量场 $A(x,y,z)=(P(x,y,z),Q(x,y,z),R(x,y,z))$
	
	散度: 
	$$div\ A=\frac{\partial P}{\partial x}+\frac{\partial Q}{\partial y}+\frac{\partial R}{\partial z}$$
	
	旋度: 
	$$rot \ A=\left| \begin{array}{lll}
		i&j&k\\\dfrac{\partial}{\partial x}&\dfrac{\partial}{\partial y}&\dfrac{\partial}{\partial z}\\P&Q&R
	\end{array}\right| $$
\end{definition}
\chapterimage{chap12.jpg}
\chapter{三重积分}
\begin{definition}[三重积分]
	$$\iiint\limits_{\Omega}f(x,y,z)d\nu$$
	
	我们将 $f(x,y,z)$ 看作空间区域 $d\nu$ 内的密度,积分表示的就是空间区域的质量,$M=\iiint\limits_{\Omega}f(x,y,z)d\nu$,特别的,当 $f(x,y,z)=1$ 时,三重积分表示的积分区域 $\Omega$ 的体积.
\end{definition}
\section{三重积分对称性}

\subsection{普通对称性}

\begin{definition}
	设 $\Omega$ 关于平面 $xoz$ 对称,我们有: 
	$$\iiint\limits_{\Omega}f(x,y,z)d\nu=\left\lbrace \begin{array}{l}
		2\iiint\limits_{\Omega_{1}}f(x,y,z)d\nu,f(x,y,z)=f(x,-y,z)\\
		0,f(x,y,z)=-f(x,-y,z)
	\end{array}\right. $$
\end{definition}
\subsection{轮换对称性}
\begin{definition}
	若将 $x,y,z$ 任意两个交换位置后 积分区域 $\Omega$ 保持不变,我们有: 
	$$\iiint\limits_{\Omega}f(x)d\nu=\iiint\limits_{\Omega}f(y)d\nu=\iiint\limits_{\Omega}f(z)d\nu$$
\end{definition}

\section{三重积分计算方法}

\subsection{直角坐标系}
\begin{definition}
	1. 先一后二: 
	$$\iiint\limits_{\Omega}f(x,y,z)d\nu=\iint\limits_{D}d\sigma \int_{z_{1}(x,y)}^{z_{2}(x,y)}f(x,y,z)dz$$
	
	适用于空间区域无侧面,能"压扁"到一个坐标平面内.
	
	2. 先二后一法: 
	$$\iiint\limits_{\Omega}f(x,y,z)d\nu=\int_{a}^{b}dz\iint\limits_{D_{z}}f(x,y,z)d\sigma$$
	
	适用于旋转体,不能“压扁”到一个坐标平面
\end{definition}
\subsection{柱面坐标系}
\begin{definition}[柱坐标替换]
	$$\iiint\limits_{\Omega}f(x,y,z)d\nu=\int_{a}^{b}dz\iint\limits_{D_{z}}f(x,y,z)d\sigma$$
	
	利用极坐标和直角坐标公式转换: 
	$$\iiint\limits_{\Omega}f(x,y,z)d\nu=\iint\limits_{D_{r\theta}}drd\theta \int_{z_{1}(r,\theta)}^{z_{2}(r,\theta)}rf(r\cos \theta,r\sin\theta,z)dz$$
\end{definition}
\subsection{球面坐标系}
\begin{definition}[球面坐标替换]
	令 $\left\lbrace \begin{array}{l}
		x=r\sin\varphi\cos\theta\\
		y=r\sin\varphi\sin\theta \\\
		z=r\cos\varphi
	\end{array}\right. $,我们有: $d\nu=r^2\sin\varphi drd\theta$
	$$\iiint\limits_{\Omega}f(x,y,z)d\nu=\iiint\limits_{\Omega}r^2\sin\varphi f(r\sin\varphi\cos\theta,r\sin\varphi\sin\theta,r\cos\varphi) drd\theta d\varphi$$
	
	其中: $\varphi\in[0,\pi],\quad \theta\in[0,2\pi]$
\end{definition}
\chapterimage{chap13.jpg}
\chapter{第一型曲线和曲面积分}
\section{第一型曲线积分}
\begin{definition}[第一型曲线积分]
	$$\int_{\Gamma}f(x,y)ds\quad \int_{\Gamma}f(x,y,z)ds$$
	我们将 $f(x,y,z)$ 称为曲线的线密度,第一型曲线积分的意义是求曲线的质量,类比定积分,定积分是在直线上积分,曲线积分则是在曲线上积分.
	
	特别的,我们有: $\int_{\Gamma}ds=L_{\Gamma}$
\end{definition}
\begin{theorem}[曲线积分的求解]
	
	1. 空间曲线
	
	$$\left\lbrace \begin{array}{l}
		x=x(t)\\
		y=y(t)\\
		z=z(t)
	\end{array}\right. ,t\in[\alpha,\beta]$$
	
	我们有: $ds=\sqrt{[x'(t)]^{2}+[y'(t)]^{2}+[z'(t)]^{2}}dt$
	
	$$\int_{L}f(x,y,z)ds=\int_{\alpha}^{\beta}f(x(t),y(t),z(t))\sqrt{[x'(t)]^{2}+[y'(t)]^{2}+[z'(t)]^{2}}dt$$
	
	2. 平面曲线
	
	(i). $L:\ y=f(x),\quad x\in[a,b]$
	
	$$\int_{L}f(x,y)ds=\int_{a}^{b}f(x,y)\sqrt{1+[f'(x)]^2}dx$$
	
	(ii). $L:\ \left\lbrace \begin{array}{l}
		x=x(t)\\
		y=y(t)
	\end{array}\right.\quad t\in[\alpha,\beta]$
	
	$$\int_{L}f(x,y)ds=\int_{\alpha}^{\beta}f(x(t),y(t))\sqrt{[x'(t)]^2+[y'(t)]^2}dt$$
	
	(iii). $L:\ r=r(\theta),\quad \theta\in[\theta_{1},\theta_{2}]$
	
	$$\int_{L}f(x,y)ds=\int_{\theta_{1}}^{\theta_{2}}f(r\cos \theta,r\sin\theta)\sqrt{[r(\theta)]^2+[r'(\theta)]^2}d\theta$$
\end{theorem}

\section{第一型曲面积分}
\begin{definition}[第一型曲面积分]
	$$\iint_{\Sigma}f(x,y,z)dS$$
	我们将 $f(x,y,z)$ 称为曲面的面密度,第一型曲面积分的意义是求曲面的质量,类比二重积分,二重积分是在平面上积分,曲面积分则是在曲面上积分.
	
	特别的,我们有: $\iint_{\Sigma}dS=S_{\Sigma}$
\end{definition}
\begin{theorem}
	$$z=f(x,y)\quad F(x,y,z)=0$$
	
	我们将曲面 $\Sigma$ 投影到任意一个平面,这里以 $xoy$ 为例,$dS=\sqrt{1+(z_{x}')^2+(z_{y}')^2}d\sigma$
	
	$$\iint_{\Sigma}f(x,y,z)dS=\iint_{D_{xy}}f(x,y,z)\sqrt{1+(z_{x}')^2+(z_{y}')^2}d\sigma$$
\end{theorem}
\subsection{应用}
1. 重心、形心

2. 转动惯量
\begin{definition}[转动惯量: $I=mr^2$]
	
	(i). 平面物体: 
	
	对 $x$ 轴: $I_{x}=\iint\limits_{D}y^2\rho(x,y)d\sigma$
	
	对 $y$ 轴: $I_{y}=\iint\limits_{D}x^2\rho(x,y)d\sigma$
	
	对 坐标原点$O$: $I_{O}=\iint\limits_{D}(x^2+y^2)\rho(x,y)d\sigma$
	
	(ii). 空间物体: 
	
	对 $x$ 轴: $I_{x}=\iiint\limits_{\Omega}(y^2+z^2)\rho(x,y,z)d\nu$
	
	对 $y$ 轴: $I_{y}=\iiint\limits_{\Omega}(x^2+z^2)\rho(x,y,z)d\nu$
	
	对 $z$ 轴: $I_{z}=\iiint\limits_{\Omega}(x^2+y^2)\rho(x,y,z)d\nu$
	
	对坐标原点 $O$: $I_{O}=\iiint\limits_{\Omega}(x^2+y^2+z^2)\rho(x,y,z)d\nu$
	
	(iii). 光滑曲线
	
	对 $x$ 轴: $I_{x}=\int\limits_{L}(y^2+z^2)\rho(x,y,z)ds$
	
	对 $y$ 轴: $I_{y}=\int\limits_{L}(x^2+z^2)\rho(x,y,z)ds$
	
	对 $z$ 轴: $I_{z}=\int\limits_{L}(x^2+y^2)\rho(x,y,z)ds$
	
	对坐标原点 $O$: $I_{O}=\int\limits_{L}(x^2+y^2+z^2)\rho(x,y,z)ds$
	
	(iiii). 曲面
	
	对 $x$ 轴: $I_{x}=\iint\limits_{\Sigma}(y^2+z^2)\rho(x,y,z)dS$
	
	对 $y$ 轴: $I_{y}=\iint\limits_{\Sigma}(x^2+z^2)\rho(x,y,z)dS$
	
	对 $z$ 轴: $I_{z}=\iint\limits_{\Sigma}(x^2+y^2)\rho(x,y,z)dS$
	
	对坐标原点 $O$: $I_{O}=\iint\limits_{\Sigma}(x^2+y^2+z^2)\rho(x,y,z)dS$
	
\end{definition}

3. 引力
\begin{definition}[引力公式: $F=\frac{GMm}{r^2}$]
	
	(i). $xoy$ 平面
	$$F_{x}=GM\iint\limits_{D}\frac{\rho(x,y)(x-x_{0})}{[(x-x_{0})^2+(y-y_{0})^2+(z-z_{0})^2]^{\frac{3}{2}}}d\sigma$$
	$$F_{y}=GM\iint\limits_{D}\frac{\rho(x,y)(y-y_{0})}{[(x-x_{0})^2+(y-y_{0})^2+(z-z_{0})^2]^{\frac{3}{2}}}d\sigma$$
	$$F_{z}=GM\iint\limits_{D}\frac{\rho(x,y)(z-z_{0})}{[(x-x_{0})^2+(y-y_{0})^2+(z-z_{0})^2]^{\frac{3}{2}}}d\sigma,z=0$$
	
	(ii). 空间物体
	$$F_{x}=GM\iiint\limits_{\Omega}\frac{\rho(x,y)(x-x_{0})}{[(x-x_{0})^2+(y-y_{0})^2+(z-z_{0})^2]^{\frac{3}{2}}}d\nu$$
	$$F_{y}=GM\iiint\limits_{\Omega}\frac{\rho(x,y)(y-y_{0})}{[(x-x_{0})^2+(y-y_{0})^2+(z-z_{0})^2]^{\frac{3}{2}}}d\nu$$
	$$F_{z}=GM\iiint\limits_{\Omega}\frac{\rho(x,y)(z-z_{0})}{[(x-x_{0})^2+(y-y_{0})^2+(z-z_{0})^2]^{\frac{3}{2}}}d\nu$$
	
	(iii). 曲线
	$$F_{x}=GM\int\limits_{L}\frac{\rho(x,y)(x-x_{0})}{[(x-x_{0})^2+(y-y_{0})^2+(z-z_{0})^2]^{\frac{3}{2}}}ds$$
	$$F_{y}=GM\int\limits_{L}\frac{\rho(x,y)(y-y_{0})}{[(x-x_{0})^2+(y-y_{0})^2+(z-z_{0})^2]^{\frac{3}{2}}}ds$$
	$$F_{z}=GM\int\limits_{L}\frac{\rho(x,y)(z-z_{0})}{[(x-x_{0})^2+(y-y_{0})^2+(z-z_{0})^2]^{\frac{3}{2}}}ds$$
	
	(iiii). 曲面
	$$F_{x}=GM\iint\limits_{\Sigma}\frac{\rho(x,y)(x-x_{0})}{[(x-x_{0})^2+(y-y_{0})^2+(z-z_{0})^2]^{\frac{3}{2}}}dS$$
	$$F_{y}=GM\iint\limits_{\Sigma}\frac{\rho(x,y)(y-y_{0})}{[(x-x_{0})^2+(y-y_{0})^2+(z-z_{0})^2]^{\frac{3}{2}}}dS$$
	$$F_{z}=GM\iint\limits_{\Sigma}\frac{\rho(x,y)(z-z_{0})}{[(x-x_{0})^2+(y-y_{0})^2+(z-z_{0})^2]^{\frac{3}{2}}}dS$$
\end{definition}

\chapterimage{chap14.jpg}
\chapter{第二型曲线和曲面积分}
\section{第二型曲线积分}
\begin{definition}[第二型曲线积分]
	物理意义: 变力沿曲线做功
	$$\int_{L}P(x,y)dx+Q(x,y)dy \quad \int_{\Gamma}P(x,y,z)dx+Q(x,y,z)dy+R(x,y,z)dz$$
\end{definition}
\subsection{格林公式}
\begin{theorem}
	格林公式: (第二型曲线积分 $\rightarrow$ 二重积分 )
	$$\oint_{L}P(x,y)dx+Q(x,y)dy=\oiint\limits_{D}(\frac{\partial Q}{\partial x}-\frac{\partial P}{\partial y})d\sigma$$
	
	前提条件: $L$取正向,左手在内侧,$L$闭合.一般适用于平面曲线
\end{theorem}
\subsection{斯托克斯公式}
\begin{theorem}[斯托克斯公式]
	斯托克斯公式: (第二型曲线积分 $\rightarrow$ 第一型曲面积分 )
	$$\oint_{\Gamma}Pdx+Qdy+Rdz=\iint\limits_{\Sigma}\left| \begin{array}{lll}
		\cos\alpha&\cos\beta&\cos\gamma\\
		\dfrac{\partial}{\partial x}&\dfrac{\partial }{\partial y}&\dfrac{\partial}{\partial z}\\
		P&Q&R
	\end{array}\right|dS $$
\end{theorem}
\section{第二型曲面积分}
\begin{definition}[第二型曲面积分]
	物理意义: 向量场通过一个曲面的通量
	$$\iint\limits_{\Omega}P(x,y,z)dydz+Q(x,y,z)dxdz+R(x,y,z)dxdy$$
\end{definition}
\subsection{高斯公式}
\begin{theorem}[高斯公式]
	高斯公式: (第二型曲面积分 $\rightarrow$ 三重积分)
	$$\iint\limits_{\Omega}P(x,y,z)dydz+Q(x,y,z)dxdz+R(x,y,z)dxdy=\iiint\limits_{\Omega}(\frac{\partial P}{\partial x}+\frac{\partial Q}{\partial y}+\frac{\partial R}{\partial z})d\nu$$
\end{theorem}