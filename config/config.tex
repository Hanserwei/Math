
% 封面设置
\title{\LaTeX\quad \quad \textbf{\textit{main}}} % 标题
\subtitle{{\fontspec{Times New Roman}\textit{Sun for morning, moon for night, and you forever.}}} % 副标题
\author{Lyshmily.Y \& 木易} % 作者
\institute{Lyshmily.Y} % 机构
\date{\today} % 日期
\version{V.1.0} % 版本
\bioinfo{邮箱}{\email{yjlpku.outlook.com} \& \email{845307723@qq.com}} % 信息
\extrainfo{在没有结束前,总要做很多没有意义的事,这样才可以在未来某一天,用这些无意义的事去堵住那些讨厌的缺口} % 箴言
\logo{pku.jpg} % 徽标
\cover{coverc.jpg} % 封面
% 颜色包
\usepackage{soul, color, xcolor}
% 表格, `diagbox` 斜对角线, `makecell` 格式化单元格
\usepackage{array, tabularx, tabularray, longtable, diagbox, makecell,tcolorbox}
% 文字字体段落
\usepackage{silence}
\WarningFilter{latexfont}{Font shape}
\WarningFilter{latexfont}{Some font}
% 支持插入动画
\usepackage{animate}
% 支持插入图片, 子图, 控制浮动体位置, 虚线
\makeatletter
\let\c@lofdepth\relax
\let\c@lotdepth\relax
\makeatother
\usepackage{graphicx, subfigure, float, arydshln}
% 数学包和箭头选项
\usepackage{amsmath, amssymb, extarrows}
% 页眉页脚
\usepackage{fancyhdr}
% 超链接和目录
\usepackage{hyperref, tocbibind}
% tikz 绘图
\usepackage{tikz}
\usetikzlibrary{arrows.meta}
\usetikzlibrary{patterns}

%% 设置
\setcounter{tocdepth}{2} % 目录深度
\graphicspath{ {figure/},{../figure/}, {figure/cover}, {../config/} } % 图片位置
\renewcommand{\arraystretch}{1.2} % 表格行高

% 自定义颜色
\definecolor{orange}{HTML}{ff9f1a}
\definecolor{skyblue}{HTML}{0097e6}
\definecolor{cyan}{HTML}{1289A7}
\definecolor{slate}{HTML}{7158e2}
\definecolor{violet}{HTML}{ED4C67}
\definecolor{turquoise}{HTML}{487eb0}
\definecolor{purplea}{HTML}{5758BB}
\definecolor{purpleb}{HTML}{833471}
\definecolor{purplec}{HTML}{006266}
\definecolor{crimson}{HTML}{EA2027}
\colorlet{coverlinecolor}{cyan}

%% 自定义命令

\def\d{\textup{d}} % 直立体 d 用于微分符号 dx
\def\artanh{\operatorname{artanh}}  % 定义artanh函数
\def\arsinh{\operatorname{arsinh}}  % 定义arsinh函数
\def\arcosh{\operatorname{arcosh}}  % 定义arcosh函数

% 指定颜色的数学公式框
\newcommand{\mathcolorbox}[2]{\colorbox{#1}{$\displaystyle #2$}}
% 空行 \myspace{n}
\newcommand{\myspace}[1]{\par\vspace{#1\baselineskip}}
% 大小写罗马数字 \Rmnum{n} \rmnum{n}
\makeatletter
\newcommand{\rmnum}[1]{\romannumeral #1}
\newcommand{\Rmnum}[1]{\expandafter\@slowromancap\romannumeral #1@}
\makeatother